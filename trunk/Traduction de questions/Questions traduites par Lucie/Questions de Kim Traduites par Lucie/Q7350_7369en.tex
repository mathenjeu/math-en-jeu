\documentclass[letterpaper, 12pt]{article}
\usepackage[french]{babel}

\usepackage{amsmath,amsfonts,amsthm,amssymb,graphicx,multirow,hyperref,color}
\usepackage[latin1]{inputenc}

\pagestyle{plain}

\setlength{\topmargin}{-2cm}
\setlength{\textheight}{23.5cm}
\setlength{\textwidth}{12cm}
\setlength{\oddsidemargin}{-1cm}
\setlength{\parindent}{0pt}

\begin{document}

7350-- Which digit is 9 in the number 319,56?\\

a) hundreds digit\\
b) hundredths digit\\
c) tenths digit\\
d) units digit\\

Answer : d)\\

Feedback :\\
\begin{center}
% use packages: array
\begin{tabular}{|rrr|rrr|rrr|}
\hline
\multicolumn{6}{|c|}{integers} &\multicolumn{3}{|c|}{decimals} \\
\hline
\multicolumn{3}{|c|}{Class of} &\multicolumn{3}{|c|}{Class of} &  \multicolumn{3}{c|}{} \\
\multicolumn{3}{|c|}{thousands} &\multicolumn{3}{|c|}{units} &  \multicolumn{3}{c|}{} \\
\hline
H & T & U &H & T & U, & T\up{th} & \textbf{H\up{th}} & K\up{th} \\
\hline
\hline
 & & & 3 & 1 & \textbf{9}, & 5 & 6 & \\
\hline
\end{tabular}
\end{center}

\scriptsize
\begin{center}
% use packages: array
\begin{tabular}{ll}
H : Hundreds & T\up{th} : Tenths\\
T : Tens & H\up{th} :Hundredths\\
U : Units & K\up{e} : Thousandths\\
\end{tabular}
\end{center}

\normalsize
The answer is d).\\

7351-- Which digit is 0 in the number 125 410?\\

a) hundreds digit\\
b) hundredths digit\\
c) tenths digit\\
d) units digit\\

Answer : d)\\

Feedback :\\
\begin{center}
% use packages: array
\begin{tabular}{|rrr|rrr|rrr|}
\hline
\multicolumn{6}{|c|}{integers} &\multicolumn{3}{|c|}{decimals} \\
\hline
\multicolumn{3}{|c|}{Class of} &\multicolumn{3}{|c|}{Class of} &  \multicolumn{3}{c|}{} \\
\multicolumn{3}{|c|}{thousands} &\multicolumn{3}{|c|}{units} &  \multicolumn{3}{c|}{} \\
\hline
H & T & U &H & T & U, & T\up{th} & \textbf{H\up{th}} & K\up{th} \\
\hline
\hline
 1 & 2 & 5 & 4 & 1 & \textbf{0} & & &  \\
\hline
\end{tabular}
\end{center}

\scriptsize
\begin{center}
% use packages: array
\begin{tabular}{ll}
H : Hundreds & T\up{th} : Tenths\\
T : Tens & H\up{th} :Hundredths\\
U : Units & K\up{e} : Thousandths\\
\end{tabular}
\end{center}

\normalsize
The answer is d).\\

7352-- Which digit is 3 in the number 132,06?\\

a) hundreds digit\\
b) hundredths digit\\
c) tens digit\\
d) units digit\\

Answer : c)\\

Feedback :\\
\begin{center}
% use packages: array
\begin{tabular}{|rrr|rrr|rrr|}
\hline
\multicolumn{6}{|c|}{integers} &\multicolumn{3}{|c|}{decimals} \\
\hline
\multicolumn{3}{|c|}{Class of} &\multicolumn{3}{|c|}{Class of} &  \multicolumn{3}{c|}{} \\
\multicolumn{3}{|c|}{thousands} &\multicolumn{3}{|c|}{units} &  \multicolumn{3}{c|}{} \\
\hline
H & T & U &H & T & U, & T\up{th} & \textbf{H\up{th}} & K\up{th} \\
\hline
\hline
 & & &1 & \textbf{3} & 2, & 0 & 6 & \\
\hline
\end{tabular}
\end{center}

\scriptsize
\begin{center}
% use packages: array
\begin{tabular}{ll}
H : Hundreds & T\up{th} : Tenths\\
T : Tens & H\up{th} :Hundredths\\
U : Units & K\up{e} : Thousandths\\
\end{tabular}
\end{center}

\normalsize
The answer is c).\\



7353-- Which digit is  1 in the number 409 917?\\

a) hundreds digit\\
b) tens digit\\
c) units digit\\
d) thousands digit\\

Answer : b)\\

Feedback :\\
\begin{center}
% use packages: array
\begin{tabular}{|rrr|rrr|rrr|}
\hline
\multicolumn{6}{|c|}{integers} &\multicolumn{3}{|c|}{decimals} \\
\hline
\multicolumn{3}{|c|}{Class of} &\multicolumn{3}{|c|}{Class of} &  \multicolumn{3}{c|}{} \\
\multicolumn{3}{|c|}{thousands} &\multicolumn{3}{|c|}{units} &  \multicolumn{3}{c|}{} \\
\hline
H & T & U &H & T & U, & T\up{th} & \textbf{H\up{th}} & K\up{th} \\
\hline
\hline
4 & 0 & 9 & 9 & \textbf{1} & 7 & & & \\
\hline
\end{tabular}
\end{center}

\scriptsize
\begin{center}
% use packages: array
\begin{tabular}{ll}
H : Hundreds & T\up{th} : Tenths\\
T : Tens & H\up{th} :Hundredths\\
U : Units & K\up{e} : Thousandths\\
\end{tabular}
\end{center}

\normalsize
The answer is b).\\





7354-- Which digit is  3 in the number 309,49?\\

a) hundreds digit\\
b) tens digit\\
c) tenths digit\\
d) units digit\\

Answer : a)\\

Feedback :\\
\begin{center}
% use packages: array
\begin{tabular}{|rrr|rrr|rrr|}
\hline
\multicolumn{6}{|c|}{integers} &\multicolumn{3}{|c|}{decimals} \\
\hline
\multicolumn{3}{|c|}{Class of} &\multicolumn{3}{|c|}{Class of} &  \multicolumn{3}{c|}{} \\
\multicolumn{3}{|c|}{thousands} &\multicolumn{3}{|c|}{units} &  \multicolumn{3}{c|}{} \\
\hline
H & T & U &H & T & U, & T\up{th} & \textbf{H\up{th}} & K\up{th} \\
\hline
\hline
& & &\textbf{3} & 0 & 9, & 4 & 9 & \\
\hline
\end{tabular}
\end{center}

\scriptsize
\begin{center}
% use packages: array
\begin{tabular}{ll}
H : Hundreds & T\up{th} : Tenths\\
T : Tens & H\up{th} :Hundredths\\
U : Units & K\up{e} : Thousandths\\
\end{tabular}
\end{center}

\normalsize
The answer is a).\\





7355-- Which digit is 0 in the number 14 075?\\

a) hundreds digit\\
b) tens digit\\
c) tens of thousands digit\\
d) thousands digit\\

Answer: a)\\

Feedback :\\
\begin{center}
% use packages: array
\begin{tabular}{|rrr|rrr|rrr|}
\hline
\multicolumn{6}{|c|}{integers} &\multicolumn{3}{|c|}{decimals} \\
\hline
\multicolumn{3}{|c|}{Class of} &\multicolumn{3}{|c|}{Class of} &  \multicolumn{3}{c|}{} \\
\multicolumn{3}{|c|}{thousands} &\multicolumn{3}{|c|}{units} &  \multicolumn{3}{c|}{} \\
\hline
H & T & U &H & T & U, & T\up{th} & \textbf{H\up{th}} & K\up{th} \\
\hline
\hline
 & 1 & 4 & \textbf{0} & 7 & 5 & & & \\
\hline
\end{tabular}
\end{center}

\scriptsize
\begin{center}
% use packages: array
\begin{tabular}{ll}
H : Hundreds & T\up{th} : Tenths\\
T : Tens & H\up{th} :Hundredths\\
U : Units & K\up{e} : Thousandths\\
\end{tabular}
\end{center}

\normalsize
The answer is a).\\






7356-- Which digit is 9 in the numbere 129 758?\\

a) hundreds of thousands digit\\
b) tens digit\\
c) tens of thousands digit\\
d) thousands digit\\

Answer : d)\\

Feedback :\\
\begin{center}
% use packages: array
\begin{tabular}{|rrr|rrr|rrr|}
\hline
\multicolumn{6}{|c|}{integers} &\multicolumn{3}{|c|}{decimals} \\
\hline
\multicolumn{3}{|c|}{Class of} &\multicolumn{3}{|c|}{Class of} &  \multicolumn{3}{c|}{} \\
\multicolumn{3}{|c|}{thousands} &\multicolumn{3}{|c|}{units} &  \multicolumn{3}{c|}{} \\
\hline
H & T & U &H & T & U, & T\up{th} & \textbf{H\up{th}} & K\up{th} \\
\hline
\hline
 1 & 2 & \textbf{9} & 7 & 5 & 8 & & & \\
\hline
\end{tabular}
\end{center}

\scriptsize
\begin{center}
% use packages: array
\begin{tabular}{ll}
H : Hundreds & T\up{th} : Tenths\\
T : Tens & H\up{th} :Hundredths\\
U : Units & K\up{e} : Thousandths\\
\end{tabular}
\end{center}

\normalsize
The answer is d).\\






7357-- Which digit is 1 in the number 1025?\\

a) hundreds digit\\
b) tens digit\\
c) units digit\\
d) thousands digit\\

Answer : d)\\

Feedback :\\
\begin{center}
% use packages: array
\begin{tabular}{|rrr|rrr|rrr|}
\hline
\multicolumn{6}{|c|}{integers} &\multicolumn{3}{|c|}{decimals} \\
\hline
\multicolumn{3}{|c|}{Class of} &\multicolumn{3}{|c|}{Class of} &  \multicolumn{3}{c|}{} \\
\multicolumn{3}{|c|}{thousands} &\multicolumn{3}{|c|}{units} &  \multicolumn{3}{c|}{} \\
\hline
H & T & U &H & T & U, & T\up{th} & \textbf{H\up{th}} & K\up{th} \\
\hline
\hline
  &  & \textbf{1} & 0 & 2 & 5 & & & \\
\hline
\end{tabular}
\end{center}

\scriptsize
\begin{center}
% use packages: array
\begin{tabular}{ll}
H : Hundreds & T\up{th} : Tenths\\
T : Tens & H\up{th} :Hundredths\\
U : Units & K\up{e} : Thousandths\\
\end{tabular}
\end{center}

\normalsize
The answer is d).\\





7358-- Which digit 5 in the numbere 950 000?\\

a) hundreds digit\\
b) tens digit\\
c) tens of thousands digit\\
d) thousands digit\\

Answer : c)\\

Feedback :\\
\begin{center}
% use packages: array
\begin{tabular}{|rrr|rrr|rrr|}
\hline
\multicolumn{6}{|c|}{integers} &\multicolumn{3}{|c|}{decimals} \\
\hline
\multicolumn{3}{|c|}{Class of} &\multicolumn{3}{|c|}{Class of} &  \multicolumn{3}{c|}{} \\
\multicolumn{3}{|c|}{thousands} &\multicolumn{3}{|c|}{units} &  \multicolumn{3}{c|}{} \\
\hline
H & T & U &H & T & U, & T\up{th} & \textbf{H\up{th}} & K\up{th} \\
\hline
\hline
 9 & \textbf{5} & 0 & 0 & 0 & 0 & & &\\
\hline
\end{tabular}
\end{center}

\scriptsize
\begin{center}
% use packages: array
\begin{tabular}{ll}
H : Hundreds & T\up{th} : Tenths\\
T : Tens & H\up{th} :Hundredths\\
U : Units & K\up{e} : Thousandths\\
\end{tabular}
\end{center}

\normalsize
The answer is c).\\





7359-- Which digit is 4 in the number 42 100?\\

a) hundreds digit\\
b) tens digit\\
c) tens of thousands digit\\
d) thousands digit\\

Answer : c)\\

Feedback :\\
\begin{center}
% use packages: array
\begin{tabular}{|rrr|rrr|rrr|}
\hline
\multicolumn{6}{|c|}{integers} &\multicolumn{3}{|c|}{decimals} \\
\hline
\multicolumn{3}{|c|}{Class of} &\multicolumn{3}{|c|}{Class of} &  \multicolumn{3}{c|}{} \\
\multicolumn{3}{|c|}{thousands} &\multicolumn{3}{|c|}{units} &  \multicolumn{3}{c|}{} \\
\hline
H & T & U &H & T & U, & T\up{th} & \textbf{H\up{th}} & K\up{th} \\
\hline
\hline
  & \textbf{4} & 2 & 1 & 0 & 0 & & & \\
\hline
\end{tabular}
\end{center}

\scriptsize
\begin{center}
% use packages: array
\begin{tabular}{ll}
H : Hundreds & T\up{th} : Tenths\\
T : Tens & H\up{th} :Hundredths\\
U : Units & K\up{e} : Thousandths\\
\end{tabular}
\end{center}

\normalsize
The answer is c).\\





7360-- Which digit is 1 in the number 100 300?\\

a) hundreds digit\\
b) hundreds of thousands digit\\
c) tens digit\\
d) units digit\\

Answer : b)\\

Feedback :\\
\begin{center}
% use packages: array
\begin{tabular}{|rrr|rrr|rrr|}
\hline
\multicolumn{6}{|c|}{integers} &\multicolumn{3}{|c|}{decimals} \\
\hline
\multicolumn{3}{|c|}{Class of} &\multicolumn{3}{|c|}{Class of} &  \multicolumn{3}{c|}{} \\
\multicolumn{3}{|c|}{thousands} &\multicolumn{3}{|c|}{units} &  \multicolumn{3}{c|}{} \\
\hline
H & T & U &H & T & U, & T\up{th} & \textbf{H\up{th}} & K\up{th} \\
\hline
\hline
 \textbf{1} & 0 & 0 & 3 & 0 & 0 & & & \\
\hline
\end{tabular}
\end{center}

\scriptsize
\begin{center}
% use packages: array
\begin{tabular}{ll}
H : Hundreds & T\up{th} : Tenths\\
T : Tens & H\up{th} :Hundredths\\
U : Units & K\up{e} : Thousandths\\
\end{tabular}
\end{center}

\normalsize
The answer is b).\\



7361-- Which digit is 6 in the number 632 900?\\

a) hundreds of thousands digit\\
b) tens of thousands digit\\
c) units digit\\
d) thousands digit\\

Answer : a)\\

Feedback :\\
\begin{center}
% use packages: array
\begin{tabular}{|rrr|rrr|rrr|}
\hline
\multicolumn{6}{|c|}{integers} &\multicolumn{3}{|c|}{decimals} \\
\hline
\multicolumn{3}{|c|}{Class of} &\multicolumn{3}{|c|}{Class of} &  \multicolumn{3}{c|}{} \\
\multicolumn{3}{|c|}{thousands} &\multicolumn{3}{|c|}{units} &  \multicolumn{3}{c|}{} \\
\hline
H & T & U &H & T & U, & T\up{th} & \textbf{H\up{th}} & K\up{th} \\
\hline
\hline
 \textbf{6} & 3 & 2 & 9 & 0 & 0 & & & \\
\hline
\end{tabular}
\end{center}

\scriptsize
\begin{center}
% use packages: array
\begin{tabular}{ll}
H : Hundreds & T\up{th} : Tenths\\
T : Tens & H\up{th} :Hundredths\\
U : Units & K\up{e} : Thousandths\\
\end{tabular}
\end{center}

\normalsize
The answer is a).\\


7362-- Put the numbers in ascending order.\\  
\begin{center}
10,90\ \ \ 10,80\ \ \ 11,01\\
\end{center}

a) 10,80\ \ \ 10,90\ \ \ 11,01\\
b) 10,80\ \ \ 11,01\ \ \ 10,90\\
c) 10,90\ \ \ 11,01\ \ \ 10,80\\
d) 11,01\ \ \ 10,90\ \ \ 10,80\\

Answer : a)\\

Feedback :\\
To put the numbers in ascending order means ordering them from lowest to highest.\\
\begin{center}
% use packages: array
\begin{tabular}{|rrr|rrr|rrr|}
\hline
\multicolumn{6}{|c|}{integers} &\multicolumn{3}{|c|}{decimals} \\
\hline
\multicolumn{3}{|c|}{Class of} &\multicolumn{3}{|c|}{Class of} &  \multicolumn{3}{c|}{} \\
\multicolumn{3}{|c|}{thousands} &\multicolumn{3}{|c|}{units} &  \multicolumn{3}{c|}{} \\
\hline
H & T & U &H & T & U, & T\up{th} & \textbf{H\up{th}} & K\up{th} \\
\hline
\hline
& & & & 1 & 0, & 8 & 0 &\\
& & & & 1 & 0, & 9 & 0 &\\
& & & & 1 & 1, & 0 & 1 &\\
\hline
\end{tabular}
\end{center}

\scriptsize
\begin{center}
% use packages: array
\begin{tabular}{ll}
H : Hundreds & T\up{th} : Tenths\\
T : Tens & H\up{th} :Hundredths\\
U : Units & K\up{e} : Thousandths\\
\end{tabular}
\end{center}

\normalsize
The answer is a).\\


7363-- Put the numbers in ascending order.\\ 
\begin{center}
1,89\ \ \ 1,90\ \ \ 1,88\\
\end{center}

a) 1,88\ \ \ 1,89\ \ \ 1,90\\
b) 1,88\ \ \ 1,90\ \ \ 1,89\\
c) 1,89\ \ \ 1,90\ \ \ 1,88\\
d) 1,90\ \ \ 1,89\ \ \ 1,88\\

Answer : a)\\

Feedaback :\\
To put the numbers in ascending order means ordering them from lowest to highest.\\
\begin{center}
% use packages: array
\begin{tabular}{|rrr|rrr|rrr|}
\hline
\multicolumn{6}{|c|}{integers} &\multicolumn{3}{|c|}{decimals} \\
\hline
\multicolumn{3}{|c|}{Class of} &\multicolumn{3}{|c|}{Class of} &  \multicolumn{3}{c|}{} \\
\multicolumn{3}{|c|}{thousands} &\multicolumn{3}{|c|}{units} &  \multicolumn{3}{c|}{} \\
\hline
H & T & U &H & T & U, & T\up{th} & \textbf{H\up{th}} & K\up{th} \\
\hline
\hline
& & & &  & 1, & 8 & 8 &\\
& & & &  & 1, & 8 & 9 &\\
& & & &  & 1, & 9 & 0 &\\
\hline
\end{tabular}
\end{center}

\scriptsize
\begin{center}
% use packages: array
\begin{tabular}{ll}
H : Hundreds & T\up{th} : Tenths\\
T : Tens & H\up{th} :Hundredths\\
U : Units & K\up{e} : Thousandths\\
\end{tabular}
\end{center}

\normalsize
The answer is a).\\


7364--Put the numbers in ascending order.\\ 
\begin{center}
985 541, 309 405, 196 000\\
\end{center}

a) 196 000, 309 405, 985 541\\
b) 309 405, 196 000, 985 541\\
c) 309 405, 985 541, 196 000\\
d) 985 541, 196 000, 309 405\\

Answer : a)\\

Feedback :\\
To put the numbers in ascending order means ordering them from lowest to highest.\\
\begin{center}
% use packages: array
\begin{tabular}{|rrr|rrr|rrr|}
\hline
\multicolumn{6}{|c|}{integers} &\multicolumn{3}{|c|}{decimals} \\
\hline
\multicolumn{3}{|c|}{Class of} &\multicolumn{3}{|c|}{Class of} &  \multicolumn{3}{c|}{} \\
\multicolumn{3}{|c|}{thousands} &\multicolumn{3}{|c|}{units} &  \multicolumn{3}{c|}{} \\
\hline
H & T & U &H & T & U, & T\up{th} & \textbf{H\up{th}} & K\up{th} \\
\hline
\hline
1 & 9 & 6 & 0 & 0 & 0 & & &\\
3 & 0 & 9 & 4 & 0 & 5 & & &\\
9 & 8 & 5 & 5 & 4 & 1 & & &\\
\hline
\end{tabular}
\end{center}

\scriptsize
\begin{center}
% use packages: array
\begin{tabular}{ll}
H : Hundreds & T\up{th} : Tenths\\
T : Tens & H\up{th} :Hundredths\\
U : Units & K\up{e} : Thousandths\\
\end{tabular}
\end{center}

\normalsize
The answer is a).\\


7365-- Put the numbers in decreasing order.\\ 
\begin{center}
5,55\ \ \ 6,54\ \ \ 5,56\\
\end{center}

a) 5,55\ \ \ 5,56\ \ \ 6,54\\
b) 5,56\ \ \ 5,55\ \ \ 6,54\\
c) 6,54\ \ \ 5,55\ \ \ 5,56\\
d) 6,54\ \ \ 5,56\ \ \ 5,55\\

Answer : d)\\

Feedback :\\
To put the numbers in decreasing order means ordering them from highest to lowest.\\
\begin{center}
% use packages: array
\begin{tabular}{|rrr|rrr|rrr|}
\hline
\multicolumn{6}{|c|}{integers} &\multicolumn{3}{|c|}{decimals} \\
\hline
\multicolumn{3}{|c|}{Class of} &\multicolumn{3}{|c|}{Class of} &  \multicolumn{3}{c|}{} \\
\multicolumn{3}{|c|}{thousands} &\multicolumn{3}{|c|}{units} &  \multicolumn{3}{c|}{} \\
\hline
H & T & U &H & T & U, & T\up{th} & \textbf{H\up{th}} & K\up{th} \\
\hline
\hline
& & &  &  & 6, & 5 & 4 &\\
& & &  &  & 5, & 5 & 6 &\\
& & &  &  & 5, & 5 & 5 &\\
\hline
\end{tabular}
\end{center}

\scriptsize
\begin{center}
% use packages: array
\begin{tabular}{ll}
H : Hundreds & T\up{th} : Tenths\\
T : Tens & H\up{th} :Hundredths\\
U : Units & K\up{e} : Thousandths\\
\end{tabular}
\end{center}

\normalsize
The answer is d).\\


7366-- Put the numbers in decreasing order.\\ 
\begin{center}
196 541, 197 301, 231 200.\\
\end{center}

a) 196 541, 197 301, 231 200\\
b) 197 301, 231 200, 196 541\\
c) 231 200, 196 541, 197 301\\
d) 231 200, 197 301, 196 541\\

Answer : d)\\

Feedback :\\
To put the numbers in decreasing order means ordering them from highest to lowest.\\
\begin{center}
% use packages: array
\begin{tabular}{|rrr|rrr|rrr|}
\hline
\multicolumn{6}{|c|}{integers} &\multicolumn{3}{|c|}{decimals} \\
\hline
\multicolumn{3}{|c|}{Class of} &\multicolumn{3}{|c|}{Class of} &  \multicolumn{3}{c|}{} \\
\multicolumn{3}{|c|}{thousands} &\multicolumn{3}{|c|}{units} &  \multicolumn{3}{c|}{} \\
\hline
H & T & U &H & T & U, & T\up{th} & \textbf{H\up{th}} & K\up{th} \\
\hline
\hline
 2& 3 & 1 & 2 & 0 & 0 & & &\\
 1& 9 & 7 & 3 & 0 & 1 & & &\\
 1& 9 & 6 & 5 & 4 & 1 & & &\\
\hline
\end{tabular}
\end{center}

\scriptsize
\begin{center}
% use packages: array
\begin{tabular}{ll}
H : Hundreds & T\up{th} : Tenths\\
T : Tens & H\up{th} :Hundredths\\
U : Units & K\up{e} : Thousandths\\
\end{tabular}
\end{center}

\normalsize
The answer is d).\\


7367-- Choose the circular diagram that corresponds to the statement.\\ 
\begin{center}
In Mrs Annie's class, two students out of five have a cat at home.\\
\end{center}

a)\\
\includegraphics[width=2cm]{Q7367a.eps}
% Q7367a.eps : 300dpi, width=3.39cm, height=3.39cm, bb=0 0 400 400
\\

b)\\
\includegraphics[width=2cm]{Q7367b.eps}
% Q7367b.eps : 300dpi, width=3.39cm, height=3.39cm, bb=0 0 400 400
\\

c)\\
\includegraphics[width=2cm]{Q7367c.eps}
% Q7367c.eps : 300dpi, width=3.39cm, height=3.39cm, bb=0 0 400 400
\\

d)\\
\includegraphics[width=2cm]{Q7367d.eps}
% Q7367d.eps : 300dpi, width=3.39cm, height=3.39cm, bb=0 0 400 400
\\

Answer : b)\\

Feedback :\\
\begin{center}
\includegraphics[width=5cm]{R7367.eps}
% R7367.eps : 300dpi, width=3.39cm, height=3.39cm, bb=0 0 400 400
\end{center}
The answer is b).\\

7368-- Complete the sentence.
\begin{center}
Any number is divisible by two without remainder if the units digit is evenly divisible by \underline{\quad\quad}.\\
\end{center}
 
a) two\\
b) four\\
c) three\\
d) one\\

Answer: a)\\

Feedback : \\
\begin{center}
Any number is divisable by two without remainder if the units 
digit is evenly divisable by \textbf{deux}.\\
\end{center}

Exemple : 986 556\\
\begin{eqnarray*}
\frac{6}{2}= 3
\end{eqnarray*}
\begin{eqnarray*}
\frac{986\ 556}{2}= 493\ 278
\end{eqnarray*}


The answer is a).\\

7369-- Complete the following sentence.
\begin{center}
Any number is divisable by three without remainder if the sum of its digits is evenly divisable by \underline{\quad\quad}.\\
\end{center}

a) two\\
b) four\\
c) three\\
d) one\\

Answer : c)\\

Feedback :
\begin{center}
Any number is divisable by three without remainder if the sum of its digits is evenly divisable by \textbf{trois}.\\
\end{center}

Example : 986 556\\
\begin{eqnarray*}
9+8+6+5+5+6 = 39
\end{eqnarray*}
\begin{eqnarray*}
\frac{39}{3}=13
\end{eqnarray*}
\begin{eqnarray*}
\frac{986\ 556}{3}= 328\ 852
\end{eqnarray*}

The answer is c).\\



\end{document}
