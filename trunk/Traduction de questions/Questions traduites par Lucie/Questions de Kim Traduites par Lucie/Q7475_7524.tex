\documentclass[letterpaper, 12pt]{article}
\usepackage[french]{babel}

\usepackage{amsmath,amsfonts,amsthm,amssymb,graphicx,multirow,hyperref,color}
\usepackage[latin1]{inputenc}

\pagestyle{plain}

\setlength{\topmargin}{-2cm}
\setlength{\textheight}{23.5cm}
\setlength{\textwidth}{12cm}
\setlength{\oddsidemargin}{-1cm}
\setlength{\parindent}{0pt}


\begin{document}


7475-- Round each number to the nearest unit and do the operation.\\

\begin{eqnarray*}
6,45 \times 3,09\\
\end{eqnarray*}

a) 18\\
b) 21\\
c) 24\\
d) 28\\

Answer : a)\\

Feedback :\\
First, we round the number 6,45 to 6, because the tenths digit is less than 5.\\

Then, we round 3,09 to 3, because the tenths digit is less than 5.\\

Finally, we perform multiplication.\\

$6\times3=18$

The answer is a).\\




7476--Round each number to the nearest tens and do the operation.\\

\begin{eqnarray*}
59 + 43\\
\end{eqnarray*}

a) 70\\
b) 80\\
c) 90\\
d) 100\\

Answer : d)\\

Feedback :\\
First, we round 59 to 60, because the units digit is greater than or equal to 5.\\

Than, we round 43 to 40, because the units digit is less than 5.\\

Finally, we perform the addition.\\

\begin{center}
 \begin{tabular}{r r}
 & 60\\
+ & 40\\
\hline
 & 100
\end{tabular}
\end{center}
The answer is d).\\




7477-- Round each number to the nearest tens and do the operation.\\

\begin{eqnarray*}
233,4 + 51,8\\
\end{eqnarray*}

a) 280\\
b) 284\\
c) 285\\
d) 300\\

Answer : a)\\

Feedback :\\
First, we round 233,4 to 230, because the units digit is less than 5.\\

Then, we round 51,8 to 50, because the units digit is less than 5.\\

Finally, we perform the addition.\\

\begin{center}
 \begin{tabular}{r r}
 & 230\\
+ & 50\\
\hline
 & 280
\end{tabular}
\end{center}
The answer is a).\\




7478--Round each number to the nearest tens and do the operation.\\

\begin{eqnarray*}
16 - 14\\
\end{eqnarray*}

a) 0\\
b) 2\\
c) 6\\
d) 10\\

Answer : d)\\

Feedback :\\
First, we round 16 to 20, because the units digit is less than or equal to 5.\\

Then, we round 14 to 10, because the units digit is less than 5.\\

Finally, we perform the subtraction.\\

$20 - 10 = 10$\\

The answer is d).\\




7479--Round each number to the nearest hundreds and do the operation.\\

\begin{eqnarray*}
991 - 405\\
\end{eqnarray*}

a) 500\\
b) 580\\
c) 600\\
d) 1400\\

Answer : c)\\

Feedback :\\
First, we round 991 to 1000, because the tens digit is greater than or equal to 5.\\

Then, we round 405 to 400, because the tens digit is less than 5.\\

Finally, we perform the subtraction.\\

$1000 - 400 = 600$\\

The answer is c).\\




7480--Round each number to the nearest unit and do the operation.\\

\begin{eqnarray*}
5,9 � 2,09\\
\end{eqnarray*}

a) 2\\
b) 3\\
c) 4\\
d) 8\\

Answer : b)\\

Feedback :\\
First, we round 5,9 to 6, because the tenths digit is greater than or equal to 5.\\

Then, we round 2,09 to 2, because the tenths digit is less than 5.\\

Finally, we perform the division.\\

$6 � 2 = 3$\\

The answer is b).\\




7481-- Considering the order of operations, by which one should we start?\\
\begin{eqnarray*}
81+2-10\div3\\
\end{eqnarray*}


a) $2-10$\\
b) $10\div3$\\
c) $81+2$\\
d) $81-10$\\

Answer : b)\\

Feedback :\\
First, to determine the order, we have to look from left to right. Then, we need to start with the first multiplication or division we come accross.\\
\begin{eqnarray*}
10\div3\\
\end{eqnarray*}
The Answer is b).\\



7482-- Considering the order of operations, by which one should we start?\\
\begin{eqnarray*}
35+1\times2-4\\
\end{eqnarray*}


a) $1\times2$\\
b) $2-4$\\
c) $35+1$\\
d) $35-4$\\

Answer : a)\\

Feedback :\\
First, to determine the order, we have to look from left to right. Then, we need to start with the first multiplication or division we come accross.\\
\begin{eqnarray*}
1\times2\\
\end{eqnarray*}
The answer is a).\\






7483-- Considering the order of operations, by which one should we start?\\

\begin{eqnarray*}
8\times2+(12-3)\div2\\
\end{eqnarray*}

a) $2+12$\\
b) $3\div2$\\
c) $8\times2$\\
d) $12-3$\\

Answer : d)\\

Feedback :\\
We have to start with the operation in between parentheses.\\
\begin{eqnarray*}
12-3
\end{eqnarray*}
The answer is d).\\



7484-- Considering the order of operations, by which one should we start?\\

\begin{eqnarray*}
5+2\times(50-3\times4)+10\\
\end{eqnarray*}

a) $2\times50$\\
b) $3\times4$\\
c) $5+2$\\
d) $50-3$\\

Answer : b)\\

Feedback :\\
We have to start with the operation in between parentheses.\\
\begin{eqnarray*}
(50-3\times4)\\
\end{eqnarray*}
First, to determine the order, we have to look from left to right. Then, we need to start with the first multiplication or division we come accross.\\
\begin{eqnarray*}
3\times4\\
\end{eqnarray*}
The answer is b).\\


7485-- Considering the order of operations, by which one should we start?\\

\begin{eqnarray*}
9+(39-6\div3-3)\\
\end{eqnarray*}

a) $6\div3$\\
b) $3-3$\\
c) $9+39$\\
d) $39-6$\\

Answer : a)\\

Feedback :\\
We have to start with the operation in between parentheses..\\
\begin{eqnarray*}
(39-6\div3-3)\\
\end{eqnarray*}
First, to determine the order, we have to look from left to right. Then, we need to start with the first multiplication or division we come accross.\\
\begin{eqnarray*}
6\div3\\
\end{eqnarray*}
The answer is a).\\




7486-- Perform the operation.\\
\begin{eqnarray*}
\frac{5}{7}+\frac{1}{7}\\
\end{eqnarray*}

a) $\frac{5}{49}$\\
\\
b) $\frac{6}{14}$\\
\\
c) $\frac{6}{7}$\\
\\
d) $\frac{35}{7}$\\

Answer : c)\\

Feedback :\\
\begin{eqnarray*}
\frac{5}{7}+\frac{1}{7}=\frac{6}{7}
\end{eqnarray*}

The answer is c).\\




7487-- Perform the operation.\\
\begin{eqnarray*}
\frac{4}{11}+\frac{3}{11}\\
\end{eqnarray*}

a) $\frac{12}{121}$\\
\\
b) $\frac{7}{22}$\\
\\
c) $\frac{7}{11}$\\
\\
d) $\frac{44}{33}$\\

Answer : c)\\

Feedback :\\
\begin{eqnarray*}
\frac{4}{11}+\frac{3}{11}=\frac{7}{11}
\end{eqnarray*}
The answer is c).\\




7488-- Perform the operation.\\
\begin{eqnarray*}
\frac{3}{4}+\frac{1}{2}\\
\end{eqnarray*}

a) $\frac{3}{6}$\\
\\
b) $\frac{4}{6}$\\
\\
c) $\frac{5}{4}$\\
\\
d) $\frac{3}{2}$\\

Answer : c)\\

Feedback :\\
First, we need to put both fractions on the same denominator.\\
\begin{center}
\begin{tabular}{c}
$\times2$  \\
\LARGE $\curvearrowright$  \\
\Large $\frac{1}{2}$  =  \Large $\frac{2}{4}$ \\
\rotatebox{180}{\LARGE$\curvearrowleft$}\\
$\times2$  \\
\end{tabular}
\end{center}

then, we perform the addition.\\
\begin{eqnarray*}
\frac{3}{4}+\frac{2}{4}=\frac{5}{4}
\end{eqnarray*}

The answer is c).\\




7489-- Perform the operation.\\
\begin{eqnarray*}
\frac{3}{10}+\frac{11}{20}\\
\end{eqnarray*}

a) $\frac{14}{30}$\\
\\
b) $\frac{21}{26}$\\
\\
c) $\frac{17}{20}$\\
\\
d) $\frac{26}{21}$\\

Answer : c)\\

Feedback :\\
First, we have to put both fractions on the same denominator.\\
\begin{center}
\begin{tabular}{c}
$\times2$  \\
\LARGE $\curvearrowright$  \\
\Large $\frac{3}{10}$  =  \Large $\frac{6}{20}$ \\
\rotatebox{180}{\LARGE$\curvearrowleft$}\\
$\times2$  \\
\end{tabular}
\end{center}

Then, we perform the addition.
\begin{eqnarray*}
\frac{6}{20}+\frac{11}{20}=\frac{17}{20}
\end{eqnarray*}

The answer is c).\\





7490-- Perform the operation.\\
\begin{eqnarray*}
\frac{5}{6}-\frac{4}{6}\\
\end{eqnarray*}

a) $\frac{1}{6}$\\
\\
b) $\frac{20}{36}$\\
\\
c) $\frac{30}{24}$\\
\\
d) $\frac{9}{6}$\\

Answer : a)\\

Feedback :\\
\begin{eqnarray*}
\frac{5}{6}-\frac{4}{6}=\frac{1}{6}
\end{eqnarray*}

The answer is a).\\



7491-- Perform the operation.\\
\begin{eqnarray*}
\frac{7}{9}-\frac{3}{9}\\
\end{eqnarray*}

a) $\frac{21}{81}$\\
\\
b) $\frac{4}{9}$\\
\\
c) $\frac{27}{49}$\\
\\
d) $\frac{10}{9}$\\

Answer : b)\\

Feedback :\\
\begin{eqnarray*}
\frac{7}{9}-\frac{3}{9}=\frac{4}{9}
\end{eqnarray*}

The answer is b).\\




7492-- Perform the operation.\\
\begin{eqnarray*}
\frac{3}{4}-\frac{3}{12}\\
\end{eqnarray*}

a) $\frac{9}{48}$\\
\\
b) $\frac{2}{4}$\\
\\
c) 0\\
\\
d) $\frac{36}{12}$\\

Answer : b)\\

Feedback :\\
Here are two different possible ways of performing the operation:\\
\\
\textbf{First way}\\
First, we have to put both fractions on the same denominator.\\
\begin{center}
\begin{tabular}{c}
$\div3$  \\
\LARGE $\curvearrowright$  \\
\Large $\frac{3}{12}$  =  \Large $\frac{1}{4}$ \\
\rotatebox{180}{\LARGE$\curvearrowleft$}\\
$\div3$  \\
\end{tabular}
\end{center}

Then, we perform the subtraction.\\
\begin{eqnarray*}
\frac{3}{4}-\frac{1}{4}=\frac{2}{4}
\end{eqnarray*}
\\
\textbf{Second way}\\
First, we have to put both fractions on the same denominator.\\
\begin{center}
\begin{tabular}{c}
$\times3$  \\
\LARGE $\curvearrowright$  \\
\Large $\frac{3}{4}$  =  \Large $\frac{9}{12}$ \\
\rotatebox{180}{\LARGE$\curvearrowleft$}\\
$\times3$  \\
\end{tabular}
\end{center}

Then, we perform the subtraction.\\
\begin{eqnarray*}
\frac{9}{12}-\frac{3}{12}=\frac{6}{12}
\end{eqnarray*}

Finally, we find an equivalent to $\frac{6}{12}$ in the given choices.\\

\begin{center}
\begin{tabular}{c}
$\div3$  \\
\LARGE $\curvearrowright$  \\
\Large $\frac{6}{12}$  =  \Large $\frac{2}{4}$ \\
\rotatebox{180}{\LARGE$\curvearrowleft$}\\
$\div3$  \\
\end{tabular}
\end{center}
So {6}{12}=\frac{2}{4}$\\

The answer is b).\\



7493-- Perform the operation.\\
\begin{eqnarray*}
\frac{1}{1}-\frac{2}{6}\\
\end{eqnarray*}

a) $\frac{1}{5}$\\
\\
b) $\frac{3}{7}$\\
\\
c) $\frac{3}{6}$\\
\\
d) $\frac{4}{6}$\\

Answer : d)\\

Feedback :\\

First, we have to put both fractions on the same denominator.\\
\begin{center}
\begin{tabular}{c}
$\times6$  \\
\LARGE $\curvearrowright$  \\
\Large $\frac{1}{1}$  =  \Large $\frac{6}{6}$ \\
\rotatebox{180}{\LARGE$\curvearrowleft$}\\
$\times6$  \\
\end{tabular}
\end{center}

Then, we perform the subtraction.\\
\begin{eqnarray*}
\frac{6}{6}-\frac{2}{6}=\frac{4}{6}
\end{eqnarray*}
The answer is d).\\





7494-- Perform the operation.\\
\begin{eqnarray*}
\frac{1}{4}\times\frac{2}{4}\\
\end{eqnarray*}

a) $\frac{2}{16}$\\
\\
b) $\frac{2}{8}$\\
\\
c) $\frac{2}{4}$\\
\\
d) $\frac{8}{4}$\\

Answer : a)\\

Feedback :\\
We have to multiply the numerators with the numerators and the denominators with the denominators.\\
\begin{center}
\begin{tabular}{cc}
$\times$  \\
\LARGE $\curvearrowright$  \\
\Large $\frac{1}{4}$  x  \Large $\frac{2}{4}$ &= \Large$\frac{2}{16}$\\
\rotatebox{180}{\LARGE$\curvearrowleft$}\\
$\times$  \\
\end{tabular}
\end{center}

The Answer is a).\\



7495-- Perform the operation.\\
\begin{eqnarray*}
\frac{3}{5}\times\frac{2}{4}\\
\end{eqnarray*}

a) $\frac{6}{20}$\\
\\
b) $\frac{5}{9}$\\
\\
c) $\frac{10}{12}$\\
\\
d) $\frac{12}{10}$\\

Answer : a)\\

Feedback :\\
We have to multiply the numerators with the numerators and the denominators with the denominators.\\
\begin{center}
\begin{tabular}{cc}
$\times$  \\
\LARGE $\curvearrowright$  \\
\Large $\frac{3}{5}$  x  \Large $\frac{2}{4}$ &= \Large$\frac{6}{20}$\\
\rotatebox{180}{\LARGE$\curvearrowleft$}\\
$\times$  \\
\end{tabular}
\end{center}

The answer is a).\\




7496-- Perform the operation.\\
\begin{eqnarray*}
\frac{9}{12}\times\frac{5}{12}\\
\end{eqnarray*}

a) $\frac{40}{144}$\\
\\
b) $\frac{45}{144}$\\
\\
c) $\frac{60}{108}$\\
\\
d) $\frac{14}{12}$\\

Answer : b)\\

Feedback :\\
We have to multiply the numerators with the numerators and the denominators with the denominators.\\
\begin{center}
\begin{tabular}{cc}
$\times$  \\
\LARGE $\curvearrowright$  \\
\Large $\frac{9}{12}$  x  \Large $\frac{5}{12}$ &= \Large$\frac{45}{144}$\\
\rotatebox{180}{\LARGE$\curvearrowleft$}\\
$\times$  \\
\end{tabular}
\end{center}

The answer is b).\\



7497-- Perform the operation.\\
\begin{eqnarray*}
\frac{1}{2}\times\frac{4}{6}\\
\end{eqnarray*}

a) $\frac{1}{3}$\\
\\
b) $\frac{5}{12}$\\
\\
c) $\frac{4}{8}$\\
\\
d) $\frac{8}{6}$\\

Answer : a)\\

Feedback :\\
First, we have to multiply the numerators with the numerators and the denominators with the denominators.\\
\begin{center}
\begin{tabular}{cc}
$\times$  \\
\LARGE $\curvearrowright$  \\
\Large $\frac{1}{2}$  x  \Large $\frac{4}{6}$ &= \Large$\frac{4}{12}$\\
\rotatebox{180}{\LARGE$\curvearrowleft$}\\
$\times$  \\
\end{tabular}
\end{center}

Then, we have to reduce the product.\\
\begin{center}
\begin{tabular}{c}
$\div4$  \\
\LARGE $\curvearrowright$  \\
\Large $\frac{4}{12}$  =  \Large $\frac{1}{3}$\\
\rotatebox{180}{\LARGE$\curvearrowleft$}\\
$\div4$  \\
\end{tabular}
\end{center}

The answer is a).\\




7498-- Solve.\\
\begin{eqnarray*}
20+10\times3\div1-4\\
\end{eqnarray*}

a) 46\\
b) 56\\
c) 76\\
d) 86\\

Answer : a)\\

Feedback :\\
\begin{center}
\begin{tabular}{ccccccccc}
20 & + & 10 & $\times$ & 3 & $\div$ & 1 & - & 4\\
20 & + & \multicolumn{3}{c}{30} & $\div$ & 1 & - & 4\\
20 & + & \multicolumn{5}{c}{30} & - & 4\\
\multicolumn{7}{c}{50} & - & 4\\
\multicolumn{9}{c}{46}\\
\end{tabular}
\end{center}

To determine the order, we have to look from left to right. Then, we need to start with the first multiplication or division we come accross. Then, we continue looking from left to right until there are no more. Finally, we perform the additions and subtraction.\\
The answer is a).\\



7499-- Solve.\\
\begin{eqnarray*}
20+10\times3\div(1+4)\\
\end{eqnarray*}

a) 18\\
b) 26\\
c) 54\\
d) 94\\

Answer : b)\\

Feedback :\\
\begin{center}
\begin{tabular}{ccccccccc}
20 & + & 10 & $\times$ & 3 & $\div$ & (1 & + & 4)\\
20 & + & 10 & $\times$ & 3 & $\div$ & \multicolumn{3}{c}{5}\\
20 & + & \multicolumn{3}{c}{30} & $\div$ & \multicolumn{3}{c}{5}\\
20 & +& \multicolumn{7}{c}{6}\\
\multicolumn{9}{c}{26}\\
\end{tabular}
\end{center}

To determine the order, we have to look from left to right. Then, we need to start with the first multiplication or division we come accross. Then, we continue looking from left to right until there are no more. Finally, we perform the additions and subtraction.\\
The answer is b).


7500-- Solve.\\
\begin{eqnarray*}
4+6-10\div2\\
\end{eqnarray*}

a) 0\\
b) 5\\
c) 10\\
d) 15\\

Answer : b)\\

Feedback :\\
\begin{center}
\begin{tabular}{ccccccc}
4 & + & 6 & - & 10 & $\div$ & 2\\
4 & + & 6 & - & \multicolumn{3}{c}{5}\\
\multicolumn{3}{c}{10} & - & \multicolumn{3}{c}{5}\\
\multicolumn{7}{c}{5}\\
\end{tabular}
\end{center}

To determine the order, we have to look from left to right. Then, we need to start with the first multiplication or division we come accross. Then, we continue looking from left to right until there are no more. Finally, we perform the additions and subtraction.\\
The answer is b).\\




7501-- Solve.\\
\begin{eqnarray*}
5+2\times6-4\\
\end{eqnarray*}

a) 9\\
b) 13\\
c) 14\\
d) 38\\

Answer : b)\\

Feedback :\\
\begin{center}
\begin{tabular}{ccccccc}
5 & + & 2 & $\times$ & 6 & - & 4\\
5 & + & \multicolumn{3}{c}{12} & - & 4\\
\multicolumn{5}{c}{17} & - & 4\\
\multicolumn{7}{c}{13}\\
\end{tabular}
\end{center}

To determine the order, we have to look from left to right. Then, we need to start with the first multiplication or division we come accross. Then, we continue looking from left to right until there are no more. Finally, we perform the additions and subtraction.\\
The answer is b).\\





7502-- Solve.\\
\begin{eqnarray*}
8\times2+(12-4)\div2\\
\end{eqnarray*}

a) 0\\
b) 12\\
c) 20\\
d) 26\\

Answer : c)\\

Feedback :\\
\begin{center}
\begin{tabular}{ccccccccc}
8 & $\times$ & 2 & + & (12 & - & 4) & $\div$ & 2\\
8 & $\times$ & 2 & + & \multicolumn{3}{c}{8} & $\div$ & 2\\
\multicolumn{3}{c}{16} & + & \multicolumn{3}{c}{8} & $\div$ & 2\\
\multicolumn{3}{c}{16} & + & \multicolumn{5}{c}{4}\\
\multicolumn{9}{c}{20}\\
\end{tabular}
\end{center}

To determine the order, we have to look from left to right. Then, we need to start with the first multiplication or division we come accross. Then, we continue looking from left to right until there are no more. Finally, we perform the additions and subtraction.\\
The answer is c).\\

7503-- Solve.\\
\begin{eqnarray*}
5+2\times(8-3\times2)+1\\
\end{eqnarray*}

a) 10\\
b) 26\\
c) 71\\
d) 113\\

Answer : a)\\

Feedback :\\
\begin{center}
\begin{tabular}{ccccccccccc}
5 & + & 2 & $\times$ & (8 & - & 3 & $\times$ & 2)& + & 1\\
5 & + & 2 & $\times$ & (8 & - & \multicolumn{3}{c}{6})& + & 1\\
5 & + & 2 & $\times$ & \multicolumn{5}{c}{2} & + & 1\\
5 & + & \multicolumn{7}{c}{4} & + & 1\\
\multicolumn{9}{c}{9} & + & 1\\
\multicolumn{11}{c}{10}\\
\end{tabular}
\end{center}

To determine the order, we have to look from left to right. Then, we need to start with the first multiplication or division we come accross. Then, we continue looking from left to right until there are no more. Finally, we perform the additions and subtraction.\\
The answer is a).\\



7504-- Solve.\\
\begin{eqnarray*}
6+(9-6\div3-1)\\
\end{eqnarray*}

a) 3\\
b) 4\\
c) 6\\
d) 12\\

Answer : d)\\

Feedback :\\
\begin{center}
\begin{tabular}{ccccccccc}
6 & + & (9 & - & 6 & $\div$ & 3 & - & 1)\\
6 & + & (9 & - & \multicolumn{3}{c}{2} & - & 1)\\
6 & + & \multicolumn{5}{c}{(7} & - & 1)\\
6 & + & \multicolumn{7}{c}{6}\\
\multicolumn{9}{c}{12}\\
\end{tabular}
\end{center}

To determine the order, we have to look from left to right. Then, we need to start with the first multiplication or division we come accross. Then, we continue looking from left to right until there are no more. Finally, we perform the additions and subtraction.\\
The answer is d).\\


7505--Find the number that follows seven hundred eight.\\

a) 700\\
b) 707\\
c) 708\\
d) 709\\

Answer : d)\\

Feedback :\\
\begin{center}
% use packages: array
\begin{tabular}{|rrr|rrr|rrr|rrr|}
\hline
\multicolumn{6}{|c|}{integers} &\multicolumn{3}{|c|}{decimals} \\
\hline
\multicolumn{3}{|c|}{Class of} &\multicolumn{3}{|c|}{Class of} &  \multicolumn{3}{c|}{} \\
\multicolumn{3}{|c|}{thousands} &\multicolumn{3}{|c|}{units} &  \multicolumn{3}{c|}{} \\
\hline
H & T & U &H & T & U, & T\up{th} & \textbf{H\up{th}} & K\up{th} \\
\hline
\hline
& & & & & &  &  & 4, & 1 & 0 &\\
& & & & & &  &  & 4, & 1 & 2 &\\
& & & & & &  &  & 4, & 5 &   & \\
\hline
\end{tabular}
\end{center}

\scriptsize
\begin{center}
% use packages: array
\begin{tabular}{ll}
H : Hundreds & T\up{th} : Tenths\\
T : Tens & H\up{th} :Hundredths\\
U : Units & K\up{e} : Thousandths\\
\end{tabular}
\end{center}

\normalsize
The answer is d).\\




7506--Find the number that precedes seven hundred eight.\\

a) 700\\
b) 707\\
c) 708\\
d) 709\\

Answer : b)\\

Feedback :\\
\begin{center}
% use packages: array
\begin{tabular}{|rrr|rrr|rrr|rrr|}
\hline
\multicolumn{6}{|c|}{integers} &\multicolumn{3}{|c|}{decimals} \\
\hline
\multicolumn{3}{|c|}{Class of} &\multicolumn{3}{|c|}{Class of} &  \multicolumn{3}{c|}{} \\
\multicolumn{3}{|c|}{thousands} &\multicolumn{3}{|c|}{units} &  \multicolumn{3}{c|}{} \\
\hline
H & T & U &H & T & U, & T\up{th} & \textbf{H\up{th}} & K\up{th} \\
\hline
\hline
& & & & & &  &  & 4, & 1 & 0 &\\
& & & & & &  &  & 4, & 1 & 2 &\\
& & & & & &  &  & 4, & 5 &   & \\
\hline
\end{tabular}
\end{center}

\scriptsize
\begin{center}
% use packages: array
\begin{tabular}{ll}
H : Hundreds & T\up{th} : Tenths\\
T : Tens & H\up{th} :Hundredths\\
U : Units & K\up{e} : Thousandths\\
\end{tabular}
\end{center}

\normalsize

The answer is b).\\




7507--Find the number that follows fifty thousand three hundred nineteen.\\

a) 50 318\\
b) 50 319\\
c) 50 320\\
d) 50 419\\

Answer : c)\\

Feedback :\\
\begin{center}
% use packages: array
\begin{tabular}{|rrr|rrr|rrr|rrr|}
\hline
\multicolumn{6}{|c|}{integers} &\multicolumn{3}{|c|}{decimals} \\
\hline
\multicolumn{3}{|c|}{Class of} &\multicolumn{3}{|c|}{Class of} &  \multicolumn{3}{c|}{} \\
\multicolumn{3}{|c|}{thousands} &\multicolumn{3}{|c|}{units} &  \multicolumn{3}{c|}{} \\
\hline
H & T & U &H & T & U, & T\up{th} & \textbf{H\up{th}} & K\up{th} \\
\hline
\hline
& & & & & &  &  & 4, & 1 & 0 &\\
& & & & & &  &  & 4, & 1 & 2 &\\
& & & & & &  &  & 4, & 5 &   & \\
\hline
\end{tabular}
\end{center}

\scriptsize
\begin{center}
% use packages: array
\begin{tabular}{ll}
H : Hundreds & T\up{th} : Tenths\\
T : Tens & H\up{th} :Hundredths\\
U : Units & K\up{e} : Thousandths\\
\end{tabular}
\end{center}

\normalsize

The answer is c).\\






7508--\mbox{Find the number that precedes fifty thousand three hundred nineteen.}\\

a) 50 318\\
b) 50 319\\
c) 50 320\\
d) 50 419\\

Answer : a)\\

Feedback :\\
\begin{center}
% use packages: array
\begin{tabular}{|rrr|rrr|rrr|rrr|}
\hline
\multicolumn{6}{|c|}{integers} &\multicolumn{3}{|c|}{decimals} \\
\hline
\multicolumn{3}{|c|}{Class of} &\multicolumn{3}{|c|}{Class of} &  \multicolumn{3}{c|}{} \\
\multicolumn{3}{|c|}{thousands} &\multicolumn{3}{|c|}{units} &  \multicolumn{3}{c|}{} \\
\hline
H & T & U &H & T & U, & T\up{th} & \textbf{H\up{th}} & K\up{th} \\
\hline
\hline
& & & & & &  &  & 4, & 1 & 0 &\\
& & & & & &  &  & 4, & 1 & 2 &\\
& & & & & &  &  & 4, & 5 &   & \\
\hline
\end{tabular}
\end{center}

\scriptsize
\begin{center}
% use packages: array
\begin{tabular}{ll}
H : Hundreds & T\up{th} : Tenths\\
T : Tens & H\up{th} :Hundredths\\
U : Units & K\up{e} : Thousandths\\
\end{tabular}
\end{center}

\normalsize

The answer is a).\\



7509--Find the number that follows two hundred thirty-nine thousand six.\\

a) 239 007\\
b) 239 016\\
c) 239 599\\
d) 239 701\\

Answer : a)\\

Feedback :\\
\begin{center}
% use packages: array
\begin{tabular}{|rrr|rrr|rrr|rrr|}
\hline
\multicolumn{6}{|c|}{integers} &\multicolumn{3}{|c|}{decimals} \\
\hline
\multicolumn{3}{|c|}{Class of} &\multicolumn{3}{|c|}{Class of} &  \multicolumn{3}{c|}{} \\
\multicolumn{3}{|c|}{thousands} &\multicolumn{3}{|c|}{units} &  \multicolumn{3}{c|}{} \\
\hline
H & T & U &H & T & U, & T\up{th} & \textbf{H\up{th}} & K\up{th} \\
\hline
\hline
& & & & & &  &  & 4, & 1 & 0 &\\
& & & & & &  &  & 4, & 1 & 2 &\\
& & & & & &  &  & 4, & 5 &   & \\
\hline
\end{tabular}
\end{center}

\scriptsize
\begin{center}
% use packages: array
\begin{tabular}{ll}
H : Hundreds & T\up{th} : Tenths\\
T : Tens & H\up{th} :Hundredths\\
U : Units & K\up{e} : Thousandths\\
\end{tabular}
\end{center}

\normalsize

The answer is a).\\






7510--Find the number that follows twelve thousand one hundred ninety-nine.\\

a) 12 100\\
b) 12 190\\
c) 12 200\\
d) 12 299\\

Answer : c)\\

Feedback :\\
\begin{center}
% use packages: array
\begin{tabular}{|rrr|rrr|rrr|rrr|}
\hline
\multicolumn{6}{|c|}{integers} &\multicolumn{3}{|c|}{decimals} \\
\hline
\multicolumn{3}{|c|}{Class of} &\multicolumn{3}{|c|}{Class of} &  \multicolumn{3}{c|}{} \\
\multicolumn{3}{|c|}{thousands} &\multicolumn{3}{|c|}{units} &  \multicolumn{3}{c|}{} \\
\hline
H & T & U &H & T & U, & T\up{th} & \textbf{H\up{th}} & K\up{th} \\
\hline
\hline
& & & & & &  &  & 4, & 1 & 0 &\\
& & & & & &  &  & 4, & 1 & 2 &\\
& & & & & &  &  & 4, & 5 &   & \\
\hline
\end{tabular}
\end{center}

\scriptsize
\begin{center}
% use packages: array
\begin{tabular}{ll}
H : Hundreds & T\up{th} : Tenths\\
T : Tens & H\up{th} :Hundredths\\
U : Units & K\up{e} : Thousandths\\
\end{tabular}
\end{center}

\normalsize

The answer is c).\\






7511--\mbox{Find the number the precedes twelve thousand one hundred ninety-nine.}\\

a) 11 199\\
b) 12 099\\
c) 12 189\\
d) 12 198\\

Answer : d)\\

Feedback :\\
\begin{center}
% use packages: array
\begin{tabular}{|rrr|rrr|rrr|rrr|}
\hline
\multicolumn{6}{|c|}{integers} &\multicolumn{3}{|c|}{decimals} \\
\hline
\multicolumn{3}{|c|}{Class of} &\multicolumn{3}{|c|}{Class of} &  \multicolumn{3}{c|}{} \\
\multicolumn{3}{|c|}{thousands} &\multicolumn{3}{|c|}{units} &  \multicolumn{3}{c|}{} \\
\hline
H & T & U &H & T & U, & T\up{th} & \textbf{H\up{th}} & K\up{th} \\
\hline
\hline
& & & & & &  &  & 4, & 1 & 0 &\\
& & & & & &  &  & 4, & 1 & 2 &\\
& & & & & &  &  & 4, & 5 &   & \\
\hline
\end{tabular}
\end{center}

\scriptsize
\begin{center}
% use packages: array
\begin{tabular}{ll}
H : Hundreds & T\up{th} : Tenths\\
T : Tens & H\up{th} :Hundredths\\
U : Units & K\up{e} : Thousandths\\
\end{tabular}
\end{center}

\normalsize

The answer is d).\\




7512--Find the number that follows thirty-two thousand six hundred.\\

a) 32 599\\
b) 32 601\\
c) 32 610\\
d) 32 700\\

Answer : b)\\

Feedback :\\
\begin{center}
% use packages: array
\begin{tabular}{|rrr|rrr|rrr|rrr|}
\hline
\multicolumn{6}{|c|}{integers} &\multicolumn{3}{|c|}{decimals} \\
\hline
\multicolumn{3}{|c|}{Class of} &\multicolumn{3}{|c|}{Class of} &  \multicolumn{3}{c|}{} \\
\multicolumn{3}{|c|}{thousands} &\multicolumn{3}{|c|}{units} &  \multicolumn{3}{c|}{} \\
\hline
H & T & U &H & T & U, & T\up{th} & \textbf{H\up{th}} & K\up{th} \\
\hline
\hline
& & & & & &  &  & 4, & 1 & 0 &\\
& & & & & &  &  & 4, & 1 & 2 &\\
& & & & & &  &  & 4, & 5 &   & \\
\hline
\end{tabular}
\end{center}

\scriptsize
\begin{center}
% use packages: array
\begin{tabular}{ll}
H : Hundreds & T\up{th} : Tenths\\
T : Tens & H\up{th} :Hundredths\\
U : Units & K\up{e} : Thousandths\\
\end{tabular}
\end{center}

\normalsize

The answer is b).\\





7513--Find the number that precedes thirty-two thousand six hundred.\\

a) 32 500\\
b) 32 590\\
c) 32 599\\
d) 32 601\\

Answer : c)\\

Feedback :\\
\begin{center}
% use packages: array
\begin{tabular}{|rrr|rrr|rrr|rrr|}
\hline
\multicolumn{6}{|c|}{integers} &\multicolumn{3}{|c|}{decimals} \\
\hline
\multicolumn{3}{|c|}{Class of} &\multicolumn{3}{|c|}{Class of} &  \multicolumn{3}{c|}{} \\
\multicolumn{3}{|c|}{thousands} &\multicolumn{3}{|c|}{units} &  \multicolumn{3}{c|}{} \\
\hline
H & T & U &H & T & U, & T\up{th} & \textbf{H\up{th}} & K\up{th} \\
\hline
\hline
& & & & & &  &  & 4, & 1 & 0 &\\
& & & & & &  &  & 4, & 1 & 2 &\\
& & & & & &  &  & 4, & 5 &   & \\
\hline
\end{tabular}
\end{center}

\scriptsize
\begin{center}
% use packages: array
\begin{tabular}{ll}
H : Hundreds & T\up{th} : Tenths\\
T : Tens & H\up{th} :Hundredths\\
U : Units & K\up{e} : Thousandths\\
\end{tabular}
\end{center}

\normalsize

The answer is c).\\




7514--Find the sum.\\
\begin{center}
% use packages: array
\begin{tabular}{rrrrrrr}
  & 6 & 2 && 3 & 1 & 5\\
+ & 1 & 9 && 4 & 2 & 7\\
\end{tabular}
\end{center}

a) 42 978\\
b) 71 732\\
c) 71 741\\
d) 81 742\\

Answer : d)\\

Feedback :\\
\begin{center}
% use packages: array
\begin{tabular}{rrrrrrr}
  & 1 &   &&   & 1 &  \\
  & 6 & 2 && 3 & 1 & 5\\
+ & 1 & 9 && 4 & 2 & 7\\
\hline
  & 8 & 1 && 7 & 4 & 2\\
\end{tabular}
\end{center}
The answer is d).\\


7515--Find the sum.\\
\begin{center}
% use packages: array
\begin{tabular}{rrrrrrr}
  & 2 & 9 && 8 & 2 & 1\\
+ & 4 & 7 && 7 & 7 & 2\\
\end{tabular}
\end{center}

a) 66 593\\
b) 67 493\\
c) 76 493\\
d) 77 593\\

Answer : d)\\

Feedback :\\
\begin{center}
% use packages: array
\begin{tabular}{rrrrrrr}
  & 1 & 1 &&   &   &  \\
  & 2 & 9 && 8 & 2 & 1\\
+ & 4 & 7 && 7 & 7 & 2\\
\hline
  & 7 & 7 && 5 & 9 & 3\\
\end{tabular}
\end{center}
The answer is d).\\

7516--Find the sum.\\
\begin{center}
% use packages: array
\begin{tabular}{rrrrrrr}
  & 3 & 2 && 4 & 0 & 1\\
+ & 1 & 7 && 0 & 6 & 3\\
  & 4 & 5 && 9 & 8 & 8\\
\end{tabular}
\end{center}

a)  4 342\\
b) 14 342\\
c) 95 452\\
d) 876 462\\

Answer : c)\\

Feedback :\\
\begin{center}
% use packages: array
\begin{tabular}{rrrrrrr}
  & 1 & 1 && 1 & 1 &  \\
  & 3 & 2 && 4 & 0 & 1\\
+ & 1 & 7 && 0 & 6 & 3\\
  & 4 & 5 && 9 & 8 & 8\\
\hline
  & 9 & 5 && 4 & 5 & 2\\
\end{tabular}
\end{center}
The answer is c).\\




7517--Find the sum.\\
\begin{center}
% use packages: array
\begin{tabular}{rrrrr}
   & 1 & 3 & 1 & 5\\
+  & 6 & 7 & 1 & 5\\
   & 1 & 0 & 0 & 0\\
\end{tabular}
\end{center}

a) 8 020\\
b) 9 030\\
c) 75 035\\
d) 76 025\\

Answer : b)\\

Feedback :\\
\begin{center}
% use packages: array
\begin{tabular}{rrrrr}
   & 1 &   & 1 &  \\
   & 1 & 3 & 1 & 5\\
+  & 6 & 7 & 1 & 5\\
   & 1 & 0 & 0 & 0\\
\hline
   & 9 & 0 & 3 & 0\\
\end{tabular}
\end{center}
The answer is b).\\




7518--Perform the subtraction.\\
\begin{center}
% use packages: array
\begin{tabular}{rrrrr}
  & 4 & 0 & 0 & 0\\
- & 2 & 6 & 0 & 0\\
\end{tabular}
\end{center}

a) 1 400\\
b) 2 400\\
c) 2 600\\
d) 42 641\\

Answer : a)\\

Feedback :\\
\begin{center}
% use packages: array
\begin{tabular}{rrrrr}
  & 3 &   &   &  \\
  & \xout{4} & $^{1}$0  & 0 & 0\\
- & 2 & 6 & 0 & 0\\
\hline
  & 1 & 4 & 0 & 0\\
\end{tabular}
\end{center}
The answer is a).\\



7519--Perform the subtraction.\\
\begin{center}
% use packages: array
\begin{tabular}{rrrrr}
   & 7 & 2 & 6 & 0\\
-  & 3 & 1 & 4 & 0\\
\end{tabular}
\end{center}

a)  4 120\\
b) 10 300\\
c) 10 400\\
d) 50 400\\

Answer : a)\\

Feedback :\\
\begin{center}
% use packages: array
\begin{tabular}{rrrrr}
   & 7 & 2 & 6 & 0\\
-  & 3 & 1 & 4 & 0\\
\hline
   & 4 & 1 & 2 & 0\\
\end{tabular}
\end{center}
The answer is a).\\



7520--Perform the subtraction.\\
\begin{center}
% use packages: array
\begin{tabular}{rrrrr}
   & 7 & 4 & 7 & 2\\
-  & 7 & 4 & 7 & 1\\
\end{tabular}
\end{center}

a) 0\\
b) 1\\
c) 7473\\
d) 14 943\\

Answer : b)\\

Feedback :\\
\begin{center}
% use packages: array
\begin{tabular}{rrrrr}
   & 7 & 4 & 7 & 2\\
-  & 7 & 4 & 7 & 1\\
\hline
   &   &   &   & 1\\
\end{tabular}
\end{center}
The answer is b).\\





7521--Perform the subtraction.\\
\begin{center}
% use packages: array
\begin{tabular}{rrrrr}
  & 5 & 9 & 1 & 7\\
- &   & 7 & 0 & 2\\
\end{tabular}
\end{center}

a) 212\\
b) 5200\\
c) 5215\\
d) 5325\\

Answer : c)\\

Feedback :\\
\begin{center}
% use packages: array
\begin{tabular}{rrrrr}
  & 5 & 9 & 1 & 7\\
- &   & 7 & 0 & 2\\
\hline
  & 5 & 2 & 1 & 5\\
\end{tabular}
\end{center}
The answer is c).\\


7522--Perform the multiplication.\\
\begin{center}
% use packages: array
\begin{tabular}{rrr}
  & 7 & 0\\
X &   & 8\\
\end{tabular}
\end{center}

a) 56\\
b) 78\\
c) 540\\
d) 560\\

Answer : d)\\

Feedback :\\
\begin{center}
% use packages: array
\begin{tabular}{rrr}
  & 7 & 0\\
X &   & 8\\
\hline
5 & 6 & 0\\
\end{tabular}
\end{center}
The answer is d).\\



7523--Perform the multiplication.\\
\begin{center}
% use packages: array
\begin{tabular}{rrr}
  & 4 & 2\\
X &   & 3\\
\end{tabular}
\end{center}

a) 18\\
b) 45\\
c) 75\\
d) 126\\

Answer : d)\\

Feedback :\\
\begin{center}
% use packages: array
\begin{tabular}{rrr}
  & 4 & 2\\
X &   & 3\\
\hline
1 & 2 & 6\\
\end{tabular}
\end{center}
The answer is d).\\




7524--Perform the multiplication.\\
\begin{center}
% use packages: array
\begin{tabular}{rrrr}
  & 3 & 4 & 5\\
X &   &   & 2\\
\end{tabular}
\end{center}

a) 645\\
b) 680\\
c) 685\\
d) 690\\

Answer : d)\\

Feedback :\\
\begin{center}
% use packages: array
\begin{tabular}{rrrr}
  &   & 1 &  \\
  & 3 & 4 & 5\\
X &   &   & 2\\
\hline
  & 6 & 9 & 0\\
\end{tabular}
\end{center}
The answer is d).\\


\end{document}

