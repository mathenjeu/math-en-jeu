\documentclass[letterpaper, 12pt]{article}
\usepackage[french]{babel}

\usepackage{amsmath,amsfonts,amsthm,amssymb,graphicx,multirow,hyperref,color}
\usepackage[latin1]{inputenc}

\pagestyle{plain}

\setlength{\topmargin}{-2cm}
\setlength{\textheight}{23.5cm}
\setlength{\textwidth}{12cm}
\setlength{\oddsidemargin}{-1cm}
\setlength{\parindent}{0pt}


\begin{document}

7450-- Ludovic has 5 dollars. Marco three times more money than Ludovic. How much moneu does Marco have?\\

a) 2\$\\
b) 8\$\\
c) 10\$\\
d) 15\$\\

Answer : d)\\

Feedback :\\
Marco has three \textbf{times more} money than Ludovic.\\
\begin{center}
$3\times 5 = 15$
\end{center}
Thus, Marco has 15\$.\\
The answer is d).\\



7451-- Marianne's dog had six puppies. Three puppies have been adopted, how many puppies are left?\\

a) 2\\
b) 3\\
c) 9\\
d) 18\\

Answer : b)\\

Feedback :\\
Three puppies have been adopted, on a total of six puppies. So, we subtract 3 puppies from the six. Thus, there are three puppies left.\\
\begin{center}
$6-3 = 3$
\end{center}
The answer is b).\\

7452-- Express the following multiplication in terms of powers.\\.
\begin{center}
$3\times3\times3\times3$
\end{center}

a) $ 3^{3}$\\
b) $ 3^{4}$\\
c) $ 4^{3}$\\
d) $ 4^{4}$\\

Answer : b)\\

Feedback :\\
\begin{center}
$ 3^{4}$ = $3\times3\times3\times3$\\
\end{center}
The answer is b).\\



7453-- Express the following multiplication in terms of powers.\\
\begin{center}
$6\times6\times6\times6\times6\times6\times6$
\end{center}


a) $ 6^{6}$\\
b) $ 6^{7}$\\
c) $ 7^{6}$\\
d) $ 7^{7}$\\

Answer : b)\\

Feedback :\\
\begin{center}
$ 6^{7}$ = $6\times6\times6\times6\times6\times6\times6$\\
\end{center}
The answer is b).\\



7454-- Express the following multiplication in terms of powers.\\
\begin{center}
$2\times2\times8\times2\times8$
\end{center}

a) $ 2^{2}\times8^{3}$\\
b) $ 2^{3}\times3^{8}$\\
c) $ 2^{3}\times8^{2}$\\
d) $ 2^{8}\times3^{2}$\\

Answer : c)\\

Feedback :\\
Tout d'abord, gr\^ace au caract\`ere commutatif de la multiplication, on rassemble les m\^emes chiffres ensemble. Ensuite, on conserve un seul chiffre par groupe et on ajoute les puissances.

\begin{eqnarray*}
2\times2\times8\times2\times8
&=&2\times2\times2\times8\times8\\
&=&2^{3}\times8^{2}\\
\end{eqnarray*}
The answer is c).\\

7455-- Exprime sous forme de puissance la multiplication.\\
\begin{center}
$4\times4\times6\times9\times9\times4\times6$
\end{center}

a) $ 2^{9}\times3^{4}\times2^{6}$\\
b) $ 2^{9}\times6^{2}\times9^{2}$\\
c) $ 4^{3}\times6\times9$\\
d) $ 4^{3}\times6^{2}\times9^{2}$\\

Answer : d)\\

Feedback :\\
First, because of the commutativity of multiplications, the same numbers are put together. 
Afterwards, only one number from each group is kept and the powers are added..

\begin{eqnarray*}
4\times4\times6\times9\times9\times4\times6
&=&4\times4\times4\times6\times6\times9\times9\\
&=&4^{3}\times6^{2}\times9^{2}\\                                                                           \end{eqnarray*}
The answer is d).\\



7456-- Continue the sequence.\\
\begin{center}
100, 200, 300, 400, 500, \underline{\quad\quad}
\end{center}

a) 400\\
b) 550\\
c) 600\\
d) 1000\\

Answer : c)\\

Feedback :\\
\begin{center}
100, 200, 300, 400, 500, \textbf{600}\\
\end{center}
The consistency is of + 100.\\

\begin{center}
$100+100=200$\\
$200+100=300$\\
$300+100=400$\\
$400+100=500$\\
$500+100=\textbf{600}$\\
\end{center}
The answer is c).\\


7457-- Continue the following sequence of numbers..\\
\begin{center}
0, 1, 3, 4, 12, 13, 39, \underline{\quad\quad}
\end{center}

a) 30\\
b) 40\\
c) 42\\
d) 117\\

Answer : b)\\

Feedback :\\
\begin{center}
0, 1, 3, 4, 12, 13, 39, \textbf{40}\\
\end{center}
The consistency is of $+ 1\times3$.\\

\begin{center}
$0+1=1$\\
$1\times3=3$\\
\end{center}
\begin{center}
$3+1=4$\\
$4\times3=12$\\
\end{center}
\begin{center}
$12+1=13$\\
$13\times3=39$\\
\end{center}
\begin{center}
$39+1=\textbf{40}$\\
\end{center}
The answer is b).\\



7458-- Find the mistake in the multiplication table.\\

\begin{center}
% use packages: array
\begin{tabular}{c|cc}
X & 11 & 12 \\
\hline
2 & 22 & 24\\
10 & 110 & 120\\
11 & 121 & 131\\

\end{tabular}
\end{center}

a) 22\\
b) 24\\
c) 110\\
d) 131\\

Answer : d)\\

Feedback :\\
\begin{center}
% use packages: array
\begin{tabular}{c|cc}
X & 11 & \textbf{12} \\
\hline
2 & 22 & 24\\
10 & 110 & 120\\
\textbf{11}& 121 & \textbf{132}\\

\end{tabular}
\end{center}


\begin{center}
% use packages: array
\begin{tabular}{ccc}
 & 1 & 1\\
X &1 & 2\\
\hline
 & 2 & 2\\
 1 & 1 & \\
\hline
1 & 3 & 2
\end{tabular}
\end{center}
The answer is d).\\


7459-- Find the mistake in the multiplication table.\\

\begin{center}
% use packages: array
\begin{tabular}{c|cc}
X & 6 & 7 \\
\hline
7 & 42 & 49\\
8 & 44 & 56\\
9 & 54 & 63\\

\end{tabular}
\end{center}

a) 44\\
b) 54\\
c) 56\\
d) 63\\

Answer : a)\\

Feedback :\\
\begin{center}
% use packages: array
\begin{tabular}{c|cc}
X & \textbf{6} & 7 \\
\hline
7 & 42 & 49\\
\textbf{8} & \textbf{48} & 56\\
9 & 54 & 63\\

\end{tabular}
\end{center}


\begin{eqnarray*}
6\times8=48
\end{eqnarray*}
The answer is a).\\



7460-- Find the mistake in the multiplication table.\\

\begin{center}
% use packages: array
\begin{tabular}{c|cc}
X & 2 & 3 \\
\hline
1 & 2 & 3\\
2 & 4 & 6\\
3 & 8 & 9\\

\end{tabular}
\end{center}

a) 2\\
b) 4\\
c) 8\\
d) 9\\

Answer : c)\\

Feedback :\\
\begin{center}
% use packages: array
\begin{tabular}{c|cc}
X & \textbf{2} & 3 \\
\hline
1 & 2 & 3\\
2 & 4 & 6\\
\textbf{3} & \textbf{6}& 9\\

\end{tabular}
\end{center}


\begin{eqnarray*}
3\times2=6
\end{eqnarray*}
The answer is c).\\



7461-- Continue the following sequence.\\
\begin{center}
800, 400, 200, 100, 50, \underline{\quad\quad}
\end{center}

a) 0\\
b) 25\\
c) 50\\
d) 100\\

Answer : b)\\

Feedback :\\
\begin{center}
800, 400, 200, 100, 50, \textbf{25}\\
\end{center}
The consistency is of $� 2$.\\

\begin{center}
$800�2=400$\\
$400�2=200$\\
$200�2=100$\\
$100�2=50$\\
$50�2=\textbf{25}$\\
\end{center}
The answer is b).\\




7462-- Continue the following sequence\\
\begin{center}
$\frac{1}{8}, \frac{2}{8}, \frac{3}{8}, \frac{4}{8}, \frac{5}{8},$ \underline{\quad\quad}
\end{center}

a) $\frac{2}{3}$\\\\
b) $\frac{6}{8}$\\\\
c) $\frac{7}{8}$\\\\
d) 1\\\\

Answer : b)\\

Feedback :\\
\begin{center}
$\frac{1}{8}, \frac{2}{8}, \frac{3}{8}, \frac{4}{8}, \frac{5}{8}, \frac{\textbf{6}}{\textbf{8}}$
\end{center}
The consistency is of + $\frac{1}{8}$ .\\


The answer is b).\\





7463-- Continue the following sequence.\\
\begin{center}
$\frac{2}{6}, \frac{1}{6}, \frac{3}{6}, \frac{2}{6},\frac{4}{6}$, \underline{\quad\quad}\\
\end{center}

a) $\frac{1}{6}$\\\\
b) $\frac{3}{6}$\\\\
c) $\frac{6}{6}$\\\\
d) 1\\\\

Answer : b)\\

Feedback :\\
\begin{center}
$\frac{2}{6}, \frac{1}{6}, \frac{3}{6}, \frac{2}{6},\frac{4}{6}, \frac{\textbf{3}}{\textbf{6}}$\\
\end{center}
The consistency is of - $\frac{1}{6}$ + $\frac{2}{6}$ .\\

\begin{center}
$\frac{2}{6}-\frac{1}{6}=\frac{1}{6}$\\
\end{center}
\begin{center}
$\frac{1}{6}+\frac{2}{6}=\frac{3}{6}$\\
\end{center}
\begin{center}
$\frac{3}{6}-\frac{1}{6}=\frac{2}{6}$\\
\end{center}
\begin{center}
$\frac{2}{6}+\frac{2}{6}=\frac{4}{6}$\\
\end{center}
\begin{center}
$\frac{4}{6}-\frac{1}{6}=\frac{\textbf{3}}{\textbf{6}}$\\
\end{center}

The answer is b).\\







7464-- Continue the following sequence.\\
\begin{center}
$\frac{1}{5}, \frac{3}{5}, \frac{5}{5}, \frac{7}{5}, \frac{9}{5}$ \underline{\quad\quad}
\end{center}

a) $\frac{5}{9}$\\\\
b) $\frac{11}{9}$\\\\
c) $\frac{11}{5}$\\\\
d) $\frac{13}{5}$\\\\

Answer : c)\\

Feedback :\\
\begin{center}
$\frac{1}{5}, \frac{3}{5}, \frac{5}{5}, \frac{7}{5}, \frac{9}{5}, \frac{\textbf{11}}{\textbf{5}}$
\end{center}
La r\'egularit\'e est de + $\frac{2}{5}$.\\

\begin{center}
$\frac{1}{5}+\frac{2}{5}=\frac{3}{5}$\\
\end{center}
\begin{center}
$\frac{3}{5}+\frac{2}{5}=\frac{5}{5}$\\
\end{center}
\begin{center}
$\frac{5}{5}+\frac{2}{5}=\frac{7}{5}$\\
\end{center}
\begin{center}
$\frac{7}{5}+\frac{2}{5}=\frac{9}{5}$\\
\end{center}
\begin{center}
$\frac{9}{5}+\frac{2}{5}=\frac{\textbf{11}}{\textbf{5}}$\\
\end{center}

The answer is c).\\




7465-- Continue the following sequence.\\
\begin{center}
\ \ \ 0 \ \ \ 0,25 \ \ \ 0,5 \ \ \ 0,75 \ \ \   \underline{\quad\quad}
\end{center}

a) 0,1\\
b) 0,5\\
c) 1\\
d) 1,25\\

Answer : c)\\

Feedback :\\
\begin{center}
\ \ \ 0 \ \ \ 0,25 \ \ \ 0,5 \ \ \ 0,75 \ \ \   \textbf{1}
\end{center}
The consistency is of + 0,25.\\

\begin{center}
$0+0,25=0,25$\\
\end{center}
\begin{center}
$0,25+0,25=0,5$\\
\end{center}
\begin{center}
$0,5+0,25=0,75$\\
\end{center}
\begin{center}
$0,75+0,25=\textbf{1}$\\
\end{center}
The answer is c).\\



7466-- What is the result of the addition?\\
\begin{center}
3011 + 1009
\end{center}

a) 3119\\
b) 3120\\
c) 4019\\
d) 4020\\

Answer : d)\\

Feedback :\\
\begin{center}
% use packages: array
\begin{tabular}{ccccc}
&  &  & 1 & \\
& 3 & 0 & 1 & 1\\
+ & 1 & 0 & 0 & 9\\
\cline {2-5}
& 4 & 0 & 2 & 0\\
\end{tabular}
\end{center}
The answer is d).\\



7467-- What is the result of the addition?\\
\begin{center}
783 + 538
\end{center}

a) 1211\\
b) 1221\\
c) 1311\\
d) 1321\\

Answer : d)\\

Feedback :\\
\begin{center}
% use packages: array
\begin{tabular}{ccccc}
  & 1 & 1 & 1 & \\
  &   & 7 & 8 & 3\\
+ &   & 5 & 3 & 8\\
\cline {2-5}
  & 1 & 3 & 2 & 1\\
\end{tabular}
\end{center}
The answer is d).\\


7468-- What is the result of the addition?\\
\begin{center}
1400 + 3090 + 801
\end{center}

a) 845\\
b) 4090\\
c) 5090\\
d) 5291\\

Answer : d)\\

Feedback :\\
\begin{center}
% use packages: array
\begin{tabular}{ccccc}
  & 1 &   &   & \\
  & 1 & 4 & 0 & 0\\
+ & 3 & 0 & 9 & 0\\
  &   & 8 & 0 & 1\\
\cline {2-5}
  & 5 & 2 & 9 & 1\\
\end{tabular}
\end{center}
The answer is d).\\




7469-- What is the result of the addition?\\
\begin{center}
10,75 + 2,63
\end{center}

a) 12,138\\
b) 12,38\\
c) 13,138\\
d) 13,38\\

Answer : d)\\

Feedback :\\
\begin{center}
% use packages: array
\begin{tabular}{ccccc}
  &   & 1  &   &   \\
  & 1 & 0, & 7 & 5 \\
+ &   & 2, & 6 & 3 \\
\cline {2-5}
  & 1 & 3, & 3  & 8 \\
\end{tabular}
\end{center}
The answer is d).\\



7470-- What is the result of the addition?\\
\begin{center}
5,01 + 2,01
\end{center}

a) 5,21\\
b) 5,22\\
c) 7,02\\
d) 7,2\\

Answer : c)\\

Feedback :\\
\begin{center}
% use packages: array
\begin{tabular}{cccc}

   & 5, & 0 & 1 \\
+  & 2, & 0 & 1 \\
\cline {2-4}
   & 7, & 0 & 2 \\
\end{tabular}
\end{center}
The answer is c).\\



7471-- What is the result of the addition?\\
\begin{center}
3 + 0,25
\end{center}

a) 3,25\\
b) 5,5\\
c) 28\\
d) 30,25\\

Answer : a)\\

Feedback :\\
\begin{center}
% use packages: array
\begin{tabular}{cccc}

   & 3  &   &  \\
+  & 0, & 2 & 5 \\
\cline {2-4}
   & 3, & 2 & 5 \\
\end{tabular}
\end{center}
The answer is a).\\





7472--Find the number that corresponds to \\
\begin{center}
three hundred fifteen thousand twelve
\end{center}

a) 12 315\\
b) 31 512\\
c) 31 205\\
d) 315 012\\

Answer : d)\\

Feedback :\\
\begin{center}
% use packages: array
\begin{tabular}{|rrr|rrr|rrr|}
\hline
\multicolumn{6}{|c|}{integers} &\multicolumn{3}{|c|}{decimals} \\
\hline
\multicolumn{3}{|c|}{Class of} &\multicolumn{3}{|c|}{Class of} &  \multicolumn{3}{c|}{} \\
\multicolumn{3}{|c|}{thousands} &\multicolumn{3}{|c|}{units} &  \multicolumn{3}{c|}{} \\
\hline
H & T & U &H & T & U, & T\up{th} & \textbf{H\up{th}} & K\up{th} \\
\hline
\hline
8 & 0 & 0 & 0 & 0 & 0 &  & &\\
 & 7 & 0 & 0 & 0 & 0 &  & &\\
+ &  &  & 5 & 0 & 0 &  & &\\
 &  &  &  & 2 & 0 &  & &\\
 &  &  &  &  & 3 &  & &\\
\hline
\hline
 8 & 7 & 0 & 5 & 2 & 3 &  & &
\\
\hline
\end{tabular}
\end{center}

\scriptsize
\begin{center}
% use packages: array
\begin{tabular}{ll}
H : Hundreds & T\up{th} : Tenths\\
T : Tens & H\up{th} :Hundredths\\
U : Units & K\up{e} : Thousandths\\
\end{tabular}
\end{center}

\normalsize

The answer is d).\\





7473--Fin the number that corresponds to \\
\begin{center}
thirty-nine thousand four hundred fifty-two
\end{center}

a) 39 452\\
b) 49 452\\
c) 394 052\\
d) 452 039\\

Answer : a)\\

Feedback :\\
\begin{center}
% use packages: array
\begin{tabular}{|rrr|rrr|rrr|}
\hline
\multicolumn{6}{|c|}{integers} &\multicolumn{3}{|c|}{decimals} \\
\hline
\multicolumn{3}{|c|}{Class of} &\multicolumn{3}{|c|}{Class of} &  \multicolumn{3}{c|}{} \\
\multicolumn{3}{|c|}{thousands} &\multicolumn{3}{|c|}{units} &  \multicolumn{3}{c|}{} \\
\hline
H & T & U &H & T & U, & T\up{th} & \textbf{H\up{th}} & K\up{th} \\
\hline
\hline
8 & 0 & 0 & 0 & 0 & 0 &  & &\\
 & 7 & 0 & 0 & 0 & 0 &  & &\\
+ &  &  & 5 & 0 & 0 &  & &\\
 &  &  &  & 2 & 0 &  & &\\
 &  &  &  &  & 3 &  & &\\
\hline
\hline
 8 & 7 & 0 & 5 & 2 & 3 &  & &
\\
\hline
\end{tabular}
\end{center}

\scriptsize
\begin{center}
% use packages: array
\begin{tabular}{ll}
H : Hundreds & T\up{th} : Tenths\\
T : Tens & H\up{th} :Hundredths\\
U : Units & K\up{e} : Thousandths\\
\end{tabular}
\end{center}

\normalsize
The answer is a).\\



7474--Find the number that corresponds to \\
\begin{center}
seven hundred thousand eighty-nine
\end{center}

a) 89 700\\
b) 89 789\\
c) 700 089\\
d) 789 000\\

Answer : c)\\

Feedback :\\
\begin{center}
% use packages: array
\begin{tabular}{|rrr|rrr|rrr|}
\hline
\multicolumn{6}{|c|}{integers} &\multicolumn{3}{|c|}{decimals} \\
\hline
\multicolumn{3}{|c|}{Class of} &\multicolumn{3}{|c|}{Class of} &  \multicolumn{3}{c|}{} \\
\multicolumn{3}{|c|}{thousands} &\multicolumn{3}{|c|}{units} &  \multicolumn{3}{c|}{} \\
\hline
H & T & U &H & T & U, & T\up{th} & \textbf{H\up{th}} & K\up{th} \\
\hline
\hline
8 & 0 & 0 & 0 & 0 & 0 &  & &\\
 & 7 & 0 & 0 & 0 & 0 &  & &\\
+ &  &  & 5 & 0 & 0 &  & &\\
 &  &  &  & 2 & 0 &  & &\\
 &  &  &  &  & 3 &  & &\\
\hline
\hline
 8 & 7 & 0 & 5 & 2 & 3 &  & &
\\
\hline
\end{tabular}
\end{center}

\scriptsize
\begin{center}
% use packages: array
\begin{tabular}{ll}
H : Hundreds & T\up{th} : Tenths\\
T : Tens & H\up{th} :Hundredths\\
U : Units & K\up{e} : Thousandths\\
\end{tabular}
\end{center}

\normalsize

The answer is c).\\



\end{document} 