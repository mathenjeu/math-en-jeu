\documentclass[letterpaper, 12pt]{article}
\usepackage[french]{babel}

\usepackage{amsmath,amsfonts,amsthm,amssymb,graphicx,multirow,hyperref,color}
\usepackage[latin1]{inputenc}

\pagestyle{plain}

\setlength{\topmargin}{-2cm}
\setlength{\textheight}{23.5cm}
\setlength{\textwidth}{12cm}
\setlength{\oddsidemargin}{-1cm}
\setlength{\parindent}{0pt}


\begin{document}


7575--What is the greatest common divisor (GCD) of numbers 2 and 6?\\
Example:\\
\begin{center}
% use packages: array
\begin{tabular}{|c|}
\hline
Divisors of 4 : \{1, 2, 4\}\\
Divisors of 6 : \{1, 2, 3, 6\}\\
The GCD is 2.\\
\hline
\end{tabular}
\end{center}

a) 0\\
b) 2\\
c) 10\\
d) 12\\

Answer : b)\\

Feedback :\\
First, we have to find the divisors for each number.\\
Then, we determine the common divisors.\\
Finally, we select the greatest common divisor of the to numbers.\\
\begin{center}
\includegraphics[width=3cm]{R7575.eps}\\
% R7575.eps : 300dpi, width=3.39cm, height=3.39cm, bb=0 0 400 400\\
\end{center}
The answer is b).\\



7576--Find an equivalent fraction of $\frac{1}{6}$.\\

a) $\frac{3}{18}$\\
\\
b) $\frac{4}{20}$\\
\\
c) $\frac{6}{12}$\\
\\
d) $\frac{2}{3}$\\

Answer : a)\\

Feedback :\\
\begin{center}
\begin{tabular}{c}
$\times 3$  \\
\LARGE $\curvearrowright$  \\
\Large $\frac{1}{6}$  =  \Large $\frac{3}{18}$ \\
\rotatebox{180}{\LARGE$\curvearrowleft$}\\
$\times 3$  \\
\end{tabular}
\end{center}
The answer is a).\\



7577--Find an equivalent fraction of $\frac{2}{7}$.\\

a) $\frac{1}{4}$\\
\\
b) $\frac{10}{35}$\\
\\
c) $\frac{7}{14}$\\
\\
d) $\frac{6}{9}$\\

Answer : b)\\

Feedback :\\
\begin{center}
\begin{tabular}{c}
$\times 5$  \\
\LARGE $\curvearrowright$  \\
\Large $\frac{2}{7}$  =  \Large $\frac{10}{35}$ \\
\rotatebox{180}{\LARGE$\curvearrowleft$}\\
$\times 5$  \\
\end{tabular}
\end{center}
The answer is b).\\



7578--Find an equivalent fraction of $\frac{1}{2}$.\\

a) $\frac{1}{8}$\\
\\
b) $\frac{2}{6}$\\
\\
c) $\frac{3}{9}$\\
\\
d) $\frac{2}{4}$\\

Answer : d)\\

Feedback :\\
\begin{center}
\begin{tabular}{c}
$\times 2$  \\
\LARGE $\curvearrowright$  \\
\Large $\frac{1}{2}$  =  \Large $\frac{2}{4}$ \\
\rotatebox{180}{\LARGE$\curvearrowleft$}\\
$\times 2$  \\
\end{tabular}
\end{center}
The answer is d).\\



7579--Find an equivalent fraction of $\frac{3}{4}$.\\

a) $\frac{4}{16}$\\
\\
b) $\frac{1}{2}$\\
\\
c) $\frac{20}{30}$\\
\\
d) $\frac{18}{24}$\\

Answer : d)\\

Feedback :\\
\begin{center}
\begin{tabular}{c}
$\times 6$  \\
\LARGE $\curvearrowright$  \\
\Large $\frac{3}{4}$  =  \Large $\frac{18}{24}$ \\
\rotatebox{180}{\LARGE$\curvearrowleft$}\\
$\times 6$  \\
\end{tabular}
\end{center}
The answer is d).\\




7580--Find the equivalent fraction.\\
\begin{center}
\includegraphics[width=2cm]{Q7580e.eps}\\
% Q7580e.eps : 300dpi, width=3.39cm, height=3.39cm, bb=0 0 400 400\\
\end{center}

a)\\
 \includegraphics[width=2cm]{Q7580a.eps}\\
% Q7580a.eps : 300dpi, width=3.39cm, height=3.39cm, bb=0 0 400 400\\
\\
b)\\
\includegraphics[width=2cm]{Q7580b.eps}\\
% Q7580b.eps : 300dpi, width=3.39cm, height=3.39cm, bb=0 0 400 400\\
\\
c)\\
\includegraphics[width=2cm]{Q7580c.eps}\\
% Q7580c.eps : 300dpi, width=3.39cm, height=3.39cm, bb=0 0 400 400\\
\\
d)\\
\includegraphics[width=2cm]{Q7580d.eps}\\
% Q7580d.eps : 300dpi, width=3.39cm, height=3.39cm, bb=0 0 400 400\\
\\

R�ponse : d)\\

R�troaction :\\
\begin{center}
\includegraphics[width=5cm]{R7580.eps}\\
% R7580.eps : 300dpi, width=3.39cm, height=3.39cm, bb=0 0 400 400\\
\end{center}
The answer is d).\\




7581--Find the equivalent fraction.\\
\begin{center}
\includegraphics[width=2cm]{Q7581e.eps}\\
% Q7581e.eps : 300dpi, width=3.39cm, height=3.39cm, bb=0 0 400 400\\
\end{center}

a)\\
 \includegraphics[width=2cm]{Q7581a.eps}\\
% Q7581a.eps : 300dpi, width=3.39cm, height=3.39cm, bb=0 0 400 400\\
\\
b)\\
\includegraphics[width=2cm]{Q7581b.eps}\\
% Q7581b.eps : 300dpi, width=3.39cm, height=3.39cm, bb=0 0 400 400\\
\\
c)\\
\includegraphics[width=2cm]{Q7581c.eps}\\
% Q7581c.eps : 300dpi, width=3.39cm, height=3.39cm, bb=0 0 400 400\\
\\
d)\\
\includegraphics[width=2cm]{Q7581d.eps}\\
% Q7581d.eps : 300dpi, width=3.39cm, height=3.39cm, bb=0 0 400 400\\
\\

R�ponse : c)\\

R�troaction :\\
\begin{center}
\includegraphics[width=5cm]{R7581.eps}\\
% R7581.eps : 300dpi, width=3.39cm, height=3.39cm, bb=0 0 400 400\\
\end{center}
The answer is c).\\



7582--Find the equivalent fraction.\\
\begin{center}
\includegraphics[width=2cm]{Q7582e.eps}\\
% Q7582e.eps : 300dpi, width=3.39cm, height=3.39cm, bb=0 0 400 400\\
\end{center}

a)\\
 \includegraphics[width=2cm]{Q7582a.eps}\\
% Q7582a.eps : 300dpi, width=3.39cm, height=3.39cm, bb=0 0 400 400\\
\\
b)\\
\includegraphics[width=2cm]{Q7582b.eps}\\
% Q7582b.eps : 300dpi, width=3.39cm, height=3.39cm, bb=0 0 400 400\\
\\
c)\\
\includegraphics[width=2cm]{Q7582c.eps}\\
% Q7582c.eps : 300dpi, width=3.39cm, height=3.39cm, bb=0 0 400 400\\
\\
d)\\
\includegraphics[width=2cm]{Q7582d.eps}\\
% Q7582d.eps : 300dpi, width=3.39cm, height=3.39cm, bb=0 0 400 400\\
\\

R�ponse : a)\\

R�troaction :\\
\begin{center}
\includegraphics[width=5cm]{R7582.eps}\\
% R7582.eps : 300dpi, width=3.39cm, height=3.39cm, bb=0 0 400 400\\
\end{center}
The answer is a).\\



7583--Find the equivalent fraction.\\
\begin{center}
\includegraphics[width=2cm]{Q7583e.eps}\\
% Q7583e.eps : 300dpi, width=3.39cm, height=3.39cm, bb=0 0 400 400\\
\end{center}

a)\\
 \includegraphics[width=2cm]{Q7583a.eps}\\
% Q7583a.eps : 300dpi, width=3.39cm, height=3.39cm, bb=0 0 400 400\\
\\
b)\\
\includegraphics[width=2cm]{Q7583b.eps}\\
% Q7583b.eps : 300dpi, width=3.39cm, height=3.39cm, bb=0 0 400 400\\
\\
c)\\
\includegraphics[width=2cm]{Q7583c.eps}\\
% Q7583c.eps : 300dpi, width=3.39cm, height=3.39cm, bb=0 0 400 400\\
\\
d)\\
\includegraphics[width=2cm]{Q7583d.eps}\\
% Q7583d.eps : 300dpi, width=3.39cm, height=3.39cm, bb=0 0 400 400\\
\\

Answer : c)\\

Feedback :\\
\begin{center}
\includegraphics[width=5cm]{R7583.eps}\\
% R7583.eps : 300dpi, width=3.39cm, height=3.39cm, bb=0 0 400 400\\
\end{center}
The answer is c).\\


7584--Mia needs 3 eggs to make banana bread. If she takes her eggs in a box that contains a dozen, what is the fraction that corresponds to the number of eggs she took?\\

a) $\frac{1}{6}$\\
\\
b) $\frac{1}{4}$\\
\\
c) $\frac{3}{10}$\\
\\
d) $\frac{1}{3}$\\

Answer : b)\\

Feedback :\\
A dozen means that there are 12 eggs in the box. Mia took 3 eggs on a total of 12. So she took $\frac{3}{12}$ of the eggs.\\
\begin{center}
\includegraphics[width=3cm]{R7584a.eps}\\
% R7584a.eps : 300dpi, width=3.39cm, height=3.39cm, bb=0 0 400 400\\
\end{center}

However, $\frac{3}{12}$ can be reduced to $\frac{1}{4}$.

\begin{center}
\begin{tabular}{c}
$\div 3$  \\
\LARGE $\curvearrowright$  \\
\Large $\frac{3}{12}$  =  \Large $\frac{1}{4}$ \\
\rotatebox{180}{\LARGE$\curvearrowleft$}\\
$\div 3$  \\
\end{tabular}
\end{center}

\begin{center}
\begin{tabular}{ccc}
\includegraphics[width=2cm]{R7584a.eps}
% R7584a.eps : 300dpi, width=3.39cm, height=3.39cm, bb=0 0 400 400
& = &
\includegraphics[width=2cm]{R7584b.eps}
% R7584b.eps : 300dpi, width=3.39cm, height=3.39cm, bb=0 0 400 400
\end{tabular}
\end{center}
The answer is b).\\


7585--Mia needs 3 eggs to make banana bread. If she takes her eggs in a box that contains a dozen, what is the fraction that corresponds to the number of eggs she took?\\

a) $\frac{3}{12}$\\
\\
b) $\frac{3}{10}$\\
\\
c) $\frac{3}{6}$\\
\\
d) $\frac{3}{3}$\\

Answer : a)\\

Feedback :\\
A dozen means that there are 12 eggs in the box. Mia took 3 eggs on a total of 12. So she took $\frac{3}{12}$ of the eggs\\
\begin{center}
\includegraphics[width=3cm]{R7585.eps}\\
% R7585.eps : 300dpi, width=3.39cm, height=3.39cm, bb=0 0 400 400\\
\end{center}
The answer is a).\\



7586--Alexandre wants to make Easter eggs with real eggs. He looks in the dozen eggs carton and there are only 4 left. What is the fraction that corresponds to the amount of eggs that are missing?\\

a) $\frac{4}{12}$\\
\\
b) $\frac{4}{10}$\\
\\
c) $\frac{8}{12}$\\
\\
d) $\frac{8}{10}$\\

Answer : c)\\

Feedback :\\
A dozen eggs means that there are 12 eggs. If there are 4 left, 8 are missing, because $12 - 4 = 8$. So 8 eggs are missing on a total of 12, so $\frac{8}{12}$ of the carton.\\
\begin{center}
\includegraphics[width=3cm]{R7586.eps}\\
% R7586.eps : 300dpi, width=3.39cm, height=3.39cm, bb=0 0 400 400\\
\end{center}
The answer is c).\\



7587--Ma�ka is a member of her school kermesse event. For the transoprtation-of-eggs-with-chopsticks challenge, she needs 18 eggs. How many dozen eggs does she need?\\

a) $\frac{1}{2}$\\
\\
b) $\frac{8}{10}$\\
\\
c) 1$\frac{1}{2}$\\
\\
d) 1$\frac{8}{10}$\\

Answer : c)\\

Feedback :\\
A dozen eggs means 12 eggs. If she only buys one dozen, she is missing 6 eggs.\\
\begin{eqnarray*}
18 - 12 = 6
\end{eqnarray*}
\begin{center}
\includegraphics[width=3cm]{R7587a.eps}\\
% R7587a.eps : 300dpi, width=3.39cm, height=3.39cm, bb=0 0 400 400\\
\end{center}
Ma�ka needs to buy two dozens, but she will only use 6 out of 12 for one of them. Thus, she will use $\frac{6}{12}$ of the carton, which is half.
\begin{center}
\begin{tabular}{c}
$\div 6$  \\
\LARGE $\curvearrowright$  \\
\Large $\frac{6}{12}$  =  \Large $\frac{1}{2}$ \\
\rotatebox{180}{\LARGE$\curvearrowleft$}\\
$\div 6$  \\
\end{tabular}
\end{center}
\begin{center}
\includegraphics[width=3cm]{R7587b.eps}\\
% R7587b.eps : 300dpi, width=3.39cm, height=3.39cm, bb=0 0 400 400\\
\end{center}
Ma�ka needs a whole dozen, plus half a dozen. Thus, she needs 1$\frac{1}{2}$ dozen eggs.
\begin{center}
\includegraphics[width=3cm]{R7587c.eps}\\
% R7587c.eps : 300dpi, width=3.39cm, height=3.39cm, bb=0 0 400 400\\
\end{center}
The answer is c).\\



7588--Which of the following is the lowest fraction?\\

a) $\frac{2}{10}$\\
\\
b) $\frac{2}{5}$\\
\\
c) $\frac{9}{15}$\\
\\
d) $\frac{4}{5}$\\

Answer : a)\\

Feedback :\\
First, we have to put the fracions on the same denominator. Here, the common denominator can 5.\\
\begin{center}
\begin{tabular}{c}
$\div 2$  \\
\LARGE $\curvearrowright$  \\
\Large $\frac{2}{10}$  =  \Large $\frac{1}{5}$ \\
\rotatebox{180}{\LARGE$\curvearrowleft$}\\
$\div 2$  \\
\end{tabular}
\end{center}

\begin{center}
\begin{tabular}{c}
$\div 3$  \\
\LARGE $\curvearrowright$  \\
\Large $\frac{9}{15}$  =  \Large $\frac{3}{5}$ \\
\rotatebox{180}{\LARGE$\curvearrowleft$}\\
$\div 3$  \\
\end{tabular}
\end{center}

Then, we only need to find the fraction that has the lowest numerator.\\
\\
$\frac{\textbf{1}}{5}(\frac{2}{10})$\\
\\
$\frac{2}{5}$\\
\\
$\frac{3}{5}(\frac{9}{15})$\\
\\
$\frac{4}{5}$\\
\\
The answer is a).\\



7589--Which of the following fraction is the greatest?\\

a) $\frac{2}{10}$\\
\\
b) $\frac{2}{5}$\\
\\
c) $\frac{9}{15}$\\
\\
d) $\frac{4}{5}$\\

Answer : d)\\

Feedback :\\
First, we have to put the fracions on the same denominator. Here, the common denominator can 5.\\
\begin{center}
\begin{tabular}{c}
$\div 2$  \\
\LARGE $\curvearrowright$  \\
\Large $\frac{2}{10}$  =  \Large $\frac{1}{5}$ \\
\rotatebox{180}{\LARGE$\curvearrowleft$}\\
$\div 2$  \\
\end{tabular}
\end{center}

\begin{center}
\begin{tabular}{c}
$\div 3$  \\
\LARGE $\curvearrowright$  \\
\Large $\frac{9}{15}$  =  \Large $\frac{3}{5}$ \\
\rotatebox{180}{\LARGE$\curvearrowleft$}\\
$\div 3$  \\
\end{tabular}
\end{center}

Then, we only need to find the fraction that has the greatest numerator.\\
\\
$\frac{1}{5}(\frac{2}{10})$\\
\\
$\frac{2}{5}$\\
\\
$\frac{3}{5}(\frac{9}{15})$\\
\\
$\frac{\textbf{4}}{5}$\\
\\
The answer is d).\\




7590--Which of the following fraction is the lowest?\\

a) $\frac{1}{9}$\\
\\
b) $\frac{6}{18}$\\
\\
c) $\frac{2}{3}$\\
\\
d) $\frac{5}{6}$\\

Answer : a)\\

Feedback :\\
First, we have to put the fracions on the same denominator. Here, the common denominator can 18.\\
\begin{center}
\begin{tabular}{c}
$\times 2$  \\
\LARGE $\curvearrowright$  \\
\Large $\frac{1}{9}$  =  \Large $\frac{2}{18}$ \\
\rotatebox{180}{\LARGE$\curvearrowleft$}\\
$\times 2$  \\
\end{tabular}
\end{center}

\begin{center}
\begin{tabular}{c}
$\times 6$  \\
\LARGE $\curvearrowright$  \\
\Large $\frac{2}{3}$  =  \Large $\frac{12}{18}$ \\
\rotatebox{180}{\LARGE$\curvearrowleft$}\\
$\times 6$  \\
\end{tabular}
\end{center}

\begin{center}
\begin{tabular}{c}
$\times 3$  \\
\LARGE $\curvearrowright$  \\
\Large $\frac{5}{6}$  =  \Large $\frac{15}{18}$ \\
\rotatebox{180}{\LARGE$\curvearrowleft$}\\
$\times 3$  \\
\end{tabular}
\end{center}

Then, we only need to find the fraction that has the lowest numerator.\\
\\
$\frac{\textbf{2}}{18}(\frac{1}{9})$\\
\\
$\frac{6}{18}$\\
\\
$\frac{12}{18}(\frac{2}{3})$\\
\\
$\frac{15}{18}(\frac{5}{6})$\\
\\
The answer is a).\\



7591--Which of the following is the greatest fraction?\\

a) $\frac{1}{9}$\\
\\
b) $\frac{6}{18}$\\
\\
c) $\frac{2}{3}$\\
\\
d) $\frac{5}{6}$\\

Answer : d)\\

Feedback :\\
First, we have to put the fracions on the same denominator. Here, the common denominator can 18.\\
\begin{center}
\begin{tabular}{c}
$\times 2$  \\
\LARGE $\curvearrowright$  \\
\Large $\frac{1}{9}$  =  \Large $\frac{2}{18}$ \\
\rotatebox{180}{\LARGE$\curvearrowleft$}\\
$\times 2$  \\
\end{tabular}
\end{center}

\begin{center}
\begin{tabular}{c}
$\times 6$  \\
\LARGE $\curvearrowright$  \\
\Large $\frac{2}{3}$  =  \Large $\frac{12}{18}$ \\
\rotatebox{180}{\LARGE$\curvearrowleft$}\\
$\times 6$  \\
\end{tabular}
\end{center}

\begin{center}
\begin{tabular}{c}
$\times 3$  \\
\LARGE $\curvearrowright$  \\
\Large $\frac{5}{6}$  =  \Large $\frac{15}{18}$ \\
\rotatebox{180}{\LARGE$\curvearrowleft$}\\
$\times 3$  \\
\end{tabular}
\end{center}

Then, we only need to find the fraction that has the greatest numerator.\\
\\
$\frac{2}{18}(\frac{1}{9})$\\
\\
$\frac{6}{18}$\\
\\
$\frac{12}{18}(\frac{2}{3})$\\
\\
$\frac{\textbf{15}}{18}(\frac{5}{6})$\\
\\
The answer is d).\\




7592--Amongst the following, which fraction is the lowest?\\

a) $\frac{1}{2}$\\
\\
b) $\frac{2}{3}$\\
\\
c) $\frac{5}{6}$\\
\\
d) $\frac{2}{1}$\\

Answer : a)\\

Feedback :\\
First, we have to put the fracions on the same denominator. Here, the common denominator can 6.\\
\begin{center}
\begin{tabular}{c}
$\times 3$  \\
\LARGE $\curvearrowright$  \\
\Large $\frac{1}{2}$  =  \Large $\frac{3}{6}$ \\
\rotatebox{180}{\LARGE$\curvearrowleft$}\\
$\times 3$  \\
\end{tabular}
\end{center}

\begin{center}
\begin{tabular}{c}
$\times 2$  \\
\LARGE $\curvearrowright$  \\
\Large $\frac{2}{3}$  =  \Large $\frac{4}{6}$ \\
\rotatebox{180}{\LARGE$\curvearrowleft$}\\
$\times 2$  \\
\end{tabular}
\end{center}

\begin{center}
\begin{tabular}{c}
$\times 6$  \\
\LARGE $\curvearrowright$  \\
\Large $\frac{2}{1}$  =  \Large $\frac{12}{6}$ \\
\rotatebox{180}{\LARGE$\curvearrowleft$}\\
$\times 6$  \\
\end{tabular}
\end{center}

Then, we only need to find the fraction that has the lowest numerator.\\
\\
$\frac{\textbf{3}}{6}(\frac{1}{2})$\\
\\
$\frac{4}{6}(\frac{2}{3})$\\
\\
$\frac{5}{6}$\\
\\
$\frac{12}{6}(\frac{2}{1})$\\
\\
The answer is a).\\




7593--Which of the following fractions is the greatest?\\

a) $\frac{1}{2}$\\
\\
b) $\frac{2}{3}$\\
\\
c) $\frac{5}{6}$\\
\\
d) $\frac{2}{1}$\\

Answer : d)\\

Feedback :\\
First, we have to put the fracions on the same denominator. Here, the common denominator can be 6.\\
\begin{center}
\begin{tabular}{c}
$\times 3$  \\
\LARGE $\curvearrowright$  \\
\Large $\frac{1}{2}$  =  \Large $\frac{3}{6}$ \\
\rotatebox{180}{\LARGE$\curvearrowleft$}\\
$\times 3$  \\
\end{tabular}
\end{center}

\begin{center}
\begin{tabular}{c}
$\times 2$  \\
\LARGE $\curvearrowright$  \\
\Large $\frac{2}{3}$  =  \Large $\frac{4}{6}$ \\
\rotatebox{180}{\LARGE$\curvearrowleft$}\\
$\times 2$  \\
\end{tabular}
\end{center}

\begin{center}
\begin{tabular}{c}
$\times 6$  \\
\LARGE $\curvearrowright$  \\
\Large $\frac{2}{1}$  =  \Large $\frac{12}{6}$ \\
\rotatebox{180}{\LARGE$\curvearrowleft$}\\
$\times 6$  \\
\end{tabular}
\end{center}

Then, we only need to find the fraction that has the greatest numerator.\\
\\
$\frac{3}{6}(\frac{1}{2})$\\
\\
$\frac{4}{6}(\frac{2}{3})$\\
\\
$\frac{5}{6}$\\
\\
$\frac{\textbf{12}}{6}(\frac{2}{1})$\\
\\
The answer is d).\\




7594--Which of the following fractions is the lowest?\\

a) $\frac{1}{16}$\\
\\
b) $\frac{1}{8}$\\
\\
c) $\frac{1}{4}$\\
\\
d) $\frac{1}{2}$\\

Answer : a)\\

Feedback :\\
First, we have to put the fracions on the same denominator. Here, the common denominator can be 16.\\
\begin{center}
\begin{tabular}{c}
$\times 2$  \\
\LARGE $\curvearrowright$  \\
\Large $\frac{1}{8}$  =  \Large $\frac{2}{16}$ \\
\rotatebox{180}{\LARGE$\curvearrowleft$}\\
$\times 2$  \\
\end{tabular}
\end{center}

\begin{center}
\begin{tabular}{c}
$\times 4$  \\
\LARGE $\curvearrowright$  \\
\Large $\frac{1}{4}$  =  \Large $\frac{4}{16}$ \\
\rotatebox{180}{\LARGE$\curvearrowleft$}\\
$\times 4$  \\
\end{tabular}
\end{center}

\begin{center}
\begin{tabular}{c}
$\times 8$  \\
\LARGE $\curvearrowright$  \\
\Large $\frac{1}{2}$  =  \Large $\frac{8}{16}$ \\
\rotatebox{180}{\LARGE$\curvearrowleft$}\\
$\times 8$  \\
\end{tabular}
\end{center}

Then, we only need to find the fraction that has the lowest numerator.\\
\\
$\frac{\textbf{1}}{16}$\\
\\
$\frac{2}{16}(\frac{1}{8})$\\
\\
$\frac{4}{16}(\frac{1}{4})$\\
\\
$\frac{8}{16}(\frac{1}{2})$\\
\\
The answer is a).\\




7595--Which of the following is the greatest fraction?\\

a) $\frac{1}{16}$\\
\\
b) $\frac{1}{8}$\\
\\
c) $\frac{1}{4}$\\
\\
d) $\frac{1}{2}$\\

Answer : d)\\

Feedback :\\
First, we have to put the fracions on the same denominator. Here, the common denominator can be 16.\\
\begin{center}
\begin{tabular}{c}
$\times 2$  \\
\LARGE $\curvearrowright$  \\
\Large $\frac{1}{8}$  =  \Large $\frac{2}{16}$ \\
\rotatebox{180}{\LARGE$\curvearrowleft$}\\
$\times 2$  \\
\end{tabular}
\end{center}

\begin{center}
\begin{tabular}{c}
$\times 4$  \\
\LARGE $\curvearrowright$  \\
\Large $\frac{1}{4}$  =  \Large $\frac{4}{16}$ \\
\rotatebox{180}{\LARGE$\curvearrowleft$}\\
$\times 4$  \\
\end{tabular}
\end{center}

\begin{center}
\begin{tabular}{c}
$\times 8$  \\
\LARGE $\curvearrowright$  \\
\Large $\frac{1}{2}$  =  \Large $\frac{8}{16}$ \\
\rotatebox{180}{\LARGE$\curvearrowleft$}\\
$\times 8$  \\
\end{tabular}
\end{center}

Then, we only need to find the fraction that has the greatest numerator.\\
\\
$\frac{1}{16}$\\
\\
$\frac{2}{16}(\frac{1}{8})$\\
\\
$\frac{4}{16}(\frac{1}{4})$\\
\\
$\frac{\textbf{8}}{16}(\frac{1}{2})$\\
\\
The answer is d).\\



7596--Which of the following is the lowest fraction?\\

a)\\
 \includegraphics[width=2cm]{Q7596a.eps}\\
% Q7596a.eps : 300dpi, width=3.39cm, height=3.39cm, bb=0 0 400 400\\
\\
b)\\
\includegraphics[width=2cm]{Q7596c.eps}\\
% Q7596c.eps : 300dpi, width=3.39cm, height=3.39cm, bb=0 0 400 400\\
\\
c)\\
\includegraphics[width=2cm]{Q7596b.eps}\\
% Q7596b.eps : 300dpi, width=3.39cm, height=3.39cm, bb=0 0 400 400\\
\\
d)\\
\includegraphics[width=2cm]{Q7596d.eps}\\
% Q7596d.eps : 300dpi, width=3.39cm, height=3.39cm, bb=0 0 400 400\\
\\

Answer : a)\\

Feedback :\\
\begin{center}
\includegraphics[width=3cm]{R7596.eps}\\
% R7596.eps : 300dpi, width=3.39cm, height=3.39cm, bb=0 0 400 400\\
\end{center}
The answer is a).\\


7597--Which of the following is the greatest fraction?\\

a)\\
 \includegraphics[width=2cm]{Q7597a.eps}\\
% Q7597a.eps : 300dpi, width=3.39cm, height=3.39cm, bb=0 0 400 400\\
\\
b)\\
\includegraphics[width=2cm]{Q7597c.eps}\\
% Q7597c.eps : 300dpi, width=3.39cm, height=3.39cm, bb=0 0 400 400\\
\\
c)\\
\includegraphics[width=2cm]{Q7597b.eps}\\
% Q7597b.eps : 300dpi, width=3.39cm, height=3.39cm, bb=0 0 400 400\\
\\
d)\\
\includegraphics[width=2cm]{Q7597d.eps}\\
% Q7597d.eps : 300dpi, width=3.39cm, height=3.39cm, bb=0 0 400 400\\
\\

Answer : d)\\

Feedback :\\
\begin{center}
\includegraphics[width=3cm]{R7597.eps}\\
% R7597.eps : 300dpi, width=3.39cm, height=3.39cm, bb=0 0 400 400\\
\end{center}
The answer is d).\\



7598--Which of the following is the lowest fraction?\\

a)\\
 \includegraphics[width=2cm]{Q7598a.eps}\\
% Q7598a.eps : 300dpi, width=3.39cm, height=3.39cm, bb=0 0 400 400\\
\\
b)\\
\includegraphics[width=2cm]{Q7598b.eps}\\
% Q7598b.eps : 300dpi, width=3.39cm, height=3.39cm, bb=0 0 400 400\\
\\
c)\\
\includegraphics[width=2cm]{Q7598c.eps}\\
% Q7598c.eps : 300dpi, width=3.39cm, height=3.39cm, bb=0 0 400 400\\
\\
d)\\
\includegraphics[width=2cm]{Q7598d.eps}\\
% Q7598d.eps : 300dpi, width=3.39cm, height=3.39cm, bb=0 0 400 400\\
\\

Answer : a)\\

Feedback :\\
\begin{center}
\includegraphics[width=3cm]{R7598.eps}\\
% R7598.eps : 300dpi, width=3.39cm, height=3.39cm, bb=0 0 400 400\\
\end{center}
The answer is a).\\




7599--Which of the following is the greatest fraction?\\

a)\\
 \includegraphics[width=2cm]{Q7599a.eps}\\
% Q7599a.eps : 300dpi, width=3.39cm, height=3.39cm, bb=0 0 400 400\\
\\
b)\\
\includegraphics[width=2cm]{Q7599b.eps}\\
% Q7599b.eps : 300dpi, width=3.39cm, height=3.39cm, bb=0 0 400 400\\
\\
c)\\
\includegraphics[width=2cm]{Q7599c.eps}\\
% Q7599c.eps : 300dpi, width=3.39cm, height=3.39cm, bb=0 0 400 400\\
\\
d)\\
\includegraphics[width=2cm]{Q7599d.eps}\\
% Q7599d.eps : 300dpi, width=3.39cm, height=3.39cm, bb=0 0 400 400\\
\\

Answer : d)\\

Feedback :\\
\begin{center}
\includegraphics[width=3cm]{R7599.eps}\\
% R7599.eps : 300dpi, width=3.39cm, height=3.39cm, bb=0 0 400 400\\
\end{center}
The answer is d).\\



\end{document} 