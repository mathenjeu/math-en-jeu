\documentclass[letterpaper, 12pt]{article}
\usepackage[french]{babel}

\usepackage{amsmath,amsfonts,amsthm,amssymb,graphicx,multirow,hyperref,color,ulem}
\usepackage[latin1]{inputenc}

\pagestyle{plain}

\setlength{\topmargin}{-2cm}
\setlength{\textheight}{23.5cm}
\setlength{\textwidth}{12cm}
\setlength{\oddsidemargin}{-1cm}
\setlength{\parindent}{0pt}


\begin{document}

7550--Add 9 tens to the number 451.\\

a) 361\\
b) 442\\
c) 460\\
d) 541\\

Answer : d)\\

Feedback :\\
To add, you need to do an addition.
\begin{center}
% use packages: array
\begin{tabular}{|rrr|rrr|rrr|}
\hline
\multicolumn{6}{|c|}{integers} &\multicolumn{3}{|c|}{decimals} \\
\hline
\multicolumn{3}{|c|}{Class of} &\multicolumn{3}{|c|}{Class of} &  \multicolumn{3}{c|}{} \\
\multicolumn{3}{|c|}{thousands} &\multicolumn{3}{|c|}{units} &  \multicolumn{3}{c|}{} \\
\hline
H & T & U &H & T & U, & T\up{th} & \textbf{H\up{th}} & K\up{th} \\
\hline
\hline
8 & 0 & 0 & 0 & 0 & 0 &  & &\\
 & 7 & 0 & 0 & 0 & 0 &  & &\\
+ &  &  & 5 & 0 & 0 &  & &\\
 &  &  &  & 2 & 0 &  & &\\
 &  &  &  &  & 3 &  & &\\
\hline
\hline
 8 & 7 & 0 & 5 & 2 & 3 &  & &
\\
\hline
\end{tabular}
\end{center}

\scriptsize
\begin{center}
% use packages: array
\begin{tabular}{ll}
H : Hundreds & T\up{th} : Tenths\\
T : Tens & H\up{th} :Hundredths\\
U : Units & K\up{e} : Thousandths\\
\end{tabular}
\end{center}

\normalsize
The answer is d).\\






7551--Add 10 tens to 129.\\

a) 29\\
b) 130\\
c) 139\\
d) 229\\

Answer : d)\\

Feedback :\\
To add, you need to do an addition.
\begin{center}
% use packages: array
\begin{tabular}{|rrr|rrr|rrr|}
\hline
\multicolumn{6}{|c|}{integers} &\multicolumn{3}{|c|}{decimals} \\
\hline
\multicolumn{3}{|c|}{Class of} &\multicolumn{3}{|c|}{Class of} &  \multicolumn{3}{c|}{} \\
\multicolumn{3}{|c|}{thousands} &\multicolumn{3}{|c|}{units} &  \multicolumn{3}{c|}{} \\
\hline
H & T & U &H & T & U, & T\up{th} & \textbf{H\up{th}} & K\up{th} \\
\hline
\hline
8 & 0 & 0 & 0 & 0 & 0 &  & &\\
 & 7 & 0 & 0 & 0 & 0 &  & &\\
+ &  &  & 5 & 0 & 0 &  & &\\
 &  &  &  & 2 & 0 &  & &\\
 &  &  &  &  & 3 &  & &\\
\hline
\hline
 8 & 7 & 0 & 5 & 2 & 3 &  & &
\\
\hline
\end{tabular}
\end{center}

\scriptsize
\begin{center}
% use packages: array
\begin{tabular}{ll}
H : Hundreds & T\up{th} : Tenths\\
T : Tens & H\up{th} :Hundredths\\
U : Units & K\up{e} : Thousandths\\
\end{tabular}
\end{center}

\normalsize
The answer is d).\\




7552--Deduct 1 ten of 451.\\

a) 351\\
b) 441\\
c) 450\\
d) 461\\

Answer : b)\\

Feedback :\\
To deduct, you need to do a subtraction.
\begin{center}
% use packages: array
\begin{tabular}{|rrr|rrr|rrr|}
\hline
\multicolumn{6}{|c|}{integers} &\multicolumn{3}{|c|}{decimals} \\
\hline
\multicolumn{3}{|c|}{Class of} &\multicolumn{3}{|c|}{Class of} &  \multicolumn{3}{c|}{} \\
\multicolumn{3}{|c|}{thousands} &\multicolumn{3}{|c|}{units} &  \multicolumn{3}{c|}{} \\
\hline
H & T & U &H & T & U, & T\up{th} & \textbf{H\up{th}} & K\up{th} \\
\hline
\hline
8 & 0 & 0 & 0 & 0 & 0 &  & &\\
 & 7 & 0 & 0 & 0 & 0 &  & &\\
+ &  &  & 5 & 0 & 0 &  & &\\
 &  &  &  & 2 & 0 &  & &\\
 &  &  &  &  & 3 &  & &\\
\hline
\hline
 8 & 7 & 0 & 5 & 2 & 3 &  & &
\\
\hline
\end{tabular}
\end{center}

\scriptsize
\begin{center}
% use packages: array
\begin{tabular}{ll}
H : Hundreds & T\up{th} : Tenths\\
T : Tens & H\up{th} :Hundredths\\
U : Units & K\up{e} : Thousandths\\
\end{tabular}
\end{center}

\normalsize
The answer is b).\\





7553--Deduct 4 tens from 194.\\

a) 154\\
b) 190\\
c) 198\\
d) 234\\

Answer : a)\\

Feedback :\\
To deduct, you have to do a subtraction.
\begin{center}
% use packages: array
\begin{tabular}{|rrr|rrr|rrr|}
\hline
\multicolumn{6}{|c|}{integers} &\multicolumn{3}{|c|}{decimals} \\
\hline
\multicolumn{3}{|c|}{Class of} &\multicolumn{3}{|c|}{Class of} &  \multicolumn{3}{c|}{} \\
\multicolumn{3}{|c|}{thousands} &\multicolumn{3}{|c|}{units} &  \multicolumn{3}{c|}{} \\
\hline
H & T & U &H & T & U, & T\up{th} & \textbf{H\up{th}} & K\up{th} \\
\hline
\hline
8 & 0 & 0 & 0 & 0 & 0 &  & &\\
 & 7 & 0 & 0 & 0 & 0 &  & &\\
+ &  &  & 5 & 0 & 0 &  & &\\
 &  &  &  & 2 & 0 &  & &\\
 &  &  &  &  & 3 &  & &\\
\hline
\hline
 8 & 7 & 0 & 5 & 2 & 3 &  & &
\\
\hline
\end{tabular}
\end{center}

\scriptsize
\begin{center}
% use packages: array
\begin{tabular}{ll}
H : Hundreds & T\up{th} : Tenths\\
T : Tens & H\up{th} :Hundredths\\
U : Units & K\up{e} : Thousandths\\
\end{tabular}
\end{center}

\normalsize
The answer is a).\\





7554--Deduct 6 tens from 639.\\

a) 39\\
b) 579\\
c) 633\\
d) 699\\

Answer : b)\\

Feedback :\\
To deduct, you need to do a subtraction.
\begin{center}
% use packages: array
\begin{tabular}{|rrr|rrr|rrr|}
\hline
\multicolumn{6}{|c|}{integers} &\multicolumn{3}{|c|}{decimals} \\
\hline
\multicolumn{3}{|c|}{Class of} &\multicolumn{3}{|c|}{Class of} &  \multicolumn{3}{c|}{} \\
\multicolumn{3}{|c|}{thousands} &\multicolumn{3}{|c|}{units} &  \multicolumn{3}{c|}{} \\
\hline
H & T & U &H & T & U, & T\up{th} & \textbf{H\up{th}} & K\up{th} \\
\hline
\hline
8 & 0 & 0 & 0 & 0 & 0 &  & &\\
 & 7 & 0 & 0 & 0 & 0 &  & &\\
+ &  &  & 5 & 0 & 0 &  & &\\
 &  &  &  & 2 & 0 &  & &\\
 &  &  &  &  & 3 &  & &\\
\hline
\hline
 8 & 7 & 0 & 5 & 2 & 3 &  & &
\\
\hline
\end{tabular}
\end{center}

\scriptsize
\begin{center}
% use packages: array
\begin{tabular}{ll}
H : Hundreds & T\up{th} : Tenths\\
T : Tens & H\up{th} :Hundredths\\
U : Units & K\up{e} : Thousandths\\
\end{tabular}
\end{center}

\normalsize
The answer is b).\\



7555--Deduct 10 tens from 146.\\

a) 46\\
b) 136\\
c) 246\\
d) 256\\

Answer : a)\\

Feedback :\\
To deduct, you need to do a subtraction.
\begin{center}
% use packages: array
\begin{tabular}{|rrr|rrr|rrr|}
\hline
\multicolumn{6}{|c|}{integers} &\multicolumn{3}{|c|}{decimals} \\
\hline
\multicolumn{3}{|c|}{Class of} &\multicolumn{3}{|c|}{Class of} &  \multicolumn{3}{c|}{} \\
\multicolumn{3}{|c|}{thousands} &\multicolumn{3}{|c|}{units} &  \multicolumn{3}{c|}{} \\
\hline
H & T & U &H & T & U, & T\up{th} & \textbf{H\up{th}} & K\up{th} \\
\hline
\hline
8 & 0 & 0 & 0 & 0 & 0 &  & &\\
 & 7 & 0 & 0 & 0 & 0 &  & &\\
+ &  &  & 5 & 0 & 0 &  & &\\
 &  &  &  & 2 & 0 &  & &\\
 &  &  &  &  & 3 &  & &\\
\hline
\hline
 8 & 7 & 0 & 5 & 2 & 3 &  & &
\\
\hline
\end{tabular}
\end{center}

\scriptsize
\begin{center}
% use packages: array
\begin{tabular}{ll}
H : Hundreds & T\up{th} : Tenths\\
T : Tens & H\up{th} :Hundredths\\
U : Units & K\up{e} : Thousandths\\
\end{tabular}
\end{center}

\normalsize
The answer is a).\\



7556--Add 1 hundred to 451.\\

a) 351\\
b) 452\\
c) 461\\
d) 551\\

Answer : d)\\

Feedback :\\
To add, you need to do an addition.
\begin{center}
% use packages: array
\begin{tabular}{|rrr|rrr|rrr|}
\hline
\multicolumn{6}{|c|}{integers} &\multicolumn{3}{|c|}{decimals} \\
\hline
\multicolumn{3}{|c|}{Class of} &\multicolumn{3}{|c|}{Class of} &  \multicolumn{3}{c|}{} \\
\multicolumn{3}{|c|}{thousands} &\multicolumn{3}{|c|}{units} &  \multicolumn{3}{c|}{} \\
\hline
H & T & U &H & T & U, & T\up{th} & \textbf{H\up{th}} & K\up{th} \\
\hline
\hline
8 & 0 & 0 & 0 & 0 & 0 &  & &\\
 & 7 & 0 & 0 & 0 & 0 &  & &\\
+ &  &  & 5 & 0 & 0 &  & &\\
 &  &  &  & 2 & 0 &  & &\\
 &  &  &  &  & 3 &  & &\\
\hline
\hline
 8 & 7 & 0 & 5 & 2 & 3 &  & &
\\
\hline
\end{tabular}
\end{center}

\scriptsize
\begin{center}
% use packages: array
\begin{tabular}{ll}
H : Hundreds & T\up{th} : Tenths\\
T : Tens & H\up{th} :Hundredths\\
U : Units & K\up{e} : Thousandths\\
\end{tabular}
\end{center}

\normalsize
The answer is d).\\




7557--Add 5 hundreds to  451.\\

a) 401\\
b) 456\\
c) 501\\
d) 951\\

Answer : d)\\

Feedback :\\
To add, you need to do an addition.
\begin{center}
% use packages: array
\begin{tabular}{|rrr|rrr|rrr|}
\hline
\multicolumn{6}{|c|}{integers} &\multicolumn{3}{|c|}{decimals} \\
\hline
\multicolumn{3}{|c|}{Class of} &\multicolumn{3}{|c|}{Class of} &  \multicolumn{3}{c|}{} \\
\multicolumn{3}{|c|}{thousands} &\multicolumn{3}{|c|}{units} &  \multicolumn{3}{c|}{} \\
\hline
H & T & U &H & T & U, & T\up{th} & \textbf{H\up{th}} & K\up{th} \\
\hline
\hline
8 & 0 & 0 & 0 & 0 & 0 &  & &\\
 & 7 & 0 & 0 & 0 & 0 &  & &\\
+ &  &  & 5 & 0 & 0 &  & &\\
 &  &  &  & 2 & 0 &  & &\\
 &  &  &  &  & 3 &  & &\\
\hline
\hline
 8 & 7 & 0 & 5 & 2 & 3 &  & &
\\
\hline
\end{tabular}
\end{center}

\scriptsize
\begin{center}
% use packages: array
\begin{tabular}{ll}
H : Hundreds & T\up{th} : Tenths\\
T : Tens & H\up{th} :Hundredths\\
U : Units & K\up{e} : Thousandths\\
\end{tabular}
\end{center}

\normalsize
The answer is d).\\




7558--Add 6 hundreds to 923.\\

a) 323\\
b) 983\\
c) 1523\\
d) 6923\\

Answer : c)\\

Feedback :\\
To add, you need to do an addition.
\begin{center}
% use packages: array
\begin{tabular}{|rrr|rrr|rrr|}
\hline
\multicolumn{6}{|c|}{integers} &\multicolumn{3}{|c|}{decimals} \\
\hline
\multicolumn{3}{|c|}{Class of} &\multicolumn{3}{|c|}{Class of} &  \multicolumn{3}{c|}{} \\
\multicolumn{3}{|c|}{thousands} &\multicolumn{3}{|c|}{units} &  \multicolumn{3}{c|}{} \\
\hline
H & T & U &H & T & U, & T\up{th} & \textbf{H\up{th}} & K\up{th} \\
\hline
\hline
8 & 0 & 0 & 0 & 0 & 0 &  & &\\
 & 7 & 0 & 0 & 0 & 0 &  & &\\
+ &  &  & 5 & 0 & 0 &  & &\\
 &  &  &  & 2 & 0 &  & &\\
 &  &  &  &  & 3 &  & &\\
\hline
\hline
 8 & 7 & 0 & 5 & 2 & 3 &  & &
\\
\hline
\end{tabular}
\end{center}

\scriptsize
\begin{center}
% use packages: array
\begin{tabular}{ll}
H : Hundreds & T\up{th} : Tenths\\
T : Tens & H\up{th} :Hundredths\\
U : Units & K\up{e} : Thousandths\\
\end{tabular}
\end{center}

\normalsize
The answer is c).\\




7559--Add 10 hundreds to 178.\\

a) 78\\
b) 278\\
c) 1178\\
d) 10 178\\

Answer : c)\\

Feedback :\\
To add, you need to do an addition.
\begin{center}
% use packages: array
\begin{tabular}{|rrr|rrr|rrr|}
\hline
\multicolumn{6}{|c|}{integers} &\multicolumn{3}{|c|}{decimals} \\
\hline
\multicolumn{3}{|c|}{Class of} &\multicolumn{3}{|c|}{Class of} &  \multicolumn{3}{c|}{} \\
\multicolumn{3}{|c|}{thousands} &\multicolumn{3}{|c|}{units} &  \multicolumn{3}{c|}{} \\
\hline
H & T & U &H & T & U, & T\up{th} & \textbf{H\up{th}} & K\up{th} \\
\hline
\hline
8 & 0 & 0 & 0 & 0 & 0 &  & &\\
 & 7 & 0 & 0 & 0 & 0 &  & &\\
+ &  &  & 5 & 0 & 0 &  & &\\
 &  &  &  & 2 & 0 &  & &\\
 &  &  &  &  & 3 &  & &\\
\hline
\hline
 8 & 7 & 0 & 5 & 2 & 3 &  & &
\\
\hline
\end{tabular}
\end{center}

\scriptsize
\begin{center}
% use packages: array
\begin{tabular}{ll}
H : Hundreds & T\up{th} : Tenths\\
T : Tens & H\up{th} :Hundredths\\
U : Units & K\up{e} : Thousandths\\
\end{tabular}
\end{center}

\normalsize
The answer is c).\\






7560--Deduct 1 hundred from 178.\\

a) 78\\
b) 188\\
c) 278\\
d) 1178\\

Answer : a)\\

Feedback :\\
To deduct, you need to do a subtraction.
\\begin{center}
% use packages: array
\begin{tabular}{|rrr|rrr|rrr|}
\hline
\multicolumn{6}{|c|}{integers} &\multicolumn{3}{|c|}{decimals} \\
\hline
\multicolumn{3}{|c|}{Class of} &\multicolumn{3}{|c|}{Class of} &  \multicolumn{3}{c|}{} \\
\multicolumn{3}{|c|}{thousands} &\multicolumn{3}{|c|}{units} &  \multicolumn{3}{c|}{} \\
\hline
H & T & U &H & T & U, & T\up{th} & \textbf{H\up{th}} & K\up{th} \\
\hline
\hline
8 & 0 & 0 & 0 & 0 & 0 &  & &\\
 & 7 & 0 & 0 & 0 & 0 &  & &\\
+ &  &  & 5 & 0 & 0 &  & &\\
 &  &  &  & 2 & 0 &  & &\\
 &  &  &  &  & 3 &  & &\\
\hline
\hline
 8 & 7 & 0 & 5 & 2 & 3 &  & &
\\
\hline
\end{tabular}
\end{center}

\scriptsize
\begin{center}
% use packages: array
\begin{tabular}{ll}
H : Hundreds & T\up{th} : Tenths\\
T : Tens & H\up{th} :Hundredths\\
U : Units & K\up{e} : Thousandths\\
\end{tabular}
\end{center}

\normalsize
The answer is a).\\




7561--Deduct 2 hundreds from 692.\\

a) 292\\
b) 492\\
c) 672\\
d) 892\\

Answer : b)\\

Feedback :\\
To deduct, you need to do a subtraction.
\begin{center}
% use packages: array
\begin{tabular}{|rrr|rrr|rrr|}
\hline
\multicolumn{6}{|c|}{integers} &\multicolumn{3}{|c|}{decimals} \\
\hline
\multicolumn{3}{|c|}{Class of} &\multicolumn{3}{|c|}{Class of} &  \multicolumn{3}{c|}{} \\
\multicolumn{3}{|c|}{thousands} &\multicolumn{3}{|c|}{units} &  \multicolumn{3}{c|}{} \\
\hline
H & T & U &H & T & U, & T\up{th} & \textbf{H\up{th}} & K\up{th} \\
\hline
\hline
8 & 0 & 0 & 0 & 0 & 0 &  & &\\
 & 7 & 0 & 0 & 0 & 0 &  & &\\
+ &  &  & 5 & 0 & 0 &  & &\\
 &  &  &  & 2 & 0 &  & &\\
 &  &  &  &  & 3 &  & &\\
\hline
\hline
 8 & 7 & 0 & 5 & 2 & 3 &  & &
\\
\hline
\end{tabular}
\end{center}

\scriptsize
\begin{center}
% use packages: array
\begin{tabular}{ll}
H : Hundreds & T\up{th} : Tenths\\
T : Tens & H\up{th} :Hundredths\\
U : Units & K\up{e} : Thousandths\\
\end{tabular}
\end{center}

\normalsize
The answer is b).\\





7562--Deduct 9 hundreds from 2410.\\

a) 1510\\
b) 2401\\
c) 2380\\
d) 3310\\

Answer : a)\\

Feedback :\\
To deduct, you need to make a subtraction.
\begin{center}
% use packages: array
\begin{tabular}{|rrr|rrr|rrr|}
\hline
\multicolumn{6}{|c|}{integers} &\multicolumn{3}{|c|}{decimals} \\
\hline
\multicolumn{3}{|c|}{Class of} &\multicolumn{3}{|c|}{Class of} &  \multicolumn{3}{c|}{} \\
\multicolumn{3}{|c|}{thousands} &\multicolumn{3}{|c|}{units} &  \multicolumn{3}{c|}{} \\
\hline
H & T & U &H & T & U, & T\up{th} & \textbf{H\up{th}} & K\up{th} \\
\hline
\hline
8 & 0 & 0 & 0 & 0 & 0 &  & &\\
 & 7 & 0 & 0 & 0 & 0 &  & &\\
+ &  &  & 5 & 0 & 0 &  & &\\
 &  &  &  & 2 & 0 &  & &\\
 &  &  &  &  & 3 &  & &\\
\hline
\hline
 8 & 7 & 0 & 5 & 2 & 3 &  & &
\\
\hline
\end{tabular}
\end{center}

\scriptsize
\begin{center}
% use packages: array
\begin{tabular}{ll}
H : Hundreds & T\up{th} : Tenths\\
T : Tens & H\up{th} :Hundredths\\
U : Units & K\up{e} : Thousandths\\
\end{tabular}
\end{center}

\normalsize
The answer is a).\\





7563--Deduct 10 hundreds from 2410.\\

a) 1410\\
b) 2310\\
c) 3410\\
d) 3510\\

Answer : a)\\

Feedback :\\
To deduct, you need to do a subtraction.
\begin{center}
% use packages: array
\begin{tabular}{|rrr|rrr|rrr|}
\hline
\multicolumn{6}{|c|}{integers} &\multicolumn{3}{|c|}{decimals} \\
\hline
\multicolumn{3}{|c|}{Class of} &\multicolumn{3}{|c|}{Class of} &  \multicolumn{3}{c|}{} \\
\multicolumn{3}{|c|}{thousands} &\multicolumn{3}{|c|}{units} &  \multicolumn{3}{c|}{} \\
\hline
H & T & U &H & T & U, & T\up{th} & \textbf{H\up{th}} & K\up{th} \\
\hline
\hline
8 & 0 & 0 & 0 & 0 & 0 &  & &\\
 & 7 & 0 & 0 & 0 & 0 &  & &\\
+ &  &  & 5 & 0 & 0 &  & &\\
 &  &  &  & 2 & 0 &  & &\\
 &  &  &  &  & 3 &  & &\\
\hline
\hline
 8 & 7 & 0 & 5 & 2 & 3 &  & &
\\
\hline
\end{tabular}
\end{center}

\scriptsize
\begin{center}
% use packages: array
\begin{tabular}{ll}
H : Hundreds & T\up{th} : Tenths\\
T : Tens & H\up{th} :Hundredths\\
U : Units & K\up{e} : Thousandths\\
\end{tabular}
\end{center}

\normalsize
The answer is a).\\






7564--Add 1 thousand to 5075.\\

a) 5076\\
b) 5085\\
c) 5175\\
d) 6075\\

Answer : d)\\

Feedback :\\
To add, you need to do an addition.
\begin{center}
% use packages: array
\begin{tabular}{|rrr|rrr|rrr|}
\hline
\multicolumn{6}{|c|}{integers} &\multicolumn{3}{|c|}{decimals} \\
\hline
\multicolumn{3}{|c|}{Class of} &\multicolumn{3}{|c|}{Class of} &  \multicolumn{3}{c|}{} \\
\multicolumn{3}{|c|}{thousands} &\multicolumn{3}{|c|}{units} &  \multicolumn{3}{c|}{} \\
\hline
H & T & U &H & T & U, & T\up{th} & \textbf{H\up{th}} & K\up{th} \\
\hline
\hline
8 & 0 & 0 & 0 & 0 & 0 &  & &\\
 & 7 & 0 & 0 & 0 & 0 &  & &\\
+ &  &  & 5 & 0 & 0 &  & &\\
 &  &  &  & 2 & 0 &  & &\\
 &  &  &  &  & 3 &  & &\\
\hline
\hline
 8 & 7 & 0 & 5 & 2 & 3 &  & &
\\
\hline
\end{tabular}
\end{center}

\scriptsize
\begin{center}
% use packages: array
\begin{tabular}{ll}
H : Hundreds & T\up{th} : Tenths\\
T : Tens & H\up{th} :Hundredths\\
U : Units & K\up{e} : Thousandths\\
\end{tabular}
\end{center}

\normalsize
The answer is d).\\





7565--Add 5 thousands to 3174.\\

a) 3124\\
b) 3179\\
c) 8174\\
d) 9174\\

Answer : c)\\

Feedback :\\
To add, you need to do an addition.
\begin{center}
% use packages: array
\begin{tabular}{|rrr|rrr|rrr|}
\hline
\multicolumn{6}{|c|}{integers} &\multicolumn{3}{|c|}{decimals} \\
\hline
\multicolumn{3}{|c|}{Class of} &\multicolumn{3}{|c|}{Class of} &  \multicolumn{3}{c|}{} \\
\multicolumn{3}{|c|}{thousands} &\multicolumn{3}{|c|}{units} &  \multicolumn{3}{c|}{} \\
\hline
H & T & U &H & T & U, & T\up{th} & \textbf{H\up{th}} & K\up{th} \\
\hline
\hline
8 & 0 & 0 & 0 & 0 & 0 &  & &\\
 & 7 & 0 & 0 & 0 & 0 &  & &\\
+ &  &  & 5 & 0 & 0 &  & &\\
 &  &  &  & 2 & 0 &  & &\\
 &  &  &  &  & 3 &  & &\\
\hline
\hline
 8 & 7 & 0 & 5 & 2 & 3 &  & &
\\
\hline
\end{tabular}
\end{center}

\scriptsize
\begin{center}
% use packages: array
\begin{tabular}{ll}
H : Hundreds & T\up{th} : Tenths\\
T : Tens & H\up{th} :Hundredths\\
U : Units & K\up{e} : Thousandths\\
\end{tabular}
\end{center}

\normalsize
The answer is c).\\




7566--Add 7 thousands to 3174.\\

a) 3181\\
b) 9174\\
c) 10 174\\
d) 73 174\\

Answer : c)\\

Feedback :\\
To add, you need to do an addition.
\begin{center}
% use packages: array
\begin{tabular}{|rrr|rrr|rrr|}
\hline
\multicolumn{6}{|c|}{integers} &\multicolumn{3}{|c|}{decimals} \\
\hline
\multicolumn{3}{|c|}{Class of} &\multicolumn{3}{|c|}{Class of} &  \multicolumn{3}{c|}{} \\
\multicolumn{3}{|c|}{thousands} &\multicolumn{3}{|c|}{units} &  \multicolumn{3}{c|}{} \\
\hline
H & T & U &H & T & U, & T\up{th} & \textbf{H\up{th}} & K\up{th} \\
\hline
\hline
8 & 0 & 0 & 0 & 0 & 0 &  & &\\
 & 7 & 0 & 0 & 0 & 0 &  & &\\
+ &  &  & 5 & 0 & 0 &  & &\\
 &  &  &  & 2 & 0 &  & &\\
 &  &  &  &  & 3 &  & &\\
\hline
\hline
 8 & 7 & 0 & 5 & 2 & 3 &  & &
\\
\hline
\end{tabular}
\end{center}

\scriptsize
\begin{center}
% use packages: array
\begin{tabular}{ll}
H : Hundreds & T\up{th} : Tenths\\
T : Tens & H\up{th} :Hundredths\\
U : Units & K\up{e} : Thousandths\\
\end{tabular}
\end{center}

\normalsize
The answer is c).\\




7567--Add 10 thousands to 8406.\\

a) 8416\\
b) 9406\\
c) 18 406\\
d) 108 406\\

Answer : c)\\

Feedback :\\
To add, you need to do an addition.
\begin{center}
% use packages: array
\begin{tabular}{|rrr|rrr|rrr|}
\hline
\multicolumn{6}{|c|}{integers} &\multicolumn{3}{|c|}{decimals} \\
\hline
\multicolumn{3}{|c|}{Class of} &\multicolumn{3}{|c|}{Class of} &  \multicolumn{3}{c|}{} \\
\multicolumn{3}{|c|}{thousands} &\multicolumn{3}{|c|}{units} &  \multicolumn{3}{c|}{} \\
\hline
H & T & U &H & T & U, & T\up{th} & \textbf{H\up{th}} & K\up{th} \\
\hline
\hline
8 & 0 & 0 & 0 & 0 & 0 &  & &\\
 & 7 & 0 & 0 & 0 & 0 &  & &\\
+ &  &  & 5 & 0 & 0 &  & &\\
 &  &  &  & 2 & 0 &  & &\\
 &  &  &  &  & 3 &  & &\\
\hline
\hline
 8 & 7 & 0 & 5 & 2 & 3 &  & &
\\
\hline
\end{tabular}
\end{center}

\scriptsize
\begin{center}
% use packages: array
\begin{tabular}{ll}
H : Hundreds & T\up{th} : Tenths\\
T : Tens & H\up{th} :Hundredths\\
U : Units & K\up{e} : Thousandths\\
\end{tabular}
\end{center}

\normalsize
The answer is c).\\


7568--Find the least common multiple (LCM), with the exception of zero, for numbers 3 and 8.\\
Example:\\
\begin{center}
% use packages: array
\begin{tabular}{|c|}
\hline
Multiples of 3 : \{0, 3, 6, 9, 12, 15, ...\}\\
Multiples of 4 : \{0, 4, 8, 12, 16, 20, ...\}\\
The LCM is 12.\\
\hline
\end{tabular}
\end{center}

a) 1\\
b) 11\\
c) 16\\
d) 24\\

Answer : d)\\

Feedback :\\
First, we need to come up with a sufficiant amount of multiples for each number.\\
3 : 0, 3, 6, 9, 12, 15, 18, 21, 24, 27, ...\\
8 : 0, 8, 16, 24, 32, 40, 48, ...\\

Then, we need to find the lowest common multiple, thus 24.\\
The answer is d).\\




7569--Find the least common multiple (LCM), with the exception of zero, for numbers 4 and 10.\\
Example:\\
\begin{center}
% use packages: array
\begin{tabular}{|c|}
\hline
Multiples of 3 : \{0, 3, 6, 9, 12, 15, ...\}\\
Multiples of 4 : \{0, 4, 8, 12, 16, 20, ...\}\\
The LCM is 12.\\
\hline
\end{tabular}
\end{center}

a) 10\\
b) 20\\
c) 30\\
d) 40\\

Answer : b)\\

Feedback :\\
First, we need to come up with a sufficiant amount of multiples for each number.\\
4 : 0, 4, 8, 12, 16, 20, 24, 28, 32, ...\\
10 : 0, 10, 20, 30, 40, ...\\

Then, we need to find the lowest common multiple, thus 20.\\
The answer is b).\\



7570--Find the least common multiple (LCM), with the exception of zero, for numbers 2 and 3.\\
Example:\\
\begin{center}
% use packages: array
\begin{tabular}{|c|}
\hline
Multiples of 3 : \{0, 3, 6, 9, 12, 15, ...\}\\
Multiples of 4 : \{0, 4, 8, 12, 16, 20, ...\}\\
The LCM is 12.\\
\hline
\end{tabular}
\end{center}

a) 6\\
b) 9\\
c) 12\\
d) 18\\

Answer : a)\\

Feedback :\\
First, we need to come up with a sufficiant amount of multiples for each number.\\
2 : 0, 2, 4, 6, 8, 10, 12, 14, ...\\
3 : 0, 3, 6, 9, 12, 15, ...\\

Then, we need to find the lowest common multiple, thus 6.\\
The answer is a).\\


7571--What is the greatest common divisor (GCD) for numbers 18 and 36?\\
Example:\\
\begin{center}
% use packages: array
\begin{tabular}{|c|}
\hline
Diviors of 4 : \{1, 2, 4\}\\
Divisors of 6 : \{1, 2, 3, 6\}\\
The GCD is 2.\\
\hline
\end{tabular}
\end{center}

a) 1\\
b) 6\\
c) 18\\
d) 36\\

Answer : c)\\

Feedback :\\
First, we have to find the divisors for each number.\\
Then, we determine the common divisors.\\
Finally, we select the greatest common divisor of the to numbers.\\
\begin{center}
\includegraphics[width=3cm]{R7571.eps}\\
% R7571.eps : 300dpi, width=3.39cm, height=3.39cm, bb=0 0 400 400\\
\end{center}
The answer is c).\\




7572--What is the greatest common divisor (GCD) of numbers 12 and 16?\\
Example:\\
\begin{center}
% use packages: array
\begin{tabular}{|c|}
\hline
Divisors of 4 : \{1, 2, 4\}\\
Divisors of 6 : \{1, 2, 3, 6\}\\
The GCD  is 2.\\
\hline
\end{tabular}
\end{center}

a) 2\\
b) 4\\
c) 6\\
d) 8\\

Answer : b)\\

Feedback :\\
First, we have to find the divisors for each number.\\
Then, we determine the common divisors.\\
Finally, we select the greatest common divisor of the to numbers.\\
\begin{center}
\includegraphics[width=3cm]{R7572.eps}\\
% R7572.eps : 300dpi, width=3.39cm, height=3.39cm, bb=0 0 400 400\\
\end{center}
The answer is b).\\




7573--What is the greatest common divisor (GCD) for numbers 15 and 18?\\
Example:\\
\begin{center}
% use packages: array
\begin{tabular}{|c|}
\hline
Divisors of 4 : \{1, 2, 4\}\\
Divisors of 6 : \{1, 2, 3, 6\}\\
The GCD is 2.\\
\hline
\end{tabular}
\end{center}

a) 1\\
b) 3\\
c) 5\\
d) 9\\

Answer : b)\\

Feedback :\\
First, we have to find the divisors for each number.\\
Then, we determine the common divisors.\\
Finally, we select the greatest common divisor of the to numbers.\\
\begin{center}
\includegraphics[width=3cm]{R7573.eps}\\
% R7573.eps : 300dpi, width=3.39cm, height=3.39cm, bb=0 0 400 400\\
\end{center}
The answer is b).\\



7574--What is the greatest common divisor (GCD) for numbers 3 and 9?\\
Example:\\
\begin{center}
% use packages: array
\begin{tabular}{|c|}
\hline
Divisors of 4 : \{1, 2, 4\}\\
Divisors of 6 : \{1, 2, 3, 6\}\\
The GCC is 2.\\
\hline
\end{tabular}
\end{center}

a) 1\\
b) 3\\
c) 6\\
d) 9\\

Answer : b)\\

Feedback :\\
First, we have to find the divisors for each number.\\
Then, we determine the common divisors.\\
Finally, we select the greatest common divisor of the to numbers.\\
\begin{center}
\includegraphics[width=3cm]{R7574.eps}\\
% R7574.eps : 300dpi, width=3.39cm, height=3.39cm, bb=0 0 400 400\\
\end{center}
The answer is b).\\




\end{document} 