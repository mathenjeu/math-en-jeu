\documentclass{article}
\usepackage{amsmath,amssymb,graphicx,enumitem,wrapfig}

\begin{document}
\parindent=0cm
\parskip=6pt
\pagestyle{empty}

%Begin
%Language English
%Source Calgary Junior High School Mathematics Contest
%Title 1998
%Question A1
%Subject algebra
%Category modelling
%Type MC
%Choices 5
%Answer C
%Creator Victor Semenoff
%Rdifficulty 15
%Qtext

\scriptsize
Source: Calgary Junior High School Mathematics Contest

\normalsize
%\begin{wrapfigure}[2]{r}[0pt]{0pt}
%	\includegraphics[width=30mm,viewport=]{CCJ78-04}
%\end{wrapfigure}
The ``Top 100 Hits of All Time'' are played one after the other on a radio station, with no breaks in between songs. Each hit is three minutes long. How many hours does it take to play all the hits?\\
%ChoiceA
(A) 3\\
%ChoiceB
(B) 4\\
%ChoiceC
(C) 5\\
%ChoiceD
(D) 6\\
%ChoiceE
(E) None of the above\\
%Ftext

%\begin{wrapfigure}{r}[0pt]{0pt}
%	\includegraphics[width=30mm,viewport=]{CCJ78-04}
%\end{wrapfigure}

\textbf{The correct answer is (C): 5}\\[1 ex]
100 three minute songs will take 300 minutes to play, and 300 minutes is 5 hours.
%End
\\[5 ex]
%Begin
%Language English
%Source Calgary Junior High School Mathematics Contest
%Title 1998
%Question A2
%Subject algebra
%Category modelling
%Type MC
%Choices 5
%Answer E
%Creator Victor Semenoff
%Rdifficulty 17
%Qtext

\scriptsize
Source: Calgary Junior High School Mathematics Contest

\normalsize
%\begin{wrapfigure}[2]{r}[0pt]{0pt}
%	\includegraphics[width=30mm,viewport=]{CCJ78-04}
%\end{wrapfigure}
Water is flowing into Len's basement at a rate of 50 litres per hour. Len can bail the water out at a rate of 60 litres per hour. If the water has been flowing in for two hours before Len begins bailing, how many hours will it take him to bail out all the water from his basement?\\
%ChoiceA
(A) 5\\
%ChoiceB
(B) 8\\
%ChoiceC
(C) 9\\
%ChoiceD
(D) 12\\
%ChoiceE
(E) 10\\
%Ftext

%\begin{wrapfigure}{r}[0pt]{0pt}
%	\includegraphics[width=30mm,viewport=]{CCJ78-04}
%\end{wrapfigure}

\textbf{The correct answer is (E): 10}\\[1 ex]
In the two hours before Len starts bailing, 100 litres will have flowed into the basement. Thus the amount of water that has flowed into the basment at some time $t$ after Len starts bailing (just the flow in, not the amount removed) is $100+50t$ litres.

Len bails water at 60 litres per hour, so a time $t$ after he starts bailing, Len has bailed $60t$ litres of water. Setting the flow in equal to the flow out, we have
\begin{align*}
100+50t&=60t\\
t&=10.
\end{align*}
Therefore he has bailed out all the water 10 hours after starting.
%End
\\[5 ex]
%Begin
%Language English
%Source Calgary Junior High School Mathematics Contest
%Title 1997
%Question A3
%Subject arithmetic
%Category integers
%Type SA
%Answer 2012
%Creator Victor Semenoff
%Rdifficulty 15 
%Qtext

\scriptsize
Source: Calgary Junior High School Mathematics Contest

\normalsize
%\begin{wrapfigure}[2]{r}[0pt]{0pt}
%	\includegraphics[width=30mm,viewport=]{CCJ78-04}
%\end{wrapfigure}
Notice that the year $1998$ is divisible by exactly three of its digits, because
$${1998\over 1}\ , \quad {1998\over 9}\ ,\quad \mbox{and}\quad
{1998\over 9}$$
are all whole numbers, but $
{1998\over 8}$ is not a whole number.
 What is the next year which is divisible by exactly three of its digits?\\
%Ftext

%\begin{wrapfigure}{r}[0pt]{0pt}
%	\includegraphics[width=30mm,viewport=]{CCJ78-04}
%\end{wrapfigure}

\textbf{The correct answer is 2012}\\[1 ex]
1999 doesn't work. The next number that doesn't have two zeros in it is 2011, but this doesn't work either (not divisible by two or zero). 2012, however does work (it is divisible by 2 and 2 and 1).
%End
\\[5 ex]
%Begin
%Language English
%Source Calgary Junior High School Mathematics Contest
%Title 1998
%Question A4
%Subject statistics
%Category concepts
%Type MC
%Choices 5
%Answer B
%Creator Victor Semenoff
%Rdifficulty 18
%Qtext

\scriptsize
Source: Calgary Junior High School Mathematics Contest

\normalsize
%\begin{wrapfigure}[2]{r}[0pt]{0pt}
%	\includegraphics[width=30mm,viewport=]{CCJ78-04}
%\end{wrapfigure}
Yesterday at least 90\% of the students in a certain class showed up for school. Today it snowed, and at most 75\% of the students in the class showed up. There were 6 fewer students in class today than yesterday. What is the largest possible number of students in the class?\\
%ChoiceA
(A) 39\\
%ChoiceB
(B) 40\\
%ChoiceC
(C) 25\\
%ChoiceD
(D) 45\\
%ChoiceE
(E) None of the above.\\
%Ftext

%\begin{wrapfigure}{r}[0pt]{0pt}
%	\includegraphics[width=30mm,viewport=]{CCJ78-04}
%\end{wrapfigure}

\textbf{The correct answer is (B): 40}\\[1 ex]
Assume exactly $75\%$ of the students show up today. Then $90\%-75\%=15\%$ of the class is equal to 6 students, and each student accounts for $\frac{15\%}{6}=\frac{5\%}{2}$. Therefore there are $\frac{100\%}{\frac{5\%}{2}}=40$ students total in the class.

This is the largest amount of students possible because the maximum number of students occurs when we minimize the percent of the class that each student accounts for. This is done by minimizing the difference between the percentage of students who make it on yesterday and the percentage who make it today, thus $75\%$ today.
%End
\\[5 ex]
%Begin
%Language English
%Source Calgary Junior High School Mathematics Contest
%Title 1998
%Question A5
%Subject geometry
%Category 3D
%Type MC
%Choices 5
%Answer E
%Creator Victor Semenoff
%Rdifficulty 18
%Qtext

\scriptsize
Source: Calgary Junior High School Mathematics Contest

\normalsize
%\begin{wrapfigure}[2]{r}[0pt]{0pt}
%	\includegraphics[width=30mm,viewport=]{CCJ78-04}
%\end{wrapfigure}
One bath oil bead contains 1 cubic centimetre of bath oil.
You drop one bath oil bead in your bathtub filled with water,
and the oil spreads evenly over the surface of the water, which
is a rectangle 1 metre wide and 2 metres long. How thick
\textbf{in millimetres} is the layer of bath oil?\\
%ChoiceA
(A) 1\\[1 ex]
%ChoiceB
(B) $\frac{1}{1000000}$\\[1 ex]
%ChoiceC
(C) $\frac{1}{1000}$\\[1 ex]
%ChoiceD
(D) $\frac{1}{2000000}$\\[1 ex]
%ChoiceE
(E) $\frac{1}{2000}$\\
%Ftext

%\begin{wrapfigure}{r}[0pt]{0pt}
%	\includegraphics[width=30mm,viewport=]{CCJ78-04}
%\end{wrapfigure}

\textbf{The correct answer is (E): $\frac{1}{2000}$}\\[1 ex]
The volume of the oil on the water must equal the volume of the oil before it enters the water, thus letting $t$ be the thickness of the oil, we have
\begin{align*}
1000mm\times2000mm\times t&=1000mm^3\\
t&=\frac{1}{2000}mm.
\end{align*}
%End
\\[5 ex]
%Begin
%Language English
%Source Calgary Junior High School Mathematics Contest
%Title 1997
%Question A6
%Subject arithmetic
%Category integers
%Type SA
%Answer 37
%Creator Victor Semenoff
%Rdifficulty 15
%Qtext

\scriptsize
Source: Calgary Junior High School Mathematics Contest

\normalsize
%\begin{wrapfigure}[2]{r}[0pt]{0pt}
%	\includegraphics[width=30mm,viewport=]{CCJ78-04}
%\end{wrapfigure}
What is the largest prime number that divides evenly into 1998?\\
%Ftext

%\begin{wrapfigure}{r}[0pt]{0pt}
%	\includegraphics[width=30mm,viewport=]{CCJ78-04}
%\end{wrapfigure}

\textbf{The correct answer is 37}\\[1 ex]
Lets factor 1998: $\frac{1998}{2}=999$, $\frac{999}{3}=111$, $\frac{111}{3}=37$, and 37 is prime, thus 37 is the largest prime factor.
%End
\\[5 ex]
%Begin
%Language English
%Source Calgary Junior High School Mathematics Contest
%Title 1997
%Question A8
%Subject statistics
%Category concepts
%Type SA
%Answer 13
%Creator Victor Semenoff
%Rdifficulty 15
%Qtext

\scriptsize
Source: Calgary Junior High School Mathematics Contest

\normalsize
%\begin{wrapfigure}[2]{r}[0pt]{0pt}
%	\includegraphics[width=30mm,viewport=]{CCJ78-04}
%\end{wrapfigure}
The average of seven numbers is 21. An eighth number is added, and the average becomes 20. What is the eighth number?\\
%Ftext

%\begin{wrapfigure}{r}[0pt]{0pt}
%	\includegraphics[width=30mm,viewport=]{CCJ78-04}
%\end{wrapfigure}

\textbf{The correct answer is 13}\\[1 ex]
The first seven numbers have and average of 21, thus we are looking for $n$ such that the average of the seven numbers and n is 20, so
\begin{align*}
\frac{7\times21+1\times n}{8}&=20\\
n&=160-147=13.
\end{align*}
%End
\\[5 ex]
%Begin
%Language English
%Source Calgary Junior High School Mathematics Contest
%Title 1998
%Question A9
%Subject geometry
%Category area
%Type MC
%Choices 5
%Answer A
%Creator Victor Semenoff
%Rdifficulty 22
%Qtext

\scriptsize
Source: Calgary Junior High School Mathematics Contest

\normalsize
\begin{wrapfigure}[4]{r}[0pt]{0pt}
	\includegraphics[width=30mm,viewport=45 166 503 432]{CJMC98-A9pic}
\end{wrapfigure}
The triangle $ABC$ is a right-angled triangle (with the right angle at $B$) in which side $AB$
has length two cm, and side $BC$ has length one cm. What is the area (in
cm$^{2}$) of the \textbf{square} $BDEF$?\\
%ChoiceA
(A) $\frac{4}{9}$\\[1 ex]
%ChoiceB
(B) $\frac{9}{16}$\\[1 ex]
%ChoiceC
(C) $\frac{1}{4}$\\[1 ex]
%ChoiceD
(D) $\frac{9}{25}$\\[1 ex]
%ChoiceE
(E) Not enough information.\\
%Ftext

\begin{wrapfigure}{r}[0pt]{0pt}
	\includegraphics[width=30mm,viewport=45 166 503 432]{CJMC98-A9pic}
\end{wrapfigure}

\textbf{The correct answer is (D): $\frac{4}{9}$}\\[1 ex]
The ratio of $BC$ to $AB$ is $\frac{1}{2}$, thus the ratio of $DF$ to $AD$ is also $\frac{1}{2}$. Also, because $BDFE$ is a square, $BD=DF$. Therefore, \begin{align*}
\frac{DF}{2-BD}&=\frac{1}{2}\\
\frac{BD}{2-BD}&=\frac{1}{2}\\
2BD&=2-BD\\
BD&=\frac{2}{3}.
\end{align*}
Thus the area of the square is $BD^2=\frac{4}{9}$.
%End
\\[5 ex]
%Begin
%Language English
%Source Calgary Junior High School Mathematics Contest
%Title 1998
%Question B1
%Subject geometry
%Category area
%Type MC
%Choices 5
%Answer C
%Creator Victor Semenoff
%Rdifficulty 17
%Qtext

\scriptsize
Source: Calgary Junior High School Mathematics Contest

\normalsize
\begin{wrapfigure}[4]{r}[0pt]{0pt}
	\includegraphics[width=30mm,viewport=77 29 473 701]{CJMC98-B1pic}
\end{wrapfigure}
Here is a drawing of the deck of the ship \textit{Titanic}, with the shape and dimensions as shown. the stern is a semicircle, and the bow is an isosceles triangle. Find the area of the deck.
$y$?\\
%ChoiceA
(A) $15\pi+6450\textbf{m}^2$\\[1 ex]
%ChoiceB
(B) $6450+\frac{225\pi}{2}\textbf{m}^2$\\[1 ex]
%ChoiceC
(C) $6750+\frac{225\pi}{2}\textbf{m}^2$\\[1 ex]
%ChoiceD
(D) $15\pi+480\textbf{m}^2$\\[1 ex]
%ChoiceE
(E) Not enough information.\\
%Ftext

\begin{wrapfigure}{r}[0pt]{0pt}
	\includegraphics[width=20mm,viewport=77 29 473 701]{CJMC98-B1pic}
\end{wrapfigure}

\textbf{The correct answer is (C): $6750+\frac{225\pi}{2}\textbf{m}^2$}\\[1 ex]
The radius of the semicircle is 15m, thus $BD=15$m and $FE=15$m.  Therefore the height of the triangle is $FH=250-215-15=20$m, and the area of the triangle is 
\begin{equation*}
\frac{1}{2}\times30\times20=300.
\end{equation*}
The area of the semicircle is $\frac{1}{2}\pi(15)^2$ and the area of the rectangle is $215\times30=6450$. Therefore the area of the deck is 
\begin{equation*}
300+\frac{225\pi}{2}+6450=6750+\frac{225\pi}{2}\textbf{m}^2.
\end{equation*}
%End
\\[5 ex]
%Begin
%Language English
%Source Calgary Junior High School Mathematics Contest
%Title 1998
%Question B2
%Subject algebra
%Category modelling
%Type MC
%Choices 5
%Answer C
%Creator Victor Semenoff
%Rdifficulty 23
%Qtext

\scriptsize
Source: Calgary Junior High School Mathematics Contest

\normalsize
%\begin{wrapfigure}[2]{r}[0pt]{0pt}
%	\includegraphics[width=30mm,viewport=]{CCJ78-04}
%\end{wrapfigure}
Bruce is running a 10km race. He runs one km at 30km/h, but then walks 1/2km at 5km/h. He then runs 1km (again at 30km/h), and walks 1/2 km (again at 5km/h), and continues alternating in this fashion until he finishes the race. If Bruce were to run at a constant speed for the entire 10km distance, how fast would he have to run in order to finish the race in the same amount of time as it takes using his current strategy?\\
%ChoiceA
(A) 8km/h\\
%ChoiceB
(B) 10km/h\\
%ChoiceC
(C) 12km/h\\
%ChoiceD
(D) 14km/h\\
%ChoiceE
(E) 16km/h\\
%Ftext

%\begin{wrapfigure}{r}[0pt]{0pt}
%	\includegraphics[width=30mm,viewport=]{CCJ78-04}
%\end{wrapfigure}

\textbf{The correct answer is (C): 12km/h}\\[1 ex]
It takes Bruce 2 minutes to run 1km, and 6 minutes to walk 1/2km. To get to 10km, Bruce uses the pattern:
\begin{equation*}
1+\frac{1}{2}+1+\frac{1}{2}+1+\frac{1}{2}+1+\frac{1}{2}+1+\frac{1}{2}+1+\frac{1}{2}+1.
\end{equation*}
Thus the whole race takes him
\begin{equation*}
2+6+2+6+2+6+2+6+2+6+2+6+2=50\textrm{minutes}.
\end{equation*}
Thus he needs to run at $\frac{60}{50}\times10=12$km/h to finish in the same amount of time.
%End
\\[5 ex]
%Begin
%Language English
%Source Calgary Junior High School Mathematics Contest
%Title 1998
%Question B3
%Subject algebra
%Category modelling
%Type MC
%Choices 5
%Answer D 
%Creator Victor Semenoff
%Rdifficulty 29
%Qtext

\scriptsize
Source: Calgary Junior High School Mathematics Contest

\normalsize
%\begin{wrapfigure}[2]{r}[0pt]{0pt}
%	\includegraphics[width=30mm,viewport=38 546 527 597]{CJMC98-B3pic.eps}
%\end{wrapfigure}
Farah, Greg and Heidi live in three houses which lie in this order along a straight road. Farah leaves her house and walks toward Greg's and Heidi's houses. When Farahis 2/3 of the way to Greg's house, she is 1/2 of the way to Heidi's house. Suppose Heidi leaves her house and walks towards Farah's house. When Heidi is 2/3 of the way to Greg's house, what fraction of the way to Farah's house is she?\\
%ChoiceA
(A) 2/9\\
%ChoiceB
(B) 1/4\\
%ChoiceC
(C) 1/5\\
%ChoiceD
(D) 1/6\\
%ChoiceE
(E) Not enough information.\\
%Ftext

\begin{wrapfigure}{r}[0pt]{0pt}
	\includegraphics[width=30mm,viewport=38 546 527 597]{CJMC98-B3pic.eps}
\end{wrapfigure}

\textbf{The correct answer is (D): 1/6}\\[1 ex]
In the picture, F, G, and H are the three houses, and X is where Farah is when she is halfway to Heidi's house. So FX and XH are the same length. Since Farah is now 2/3 of the way to G, FX must be twice as lon as XG, so XG is 1/4 of the whole length FH. Thus GH is also 1/4 of FH. Now let Y be where Heidi is when she is 2/3 of the way to G. Then HY must be 2/3 of 1/4 of FH, which is 1/6 of the way to Farah's house.
%End
\\[5 ex]%Begin
%Language English
%Source Calgary Junior High School Mathematics Contest
%Title 1998
%Question B4
%Subject geometry
%Category length
%Type MC
%Choices 5
%Answer B
%Creator Victor Semenoff
%Rdifficulty 27
%Qtext

\scriptsize
Source: Calgary Junior High School Mathematics Contest

\normalsize
\begin{wrapfigure}[4]{r}[0pt]{0pt}
	\includegraphics[width=30mm,viewport=134 350 417 610]{CJMC98-B4pic.eps}
\end{wrapfigure}
We construct an octagon by cutting the corners from a 6cm$\times$6cm square, at points 2cm in from each corner of the square. What is the radius (in centimeters) of the largest circle that can be drawn inside of the octagon?\\
%ChoiceA
(A) $\sqrt{2}$\\
%ChoiceB
(B) $2\sqrt{2}$\\
%ChoiceC
(C) $\frac{3\sqrt{2}}{2}$\\
%ChoiceD
(D) 3\\
%ChoiceE
(E) None of the above.\\
%Ftext

\begin{wrapfigure}{r}[0pt]{0pt}
	\includegraphics[width=30mm,viewport=119 347 418 609]{CJMC98-B4pic-2.eps}
\end{wrapfigure}

\textbf{The correct answer is (B): $2\sqrt{2}$}\\[1 ex]
Label the picture as shown, where O is the center of the square and octagon. OC is given by the Pythagorus theorem, OC=$\sqrt{3^2+3^2}=3\sqrt{2}$. 

Because CBA and CBD are 45-45-90 triangles, BC=BA=BD. And by Pythagorus, DD=$\sqrt{2^2+2^2}=2\sqrt{2}$, thus BC=$\sqrt{2}$. Therefore, OB=OC-BC=$2\sqrt{2}$, and thus the radius of the largest circle that can be drawn inside of the octagon is $2\sqrt{2}$.
%End
\\[5 ex]%Begin
%Language English
%Source Calgary Junior High School Mathematics Contest
%Title 1998
%Question B5
%Subject statistics
%Category concepts
%Type MC
%Choices 5
%Answer C
%Creator Victor Semenoff
%Rdifficulty 29
%Qtext

\scriptsize
Source: Calgary Junior High School Mathematics Contest

\normalsize
%\begin{wrapfigure}[2]{r}[0pt]{0pt}
%	\includegraphics[width=30mm,viewport=]{CCJ78-04}
%\end{wrapfigure}
Yesterday \textbf{more than} $90\%$ of the students in a certain class showed up for school. Today it snowed, and less than $75\%$ of the students in the class showed up. There were 5 fewer students in class today than yesterday. What is the largest possible number of students in the class?\\
%ChoiceA
(A) 24\\
%ChoiceB
(B) 27\\
%ChoiceC
(C) 31\\
%ChoiceD
(D) 35\\
%ChoiceE
(E) 40\\
%Ftext

%\begin{wrapfigure}{r}[0pt]{0pt}
%	\includegraphics[width=30mm,viewport=]{CCJ78-04}
%\end{wrapfigure}

\textbf{The correct answer is (C): 31}\\[1 ex]
We are told that 5 more students were present yesterday than today. To maximize the number of students in the class, we must minimize the percentage of the total class that the 5 students account for. Therefore, we want to minimize the difference between the percent that was here yesterday and the percent that was here today. So set the percentage that was here yesterday to be $90\%$ and the percent here today to be $75\%$. Then 5 students is $15\%$ of the class, and the class has $5\frac{1}{.15}=33\frac{1}{3}$.

At this point it looks like the correct answer is 33 students, but this is \textbf{wrong}. More than $90\%$ of 33 is at least 30, and less than $75\%$ is at most 24, so the minimum difference is 6 students not 5. Similarly 32 students doesn't work either. Thus 31 is the maximum number of students.
%End
\\[5 ex]
\end{document}
