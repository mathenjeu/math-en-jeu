\documentclass{article}
\usepackage{amsmath,amssymb,graphicx,enumitem,wrapfig}

\begin{document}
\parindent=0cm
\parskip=6pt
\pagestyle{empty}

%~~~~~~~~~

%Begin
%Language English
%Source Cariboo College High School Mathematics Contest
%Title Junior Preliminary Round 1978
%Question 1
%Subject algebra
%Category manipulations
%Type MC
%Choices 5
%Answer C
%Creator Jane Lee
%Rdifficulty 20

%Qtext
\scriptsize
Source: Cariboo College High School Mathematics Contest

\normalsize
If $3x+7=44$, then $6x+7$ equals:
\begin{enumerate}[noitemsep,topsep=0mm,leftmargin=*,widest=D,label=\Alph*)]
%ChoiceA
	\item 71
%ChoiceB
	\item 78
%ChoiceC
	\item 81
%ChoiceD
	\item 88
%ChoiceE
	\item 91
\end{enumerate}

%Ftext
\textbf{The correct answer is (C): 81}
\[
6x+7 = 6x+14-7 = 2(3x+7)-7=2(44)-7=81
\]
%End

\vskip 1.5cm

%Begin
%Language English
%Source Cariboo College High School Mathematics Contest
%Title Junior Preliminary Round 1978
%Question 2
%Subject arithmetic
%Category reals
%Type MC
%Choices 5
%Answer C
%Creator Jane Lee
%Rdifficulty 21

%Qtext
\scriptsize
Source: Cariboo College High School Mathematics Contest

\normalsize
An approximate value of $\dfrac{31.78\sqrt{2600}}{(3.98)^3}$ is:
\begin{enumerate}[noitemsep,topsep=0mm,leftmargin=*,widest=D,label=\Alph*)]
%ChoiceA
	\item 2.5
%ChoiceB
	\item 5.6
%ChoiceC
	\item 25
%ChoiceD
	\item 56
%ChoiceE
	\item 150
\end{enumerate}

%Ftext
\textbf{The correct answer is (C): 25}

The idea is to split each number into components that can be more easily calculated. Hence
\begin{align*}
\frac{31.78\sqrt{2600}}{(3.98)^3} &= \frac{31.78\sqrt{26\times100}}{(3.98)^3} = \frac{31.78\times 10\sqrt{26}}{(3.98)^3}\\
&\approx \frac{32\times10\sqrt{25}}{4^3} = \frac{32\cdot10\cdot5}{64} = \frac{10\cdot5}{2} = 25
\end{align*}
%End

\vskip 1.5cm

%Begin
%Language English
%Source Cariboo College High School Mathematics Contest
%Title Junior Preliminary Round 1978
%Question 5
%Subject arithmetic
%Category fractions
%Type MC
%Choices 5
%Answer B
%Creator Jane Lee
%Rdifficulty 17

%Qtext
\scriptsize
Source: Cariboo College High School Mathematics Contest

\normalsize
The sum of the two largest numbers in the set $\bigl\{\frac{4}{5},\frac{24}{25},0.9,1.1\bigr\}$ is:
\begin{enumerate}[noitemsep,topsep=0mm,leftmargin=*,widest=D,label=\Alph*)]
%ChoiceA
	\item 2.2
%ChoiceB
	\item 2.06
%ChoiceC
	\item 2
%ChoiceD
	\item 1.96
%ChoiceE
	\item 1.9
\end{enumerate}

%Ftext
\textbf{The correct answer is (B): 2.06}

The numbers in the set written as decimals are $\{0.8,0.96,0.9,1.1\}$. The two largest numbers are 1.1 and 0.96, the sum of which is $1.1+0.96=2.06$.
%End

\vskip 1.5cm

%Begin
%Language English
%Source Cariboo College High School Mathematics Contest
%Title Junior Preliminary Round 1978
%Question 6
%Subject arithmetic
%Category naturals
%Type MC
%Choices 5
%Answer D
%Creator Jane Lee
%Rdifficulty 24

%Qtext
\scriptsize
Source: Cariboo College High School Mathematics Contest

\normalsize
The number of 9's in the product of the integer 9,999,999,999 (ten 9's) and 99 is:
\begin{enumerate}[noitemsep,topsep=0mm,leftmargin=*,widest=D,label=\Alph*)]
%ChoiceA
	\item 12
%ChoiceB
	\item 11
%ChoiceC
	\item 10
%ChoiceD
	\item 9
%ChoiceE
	\item 8
\end{enumerate}

%Ftext
\textbf{The correct answer is (D): 9}
\begin{align*}
9,999,999,999 \times 99 &= (10,000,000,000 -1) \times 99\\
&= 990,000,000,000 - 99\\
&= 989,999,999,901
\end{align*}
Therefore, the product of 9,999,999,999 and 99 has 9 nines in it.
%End

\vskip 1.5cm

%Begin
%Language English
%Source Cariboo College High School Mathematics Contest
%Title Junior Preliminary Round 1978
%Question 12
%Subject arithmetic
%Category naturals
%Type MC
%Choices 5
%Answer B
%Creator Jane Lee
%Rdifficulty 24

%Qtext
\scriptsize
Source: Cariboo College High School Mathematics Contest

\normalsize
If the 50 digit integer made up of 50 twos is divided by 9, the remainder will be:
\begin{enumerate}[noitemsep,topsep=0mm,leftmargin=*,widest=D,label=\Alph*)]
%ChoiceA
	\item 0
%ChoiceB
	\item 1
%ChoiceC
	\item 2
%ChoiceD
	\item 3
%ChoiceE
	\item 6
\end{enumerate}

%Ftext
\textbf{The correct answer is (B): 1}

The remainder when a number is divided by 9 is equal to the remainder when the sum of its digits is divided by 9. Since the sum of the digits of the integer made of 50 twos is $50\times2=100$, the remainder when it is divided by 9 will be 1.
%End

\vskip 1.5cm

%Begin
%Language English
%Source Cariboo College High School Mathematics Contest
%Title Junior Preliminary Round 1978
%Question 3
%Subject algebra 
%Category manipulations
%Type MC
%Choices 5
%Answer E
%Creator Jane Lee
%Rdifficulty 20

%Qtext
\scriptsize
Source: Cariboo College High School Mathematics Contest

\normalsize
A kiss is worth 2 lemons and 1 orange, while a hug is worth 1 kiss and 1 orange. The number of hugs (no kisses) available for 6 lemons and 6 oranges is:
\begin{enumerate}[noitemsep,topsep=0mm,leftmargin=*,widest=D,label=\Alph*)]
%ChoiceA
	\item 8
%ChoiceB
	\item 6
%ChoiceC
	\item 5
%ChoiceD
	\item 4
%ChoiceE
	\item 3
\end{enumerate}

%Ftext
\textbf{The correct answer is (E): 3}

A hug is worth a kiss (2 lemons, 1 orange) plus 1 orange, so it's worth 2 lemons and 2 oranges in total.

Therefore 3 hugs are available for 6 lemons and 6 oranges.
%End

\vskip 1.5cm

%Begin
%Language English
%Source Cariboo College High School Mathematics Contest
%Title Junior Preliminary Round 1978
%Question 4
%Subject geometry
%Category area
%Type MC
%Choices 5
%Answer D
%Creator Jane Lee
%Rdifficulty 23

%Qtext
\scriptsize
Source: Cariboo College High School Mathematics Contest

\normalsize
\begin{wrapfigure}[2]{r}[0pt]{0pt}
	\includegraphics[width=30mm]{CCJ78-04}
\end{wrapfigure}
If, in the figure shown, $AB=2$ and the radius of each of the quarter circles is 4, then the area of the figure is:

\begin{enumerate}[noitemsep,topsep=0mm,leftmargin=*,widest=D,label=\Alph*)]
%ChoiceA
	\item $8+8\pi$
%ChoiceB
	\item $8+4\pi$
%ChoiceC
	\item $8+2\pi$
%ChoiceD
	\item 48
%ChoiceE
	\item 64
\end{enumerate}

%Ftext
\begin{wrapfigure}{r}[0pt]{0pt}
	\includegraphics[width=30mm]{CCJ78-04}
\end{wrapfigure}
\textbf{The correct answer is (D): 48}

If the quarter circle at the bottom were moved to the quarter circular space at the top, a rectangle would be formed.

The top side of the rectangle has length $AB$ plus the radius of the quarter circle. Since $AB=2$ and the radius of the quarter circle is 4, this side has length 6.

The right side of the rectangle is twice the radius of the quarter circle, or $2 \times 4 = 8$.

Hence the area of the rectangle is $8 \times 6 =48$.
%End

\vskip 1.5cm

%Begin
%Language English
%Source Cariboo College High School Mathematics Contest
%Title Junior Preliminary Round 1978
%Question 7
%Subject algebra
%Category modelling
%Type MC
%Choices 5
%Answer B
%Creator Jane Lee
%Rdifficulty 26

%Qtext
\scriptsize
Source: Cariboo College High School Mathematics Contest

\normalsize
2 pens cost the same as 3 pencils. A customer bought 1 pen and 1 pencil. The clerk said, ``The total of the digits of what you owe is 12.'' The number of cents, in change, that the customer receives from a dollar bill is:
\begin{enumerate}[noitemsep,topsep=0mm, leftmargin=*,widest=D,label=\Alph*)]
%ChoiceA
	\item 18
%ChoiceB
	\item 25
%ChoiceC
	\item 34
%ChoiceD
	\item 43
%ChoiceE
	\item none of these
\end{enumerate}

%Ftext
\textbf{The correct answer is (B): 25}

Let $p=$ cost of a pencil, and $P=$ cost of a pen. 2 pens cost the same as 3 pencils, so $2P=3p$ and $P=\frac{3}{2}p$. Then a pencil and a pen cost $C=p+\frac{3}{2}p=\frac{5}{2}p$.

Assuming both $p$ and $P$ are whole numbers, $C=\frac{5}{2}$ has to be divisible by 5, meaning $C$ ends in a 0 or a 5. The two digits add to 12, so the digits of $C$ are either $12+0$ (impossible in base 10) or 7$+5$.

So $C=75$ cents and the customer gets 25 cents change from a dollar.
%End

\vskip 1.5cm

%Begin
%Language English
%Source Cariboo College High School Mathematics Contest
%Title Junior Preliminary Round 1978
%Question 8
%Subject geometry
%Category triangles
%Type MC
%Choices 5
%Answer C
%Creator Jane Lee
%Rdifficulty 25

%Qtext
\scriptsize
Source: Cariboo College High School Mathematics Contest

\normalsize
%\begin{wrapfigure}{r}[0pt]{0pt}
%	\includegraphics[width=50mm]{CCJ78-08}
%\end{wrapfigure}
If $a$, $b$, and $c$ are the areas of the three isosceles right triangles with hypotenuses of length 3, 6, and 2, respectively, then $b$ equals:

\begin{minipage}[b]{0.5\textwidth}
\begin{enumerate}[noitemsep,topsep=0mm,leftmargin=*,widest=D,label=\Alph*)]
%ChoiceA
	\item $2a+3c$
%ChoiceB
	\item $2c+3a$
%ChoiceC
	\item $2a+4\frac{1}{2}c$
%ChoiceD
	\item $4a+c$
%ChoiceE
	\item $9c+4a$
\end{enumerate}
\end{minipage}
\begin{minipage}[b]{0.45\textwidth}
	\raggedleft
	\includegraphics[width=50mm]{CCJ78-08}
\end{minipage}

%Ftext
\begin{wrapfigure}[5]{r}[0pt]{0pt}
	\includegraphics[width=45mm]{CCJ78-08}
\end{wrapfigure}
\textbf{The correct answer is (C): $\mathbf{2a+4\frac{1}{2}c}$}

Recall that all isosceles right triangles are similar. Since the hypotenuse of $b$ is $2\times$ the hypotenuse of $a$, the other two sides of $b$ will be $2\times$ the corresponding sides of $a$. Thus $b=4a$.

Similarly we find that $b=9c$. Adding the two equations together we get:
\begin{align*}
2b&=4a+9c\\
b&=2a+4\tfrac{1}{2}c.
\end{align*}
%End

\vskip 1.5cm

%Begin
%Language English
%Source Cariboo College High School Mathematics Contest
%Title Junior Preliminary Round 1978
%Question 9
%Subject algebra
%Category modelling
%Type MC
%Choices 5
%Answer E
%Creator Jane Lee
%Rdifficulty 24

%Qtext
\scriptsize
Source: Cariboo College High School Mathematics Contest

\normalsize
In Al's car lot, 20\% of the cars are red and 60\% of the red cars are '77 models. If the number of red '77 model cars is 12, then the number of cars that are not red '77 models is:
\begin{enumerate}[noitemsep,topsep=0mm,leftmargin=*,widest=D,label=\Alph*)]
%ChoiceA
	\item 15
%ChoiceB
	\item 25
%ChoiceC
	\item 75
%ChoiceD
	\item 85
%ChoiceE
	\item 88
\end{enumerate}

%Ftext
\textbf{The correct answer is (E): 88}

Let c be the total number of cars on Al's lot. 20\% of these are red, so there are $0.2\,c$ red cars.

Then 60\% of those are '77 models, so there are $0.6(0.2\,c) = 0.12\,c$ red '77 cars. We know there are 12 of those, so
\[
c = \frac{12}{0.12} = 12 \div \frac{12}{100} = 12 \times \frac{100}{12} = 12 \times \frac{25}{3} = 4 \times 25 = 100.
\]
Since 12 of the cars on the lot are red '77 models, $100-12=88$ cars are not red '77 models.
%End

\vskip 1.5cm

%Begin
%Language English
%Source Cariboo College High School Mathematics Contest
%Title Junior Preliminary Round 1978
%Question 10
%Subject arithmetic
%Category proportions
%Type MC
%Choices 5
%Answer C
%Creator Jane Lee
%Rdifficulty 23

%Qtext
\scriptsize
Source: Cariboo College High School Mathematics Contest

\normalsize
Ann, Bill, and Chuck meet at the bank on April 5, 1978. If Ann goes to the bank every 6 weeks, Bill every 8 weeks, and Chuck every 10 weeks, then the number of weeks that pass before they all meet at the bank again will be:
\begin{enumerate}[noitemsep,topsep=0mm,leftmargin=*,widest=D,label=\Alph*)]
%ChoiceA
	\item 480
%ChoiceB
	\item 240
%ChoiceC
	\item 120
%ChoiceD
	\item 60
%ChoiceE
	\item 24
\end{enumerate}

%Ftext
\textbf{The correct answer is (C): 120}

Ann goes to the bank every week that is a multiple of 6.

Bill goes to the bank every week that is a multiple of 8.

Chuck goes to the bank every week that is a multiple of 10.

Thus they will all go to the bank on weeks that are a common multiple of 6, 8, and 10. The lowest common multiple of $6=2\times 3$, $8=2^3$, and $10=2 \times 5$ is $2^3 \times 3 \times 5 = 120$.
%End

\vskip 1.5cm

%Begin
%Language English
%Source Cariboo College High School Mathematics Contest
%Title Final Round Part A 1978
%Question 6
%Subject geometry
%Category length
%Type MC
%Choices 5
%Answer B
%Creator Jane Lee
%Rdifficulty 28

%Qtext
\scriptsize
Source: Cariboo College High School Mathematics Contest

\normalsize
\begin{wrapfigure}[4]{r}[0pt]{0pt}
	\includegraphics[width=40mm]{CCFA78-06}
\end{wrapfigure}
The diagram shows 6 nested squares obtained by joining the mid-points of the sides of each square in a suitable way. If the area of the smallest square is 1, then the perimeter of the largest square is:

\begin{enumerate}[noitemsep,topsep=0mm,leftmargin=*,widest=D,label=\Alph*)]
%ChoiceA
	\item 32
%ChoiceB
	\item $16\sqrt{2}$
%ChoiceC
	\item 16
%ChoiceD
	\item 8
%ChoiceE
	\item $4\sqrt{2}$
\end{enumerate}

%Ftext
\begin{wrapfigure}{r}[0pt]{0pt}
	\includegraphics[width=30mm]{CCFA78-06}
\end{wrapfigure}
\textbf{The correct answer is (B): $\mathbf{16\sqrt{2}}$}

Let $l_n$ be the length of the sides of the $n^\text{th}$ largest square. Consider one of the outermost right triangles. It has two legs of length $l_1/2$, so the hypotenuse $l_2$ is
\[
l_2 = \sqrt{2} \times \frac{l_1}{2} = l_1 \frac{1}{\sqrt{2}}.
\]
Repeat 4 times for successive squares:
\[
l_6 =  l_1 \left(\frac{1}{\sqrt{2}}\right)^5 = \frac{l_1}{4\sqrt{2}} \quad \Longrightarrow \quad l_1 = l_6 \times 4 \sqrt{2} 
\]
We know the area of the smallest square is 1, so $l_6=1$ and $l_1 = 4\sqrt{2}$. Then the perimeter of the largest square is $4\times 4\sqrt{2} = 16 \sqrt{2}$.
%End

\vskip 1.5cm

%Begin
%Language English
%Source Cariboo College High School Mathematics Contest
%Title Final Round Part B 1978
%Question 3a
%Subject arithmetic
%Category naturals
%Type SA
%Answer 10
%Creator Jane Lee
%Rdifficulty 30

%Qtext
\scriptsize
Source: Cariboo College High School Mathematics Contest

\normalsize
Imagine 100 lockers, all closed, and 100 students. Suppose the first student goes along and opens every locker. Then the second student goes along and shuts every second locker. Then the third student goes along and changes the state of every third locker (ie. if it is open, she shuts it, and vice versa). Then the fourth student goes along and changes the state of every fourth locker, etc. until all 100 students have passed by all lockers.

How many lockers will be open at the end?

%Ftext
\textbf{The correct answer is 10}

A locker will be open only if its state is changed an odd number of times. If $x$ is the number of lockers, then the state of a locker will change once for every divisor of $x$ between 1 and $x$ itself, inclusive.

If $n$ is a divisor of $x$, then $x/n$ is also a divisor of $n$. Thus every divisor $x$ has another divisor $n/x$ paired with it, except where $x = n/x$. This only happens in perfect squares.

Thus every opening of a locker is eventually canceled by a closing, except in perfect square lockers. Since $10^2=100$, there are ten perfect squares between 1 and 100. Hence 10 lockers will be open after the $100^\text{th}$ student goes by.
%End

\vskip 1.5cm

%Begin
%Language English
%Source Cariboo College High School Mathematics Contest
%Title Junior Preliminary Round 1978
%Question 11
%Subject algebra
%Category modelling
%Type MC
%Choices 5
%Answer D
%Creator Jane Lee
%Rdifficulty 23

%Qtext
\scriptsize
Source: Cariboo College High School Mathematics Contest

\normalsize
When Antonino goes fishing, he finds that on the trip from his home to the lake his jeep uses 10 km/L of gas, but on the return trip it uses 15 km/L of gas. If he uses 2 L of gas for the entire trip, the distance, in km, of the lake from his home must be:
\begin{enumerate}[noitemsep,topsep=0mm,leftmargin=*,widest=D,label=\Alph*)]
%ChoiceA
	\item 25
%ChoiceB
	\item 20
%ChoiceC
	\item $12\frac{1}{2}$
%ChoiceD
	\item 12
%ChoiceE
	\item 10
\end{enumerate}

%Ftext
\textbf{The correct answer is (D): 12}

Let $d$ be the distance from the home to the lake, $l_1$ be the litres of gas used on the way to the lake, and $l_2$ be the litres of gas used on the way back.

On the way to the lake, Antonino burns 10 L of gas for every km. So $l_1 = d/10$.
On the way back home, Antonino burns 15 L of gas for every km. So $l_2 = d/15$.

In total Antonino burns 2 L of gas, so $2 = l_1 + l_2 = d \left(\dfrac{1}{10}+\dfrac{1}{15}\right)$ and
\[
d = \frac{2}{1/10 + 1/15} = \frac{60}{3+2} = 12.
\]
%End
  
\vskip 1.5cm

%Begin
%Language English
%Source Cariboo College High School Mathematics Contest
%Title Junior Preliminary Round 1978
%Question 13
%Subject geometry
%Category length
%Type MC
%Choices 5
%Answer D
%Creator Jane Lee
%Rdifficulty 27

%Qtext
\scriptsize
Source: Cariboo College High School Mathematics Contest

\normalsize
\begin{wrapfigure}{r}[0pt]{0pt}
	\includegraphics[width=50mm]{CCJ78-13-1}
\end{wrapfigure}
$P$ is the midpoint of one side of a rectangle which is twice as long as it is wide, and which has a perimeter of 24. The radius of the circle inscribed in triangle $PCD$ is approximately:

\begin{enumerate}[noitemsep,topsep=0mm,leftmargin=*,widest=D,label=\Alph*)]
%ChoiceA
	\item 1 1/3
%ChoiceB
	\item 1 1/4
%ChoiceC
	\item 1 1/2
%ChoiceD
	\item 1 2/3
%ChoiceE
	\item 1 3/4
\end{enumerate}

%Ftext
\begin{wrapfigure}[7]{r}[0pt]{0pt}
	\includegraphics[width=50mm]{CCJ78-13-2}
\end{wrapfigure}
\textbf{The correct answer is (D): $\mathbf{1\frac{2}{3}}$}

Let $r$ be the radius of the circle, and $l$ be the length of the rectangle. The width of the rectangle will then be $l/2$. Since the perimeter is 24, we have
\[
24 = 2 \left(l+\frac{l}{2}\right) = 3l \quad \Longrightarrow \quad l=8.
\]
Now the triangle $PCD$ has height 4. Consider a perpendicular from $P$ down to $CD$, and a radius of the circle from the centre $O$ to $PC$, meeting $PC$ at $E$. Then triangle $POE$ is a right triangle with $EP=EO=r$, from which we see that $PO=\sqrt{2}\;r$. The remaining distance from $O$ to $CD$ is $r$, so that
\begin{align*}
\text{width of the rectangle} = 4 = \sqrt{2}\;r+r = r(\sqrt{2}+1)\\
r = \frac{4}{\sqrt{2}+1} \approx \frac{4}{14/10+1} = \frac{4}{24/10} = \frac{40}{24} = 1 \frac{2}{3}.
\end{align*}
%End

\vskip 1.5cm

%Begin
%Language English
%Source Cariboo College High School Mathematics Contest
%Title Junior Preliminary Round 1978
%Question 14
%Subject algebra
%Category modelling
%Type MC
%Choices 5
%Answer E
%Creator Jane Lee
%Rdifficulty 24

%Qtext
\scriptsize
Source: Cariboo College High School Mathematics Contest

\normalsize
The outside wheel of a chariot running on a circular track is going twice as fast as the inside one. If the wheels are 2 metres apart, then the number of metres traveled by the outside wheel in 10 circuits of the track is:
\begin{enumerate}[noitemsep,topsep=0mm,leftmargin=*,widest=D,label=\Alph*)]
%ChoiceA
	\item 20
%ChoiceB
	\item 40
%ChoiceC
	\item $40\pi$
%ChoiceD
	\item $60\pi$
%ChoiceE
	\item $80\pi$
\end{enumerate}

%Ftext
\textbf{The correct answer is (E): $\mathbf{80\pi}$}

Let $r$ be the radius of the track used by the inner wheel. The inner wheel, then, travels $2\pi r$ metres per circuit. Since the outer wheel is 2 m away, it travels $2\pi(r+2)$ metres per circuit.

The outer wheel goes twice as fast so it makes 2 turns for every 1 turn of the inner wheel. Assuming the wheels have the same radius, the outer wheel travels twice as far as the inner wheel, so that
\begin{align*}
2\pi (r+2) &= 2\times 2\pi r\\
2\pi r + 4\pi &= 4\pi r\\
2r &= 4\\
r &= 2
\end{align*}
The outer wheel then travels $2\pi(2+2) = 8\pi$ m per circuit. Therefore, after 10 circuits, the outer wheel will have traveled $80\pi$ m.
%End

\vskip 1.5cm

%Begin
%Language English
%Source Cariboo College High School Mathematics Contest
%Title Junior Preliminary Round 1978
%Question 15
%Subject arithmetic
%Category integers
%Type MC
%Choices 5
%Answer A
%Creator Jane Lee
%Rdifficulty 27

%Qtext
\scriptsize
Source: Cariboo College High School Mathematics Contest

\normalsize
If $n$ is an integer and $r$ is the smallest integer such that $252r = n^2$, then $r+n$ equals:
\begin{enumerate}[noitemsep,topsep=0mm,leftmargin=*,widest=D,label=\Alph*)]
%ChoiceA
	\item 49
%ChoiceB
	\item 51
%ChoiceC
	\item 34
%ChoiceD
	\item 40
%ChoiceE
	\item 36
\end{enumerate}

%Ftext
\textbf{The correct answer is (A): 49}

If $252r$ is to equal $n^2$, and $n$ is an integer, $252r$ must be a square. Factorization gives $252r = 2^2 3^3 7r$, so that the smallest possible value of $r$ is 7. In this case
\[
n^2 = 252\,r = 2^2\cdot3^3\cdot7^2 = (2 \cdot 3 \cdot 7)^2 = (42)^2
\]
so that $n=42$. Thus $r+n = 7+42 = 49$.
%End

\vskip 1.5cm

%Begin
%Language English
%Source Cariboo College High School Mathematics Contest
%Title Junior Preliminary Round 1978
%Question 16
%Subject algebra
%Category manipulations
%Type MC
%Choices 5
%Answer D
%Creator Jane Lee
%Rdifficulty 25

%Qtext
\scriptsize
Source: Cariboo College High School Mathematics Contest

\normalsize
If $n$ and $s$ are positive integers, then the number of solutions of the equation $3n+2s=73$ is:
\begin{enumerate}[noitemsep,topsep=0mm,leftmargin=*,widest=D,label=\Alph*)]
%ChoiceA
	\item 36
%ChoiceB
	\item 35
%ChoiceC
	\item 18
%ChoiceD
	\item 12
%ChoiceE
	\item 8
\end{enumerate}

%Ftext
\textbf{The correct answer is (D): 12}

One possible solution is $(n,s)=(1,35)$. If $n$ and $s$ are to remain positive integers in the remaining solutions, $n$ must rise by 2 for every drop of 3 in $s$. In fact, all solutions $(n,s)$ must have the form $(1+2m,35-3m)$ for integer $m$.

The smallest allowed solution is then (23,2) for $m=11$. Thus there are 12 solutions in all.
%End

\vskip 1.5cm

%Begin
%Language English
%Source Cariboo College High School Mathematics Contest
%Title Junior Preliminary Round 1978
%Question 17
%Subject geometry
%Category triangles
%Type MC
%Choices 5
%Answer D
%Creator Jane Lee
%Rdifficulty 25

%Qtext
\scriptsize
Source: Cariboo College High School Mathematics Contest

\normalsize
\begin{wrapfigure}{r}[0pt]{0pt}
	\includegraphics[width=45mm]{CCJ78-17}
\end{wrapfigure}
A box is dragged along from $S$ to $F$, a distance of 7, by a weight attached to a rope which passes over a pulley at point $P$. If $PB = 12$ and $SB=16$, then the distance the weight moves is:
\begin{enumerate}[noitemsep,topsep=0mm,leftmargin=*,widest=D,label=\Alph*)]
%ChoiceA
	\item 3
%ChoiceB
	\item 3.5
%ChoiceC
	\item 4
%ChoiceD
	\item 5
%ChoiceE
	\item 7
\end{enumerate}

%Ftext
\begin{wrapfigure}{r}[0pt]{0pt}
	\includegraphics[width=45mm]{CCJ78-17}
\end{wrapfigure}
\textbf{The correct answer is (D): 5}

The distance the weight moves is the difference in rope lengths between the line segments $SP$ and $FP$. Applying Pythagoras to the right triangle $SBP$, we find
\begin{align*}
(SP)^2 &= (SB)^2 + (BP)^2 = 16^2 + 12^2 = 400^2;\\
SP &= 20.
\end{align*}
Similarly, for triangle $FBP$ we find
\begin{align*}
(FP)^2 &= (FB)^2 + (BP)^2 = 9^2 + 12^2 = 225^2;\\
FP &= 15.
\end{align*}
Thus the distance the weight moves is $SP-FP=20-15=5$.
%End

\vskip 1.5cm

%Begin
%Language English
%Source Cariboo College High School Mathematics Contest
%Title Junior Preliminary Round 1978
%Question 18
%Subject arithmetic
%Category proportions
%Type MC
%Choices 5
%Answer B
%Creator Jane Lee
%Rdifficulty 24

%Qtext
\scriptsize
Source: Cariboo College High School Mathematics Contest

\normalsize
The times it takes three tireless boys, $A$, $B$, and $C$, to run around a circular track are 1/2 minute, 1 1/3 minutes, and 1 2/5 minutes respectively. If they are together at the starting line at noon, the number of minutes that pass before they will again all be at the starting line is:
\begin{enumerate}[noitemsep,topsep=0mm,leftmargin=*,widest=D,label=\Alph*)]
%ChoiceA
	\item 4
%ChoiceB
	\item 28
%ChoiceC
	\item 52 1/2
%ChoiceD
	\item 56
%ChoiceE
	\item 60
\end{enumerate}

%Ftext
\textbf{The correct answer is (B): 28}

Let $a$, $b$, and $c$ be the number of laps $A$, $B$, and $C$ have made, respectively, when all three are back at the starting line. The time that has passed will equal the number of laps that $A$ has made times the amount of time each lap takes, or $a\;\times$ 1/2 min. Similarly, the time will also equal $b\;\times$ 1 1/3 min and $c\;\times$ 1 2/5 min. Since they are all equal we have
\[
\frac{1}{2} a = \frac{4}{3} b = \frac{7}{5} c 
\]
Multiplying through by $2 \times 3 \times 5$, we get
\[
3 \times 5 \times a = 2^3 \times 5 \times b = 2 \times 3 \times 7 \times c
\]
The lowest common multiple of these numbers is $2^3 \times 3 \times 5 \times 7$, meaning that $a = 2^3 \times 7 = 56$. Hence $A$ will have made 56 laps when everyone is back at the starting line. This will take $56/2 = 28$ min.
%End

\vskip 1.5cm

%Begin
%Language English
%Source Cariboo College High School Mathematics Contest
%Title Junior Preliminary Round 1978
%Question 19
%Subject probability
%Category counting
%Type MC
%Choices 5
%Answer C
%Creator Jane Lee
%Rdifficulty 25

%Qtext
\scriptsize
Source: Cariboo College High School Mathematics Contest

\normalsize
A man stands with his back against a wall. If he takes 6 steps (some forward, some backward), how many different patterns of forward and backward steps can he take, if each pattern must bring him back to his starting position?
\begin{enumerate}[noitemsep,topsep=0mm,leftmargin=*,widest=D,label=\Alph*)]
%ChoiceA
	\item 3
%ChoiceB
	\item 4
%ChoiceC
	\item 5
%ChoiceD
	\item 10
%ChoiceE
	\item 20
\end{enumerate}

%Ftext
\begin{wrapfigure}{r}[0pt]{0pt}
	\includegraphics[width=30mm]{CCJ78-19}
\end{wrapfigure}
\textbf{The correct answer is (C): 5}

Refer to the accompanying diagram for the man's possible moves, and the number of paths to get to each position. Note that the number of ways to get to each position is equal to the sum of the ways to get to the positions 1 step further from the wall and 1 step closer to the wall on the previous move.
%End

\vskip 1.5cm

%Begin
%Language English
%Source Cariboo College High School Mathematics Contest
%Title Junior Preliminary Round 1978
%Question 20
%Subject geometry
%Category area
%Type MC
%Choices 5
%Answer E
%Creator Jane Lee
%Rdifficulty 26

%Qtext
\scriptsize
Source: Cariboo College High School Mathematics Contest

\normalsize

\begin{minipage}[b]{0.6\textwidth}
	Three circles, each of radius 1, are tangent to each other and thus enclose the shaded region as shown. The area of the shaded region is:
	\\
	\begin{enumerate}[itemsep=0mm,topsep=0mm,leftmargin=*,widest=D,label=\Alph*)]
%ChoiceA
		\item $\frac{\sqrt{3}}{2} - \frac{\pi}{3}$
%ChoiceB
		\item $\sqrt{3} - \pi$
%ChoiceC
		\item $\sqrt{3} - \frac{\pi}{3}$
%ChoiceD
		\item $\pi - \sqrt{3}$
%ChoiceE
		\item $\sqrt{3} - \frac{\pi}{2}$
	\end{enumerate}
\end{minipage}
\begin{minipage}[b]{0.35\textwidth}
	\raggedleft
	\includegraphics[width=40mm]{CCJ78-20}
\end{minipage}

%Ftext
\textbf{The correct answer is (E): $\mathbf{\sqrt{3} - \dfrac{\pmb{\pi}}{2}}$}

Note that a triangle drawn connecting the centres of the circles will be equilateral. The area of the shaded region, then, is the area of this triangle minus the area of the three circular sectors within it.

Since the triangle is equilateral each sector subtends a 60$^\circ$ angle. The area of each sector is
\[
\frac{60^\circ}{360^\circ} \pi 1^2 = \frac{\pi}{6}
\]
and the area of all three sectors is $3 \times \pi/6 = \pi/2$. The area of an equilateral triangle with base $b$ is $(\sqrt{3}/4)b^2$. Since $b = 2$ in our case, the area of our triangle is $\sqrt{3}$, and the area of the shaded region is $\sqrt{3} - \pi/2$.
%End
 
\vskip 1.5cm

%Begin
%Language English
%Source Cariboo College High School Mathematics Contest
%Title Senior Preliminary Round 1978
%Question 13
%Subject functions
%Category exponential
%Type MC
%Choices 5
%Answer D
%Creator Jane Lee
%Rdifficulty 24

%Qtext
\scriptsize
Source: Cariboo College High School Mathematics Contest

\normalsize
The number $64 \log 16 - 4 \log 8$ equals:
\begin{enumerate}[noitemsep,topsep=0mm,leftmargin=*,widest=D,label=\Alph*)]
%ChoiceA
	\item $16 \log 2$
%ChoiceB
	\item $60 \log 8$
%ChoiceC
	\item $60 \log 2$
%ChoiceD
	\item $244 \log 2$
%ChoiceE
	\item $256 \log 2$
\end{enumerate}

%Ftext
\textbf{The correct answer is (D): 244 log 2}
\begin{align*}
64 \log 16 - 4 \log 8	&= \log 16^{64} - \log 8^4 = \log \frac{16^{64}}{8^4} = \log \frac{(2^4\!)^{64}}{(2^3\!)^4}\\
											&= \log \frac{2^{256}}{2^{12}} = \log 2^{(256-12)} = \log 2^{244} = 244 \log 2
\end{align*}
%End

\vskip 1.5cm

%Begin
%Language English
%Source Cariboo College High School Mathematics Contest
%Title Senior Preliminary Round 1978
%Question 12
%Subject algebra
%Category manipulations
%Type MC
%Choices 5
%Answer C
%Creator Jane Lee
%Rdifficulty 28

%Qtext
\scriptsize
Source: Cariboo College High School Mathematics Contest

\normalsize
Suppose that the sum of all the positive integers from 1 to $n$, inclusive, is $S$; then the sum of all the positive integers from 1 to $2n$, inclusive, is:
\begin{enumerate}[noitemsep,topsep=0mm,leftmargin=*,widest=D,label=\Alph*)]
%ChoiceA
	\item $2S$
%ChoiceB
	\item $2S + 2n$
%ChoiceC
	\item $2S + n^2$
%ChoiceD
	\item $2S + n/2$
%ChoiceE
	\item $2S + n^2/2$
\end{enumerate}

%Ftext
\textbf{The correct answer is (C): $\mathbf{2S + n^2}$}

Let $S_{2n}$ be the sum of all the positive integers from 1 to $2n$ inclusive. Then $S_{2n}$ is $S$ plus the sum of the integers from $n+1$ to $2n$. The sum of the integers from $n+1$ to $2n$ can be found by noticing that when we subtract $n$ from each of the $n$ values, we have left the sum from 1 to $n$, which is $S$. Thus
\[
S_{2n} = S+(n^2+S)=2S+n^2.
\]
%End

\vskip 1.5cm

%Begin
%Language English
%Source Cariboo College High School Mathematics Contest
%Title Senior Preliminary Round 1978
%Question 14
%Subject functions
%Category parts
%Type MC
%Choices 5
%Answer C
%Creator Jane Lee
%Rdifficulty 25

%Qtext
\scriptsize
Source: Cariboo College High School Mathematics Contest

\normalsize
If $f(x)=3x-5$ and $g(f(x))=x$ for all $x$, then $g(x)$ equals:
\begin{enumerate}[topsep=0mm,leftmargin=*,widest=D,label=\Alph*)]
%ChoiceA
	\item $\dfrac{x}{3}+5$
%ChoiceB
	\item $\dfrac{x}{3}-5$
%ChoiceC
	\item $\dfrac{x+5}{3}$
%ChoiceD
	\item $\dfrac{1}{3x-5}$
%ChoiceE
	\item $\dfrac{1}{3x}-\dfrac{1}{5}$
\end{enumerate}

%Ftext
\textbf{The correct answer is (C): $\mathbf{\dfrac{x+5}{3}}$}

Since $g(f(x)) = g(3x-5)$, we must find a function $g(x)$ such that $g(3x-5)=x$ for all $x$. Thus we must think of a series of steps to `undo' $3x-5$ back into $x$. Thus $g(x)$ must first add 5 to $x$ and then divide by 3, or
\[
g(x) = (x+5) \div 3 = \frac{x+5}{3}.
\] 
%End

\vskip 1.5cm

%Begin
%Language English
%Source Cariboo College High School Mathematics Contest
%Title Senior Preliminary Round 1978
%Question 15
%Subject functions
%Category polynomials
%Type MC
%Choices 5
%Answer B
%Creator Jane Lee
%Rdifficulty 26

%Qtext
\scriptsize
Source: Cariboo College High School Mathematics Contest

\normalsize
The points $(1,a)$ and $(-1,b)$ are on the curve $y=4x^4+p\,x^3+q\,x^2+rx+5$. If $a+b=10$, then the value of $q$ is:
\begin{enumerate}[noitemsep,topsep=0mm,leftmargin=*,widest=D,label=\Alph*)]
%ChoiceA
	\item -9
%ChoiceB
	\item -4
%ChoiceC
	\item -1
%ChoiceD
	\item 5
%ChoiceE
	\item not determined by the information given
\end{enumerate}

%Ftext
\textbf{The correct answer is (B): -4}

Substitute $(x,y)=(1,a)$ into the equation for the curve and get
\[
a=p+q+r+9.
\]
Then substitute again with $(x,y)=(-1,b)$ and get
\[
b= -p+q-r+9.
\]
Add the above equations together to get $a+b=2q+18$. Since $a+b=10$, we can get
\begin{align*}
10&=2q+18\\
2q&=-8\\
q&=-4.
\end{align*}
%End

\vskip 1.5cm

%Begin
%Language English
%Source Cariboo College High School Mathematics Contest
%Title Senior Preliminary Round 1978
%Question 16
%Subject functions
%Category polynomials
%Type MC
%Choices 5
%Answer C
%Creator Jane Lee
%Rdifficulty 26

%Qtext
\scriptsize
Source: Cariboo College High School Mathematics Contest

\normalsize
If the point $(3,y)$ lies on the line joining $(0,\frac{3}{2}$ and $(\frac{9}{4},0)$, then $y$ equals:
\begin{enumerate}[noitemsep,topsep=0mm,leftmargin=*,widest=D,label=\Alph*)]
%ChoiceA
	\item -7/2
%ChoiceB
	\item -2
%ChoiceC
	\item -1/2
%ChoiceD
	\item 1/2
%ChoiceE
	\item 3/2
\end{enumerate}

%Ftext
\textbf{The correct answer is (C): -1/2}

If $(3,y)$ is on the line, then the slope $m_1$from $(3,y)$ to $\bigl(0,\frac{3}{2}\bigr)$ is the same as the slope $m_2$ from $\bigl(0,\frac{3}{2}\bigr)$ to $\bigl(\frac{9}{4},0\bigr)$. Hence $m_1=m_2$, or
\begin{align*}
\frac{y-\frac{3}{2}}{3-0} &= \frac{\frac{3}{2}-0}{0-\frac{9}{4}}\\
\frac{2y-3}{6} &= -\frac{2}{3}\\
2y-3&=-4\\
y&= -\frac{1}{2}
\end{align*}
%End

\vskip 1.5cm

%Begin
%Language English
%Source Cariboo College High School Mathematics Contest
%Title Senior Preliminary Round 1978
%Question 17
%Subject functions
%Category parts
%Type MC
%Choices 5
%Answer E
%Creator Jane Lee
%Rdifficulty 24

%Qtext
\scriptsize
Source: Cariboo College High School Mathematics Contest

\normalsize
If $f(0)=2$, $f(1)=3$, and $f(n+2)=2f(n)-f(n+1)$, where $n$ is an integer, then $f(5)$ equals:
\begin{enumerate}[noitemsep,topsep=0mm,leftmargin=*,widest=D,label=\Alph*)]
%ChoiceA
	\item -7
%ChoiceB
	\item -3
%ChoiceC
	\item 7
%ChoiceD
	\item 10
%ChoiceE
	\item 13
\end{enumerate}

%Ftext
\textbf{The correct answer is (E): 13}

Use an iteration process, going from $f(0)$ to $f(2)$ to $f(4)$, etc.
\begin{align*}
f(2) &= f(0+2) = 2f(0)-f(1) = 2(2) - (3) = 1\\
f(3) &= f(1+2) = 2f(1)-f(2) = 2(3) - (1) = 5\\
f(4) &= f(2+2) = 2f(2)-f(3) = 2(1)-(5) = -3\\
f(5) &= f(3+2) = 2f(3)-f(4) = 2(5)-(-3) = 13
\end{align*}
%End

\vskip 1.5cm

%Begin
%Language English
%Source Cariboo College High School Mathematics Contest
%Title Senior Preliminary Round 1978
%Question 18
%Subject functions
%Category polynomials
%Type MC
%Choices 5
%Answer E
%Creator Jane Lee
%Rdifficulty 26

%Qtext
\scriptsize
Source: Cariboo College High School Mathematics Contest

\normalsize
If $N= {\displaystyle\sum_{i=1}^{100}}\; i^2$ and $S={\displaystyle\sum_{i=1}^{100}} (i+3)^2$, then $S-N$ equals:
\begin{enumerate}[noitemsep,topsep=0mm,leftmargin=*,widest=D,label=\Alph*)]
%ChoiceA
	\item 9
%ChoiceB
	\item 609
%ChoiceC
	\item 900
%ChoiceD
	\item 30 300
%ChoiceE
	\item 31 200
\end{enumerate}

%Ftext
\textbf{The correct answer is (E): 31 200}

$S$ is the sum of the squares from $4^2$ to $103^2$, while $N$ is the sum of the squares from $1^2$ to $100^2$. Thus both sums include the squares from $4^2$ to $100^2$. The difference is then
\begin{align*}
S-N &= 101^2+102^2+103^2-1^1-2^2-3^2\\
&= (101^2-1^2)-(102^2-2^2)+(103^2-3^2)\\
&= 100 \cdot 102 + 100 \cdot 104 + 100 \cdot 106 && \text{\small(use difference of squares)}\\
&= 31\;200
\end{align*}
%End

\vskip 1.5cm

%Begin
%Language English
%Source Cariboo College High School Mathematics Contest
%Title Senior Preliminary Round 1978
%Question 19
%Subject geometry
%Category length
%Type MC
%Choices 5
%Answer E
%Creator Jane Lee
%Rdifficulty 28

%Qtext
\scriptsize
Source: Cariboo College High School Mathematics Contest

\normalsize
\begin{wrapfigure}[5]{r}[0pt]{0pt}
	\includegraphics[width=45mm]{CCS78-19-1}
\end{wrapfigure}
The diameter of each of the circles in the diagram is 2. If circle $A$, sitting at the highest point of $B$, is moved down so that it touches both $B$ and $C$, then its loss in height will be:
\begin{enumerate}[noitemsep,topsep=0mm,leftmargin=*,widest=D,label=\Alph*)]
%ChoiceA
	\item 1
%ChoiceB
	\item $3-\sqrt{2}$
%ChoiceC
	\item $2-\sqrt{2}$
%ChoiceD
	\item $3-\sqrt{3}$
%ChoiceE
	\item $2-\sqrt{3}$
\end{enumerate}

%Ftext
\begin{wrapfigure}[8]{r}[0pt]{0pt}
	\includegraphics[width=45mm]{CCS78-19-2}
\end{wrapfigure}
\textbf{The correct answer is (E): $\mathbf{2-\sqrt{3}}$}

The radius of each circle is 1. We are looking for the difference in height between $a$ and $a'$ in the diagram.

The height of $a$ is 3. Note that $a'bc$ is an equilateral triangle with base 2; hence its height is $2 \times \sin 60^{\circ} = \sqrt{3}$. The height of $a'$ is then $\sqrt{3}+1$.

The loss of height of the circle is $a-a'=3-\sqrt{3}+1=2-\sqrt{3}$. 
%End

\vskip 1.5cm

%Begin
%Language English
%Source Cariboo College High School Mathematics Contest
%Title Final Round Part A 1978
%Question 1
%Subject arithmetic
%Category fractions
%Type MC
%Choices 5
%Answer A
%Creator Jane Lee
%Rdifficulty 23

%Qtext
\scriptsize
Source: Cariboo College High School Mathematics Contest

\normalsize
The exact value of $\dfrac{(35 \times 13)+(13\times13)+26}{5\times13}$ is:
\begin{enumerate}[noitemsep,topsep=0mm,leftmargin=*,widest=D,label=\Alph*)]
%ChoiceA
	\item 10
%ChoiceB
	\item 11.6
%ChoiceC
	\item 22
%ChoiceD
	\item 25.2
%ChoiceE
	\item 202
\end{enumerate}

%Ftext
\textbf{The correct answer is (A): 10}
\[
\frac{(35 \times 13)+(13\times13)+26}{5\times13} = \frac{35+13+2}{5} = \frac{50}{5} = 10
\]
%End

\vskip 1.5cm

%Begin
%Language English
%Source Cariboo College High School Mathematics Contest
%Title Final Round Part A 1978
%Question 4
%Subject probability
%Category calculations
%Type MC
%Choices 5
%Answer B
%Creator Jane Lee
%Rdifficulty 26

%Qtext
\scriptsize
Source: Cariboo College High School Mathematics Contest

\normalsize
Bill has 3 boxes with marbles in them. 24\% of the marbles in the first box are red. The second box contains twice as many marbles as the first, and 15\% of them are red. The third box contains 3 times as many marbles as the first box and 10\% of them are red. If Bill puts all the marbles into one box, then the percentage of red marbles in this box will be:
\begin{enumerate}[noitemsep,topsep=0mm,leftmargin=*,widest=D,label=\Alph*)]
%ChoiceA
	\item 12
%ChoiceB
	\item 14
%ChoiceC
	\item 15
%ChoiceD
	\item $16 \frac{1}{3}$
%ChoiceE
	\item 49
\end{enumerate}

%Ftext
\textbf{The correct answer is (B): 14\%}

Let $m$ be the number of marbles in the first box. Then the second and third boxes have 2$m$ and $3m$ marbles, respectively. In total there will be $6m$ marbles.

The first box has 24\% red marbles, so it has $0.24m$ of them.\\
The second box has 15\% red marbles, so it has $0.15\times2m = 0.3m$ of them.\\
The third box has 10\% red marbles, so it has $0.1\times3m= 0.3m$ of them.

The total number of red marbles is $0.24m+0.3m+0.3m = 0.84m$. This puts the percentage of red marbles at
\[
\frac{0.84m}{6m}\times100\% = 14\%.
\]
%End

\vskip 1.5cm

%Begin
%Language English
%Source Cariboo College High School Mathematics Contest
%Title Final Round Part A 1978
%Question 3
%Subject geometry
%Category area
%Type MC
%Choices 5
%Answer D
%Creator Jane Lee
%Rdifficulty 26

%Qtext
\scriptsize
Source: Cariboo College High School Mathematics Contest

\normalsize
\begin{wrapfigure}[5]{r}[0pt]{0pt}
	\includegraphics[width=45mm]{CCFA78-03}
\end{wrapfigure}
In the diagram $ABCD$ is a rectangle; $EK$ and $LG$ are arcs of circles centred at $A$ and $C$ respectively. If
\[
AE=EF=FB=BL=LC=GH=2,
\]
then the area of the shaded part is:
\begin{enumerate}[noitemsep,topsep=0mm,leftmargin=*,widest=D,label=\Alph*)]
%ChoiceA
	\item $24-\pi$
%ChoiceB
	\item $24-2\pi$
%ChoiceC
	\item $20-\pi$
%ChoiceD
	\item $20-2\pi$
%ChoiceE
	\item $22-\pi$
\end{enumerate}

%Ftext
\begin{wrapfigure}[7]{r}[0pt]{0pt}
	\includegraphics[width=40mm]{CCFA78-03}
\end{wrapfigure}
\textbf{The correct answer is (D): $\mathbf{20-\boldsymbol{\pi}}$}

The area of each quarter circle is $\frac{1}{4}\pi2^2=\pi$. The area of each corner triangle is $\frac{1}{2}\times2\times2=2$. The total unshaded area is then $2\pi+4$. The area of rectangle $ABCD$ is $6\times4=24$.

The area of the shaded region is the area of the rectangle minus the area of the unshaded region, or $24-(2\pi+4)=20-2\pi$.
%End

\vskip 1.5cm

%Begin
%Language English
%Source Cariboo College High School Mathematics Contest
%Title Final Round Part A 1978
%Question 4
%Subject functions
%Category concepts
%Type MC
%Choices 5
%Answer E
%Creator Jane Lee
%Rdifficulty 23

%Qtext
\scriptsize
Source: Cariboo College High School Mathematics Contest

\normalsize
Ann is on a flat plain at the point $P$. She walks 2 km east, then 7 km north, then 6 km east, then 5 km north, and finally 3 km west. Her distance from $P$ (in km) is:
\begin{enumerate}[noitemsep,topsep=0mm,leftmargin=*,widest=D,label=\Alph*)]
%ChoiceA
	\item 23
%ChoiceB
	\item 17
%ChoiceC
	\item 15
%ChoiceD
	\item 14
%ChoiceE
	\item 13
\end{enumerate}

%Ftext
\textbf{The correct answer is (E): 13}

Imagine a Cartesian plane with origin at $P$, North at the positive y-axis, and East at the positive x-axis, with units in km. On this plane, Ann will walk from (0,0) to (2,0), then to (2,7), (8,7), (8,12), and finally (5,12). Using the distance formula, we find that Ann's final distance from $P$ is
\[
\sqrt{(5)^2+(12)^2} = \sqrt{169} = 13.
\]
%End

\vskip 1.5cm

%Begin
%Language English
%Source Cariboo College High School Mathematics Contest
%Title Final Round Part A 1978
%Question 5
%Subject arithmetic
%Category naturals
%Type MC
%Choices 5
%Answer D
%Creator Jane Lee
%Rdifficulty 27

%Qtext
\scriptsize
Source: Cariboo College High School Mathematics Contest

\normalsize
What percentage of the first 4000 positive integers give a remainder of 5 when divided by 9?
\begin{enumerate}[itemsep=0mm,topsep=0mm,leftmargin=*,widest=D,label=\Alph*)]
%ChoiceA
	\item $10 \frac{8}{9}$
%ChoiceB
	\item $10 \frac{9}{10}$
%ChoiceC
	\item $11$
%ChoiceD
	\item $11 \frac{1}{10}$
%ChoiceE
	\item $11 \frac{1}{9}$
\end{enumerate}

%Ftext
\textbf{The correct answer is (D): $\mathbf{11\frac{1}{10}}$}

The numbers that give 5 as a remainder when divided by 9 are 5, 14, 23, \ldots, \mbox{$5+9(n-1)$}. Since we are looking for numbers less than or equal to 4000, we must find $n$ such that
\begin{align*}
5+9(n-1) &\leq 4000\\
9(n-1) &\leq 3995\\
n-1 &\leq 443.88\ldots\\
n &\leq 444.88\ldots\\
n &\leq 444 &&(n\text{ is an integer.})
\end{align*}
Thus 444/4000 = 11.1\% of the first 4000 positive integers give a remainder of 5 when divided by 9.
%End

\vskip 1.5cm

%Begin
%Language English
%Source Cariboo College High School Mathematics Contest
%Title Final Round Part A 1978
%Question 7
%Subject algebra
%Category modelling
%Type MC
%Choices 5
%Answer A
%Creator Jane Lee
%Rdifficulty 26

%Qtext
\scriptsize
Source: Cariboo College High School Mathematics Contest

\normalsize
The wheels on Antonino's bicycle have a diameter of 70 cm, while those on Bill's bicycle have a diameter of 63 cm. In a 1 km ride Bill's front wheel will make more revolutions that Antonino's front wheel. Approximately how many more?
\begin{enumerate}[noitemsep,topsep=0mm,leftmargin=*,widest=D,label=\Alph*)]
%ChoiceA
	\item 50
%ChoiceB
	\item 100
%ChoiceC
	\item 150
%ChoiceD
	\item 200
%ChoiceE
	\item 700
\end{enumerate}

%Ftext
\textbf{The correct answer is (A): 50}

Each bicycle's wheels will travel the distance of one circumference for every revolution it makes. So Antonino's bicycle travels $70\pi$ cm = $0.7\pi$ m per revolution and Bill's bicycle travels $63\pi$ cm = $0.63\pi$ m per revolution. The total distance is 1 km = 1000 m, so the number of revolutions of each bicycle's wheels are given by

\begin{minipage}{0.45\textwidth}
\[r_{\text{Ant}} = \frac{1000\text{ m}}{0.7\pi \text{ m}}\]
\end{minipage}
\begin{minipage}{0.45\textwidth}
\[r_{\text{Bill}} = \frac{1000\text{ m}}{0.63\pi \text{ m}}\]
\end{minipage}

Therefore, Bill's wheel made
\begin{align*}
r_{\text{Bill}} - r_{\text{Ant}} &= \frac{1000}{0.7\pi} - \frac{1000}{0.63\pi} = \frac{1000}{\pi} \left(\frac{100}{63}-\frac{100}{70}\right)\\
&= \frac{100\;000}{7\pi}\left(\frac{1}{9}-\frac{1}{10}\right) = \frac{100\;000}{7\pi \times 90} = \frac{10\;000}{63\pi} \approx 50
\end{align*}
more revolutions than Antonino's.
%End

\vskip 1.5cm

%Begin
%Language English
%Source Cariboo College High School Mathematics Contest
%Title Final Round Part A 1978
%Question 8
%Subject geometry
%Category length
%Type MC
%Choices 5
%Answer B
%Creator Jane Lee
%Rdifficulty 28

%Qtext
\scriptsize
\begin{wrapfigure}[3]{r}[0pt]{0pt}
	\includegraphics[width=30mm]{CCFA78-08}
\end{wrapfigure}
Source: Cariboo College High School Mathematics Contest

\normalsize
In the diagram, the area of the figure $ABCDE$ equals the area of the square $FBCD$. If $ABG$ is an isosceles right triangle, then the length of $ED$ is:
\begin{enumerate}[noitemsep,topsep=0mm,leftmargin=*,widest=D,label=\Alph*)]
%ChoiceA
	\item $8\sqrt{2}-8$
%ChoiceB
	\item $16-8\sqrt{2}$
%ChoiceC
	\item 4
%ChoiceD
	\item $4\sqrt{2}$
%ChoiceE
	\item $4+\sqrt{2}/8$
\end{enumerate}

%Ftext
\textbf{The correct answer is (B): $\mathbf{16-8\sqrt{2}}$}

\begin{minipage}[b]{0.6\textwidth}
Since the areas of $ABCDE$ and $FBCD$ are equal, the areas of the two triangle $ABG$ and $EFG$ are equal and they are congruent. Let $x=\overline{FG}=\overline{EF}$. Then $\overline{EG}=\overline{BG}=x\sqrt{2}$ by Pythagoras. Since $\overline{FG}+\overline{BG}=8$, we can get an equation to solve for $x$:
\begin{align*}
x\sqrt{2}+x &= 8\\
x(\sqrt{2}+1) &= 8
\end{align*}
\end{minipage}
\hfill
\begin{minipage}[t]{0.35\textwidth}
	\raggedleft
	\includegraphics[width=30mm]{CCFA78-08}
\end{minipage}
\[x = \frac{8}{\sqrt{2}+1} = \frac{8(\sqrt{2}-1)}{(\sqrt{2}+1)(\sqrt{2}-1)} = 8\sqrt{2}-8.\]
The length of $ED$ is then $8-x=16-8\sqrt{2}$.
%End

\vskip 1.5cm

%Begin
%Language English
%Source Cariboo College High School Mathematics Contest
%Title Final Round Part A 1978
%Question 9
%Subject logic
%Category general
%Type MC
%Choices 5
%Answer C
%Creator Jane Lee
%Rdifficulty 26

%Qtext
\scriptsize
Source: Cariboo College High School Mathematics Contest

\normalsize
In a row of 200 tiles we find that every 11$^{\text{th}}$ tile is red, and every 19$^{\text{th}}$ tile is green. The rest are white. If the row starts with 10 white tiles followed by a red tile, followed by 7 white tiles, followed by a green tile, and so on, the number of times that a red tile and a green tile are side by side is:
\begin{enumerate}[noitemsep,topsep=0mm,leftmargin=*,widest=D,label=\Alph*)]
%ChoiceA
	\item 0
%ChoiceB
	\item 1
%ChoiceC
	\item 2
%ChoiceD
	\item 3
%ChoiceE
	\item 4
\end{enumerate}

%Ftext
\textbf{The correct answer is (C): 2}

The following tiles will be red:

{\centering 11, 22, 33, 44, 55, 66, 77, 88, 99, 110, 121, 132, 143, 154, 165, 176, 187, 198 \par}

The following tiles will be green:

{\centering 19, 38, 57, 76, 95, 114, 133, 171, 190 \par}

Checking the number in the second list, we find that green tile 76 is next to red tile 77, and green tile 133 is next to red tile 132.
%End

\vskip 1.5cm

%Begin
%Language English
%Source Cariboo College High School Mathematics Contest
%Title Final Round Part A 1978
%Question 10
%Subject geometry
%Category length
%Type MC
%Choices 5
%Answer C
%Creator Jane Lee
%Rdifficulty 28

%Qtext
\scriptsize
Source: Cariboo College High School Mathematics Contest

\normalsize
\begin{wrapfigure}{r}[0pt]{0pt}
	\includegraphics[width=50mm]{CCFA78-10-1}
\end{wrapfigure}
In the diagram, the line through $A$ and $C$ is parallel to the line through the centre of the two concentric circles. If the distance between the lines is 1, the radius of the smaller circle is $\sqrt{2}$, and the radius of the larger circle is 2, then the ratio of the length of arc $AB$ to the length of arc $CD$ is:
\begin{enumerate}[noitemsep,topsep=0mm,leftmargin=*,widest=D,label=\Alph*)]
%ChoiceA
	\item $\sqrt{2}:1$
%ChoiceB
	\item $1:1$
%ChoiceC
	\item $3\sqrt{2}:4$
%ChoiceD
	\item $1:\sqrt{2}$
%ChoiceE
	\item $2:\sqrt{2}$
\end{enumerate}

%Ftext
\begin{wrapfigure}{r}[0pt]{0pt}
	\includegraphics[width=50mm]{CCFA78-10-2}
\end{wrapfigure}
\textbf{The correct answer is (C): $\mathbf{3\sqrt{2}:4}$}

Consider triangle $OAB$ in the diagram, where $O$ is the centre of the concentric circles. This is a right triangle with hypotenuse $\sqrt{2}$ and height $1$, so that $\angle BOA = 45^{\circ}$. The length of arc $AB$ is then
\[
\frac{45^{\circ}}{360^{\circ}} \times 2\pi\sqrt{2} = \frac{\pi\sqrt{2}}{4}.
\]
Similarly, consider triangle $OCD$, which has hypotenuse 2 and height 1, giving $\angle DOC = 30^{\circ}$. The length of arc $CD$ is
\[
\frac{30^{\circ}}{360^{\circ}} \times 2\pi(2) = \frac{\pi}{3}.
\]
The ratio of the length of arc $AB$ to the length of arc $CD$ is
\[
\frac{\pi\sqrt{2}}{4}:\frac{\pi}{3} \quad \Longrightarrow \quad 3\sqrt{2}:4.
\]
%End

\vskip 1.5cm

%Begin
%Language English
%Source Cariboo College High School Mathematics Contest
%Title Final Round Part B 1978
%Question 2a
%Subject algebra
%Category modelling
%Type SA
%Answer 1
%Creator Jane Lee
%Rdifficulty 26

%Qtext
\scriptsize
Adapted from: Cariboo College High School Mathematics Contest

\normalsize
This is a problem similar to ones found in old Babylonian texts:

A field yields 4 loads of grain per hectare while a second field yields 3 loads of grain per hectare. The yield of the first field was 12 loads more than the second. The area of the two fields together was 17 hectares. What is the difference in area between the fields, in hectares?

%Ftext
\textbf{The correct answer is 1}

Let $A_1$ = area of first field, $A_2$ = area of second field.

The first field yields 4 loads of grain per hectare, or $4A_1$ loads in total. The second field yields 3 loads of grain per hectare, or $3A_2$ loads in total. The first field yields 12 loads more than the second, so
\[
4A_1 - 3A_2 = 12.
\]
We also know that $A_1 + A_2 = 17$ from the question. Multiply this equation by 3 and add to the previous equation to get $7A_1 = 63$, or $A_1 = 9$. Therefore $A_2 = 17-9 = 8$, and the difference between the area of the two fields is 1 hectare.
%End

\vskip 1.5cm

%Begin
%Language English
%Source Cariboo College High School Mathematics Contest
%Title Final Round Part B 1978
%Question 2b
%Subject algebra
%Category modelling
%Type SA
%Answer 180
%Creator Jane Lee
%Rdifficulty 26

%Qtext
\scriptsize
Adapted from: Cariboo College High School Mathematics Contest

\normalsize
This is a problem similar to ones found in old Babylonian texts:

I have multiplied the length and width to obtain the area of a rectangle. Then I added to the area the excess of the length over the width and obtained 183 as a result. Moreover, I added the length and the width to get 27. Find the smallest possible area of the rectangle.

%Ftext
\textbf{The correct answer is 180}

Let $l$ = width of rectangle, $w$ = width of rectangle.

We are given that the area plus the excess of the length over the width equals 183, and that the length plus width is 27:

\begin{minipage}{0.5\textwidth}
	\centering
	$lw+(l-w)=183$
\end{minipage}
\begin{minipage}{0.49\textwidth}
	\centering
	$l+w=27$
\end{minipage}

Combining the equations and solving for $l$, we get
\begin{align*}
l(27-l)+[l-(27-l)] &= 183\\
l^2-29l+210 &= 0\\
(l-15)(l-14) &= 0,
\end{align*}
implying that either $l=15$ and $w=12$, or $l=14$ and $w=13$. The first case gives an area of 180; the second case gives an area of 182. Hence the smallest possible area of the rectangle is 180.
%End

\vskip 1.5cm

%Begin
%Language English
%Source Cariboo College High School Mathematics Contest
%Title Final Round Part B 1978
%Question 3b
%Subject arithmetic
%Category proportions
%Type SA
%Answer 31
%Creator Jane Lee
%Rdifficulty 30

%Qtext
\scriptsize
Source: Cariboo College High School Mathematics Contest

\normalsize
Imagine 1000 lockers, all closed, and 1000 students. Suppose the first student goes along and opens every locker. Then the second student goes along and shuts every second locker. Then the third student goes along and changes the state of every third locker (ie. if it is open, she shuts it, and vice versa). Then the fourth student goes along and changes the state of every fourth locker, etc. until all 1000 students have passed by all lockers.

How many lockers will be open at the end?

%Ftext
\textbf{The correct answer is 31}

A locker will be open only if its state is changed an odd number of times. If $x$ is the number of lockers, then the state of a locker will change once for every divisor of $x$ between 1 and $x$ itself, inclusive.

If $n$ is a divisor of $x$, then $x/n$ is also a divisor of $n$. Thus every divisor $x$ has another divisor $n/x$ paired with it, except where $x = n/x$. This only happens in perfect squares.

Thus every opening of a locker is eventually canceled by a closing, except in perfect square lockers. Since $31^2=963<1000$ and $32^2=1024>1000$, there are 31 perfect squares between 1 and 1000. Hence 31 lockers will be open after the $1000^\text{th}$ student goes by.
%End

\vskip 1.5cm

%Begin
%Language English
%Source Cariboo College High School Mathematics Contest
%Title Final Round Part B 1978
%Question 4a
%Subject geometry
%Category length
%Type SA
%Answer 109
%Creator Jane Lee
%Rdifficulty 28

%Qtext
\scriptsize
Adapted from: Cariboo College High School Mathematics Contest

\normalsize
The distance between the parallel lines $m$ and $l$ is 2, and the distance between the points $A$ and $B$ is 10 as shown. $B$ is midway between $m$ and $l$. A ball starts at $A$ and follows the shortest path from $A$ to $B$. This path touches $l$ and $m$ a total of $n$ times, not counting $A$. The diagram shows the case where $n=2$. Assume that the angles $x$ and $y$ formed at a collision are equal.

Find the \textbf{square} of the length of the path if $n=1$ (if the ball hits $m$ and then goes directly to $B$).

\begin{center}
	\includegraphics[width=80mm]{CCFB78-04-1}
\end{center}

%Ftext
\begin{wrapfigure}{r}[0pt]{0pt}
	\includegraphics[width=60mm]{CCFB78-04-2}
\end{wrapfigure}
\textbf{The correct answer is 109}

Consider the path $AB'$ shown. For $n=1$, we can imagine reflecting the path of the ball after it has bounced off $m$ across the line $m$, so that the traveled path of the ball has the same length as $AB'$. Note that $AB'$ is the hypotenuse of a right triangle with height 3 and base 10. Using Pythagoras, we can see that

\begin{center}
	(Length of path)$^2 = \overline{AB'}\,^2 = 3^2 + 10^2 = 109$.
\end{center}
%End

\vskip 1.5cm

%Begin
%Language English
%Source Cariboo College High School Mathematics Contest
%Title Final Round Part B 1978
%Question 4b
%Subject geometry
%Category length
%Type SA
%Answer 125
%Creator Jane Lee
%Rdifficulty 29

%Qtext
\scriptsize
Adapted from: Cariboo College High School Mathematics Contest

\normalsize
The distance between the parallel lines $m$ and $l$ is 2, and the distance between the points $A$ and $B$ is 10 as shown. $B$ is midway between $m$ and $l$. A ball starts at $A$ and follows the shortest path from $A$ to $B$. This path touches $l$ and $m$ a total of $n$ times, not counting $A$. Assume that the angles $x$ and $y$ formed at a collision are equal.

Find the \textbf{square} of the length of the path if $n=2$ (if the ball hits $m$ and then goes directly to $B$).

\begin{center}
	\includegraphics[width=80mm]{CCFB78-04-1}
\end{center}

%Ftext
\textbf{The correct answer is 125}

\begin{wrapfigure}{r}[0pt]{0pt}
	\includegraphics[width=50mm]{CCFB78-04-3}
\end{wrapfigure}
Consider the path $AB'$ shown. For $n=2$ there are three segments to the ball's path. We can imagine reflecting the second segment across the line $m$, then translating the last segment up above $m'$. This produces the straight line $AB'$, which is now the hypotenuse of a right triangle with height 5 and base 10. Using Pythagoras, we can see that
\begin{align*}
	\text{(Length of path)}^2 &= \overline{AB'}\,^2\\
	&= 5^2 + 10^2 = 125.
\end{align*}
%End

%~~~~~~~~~~~

\end{document}
