\documentclass[letterpaper, 12pt]{article}

\usepackage[francais]{babel}

\usepackage{amsmath,amsfonts,amsthm,amssymb, graphicx,wasysym,multirow,enumerate}
\usepackage[latin1]{inputenc}
%\usepackage[ansinew]{inputenc}

\pagestyle{plain}

\setlength{\topmargin}{-2cm}
\setlength{\textheight}{23.5cm}
\setlength{\textwidth}{18cm}
\setlength{\oddsidemargin}{-1cm}
\setlength{\parindent}{0pt}

\begin{document}

\section{Which of these are factors of 72 written in prime numbers?}

	\begin{enumerate}
		\item $2 \times 4 \times 9$
		\item $2 \times 6^{2}$
		\item $2 \times 6^{6}$
		\item $2^{3} \times 3^{2}$
	\end{enumerate}

Answer: d)\\

Explanation:
$2 \times 4 \times 9 = 72$, but 4 and 9 are not prime numbers.\\
$2 \times 6^{2} = 2 \times 6 \times 6 = 72$, but 6 is not a prime number.\\
$2 \times 6^{6} = 2 \times 6 \times 6 \times 6 \times 6 \times 6
\times 6 = 93\,312 \neq72$.  Again,\\ 6 is not a prime number.\\
Thus, the correct answer is d).\\

\section{}
Which of these are factors of 72 written in prime numbers?
	\begin{enumerate}
		\item $2 \times 4 \times 9$
		\item $2 \times 6^{2}$
		\item $2 \times 6^{6}$
		\item $2^{3} \times 3^{2}$
	\end{enumerate}

Answer: d)\\

Explanation:
$2 \times 4 \times 9 = 72$, but 4 and 9 are not prime numbers.\\
$2 \times 6^{2} = 2 \times 6 \times 6 = 72$, but 6 is not a prime number.\\
$2 \times 6^{6} = 2 \times 6 \times 6 \times 6 \times 6 \times 6
\times 6 = 93\,312 \neq72$.  Again,\\ 6 is not a prime number.\\
Thus, the correct answer is d).\\

\end{document}