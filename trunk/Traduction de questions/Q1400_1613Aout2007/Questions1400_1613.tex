\documentstyle[12pt]{article}
\topmargin -2cm
\textheight 23.5cm
\textwidth 6.5in
\oddsidemargin -1cm
\setlength{\parindent}{0pt}



\begin{document}

1400-- Quel est le prochain terme de la suite suivante?\\
$$1,4,9,16,25,\ldots$$\\
a) 36\\
b) 39\\
c) 44\\
d) 49\\

R\'eponse : a)\\

R\'etroaction :\\
On peut remarquer que l'on est en pr\'esence de la suite des
carr\'es parfaits :
\begin{eqnarray*}
1^2&=&1 \\ 2^2&=&4 \\ 3^2&=&9 \\ 4^2&=&16 \\ 5^2&=&25 \\ &\ldots&
\end{eqnarray*}
Le terme qui suit 25 est bien $6^2=36$. La r\'eponse est donc a).\\

1401-- Un escargot avance et l'on note \`a chaque minute le nombre de
centim\`etres qu'il a parcourus. Voici les valeurs que l'on a
recueillies :
\begin{eqnarray*}
temps \,(min)&\longrightarrow&cm \\
0&\longrightarrow&5 \\ 1&\longrightarrow&8 \\ 2&\longrightarrow&11
\\ 3 &\longrightarrow& 14 \\ 4&\longrightarrow&17 \\ &\ldots&
\end{eqnarray*}
Si $n$ repr\'esente le nombre de minutes \'ecoul\'ees, d\'etermine la
r\`egle qui donne la distance totale parcourue par l'escargot.\\
a) $n+3$\\
b) $2n+1$\\
c) $3n+5$\\
d) $5n+3$\\

R\'eponse : c)\\

R\'etroaction :\\
Tu dois regarder tous les termes de la suite. L'escargot commence
son parcours \`a 5 cm et ensuite, il avance de 3 cm \`a chaque
minute. La r\`egle qui repr\'esente la distance parcourue par
l'escargot est
donc $3n+5$. La r\'eponse est c).\\

1402-- Un chauffeur de taxi fait payer ses clients selon la m\'ethode
suivante : il compte $5\,\$$ comme montant de base et il ajoute
$0,50\,\$$ pour chaque kilom\`etre parcouru. D\'etermine la r\`egle
donnant le montant total de la facture si $x$ repr\'esente le nombre
de kilom\`etres parcourus et $y$ le co\^ut total du trajet.\\
a) $y=0,50x+5$\\
b) $y=5x+0,50$\\
c) $y=5,50x$\\
d) $y=5,50+x$\\

R\'eponse : a)\\

R\'etroaction :\\
Dans cette situation, le taux de variation du montant total \`a
payer par rapport au nombre de kilom\`etres parcourus est $0,50$,
car pour chaque kilom\`etre parcouru, on ajoute $0,50\,\$$ au
montant de base. L'ordonn\'ee \`a l'origine est 5 puisque c'est le
montant que l'on doit payer m\^eme si le taxi ne parcourt aucun
kilom\`etre. On
a donc la droite $y=0,50x+5$. La r\'eponse est a).\\

1403-- Simplifie l'expression suivante :
$${\LARGE \frac {a^{16}}{a^4}}.$$
a) $4$\\
b) $a^4$\\
c) $a^{12}$\\
d) $a^{20}$\\

R\'eponse : c)\\

R\'etroaction :\\
Voici les propri\'et\'es importantes des exposants :
\begin{eqnarray*}
x^a\cdot x^b&=&x^{a+b} \\ (x^a)^b&=&x^{a \cdot b} \\
\frac{x^a}{x^b}&=& x^{a-b}.
\end{eqnarray*}
Dans le cas pr\'esent, nous avons

$${\LARGE \frac {a^{16}}{a^4}}=a^{16-4}=a^{12}.$$\\
La r\'eponse est c).\\

1404-- Simplifie au maximum l'expression suivante :
$${\LARGE \frac {a^2\cdot a^4}{a^{-2}}}.$$\\
a) $a^{-4}$\\
b) $a^4$\\
c) $a^8$\\
d) $a^{10}$\\

R\'eponse : c)\\

R\'etroaction :\\
Voici les propri\'et\'es importantes des exposants :
\begin{eqnarray*}
x^a\cdot x^b&=&x^{a+b} \\ (x^a)^b&=&x^{a \cdot b} \\
\frac{x^a}{x^b}&=& x^{a-b}.
\end{eqnarray*}
On a donc

$${\LARGE \frac {a^2\cdot a^4}{a^{-2}}}={\LARGE \frac
{a^{2+4}}{a^{-2}}}={\LARGE \frac {a^6}{a^{-2}}}={\LARGE a^{(6-
-2)}=a^8}.$$\\
La r\'eponse est c).\\

1405-- Quel est le r\'esultat de cette expression?
$$3^2+2\times6-4\times3$$\\
a) 6\\
b) 9\\
c) 54\\
d) 189\\

R\'eponse : b)\\

R\'etroaction :\\
La loi sur la priorit\'e des op\'erations mentionne que nous devons
ex\'ecuter les calculs en respectant l'ordre suivant : \vskip 10pt
$1^{\circ}$ les parenth\`eses; \vskip 10pt $2^{\circ}$ les exposants
et les radicaux; \vskip 10pt $3^{\circ}$ les multiplications et les
divisions; \vskip 10pt $4^{\circ}$ les additions et les
soustractions. \vskip 25pt \noindent Selon ces lois, on doit
r\'esoudre cette \'equation ainsi :
\begin{eqnarray*}
3^2+2\times6-4\times3 & = & 3^2+(2\times6)-(4\times3) \\ & = &
9+12-12 \\ & = & 9.
\end{eqnarray*}\\
La r\'eponse est b).\\

1406-- Tu gagnes 1 000 000 \$ \`a la loterie. Cependant, tu dois
absolument r\'epondre correctement \`a la question math\'ematique
suivante. Que vaut
$$20-3\times2\times2-2-2\times2?$$ \\
a) 2\\
b) 8\\
c) 12\\
d) 128\\

R\'eponse : a)\\

R\'etroaction :\\
D\'esol\'e. Meilleure chance la prochaine fois. La r\'eponse \'etait
2. Voici la d\'emarche :
\begin{eqnarray*}
20-3\times2\times2-2-2\times2 & = & 20-((3\times2)\times2)-2-(2\times2) \\ &
= & 20-(6\times2)-2-4 \\ & = & 20-12-2-4 \\ & = & 2
\end{eqnarray*}
La r\'eponse est a).
Fais attention \`a la priorit\'e des op\'erations!\\

1407-- Laquelle des expressions suivantes est \'equivalente \`a
celle-ci :
$$6\times6\times6\times6?$$\\
a) $4\times6$\\
b) $4^6$\\
c) $6\times4$\\
d) $6^4$\\

R\'eponse : d)\\

R\'etroaction :\\
La r\'eponse est d). Les trois autres expressions ne sont pas
\'equivalentes \`a l'expression\\
$6\times6\times6\times6$.
\begin{eqnarray*}
6^4&=&6\times6\times6\times6={\textrm{1 296}} \\ 4^6 &=&
4\times4\times4\times4\times4\times4={\textrm{4 096}}\\  4\times6 &=&24 \\
6\times4&=&24
\end{eqnarray*}\\

1408-- Laquelle des expressions suivantes vaut 1? \\
a) $\sqrt2$\\
b) $5^0$\\
c) $a^6-a^5$\\
d) {\large $\frac{a^{16}}{a{^{15}}}$}\\

R\'eponse : b)\\

R\'etroaction :\\
$5^0=1$\\
La r\'eponse est donc b).\\


1409-- Laquelle des expressions suivantes repr\'esente le volume d'un
cube de
c\^ot\'e 4?\\
a) $3^4$\\
b) $4^3$\\
c) $4\times4$\\
d) $6\times4^2$\\

R\'eponse : b)\\

R\'etroaction :\\
Le volume d'un cube se calcule par la formule suivante :
$$V=c\times c\times c=c^3.$$
Dans le cas pr\'esent, nous avons
$$4\times4\times4=4^3=64.$$\\
La r\'eponse est donc b).\\

1410-- Place les parenth\`eses aux bons endroits afin que
l'\'egalit\'e suivante soit exacte :
$$36-3\div6+5=3.$$\\
a) $36-(3\div6)+5=3$\\
b) $36-((3\div6)+5)=3$\\
c) $((36-3)\div6)+5=3$\\
d) $(36-3)\div(6+5)=3$\\

R\'eponse : d)\\

R\'etroaction :\\
La loi sur la priorit\'e des op\'erations mentionne que nous devons
ex\'ecuter les calculs en respectant l'ordre suivant : \vskip 10pt
$1^{\circ}$ les parenth\`eses; \vskip 10pt $2^{\circ}$ les exposants
et les radicaux; \vskip 10pt $3^{\circ}$ les multiplications et les
divisions; \vskip 10pt $4^{\circ}$ les additions et les
soustractions. \vskip 25pt \noindent On a donc
\begin{eqnarray*}
(36-3)\div(6+5)&=&33\div11\\&=&3.
\end{eqnarray*}\\
La r\'eponse est d).\\

1411-- Tu organises un tournoi de dards dans ta rue et tu es
responsable de compter les points. Chaque \'equipe lance 6 dards et
celle qui obtient le plus de points remporte la partie. Voici le
r\'esultat des lancers : \vskip10pt L'\'equipe A : 3 fois le (-2), 2
fois le (+5) et 1 fois le (+2); \vskip 10pt L'\'equipe B : 3 fois le
(-3), 2 fois le (+6) et 1 fois le (+8); \vskip 10pt L'\'equipe C : 3
fois le (-2), 2 fois le (-1) et 1 fois le (+12); \vskip 10pt
L'\'equipe D : 6 fois le (-10).\\

Quelle \'equipe remportera la partie?\\
a) A\\
b) B\\
c) C\\
d) D\\

R\'eponse : b)\\

R\'etroaction :\\
On peut traduire les lancers de chaque \'equipe par les expressions
suivantes :

\begin{eqnarray*}
A&:& 3\cdot(-2)+2\cdot5+1\cdot(2)=6; \\
B&:& 3\cdot(-3)+2\cdot6+1\cdot8=11;\\
C&:& 3\cdot(-2)+2\cdot(-1)+1\cdot(12)=4;\\
D&:& 6\cdot(-10)=-60.\\
\end{eqnarray*}

On voit bien que l'\'equipe B remporte le tournoi. La r\'eponse est b).\\

1412-- Que valent les deux expressions suivantes? \vskip 10pt 1)
$-7+(-5)+7-(-3)-(+15)+(-4)^2$ \vskip 10pt
2)  $-4+(-4)+(-11)-(-21)+20-4^2$\\

a) -33 et 38\\
b) -33 et 6\\
c) -1 et 38\\
d) -1 et 6\\

R\'eponse : d)\\

R\'etroaction :\\
Ce probl\`eme comporte deux difficult\'es. Tout d'abord, la
soustraction d'un entier n\'egatif et ensuite, la mise au carr\'e
d'un entier n\'egatif. Voici le bon cheminement \`a faire :

\begin{eqnarray*}
1&:&-7+(-5)+7-(-3)-(+15)+(-4)^2 \\ &=& -7-5+7+3-15+16 \\ &=& -1.
\end{eqnarray*}

\begin{eqnarray*}
2&:&-4+(-4)+(-11)-(-21)+20-4^2 \\ &=& -4-4-11+21+20-16 \\ &=& 6.
\end{eqnarray*}

Les r\'eponses sont donc -1 et 6, c'est-\`a-dire d).\\

1413-- Exprime le nombre 14 700 000 000 en notation scientifique.\\
a) $0,147\times10^{-10}$\\
b) $1,47\times10^{-10}$\\
c) $0,147\times10^{10}$\\
d) $1,47\times10^{10}$\\

R\'eponse : d)\\

R\'etroaction :\\
Pour \'ecrire un nombre en notation scientifique, on doit d\'eplacer
la virgule de telle sorte qu'elle soit situ\'ee imm\'ediatement
apr\`es le premier chiffre de gauche non nul le constituant. Ce
faisant, il faut compter le nombre de positions duquel on a d\^u la
d\'eplacer. Dans le cas pr\'esent, on a d\'eplac\'e la virgule de 10
positions vers la gauche. On obtient par cons\'equent
$1,47\times10^{10}$, car pour passer de $1,47$ \`a 14 700 000 000,
on doit
multiplier par $10^{10}$. La r\'eponse est donc d).\\

1414-- Lequel de ces 4 nombres est le plus grand?\\
a) $2,58\times10^6$\\
b) $258\times10^3$\\
c) 258 000\\
d) ${\textrm{258 100 000}}\times10^{-3}$\\

R\'eponse : a)\\

R\'etroaction :\\La r\'eponse est a). Pour faciliter la comparaison
entre les quatre nombres, nous devons tous les ramener de la
notation scientifique \`a la forme d\'ecimale. \vskip 10pt \noindent
On obtient donc a) 2 580 000, b) 258 000, c) 258 000 et d) 258 100.
Il est maintenant beaucoup plus facile de voir que 2 580 000 est le
plus grand nombre. \\


1415-- Lors de ton repas de f\^ete, les g\^ateaux n'ont pas \'et\'e
coup\'es en parts \'egales. Il reste quatre morceaux qui
repr\'esentent exactement $\frac{2}{3}$, $\frac{5}{6}$,
$\frac{5}{18}$, et $\frac{7}{9}$ de ce qu'il y avait au d\'epart.
Place ces
fractions en ordre croissant pour t'aider \`a choisir le plus gros morceau
de g\^ateau.\\[3mm]
a) $\frac{2}{3}$,  $\frac{5}{6}$,  $\frac{5}{18}$,  $\frac{7}{9}$\\[3mm]
b) $\frac{5}{18}$,  $\frac{7}{9}$,   $\frac{2}{3}$,  $\frac{5}{6}$\\[3mm]
c) $\frac{5}{6}$,   $\frac{7}{9}$,  $\frac{2}{3}$,  $\frac{5}{18}$\\[3mm]
d) $\frac{5}{18}$,  $\frac{2}{3}$,   $\frac{7}{9}$,  $\frac{5}{6}$\\

R\'eponse : d)\\

R\'etroaction :\\
Pour comparer ces fractions, nous devons les mettre sur un
d\'enominateur commun qui est ici 18.
\begin{eqnarray*}
\frac{2}{3}&=&\frac{12}{18}\\[3mm]
\frac{5}{6}&=&\frac{15}{18}\\[3mm]
\frac{7}{9}&=&\frac{14}{18}\\[3mm]
\frac{5}{18}&=&\frac{5}{18}\\[3mm]
\end{eqnarray*}
En les mettant en ordre CROISSANT, c'est-\`a-dire de la plus petite
\`a la plus grande, on
obtient $\frac{5}{18}$,  $\frac{2}{3}$,  $\frac{7}{9}$,  $\frac{5}{6}$. La
r\'eponse est d).\\

1416-- Lors d'un concours de math\'ematiques, on te demande de placer
en ordre d\'ecroissant les nombres suivants : \vskip 10pt 1)
$65\,\%$ \vskip 10pt 2) $0,47$ \vskip 10pt 3) $2,9\times10^{-1}$
\vskip 10pt
4) $0,049\times10^{2}$\\

a) 1), 2), 3), 4) \\
b) 1), 4), 2), 3) \\
c) 3), 2), 4), 1) \\
d) 4), 1), 2), 3) \\

R\'eponse : d)\\

R\'etroaction :\\
Pour mieux comparer les nombres entre eux, on doit tous les mettre
sous la m\^eme forme. \vskip10pt \noindent
\begin{center}
$0,049\times10^{2}=4,9$\\
$65\,\%=0,65$\\
$2,9\times10^{-1}=0,29$\\
$0,47=0,47$\\
\end{center}
Il est maintenant beaucoup plus facile de d\'eterminer que la r\'eponse est
d).\\

1417-- Lequel des \'enonc\'es suivants est faux?\\
a) Dans tout triangle isoc\`ele, il y a toujours deux angles et deux
c\^ot\'es congrus.\\
b) Dans tout triangle \'equilat\'eral, chaque angle mesure $60^\circ$.\\
c) Dans tout triangle scal\`ene, on retrouve un angle obtus.\\
d) Dans tout triangle rectangle, il y a toujours deux angles
compl\'ementaires.\\

R\'eponse : c)\\

R\'etroaction :\\
Un angle obtus mesure plus de $90^\circ$. Or, on peut avoir un
triangle scal\`ene dont la somme des angles est $180^\circ$ et o\`u
chacun des angles est inf\'erieur \`a
$90^\circ$.\\
Ex. : $50^\circ, 60^\circ, 70^\circ.$\\
Par cons\'equent, la r\'eponse est c).\\

1418-- Parmi les \'enonc\'es suivants, lequels sont vrais?\\

1. Tous les triangles \'equilat\'eraux sont acutangles.\\
2. Tous les triangles scal\`enes sont acutangles.\\
3. Tous les triangles rectangles poss\`edent deux angles aigus et un angle
droit.\\
4. Tous les triangles isoc\`eles sont acutangles. \\
5. Tous les triangles obtusangles poss\`edent deux angles obtus et un angle
aigu.\\
6. Tous les triangles rectangles isoc\`eles poss\`edent un angle droit et
deux angles aigus.\\

a) 1, 3, 6\\
b) 1, 3, 4, 6\\
c) 1, 2, 3, 5, 6\\
d) 1, 2, 3, 4, 5, 6\\

R\'eponse : a)\\

R\'etroaction :\\
Un triangle acutangle est un triangle poss\'edant trois angles
aigus. Les seuls triangles \'etant toujours acutangles sont les
triangles \'equilat\'eraux. Les \'enonc\'es 2 et 4 sont donc faux.
La r\'eponse est a).\\

1419-- Dans un parall\'elogramme donn\'e, un angle mesure $60^\circ$.
Quel est la
mesure d'un de ses deux angles adjacents?\\
a) $60^\circ$\\
b) $90^\circ$\\
c) $120^\circ$\\
d) $180^\circ$\\

R\'eponse : c)\\

R\'etroaction :\\
La somme des angles int\'erieurs d'un quadrilat\`ere est de
$360^\circ$. De plus, il y a toujours deux paires d'angles congrus
dans un parall\'elogramme. Ainsi,
$360^\circ-2\cdot60^\circ=240^\circ$. On doit diviser ce $240^\circ$
en deux
parties \'egales. L'angle vaut donc $120^\circ$ et la r\'eponse est c).\\

1420-- Quelle(s) figure(s) poss\`ede(nt) les propri\'et\'es suivantes?\\

- 4 c\^ot\'es congrus;\\
- 2 paires de c\^ot\'es parall\`eles;\\
- des angles oppos\'es congrus;\\
- des diagonales perpendiculaires l'une \`a l'autre se coupant en
leur milieu.\\

a) Losange\\
b) Losange et carr\'e\\
c) Losange, carr\'e et rectangle\\
d) Losange, carr\'e, rectangle et parall\'elogramme\\

R\'eponse : b)\\

R\'etroaction :\\
Seuls les losanges et les carr\'es respectent toutes ces conditions.
Les rectangles n'ont pas de diagonales perpendiculaires. Les
\'enonc\'es c) et
d) sont donc faux et la r\'eponse est b).\\

1421-- Votre piscine municipale est de forme rectangulaire. Sachant
que le p\'erim\`etre de cette piscine est de 75 m et que la mesure
d'un c\^ot\'e est
de 25 m, quelle est la mesure d'un de ses c\^ot\'es adjacents?\\
a) 12,5 m\\
b) 25 m \\
c) 50 m\\
d) 100 m\\

R\'eponse : a)\\

R\'etroaction :\\
La formule du p\'erim\`etre est  $P=2\cdot(L+l)$. On a donc
\begin{eqnarray*}
2\cdot(25+x)&=&75 \\ (25+x)&=&37,5 \\ x&=&12,5.
\end{eqnarray*}
Le c\^ot\'e mesure donc 12,5 m et la r\'eponse est a).\\


1422-- Combien de diagonales issues d'un sommet y a-t-il dans un octogone?\\
a) 5\\
b) 6\\
c) 7\\
d) 8\\

R\'eponse : a)\\

R\'etroaction :\\
Par d\'efinition, une diagonale est un segment de droite reliant
deux sommets non cons\'ecutifs. De n'importe quel sommet de
l'octogone sont toujours
issues cinq diagonales. La r\'eponse est a).\\

1423-- Quelle est la mesure de chacun des angles int\'erieurs d'un hexagone?\\
a) $60^\circ$\\
b) $90^\circ$\\
c) $120^\circ$\\
d) $180^\circ$\\

R\'eponse : c)\\

R\'etroaction :\\
La formule pour calculer la somme des angles int\'erieurs d'un
polygone est
$$(n-2)\times180^\circ,$$ o\`u $n$ est le nombre de c\^ot\'es du polygone.
Dans notre cas, nous avons $(6-2)\times180^\circ=720^\circ$ \`a
diviser en six angles identiques, ce qui donne $120^\circ$ pour chacun des
angles. La r\'eponse est donc c).\\

1424-- Tu ach\`etes cinq cahiers \`a 3\,\$ chacun et deux stylos \`a
4\,\$ chacun.
Si tu payes avec un billet de 50\,\$, quel montant d'argent te sera remis?\\
a) $23\,\$$\\
b) $27\,\$$\\
c) $28\,\$$\\
d) $43\,\$$\\

R\'eponse : b)\\

R\'etroaction :\\
Il s'agit de r\'esoudre l'\'equation $50-(5\times3+2\times4)$. La r\'eponse
est $27\,\$$, ce qui veut dire b).\\

1425--Un professeur veut distribuer des bonbons \`a ses \'el\`eves. Il
poss\`ede six paquets de huit bonbons, 11 paquets de 13 bonbons et
13 paquets de 18 bonbons. S'il y a 19 gar\c cons et six filles dans
sa classe, calcule le
nombre de bonbons que chacun recevra.\\
a) 17\\
b) 30\\
c) 39\\
d) 425\\

R\'eponse : a)\\

R\'etroaction :\\
Il y a au total $(6\times8+11\times13+13\times18)$ bonbons pour
$(16+9)$ \'el\`eves. Pour d\'eterminer combien chacun en recevra, il
suffit de diviser le nombre de bonbons par le nombre d'\'el\`eves.
$$\frac{6\times8+11\times13+13\times18}{16+9}=\frac{425}{25}=17$$
La r\'eponse est a).\\

1426-- Combien y a-t-il de nombres premiers de 2 \`a 25?\\
a) 7\\
b) 8\\
c) 9\\
d) 10 et plus\\

R\'eponse : c)\\

R\'etroaction :\\
Un nombre premier est un nombre qui se divise seulement par 1 et par
lui-m\^eme. De 2 \`a 25, il y a 9 nombres premiers, soit 2, 3, 5, 7,
11, 13, 17, 19 et 23. La r\'eponse est donc c).\\

1427-- Dans une maison de campagne, il y a deux horloges. La
premi\`ere sonne toutes les 12 minutes et la seconde toutes les 15
minutes. Si les horloges sonnent en m\^eme temps \`a 6h00 du matin,
\`a quelle heure sonneront-elles de nouveau simultan\'ement?\\
a) 6h30\\
b) 7h00\\
c) 8h00\\
d) 8h24\\

R\'eponse : b)\\

R\'etroaction :\\
Nous devons chercher le plus petit commun multiple (PPCM) de 12 et
de 15. Ce dernier valant 60, les horloges sonneront en m\^eme
temps 60 minutes plus tard que 6h00 du matin, c'est-\`a-dire \`a 7h00. La
r\'eponse est b).\\

1428-- Pour l'Halloween, Alex veut pr\'eparer le plus grand nombre de
sacs identiques avec 120 chocolats et 180 bonbons. Combien de sacs
semblables pourra-t-il
remplir?\\
a) 40\\
b) 60\\
c) 80\\
d) 120\\

R\'eponse : b)\\

R\'etroaction :\\
On doit trouver le plus grand commun diviseur (PGCD) de 120 et 180.
Comme ce dernier vaut 60, il y aura en tout 60 sacs contenant 2
barres de chocolat et 3 bonbons. La r\'eponse est donc b).\\

1429-- Quel est le prochain terme de la suite suivante?
$${\textrm{2, 6, 18, 54,}}
\ldots$$\\
a) 90\\
b) 108\\
c) 162\\
d) 216\\

R\'eponse : c)\\

R\'etroaction :\\
Il faut trouver la r\`egle de la suite. Ici, d'un terme \`a l'autre,
on multiplie par trois. Le terme suivant 54 est donc
$54\times3=162$. La r\'eponse est c).\\

1430-- Lequel des nombres suivants n'est pas un entier?\\
a) $-7$\\[3mm]
b) $\frac{-26}{2}$\\[3mm]
c) $\left(\frac{1}{3}\right)^0$\\[3mm]
d) $\frac{13}{5}$\\

R\'eponse : d)\\

R\'etroaction :\\
$\frac{-26}{2}$ = $-13$ est un entier.\\[3mm]
$\left(\frac{1}{3}\right)^0\,=\,1$ est un entier.\\[3mm]
$\frac{13}{5}$ = $2,6$ n'est pas un entier.\\[3mm]
La r\'eponse est donc d).\\

1431-- Laquelle des op\'erations suivantes ne donne pas huit?\\
a) $\sqrt{64}$\\
b) $-2\times-4$\\
c) $(-2)^3$\\
d) $(-\sqrt8)^2$\\

R\'eponse : c)\\

R\'etroaction :\\
Comme $(-2)^3=-2\times-2\times-2=-8$, la r\'eponse est c).\\

1432-- Sur la plan\`ete V\'enus, il fait  $35$ degr\'es v\'enusiens le
jour et $-50$ degr\'es v\'enusiens la
nuit. Quel est l'\'ecart entre la temp\'erature du jour et celle de la
nuit?\\
a) $15$ degr\'es v\'enusiens\\
b) $85$ degr\'es v\'enusiens\\
c) $-15$ degr\'es v\'enusiens\\
d) $25$ degr\'es v\'enusiens\\

R\'eponse : b)\\

R\'etroaction :\\
L'\'ecart est la diff\'erence entre les deux temp\'eratures. On
effectue donc la soustraction suivante : $35-(-50)=35+50=85$
degr\'es v\'enusiens. La r\'eponse est b).\\

1433-- Platon est n\'e en 428 av. J.-C. Quelle est l'ann\'ee de son
d\'ec\`es s'il est mort \`a 80 ans?\\
a) 508 av. J.-C.\\
b) 348 av. J.-C.\\
c) 348 apr. J.-C.\\
d) 508 apr. J.-C.\\

R\'eponse : b)\\

R\'etroaction :\\
Platon est n\'e en -428. Pour trouver l'ann\'ee de son d\'ec\`es, on
doit ajouter 80 ans \`a -428. On a donc $$-428+80=-348.$$ L'ann\'ee
-348 signifie l'an 348 av. J.-C. La r\'eponse est donc
b).\\

1434-- Lequel des groupes suivants est compos\'e uniquement de
fractions
\'equivalentes \`a $\frac{1}{3}$?\\[3mm]
a) $\frac{2}{6}$,  $\frac{3}{12}$,  $\frac{6}{18}$,  $\frac{10}{30}$\\[3mm]
b) $\frac{1}{2}$,  $\frac{2}{5}$,   $\frac{7}{21}$,  $\frac{8}{24}$\\[3mm]
c) $\frac{3}{9}$,   $\frac{15}{45}$,  $\frac{51}{153}$,
$\frac{75}{225}$\\[3mm]
d) $\frac{8}{24}$,  $\frac{11}{33}$,   $\frac{18}{48}$,  $\frac{16}{67}$\\

R\'eponse : c)\\

R\'etroaction :\\
Seule la r\'eponse c) contient uniquement des fractions \'equivalentes \`a
$\frac{1}{3}$. En a), $\frac{3}{12}$ = $\frac{1}{4}$. En b)
$\frac{2}{5}\neq\frac{1}{3}$ et en d), $\frac{16}{67}\neq\frac{1}{3}$.\\

1435-- Laquelle des fractions suivantes est irr\'eductible?\\[3mm]
a) $\frac{7}{35}$\\[3mm]
b) $\frac{17}{51}$\\[3mm]
c) $\frac{5}{10}$\\[3mm]
d) $\frac{2}{9}$\\

R\'eponse : d)\\

R\'etroaction :\\
En a), $\frac{7}{35}=\frac{1}{5}$. En b), $\frac{17}{51}=\frac{1}{3}$ et en
c), $\frac{5}{10}=\frac{1}{2}$. La r\'eponse est donc d).\\

1436-- Laquelle des propositions suivantes est vraie?\\
a) $\frac{1}{2}\leq\frac{1}{3}\leq\frac{1}{4}\leq\frac{1}{5}$\\[3mm]
b) $\frac{2}{3}\leq\frac{3}{3}\leq\frac{4}{3}\leq\frac{5}{3}$\\[3mm]
c) $\frac{1}{27}\leq\frac{1}{9}\leq\frac{1}{4}\leq\frac{1}{5}$\\[3mm]
d) $\frac{3}{25}\leq\frac{3}{30}\leq\frac{2}{9}\leq\frac{1}{3}$\\[3mm]

R\'eponse : b)\\

R\'etroaction :\\
Lorsque le d\'enominateur ne change pas et que le num\'erateur
augmente, la fraction devient de plus en plus grande. Voici un petit
truc pour t'aider \`a comparer les fractions ensemble : mets-les sur
un d\'enominateur commun. La r\'eponse est donc la
proposition b).\\

1437-- Laquelle des op\'erations suivantes est fausse?\\[3mm]
a) $\frac{1}{3}+\frac{1}{4}=\frac{7}{12}$\\[3mm]
b) $\frac{5}{3}+\frac{10}{3}=5$\\[3mm]
c) $\frac{3}{7}+\frac{9}{3}=\frac{12}{10}$\\[3mm]
d) $\frac{1}{5}+\frac{4}{5}=1$\\

R\'eponse : c)\\

R\'etroaction :\\
Pour additionner ou soustraire des fractions, tu dois ABSOLUMENT
mettre ces fractions sur un d\'enominateur commun. Donc
$\frac{3}{7}+\frac{9}{3}=\frac{9}{21}+\frac{63}{21}=\frac{72}{21}=\frac{24}{7}\neq\frac{12}{10}$.\\
L'\'enonc\'e c) est faux.\\

1438-- Donne le r\'esultat des op\'erations suivantes:\\
$$\frac{1}{2}+\frac{3}{4}\times\frac{4}{5}-\frac{2}{3}.$$\\[3mm]
a) $\frac{13}{30}$\\[3mm]
b) $\frac{14}{27}$\\[3mm]
c) $\frac{11}{19}$\\[3mm]
d) $\frac{20}{17}$\\[3mm]

R\'eponse : a)\\

R\'etroaction :\\
Nous devons tenir compte de la priorit\'e des op\'erations. Il faut
commencer par la multiplication, puis mettre les fractions sur leur
d\'enominateur commun pour pouvoir effectuer les additions et les
soustractions.\\
\begin{eqnarray*}
\frac{1}{2}+\frac{3}{4}\times\frac{4}{5}-\frac{2}{3}&=&\frac{1}{2}+\frac{12}{20}-\frac{2}{3}\\\\
&=&\frac{15}{30}+\frac{18}{30}-\frac{20}{30}\\\\
&=&\frac{13}{30}
\end{eqnarray*}\\
La r\'eponse est a).\\

1439-- Un groupe de touristes vient visiter Qu\'ebec. Les
$\frac{5}{8}$ sont des femmes. Les $\frac{4}{5}$ de ces femmes
parlent anglais et les $\frac{3}{4}$ de ces derni\`eres visitent la
r\'egion pour la premi\`ere fois. Les $\frac{5}{6}$ des hommes
parlent anglais et les $\frac{2}{5}$ de ces derniers ont d\'ej\`a
visit\'e la r\'egion. Trouve la fraction du groupe repr\'esentant
les femmes parlant anglais ayant
d\'ej\`a visit\'e la r\'egion.\\[3mm]
a) $\frac{1}{8}$\\[3mm]
b) $\frac{3}{8}$\\[3mm]
c) $\frac{3}{32}$\\[3mm]
d) $\frac{67}{40}$\\[3mm]

R\'eponse : a)\\

R\'etroaction :\\
On doit faire ressortir les fractions qui repr\'esentent ce que l'on
cherche. On veut le nombre de femmes (une proportion de
$\frac{5}{8}$ du groupe) parlant anglais ($\frac{4}{5}$ des femmes)
et ayant d\'ej\`a visit\'e la r\'egion (c'est-\`a-dire
$\frac{1}{4}$). Nous devons multiplier ces fractions comme suit :\\[3mm]
$\frac{5}{8}\times\frac{4}{5}\times\frac{1}{4} = \frac{20}{160}$ =
$\frac{1}{8}$.\\[3mm] La r\'eponse est donc a).\\


1440-- Comment \'ecrit-on le nombre cent trente-deux dix-milli\`emes?\\
a) 132\\
b) 0,132\\
c) 0,013 2\\
d) 0,001 32\\

R\'eponse : c)\\

R\'etroaction :\\
Les positions apr\`es la virgule repr\'esentent respectivement les
valeurs suivantes : dixi\`eme, centi\`eme, milli\`eme,
dix-milli\`eme, ... Dans le cas pr\'esent, le $2$ \'etant le dernier
chiffre, il doit \^etre plac\'e \`a la position correspondant aux
dix-milli\`emes. On obtient donc
${\textrm{0,013 2}}$. Par cons\'equent, la r\'eponse est c).\\

1441-- Arrondis le nombre suivant au centi\`eme pr\`es : $${\textrm{346,287
54.}}$$\\
a) 300\\
b) 346,28\\
c) 346,290\\
d) 346,29\\

R\'eponse : d)\\

R\'etroaction :\\
Lorsque l'on arrondit, le dernier chiffre que l'on \'ecrit est celui
dont la position correspond \`a la valeur \`a laquelle on veut
arrondir. Dans le cas pr\'esent, il s'agit du chiffre des
centi\`emes; on gardera donc deux chiffres apr\`es la virgule.
$$346,28{\Bigg{\vert}}754$$ Si le chiffre qui suit est sup\'erieur
ou \'egal \`a 5, on arrondit vers le haut. S'il est inf\'erieur \`a 5, on
arrondit vers le bas. On obtient 346,29. La r\'eponse est d).\\

1442-- Parmi ces \'enonc\'es, lequel est vrai?\\[3mm]
a) $0,46\le\frac{46}{100}<0,086<1,2$\\[3mm]
b) $0,22<0,3<\frac{51}{100}<\frac{51}{10}$\\[3mm]
c) $1,3<\frac{42}{20}\le2,10<\frac{8}{6}$\\[3mm]
d) $0,022<0,202<0,22<0,2$\\

R\'eponse : b)\\

R\'etroaction :\\
Seul l'\'enonc\'e b) est vrai. La m\'ethode la plus facile pour
comparer les valeurs entre elles est de transformer les fractions en
nombres d\'ecimaux. En b), nous avons : $$0,22<0,3<0,51<5,1.$$\\La
r\'eponse est b).\\

1443-- Effectue la multiplication suivante : $$24,27\times7,3.$$\\
a) 31,57\\
b) 177,171\\
c) 1 771,71\\
d) 177 171\\

R\'eponse : b)\\

R\'etroaction :\\
Pour effectuer des multiplications de nombres d\'ecimaux, il faut
proc\'eder sans tenir compte des virgules. On obtient ainsi
${\textrm{177 171}}$. Par la suite, nous devons positionner la
virgule au bon endroit. Pour ce faire, il faut la placer dans la
r\'eponse de telle sorte qu'elle soit suivie du nombre total de
chiffres suivant la virgule dans les deux nombres \`a multiplier.
Ainsi, comme il y a en tout trois chiffres apr\`es la virgule dans
les deux nombres, la r\'eponse
est $177,171$, c'est-\`a-dire b).\\

1444-- Donne le r\'esultat de la division suivante : $$163,22\div9.$$\\
a) $18,14$\\
b) $27,20$\\
c) $154,22$\\
d) ${\textrm{1 468, 98}}$\\

R\'eponse : a)\\

R\'etroaction :\\
On effectue la division \`a la main exactement comme avec des entiers, mais
lorsqu'on rencontre la virgule, on doit la placer dans la r\'eponse. Ainsi,
on obtient 18,14. La r\'eponse est donc a).\\

1445-- Un \'epicier vend cinq oranges pour $2,15\,\$$ et une douzaine
de ces m\^emes oranges pour $4,68\,\$$. Quel achat est le plus
avantageux pour le client?\\
a) 5 oranges pour $2,15\,\$.$\\
b) Une douzaine pour $4,68\,\$.$\\
c) Les deux sont au m\^eme prix.\\
d) Aucun des deux achats car c'est trop cher pour des oranges!\\

R\'eponse : b)\\

R\'etroaction :\\
Nous devons calculer le prix unitaire des oranges pour chacune des
situations. \vskip10pt Premi\`ere situation :
$\frac{2,15\,\$}{5}=0,43\,\$$/orange \vskip 10pt Deuxi\`eme
situation : $\frac{4,68\,\$}{12}=0,39\,\$$/orange. \vskip 20pt
\noindent Il est maintenant tr\`es facile de voir que la douzaine
d'oranges revient \`a un prix
plus avantageux pour le client que l'achat du paquet de 5 oranges. La
r\'eponse est b).\\

1446-- Marie a obtenu $\frac{18}{30}$ \`a son examen. Quel est son
r\'esultat en
pourcentage?\\
a) 15\,\% \\
b) 50\,\% \\
c) 60\,\% \\
d) 75\,\% \\

R\'eponse : c)\\

R\'etroaction :\\
On doit effectuer la division pour obtenir la fraction en nombre
d\'ecimal. On multiplie ensuite par 100 pour obtenir la
note de Marie en pourcentage. \\

$\frac{16}{30}$$\,\times100 = 0,6\,\times\,100=60\,\%.$\\[3mm]
La r\'eponse est c).\\

1447-- Luc n'a pas obtenu de tr\`es bons r\'esultats en
math\'ematiques et il veut \'eviter d'avoir \`a suivre un cours
d'\'et\'e. Son professeur lui mentionne qu'il doit avoir 70\,\% \`a
l'examen final s'il veut r\'eussir son cours. L'examen est not\'e
sur 40 points. Quel r\'esultat Luc doit-il obtenir sur 40 pour
\'eviter le cours
d'\'et\'e? \\
a) 26\\
b) 28\\
c) 30\\
d) 70\\

R\'eponse : b)\\

R\'etroaction :\\
On doit trouver la valeur qui compl\`ete l'\'egalit\'e suivante
:$$70\,\%=\frac{???}{40}.$$\\

On cherche donc 70\,\% de 40. Le calcul est
simple :$$\frac{70}{100}\times40.$$On obtient $\frac{28}{40}$. La r\'eponse
est alors b).\\

1448-- Une compagnie emploie 120 personnes, dont 40\,\% sont
mari\'ees, 8\,\% veuves et 12\,\% divorc\'ees. Combien d'employ\'es
ne font pas partie de ces cat\'egories?\\
a) 48 employ\'es\\
b) 60 employ\'es\\
c) 72 employ\'es\\
d) 84 employ\'es\\

R\'eponse : a)\\

R\'etroaction :\\
On doit chercher le pourcentage d'employ\'es ne faisant pas partie
de ces trois
cat\'egories.$$100\,\%-(40\,\%+8\,\%+12\,\%)=40\,\%.$$\\On cherche
donc 40\,\% de 120 employ\'es :
$$\frac{40}{100}\times120=48 {\textrm{ employ\'es.}}$$ La r\'eponse est
a).\\

1449-- Lors d'un sondage estimant le nombre de fumeurs dans une
\'ecole secondaire, on interroge 300 personnes. Si le r\'esultat
affirme que 25\,\% des \'etudiants fument, combien y a-t-il
exactement de fumeurs dans
l'\'echantillon de 300 personnes?\\
a) 25 fumeurs\\
b) 75 fumeurs\\
c) 150 fumeurs\\
d) 225 fumeurs\\

R\'eponse : b)\\

R\'etroaction :\\
On cherche $25\,\%$ de $300$ personnes. Le calcul est le suivant :
$$\frac{25}{100}\,\times\,300\,=\,75 {\textrm{ personnes.}}$$ La r\'eponse
est b).\\

1450-- Laquelle des transformations suivantes utilise un axe de sym\'etrie?\\
a) Homoth\'etie\\
b) R\'eflexion\\
c) Rotation\\
d) Translation\\

R\'eponse : b)\\

R\'etroaction :\\
La r\'eflexion est la seule transformation \`a utiliser un axe de
sym\'etrie. La r\'eponse est b).\\

1451-- Laquelle des affirmations suivantes est fausse?\\
a) Un triangle isoc\`ele poss\`ede 3 axes de sym\'etrie.\\
b) Un triangle rectangle poss\`ede un angle droit et deux angles
compl\'ementaires.\\
c) Un triangle \'equilat\'eral est toujours acutangle.\\
d) Un triangle scal\`ene ne poss\`ede jamais d'angles congrus.\\

R\'eponse : a)\\

R\'etroaction :\\
La r\'eponse a) est fausse, car un triangle isoc\`ele poss\`ede un seul axe
de sym\'etrie. Les triangles \'equilat\'eraux en poss\`edent trois. Les
trois autres \'enonc\'es sont vrais.\\

1452-- Si un enseignant te mentionne qu'un des c\^ot\'es d'un triangle
donn\'e mesure 2 cm et qu'un autre de ses c\^ot\'es en mesure trois,
que
peux-tu conclure sur ce triangle?\\
a) Il est isoc\`ele.\\
b) La mesure de son dernier c\^ot\'e est inf\'erieure \`a 5.\\
c) La mesure de son dernier c\^ot\'e est sup\'erieure \`a 5.\\
d) On ne peut rien conclure avec ces informations.\\

R\'eponse : b)\\

R\'etroaction :\\
On sait par l'in\'egalit\'e triangulaire que la mesure d'un c\^ot\'e
est toujours inf\'erieure \`a la somme des mesures des deux autres
c\^ot\'es. L'\'enonc\'e b) est donc
vrai.\\

1453-- Lequel des \'enonc\'es suivants est faux?\\
a) Dans un triangle isoc\`ele, la hauteur, la m\'ediane et la
m\'ediatrice co\"incident.\\
b) La hauteur d'un triangle passe par un sommet et descend
perpendiculairement
au c\^ot\'e oppos\'e ou \`a son prolongement.\\
c) Selon la forme du triangle, la hauteur peut passer \`a
l'ext\'erieur de ce dernier.\\
d) La m\'ediatrice passe par un sommet et par le point milieu du
c\^ot\'e
oppos\'e.\\

R\'eponse : d)\\

R\'etroaction :\\
L'\'enonc\'e d) est faux puisqu'il s'agit de la d\'efinition de la
m\'ediane et non de la
m\'ediatrice. La r\'eponse est donc d).\\

1454-- Je suis une droite issue d'un sommet et perpendiculaire au
c\^ot\'e
oppos\'e. Qui suis-je?\\
a) Bissectrice\\
b) Hauteur \\
c) M\'ediane\\
d) M\'ediatrice\\

R\'eponse : b)\\

R\'etroaction :\\
La hauteur est bien une droite issue d'un sommet et qui croise le
c\^ot\'e
oppos\'e en formant un angle droit. La r\'eponse est b).\\

1455-- Quel est le nom de la droite passant par le milieu d'un
c\^ot\'e et perpendiculaire \`a ce c\^ot\'e?\\
a) Hauteur\\
b) M\'ediane\\
c) M\'ediatrice\\
d) Bissectrice\\

R\'eponse : c)\\

R\'etroaction :\\
La m\'ediatrice d'un segment n'est pas n\'ecessairement issue d'un
sommet. C'est une droite passant par le point milieu d'un c\^ot\'e
et perpendiculaire \`a ce
c\^ot\'e. La r\'eponse est c).\\

1456-- Soit un quadrilat\`ere donn\'e. Si l'angle $1$ est de
$75^{\circ}$, l'angle $2$ de $55^{\circ}$ et l'angle $3$ de
$15^{\circ}$, quelle est la
mesure de l'angle $4$?\\
a) $35^{\circ}$\\
b) $115^{\circ}$\\
c) $215^{\circ}$\\
d) $360^{\circ}$\\

R\'eponse : c)\\

R\'etroaction :\\
La somme des angles int\'erieurs d'un quadrilat\`ere vaut
$360^{\circ}$. On a donc
$$360^{\circ}-(75^{\circ}+15^{\circ}+55^{\circ})=215^{\circ}.$$\\
La r\'eponse est donc c).\\

1457-- Lequel des \'enonc\'es suivants est faux?\\
a) La somme des mesures de deux angles cons\'ecutifs d'un
parall\'elogramme est $180^{\circ}$.\\
b) Les c\^ot\'es oppos\'es d'un parall\'elogramme sont parall\`eles.\\
c) Tout losange est un carr\'e.\\
d) Tout parall\'elogramme est un trap\`eze.\\

R\'eponse : c)\\

R\'etroaction :\\
Le propre d'un losange est de poss\'eder des c\^ot\'es ayant tous la
m\^eme longueur. Le carr\'e est donc un losange. Cependant, il n'est
pas vrai que tout losange est
un carr\'e, puisque les angles d'un losange ne sont pas n\'ecessairement
droits, ce qui est le cas du carr\'e. La r\'eponse est c).\\

1458-- Mes diagonales (ou nos diagonales) sont perpendiculaires et congrues.
Qui suis-je ou qui sommes-nous?\\
a) Carr\'e\\
b) Carr\'e et losange\\
c) Carr\'e, losange et rectangle\\
d) Carr\'e, losange et trap\`eze\\

R\'eponse : a)\\

R\'etroaction :\\
Un losange a des diagonales perpendiculaires, mais non congrues. La
r\'eponse est donc a).\\

1459-- Lesquels des \'enonc\'es suivants se rapportent au calcul d'une aire?\\

1. Recouvrir un terrain de gazon;\\
2. Peindre un mur;\\
3. Faire le tour d'un terrain de course;\\
4. Remplacer un miroir.\\[3mm]
a) 1, 2\\
b) 1, 2, 3, 4\\
c) 1, 2, 4\\
d) 2, 3, 4\\

R\'eponse : c)\\

R\'etroaction :\\
L'\'enonc\'e 3 se rapporte \`a  un p\'erim\`etre et non \`a une
aire. Les
\'enonc\'es 1, 2 et 4 sont tous en lien avec le calcul de l'aire. La
r\'eponse est c).\\

1460-- Lequel des renseignements suivants doit-on absolument
conna\^itre si on
veut estimer le prix \`a payer pour l'achat d'un terrain?\\
a) Le p\'erim\`etre du terrain\\
b) L'aire du terrain\\
c) Le volume du terrain\\
d) On ne peut pas estimer le prix \`a payer pour un terrain.\\

R\'eponse : b)\\

R\'etroaction :\\
Pour d\'eterminer le prix d'un terrain, on doit absolument
conna\^itre l'aire du terrain en question. Par la suite, en sachant
le prix de chaque unit\'e de surface, on pourra d\'eterminer le prix final.
La r\'eponse est donc b).\\

1461-- Que vaut 6,8 dm? \\
a) 68 cm\\
b) 68 hm\\
c) 68 m\\
d) 68 mm \\

R\'eponse : a)\\

R\'etroaction :\\
Si on convertit des d\'ecim\`etres en centim\`etres, on se retrouve
avec une unit\'e dix fois plus petite. On doit multiplier
${\textrm{6,8 dm}}$ par 10 :$${\textrm{6,8 dm}}\times10=68 {\textrm{
cm}}.$$ \

On constate que 68 cm \'equivalent \`a 6,8 dm. La r\'eponse est donc a).\\

1462-- Un de tes amis a un peu de difficult\'e avec le syst\`eme
m\'etrique. Peux-tu l'aider \`a convertir $0,8$ d\'ecam\`etre en
centim\`etres?\\
a) 0,000 8 cm\\
b) 0,008 cm\\
c) 80 cm\\
d) 800 cm\\

R\'eponse : d)\\

R\'etroaction :\\
Dans l'ordre croissant, on a : \vskip 10pt
\begin{center}
mm,           cm,            dm,           m,           dam, hm,
  km.
\end{center}
\vskip 10pt On doit passer des d\'ecam\`etres aux centim\`etres. Il
faut multiplier par {\textrm{1 000 }}car un centim\`etre est
{\textrm{1 000}} fois plus petit qu'un d\'ecam\`etre. On a donc
$0,8\times{\textrm{1 000}}=\,800\,$cm. La r\'eponse est d).\\

1463-- Lequel des \'enonc\'es suivants est faux?\\
a) 100 mm $<$ 10 dm $<$ 0,1 m\\
b) 1 km $<$ 100 m $<$ 10 dam\\
c) 1 m $<$ 1 dam $<$ 1 hm\\
d) 1 m $<$ 1 hm $<$ 1 dam\\

R\'eponse : c)\\

R\'etroaction :\\
Pour faciliter la comparaison, il est pr\'ef\'erable de convertir
toutes les valeurs dans les m\^emes unit\'es. En c), on obtient
\begin{center}
1 m $<$ 10 m $<$ 100 m.
\end{center}
Ces in\'egalit\'es sont vraies. La r\'eponse est donc c). \\

1464-- Jean se pr\'epare pour un marathon. Lors de son entra\^inement,
il parcourt une apr\`es l'autre quatre routes diff\'erentes :\\

- route A : 1,285 km;\\
- route B : 3,747 m;\\
- route C : 43 hm;\\
- route D : 124,3 dam. \\

Combien a-t-il parcouru de kilom\`etres au total?\\
a) 10,575 km\\
b) 105,75 km\\
c) 1 057,5 km\\
d) 3 915,585 km\\

R\'eponse : a)\\

R\'etroaction :
\begin{eqnarray*}
{\textrm{ 1,285 km}} + {\textrm{ 3 747 m}}+ {\textrm{ 43 hm}} +
{\textrm{ 124,3 dam}}&=&{\textrm{ 1,285 km}}+ {\textrm{ 3,747 km}} +
{\textrm{ 4,3 km}} + {\textrm{1,243 km}}\\[2mm]
&=&{\textrm{ 10,575 km}}
\end{eqnarray*}
La r\'eponse est donc a).\\[3mm]

1465-- Mme Bou Claire doit acheter 225 cm de tissu \`a 1,30\,\$ le
m\`etre. Combien lui co\^utera ce tissu?\\
a) 2,92\,\$\\
b) 29,25\,\$\\
c) 292,50\,\$\\
d) 2 925,00\,\$\\

R\'eponse : a)\\

R\'etroaction :\\
Il s'agit de convertir 225 centim\`etres en m\`etres, car le prix du
tissu est \'evalu\'e en m\`etres.
\begin{center}
225 cm = 2,25 m
\end{center}
\begin{center}
$2,25$ m $\times$ $1,30\,\$$/m = $2,92\,\$$
\end{center}
La r\'eponse est donc a).\\

1466-- Tu veux couper des morceaux de 10,5 cm dans une planche de
bois de 25 m. Combien peux-tu en obtenir?\\
a) 2\\
b) 23\\
c) 238\\
d) 250\\

R\'eponse : c)\\

R\'etroaction :\\
On doit trouver la longueur de la planche en centim\`etres :
\begin{center}
25 m = 2 500 cm.
\end{center}
Par la suite, on divise ce r\'esultat par la longueur d\'esir\'ee
d'un morceau :
\begin{center}
$\frac{2\,500}{10,5}={238}$ morceaux.
\end{center}
La r\'eponse est c).\\[3mm]

1467-- Un triangle ABC est constitu\'e de c\^ot\'es ayant les
longueurs suivantes : \vskip 10pt $\bar{AB}= 3,8$ cm; \vskip 10pt
$\bar{BC}= 52$ mm; \vskip 10pt $\bar{CA}= 0,015$ m. \vskip15pt
\noindent
Quel est le p\'erim\`etre de ce triangle en centim\`etres?\\
a) 9,15 cm\\
b) 10,5 cm\\
c) 19,32 cm\\
d) 55,815 cm\\

R\'eponse : b)\\

R\'etroaction :\\
Convertissons toutes les valeurs en centim\`etres.
\begin{center}
\vskip 10pt 0,015
m = 1,5 cm \vskip 10pt 52 mm = 5,2 cm \vskip 15pt \noindent
\end{center}
On doit
ensuite additionner ces trois mesures, car on cherche le
p\'erim\`etre du triangle.
\begin{center}
P\,=\,\,$3,8\,+\,5,2\,+\,1,5\,=\,10,5$\,\,cm
\end{center}
La r\'eponse est donc b).\\

1468-- Soit un rectangle de c\^ot\'es 1,5 et $x$. Quelle expression
alg\'ebrique
repr\'esente le p\'erim\`etre de ce rectangle?\\
a) $1,5 + x$\\
b) $1,5\cdot x$\\
c) $3+2\cdot x$\\
d) $3\cdot x$\\

R\'eponse : c)\\

R\'etroaction :\\
Le p\'erim\`etre est la somme des mesures des quatre c\^ot\'es du
rectangle. On doit donc calculer $$1,5 + x + 1,5 + x = 3 +2\cdot x.$$ La
r\'eponse est c).\\


1469-- Combien existe-t-il de rectangles ayant un p\'erim\`etre de 12
cm et dont les dimensions sont des nombres entiers distincts?\\
a) 2\\
b) 3\\
c) 4\\
d) 5\\

R\'eponse : a)\\

R\'etroaction :\\
On doit chercher les rectangles tels que : $$2\cdot(x+y)=12.$$Ceci
\'equivaut \`a chercher les couples $(x, y)$ tels que :
$$(x+y)=6.$$ On a donc les couples (1,\,5) et (2,\,4). On doit
rejeter (3,\,3), car ce ne sont pas des nombres entiers DISTINCTS.
On doit aussi consid\'erer que les couples (1,\,5) et (5,\,1)
repr\'esentent le m\^eme rectangle. Il y a donc deux rectangles
correspondant \`a l'\'enonc\'e
de d\'epart. La r\'eponse est a).\\

1470-- M. Laverdure d\'esire poser une cl\^oture autour de son terrain
rectangulaire mesurant 24,2 m\`etres par 21,3 m\`etres. Combien
co\^utera l'installation de cette cl\^oture si le prix du bois est de
$7,25\,\$$ le m\`etre?\\

a) 91\,\$\\
b) 329,88\,\$\\
c) 659,75\,\$\\
d) 1 319,50\,\$\\

R\'eponse : c)\\

R\'etroaction :\\
On doit tout d'abord calculer le p\'erim\`etre du terrain. Il est de
$2\cdot(24,2+21,3)=91$ m. On sait que le prix est de 7,25\,\$/m.
Pour trouver le co\^ut total de l'installation, il faut multiplier
le
p\'erim\`etre par ce prix. On a donc $$91\times7,25=659,75\,\$.$$ La
r\'eponse est c).\\

1471-- Quelle est la mesure de la largeur d'un rectangle dont le
p\'erim\`etre
est de 25 m et la longueur de 8 m?\\
a) 4,5 m\\
b) 8 m\\
c) 8,5 m\\
d) 17 m \\

R\'eponse : a)\\

R\'etroaction :\\
On sait que $$2\cdot(8+x)=25.$$ Il faut donc chercher $x$ tel que
$$(8+x)=12,5.$$ En isolant $x$, on trouve que le
c\^ot\'e manquant mesure 4,5 m. La r\'eponse est donc a).\\

1472-- Ta m\`ere te demande de convertir l'aire de son plancher de
cuisine (34 m\`etres carr\'es) en centim\`etres carr\'es.
Que lui r\'epondras-tu?\\
a) 0,003 4\,cm$^{2}$\\
b) 34\,000\,cm$^{2}$\\
c) 3\,400\,cm$^{2}$\\
d) 340 000\,cm$^{2}$\\

R\'eponse : d)\\

R\'etroaction :\\
Pour passer d'une unit\'e d'aire donn\'ee \`a une unit\'e d'aire
inf\'erieure, on doit multiplier par 100. Ici, pour passer en
centim\`etres carr\'es, on doit multiplier par
$100\times100=10\,000$ car les centim\`etres carr\'es sont
inf\'erieures aux m\`etres carr\'es de deux cat\'egories d'unit\'es.
$$34\cdot10\,000=340\,000{\textrm{ cm$^2$}}$$
La r\'eponse est donc d).\\

1473-- M. Pr\'elard, le g\'erant de la quincaillerie, doit effectuer
l'addition des surfaces suivantes : 7,8\,m$^{2}$ + 0,008\,hm$^{2}$ +
640\,dm$^{2}$. Il doit donner le r\'esultat de ce calcul en
m\`etres; cependant, il a de la difficult\'e \`a convertir ces mesures.
Peux-tu aider M.\,Pr\'elard \`a trouver le r\'esultat de cette
op\'eration?\\
a) 72,6 m$^{2}$\\
b) 94,2 m$^{2}$\\
c) 158,4 m$^{2}$\\
d) 647,808 m$^{2}$\\

R\'eponse : b)\\

R\'etroaction :\\
On doit d'abord convertir toutes les mesures en m\`etres
carr\'es.\\\\
0,008\,hm$^{2}$ = 80\,m$^2$ (pour passer de \,hm$^2$ \`a\,m$^2$, il
faut multiplier par $100\times100$). \vskip 10pt 640\,dm$^{2}$
=\,6,4\,m$^2$ (pour passer de dm$^2$ \`a m$^2$, il faut diviser par
100). \vskip 15pt On doit maintenant additionner les valeurs en
m\`etres carr\'es :$$7,8+80+6,4=94,2{\textrm{ m$^2$}}.$$
La r\'eponse est donc b).\\

1474-- Le p\'erim\`etre d'un carr\'e est de 31,6 cm. Peux-tu
d\'eterminer son aire?\\
a) 22,49 cm$^2$\\
b) 25 cm$^2$\\
c) 31,6 cm$^2$\\
d) 62,41 cm$^2$\\

R\'eponse : d)\\

R\'etroaction :\\
Comme le p\'erim\`etre du carr\'e est de 31,6 cm, le carr\'e
poss\`ede un c\^ot\'e de longueur $31,6\div4=7,9$ cm. Pour calculer
l'aire, on n'a qu'\`a faire $7,9^2=62,41{\textrm{ m$^2$}}$. La r\'eponse est
d).\\

1475-- Un litre de peinture couvre une surface de 18 m$^2$. Combien de
litres de peinture doit acheter M. Sico pour peindre deux murs
rectangulaires de 9 m par 6 m et un plafond
rectangulaire de 9 m par 8 m?\\
a) 5 litres \\
b) 7 litres\\
c) 10 litres\\
d) 3 240 litres\\

R\'eponse : c)\\

R\'etroaction :\\
Nous devons calculer l'aire totale de la surface \`a peindre. Il y a
deux murs de $9\times6=54$ m$^{2}$ d'aire et un plafond de
$9\times8=72$ m$^{2}$. M. Sico a donc un total de $54+54+72=180$
m$^{2}$ \`a peindre. Comme chaque litre de peinture couvre 18 m$^2$,
il aura besoin de $180\div18=10$
L de peinture. Par cons\'equent, la r\'eponse est c).\\

1476-- Voici les r\'esultats des 15 \'el\`eves de la classe \`a un
examen de math\'ematiques :
\begin{center}
49, 51, 62, 65, 67, 69, 70, 70, 74, 77, 78, 79, 81, 87, 93.
\end{center}
Quel pourcentage des \'etudiants ont obtenu une note sup\'erieure
\`a 80\,\%?\\
a) $3\,\%$\\
b) $20\,\%$\\
c) $26,66\,\%$\\
d) $80\,\%$\\

R\'eponse : b)\\

R\'etroaction :\\
Trois \'el\`eves sur un total de 15 ont obtenu une note
sup\'erieure \`a 80\,\%. \\$$3\div15\times100=20\,\%.$$ La r\'eponse est
donc b).\\

1477-- Durant les dict\'ees, qui sont not\'ees sur 20 points, ton
professeur de fran\c cais enl\`eve deux points par faute. Quatre
dict\'ees ont eu lieu au cours du trimestre. Julien a fait deux
fautes dans la premi\`ere dict\'ee, une faute dans la deuxi\`eme,
quatre fautes dans la troisi\`eme ainsi que trois fautes dans la
quatri\`eme. Quelle est sa moyenne pour les quatre dict\'ees?\\
a) $\frac{5}{20}$\\[3mm]
b) $\frac{13}{20}$\\[3mm]
c) $\frac{14}{20}$\\[3mm]
d) $\frac{15}{20}$\\

R\'eponse : d)\\

R\'etroaction :\\
On doit tout d'abord calculer les r\'esultats de Julien pour chacune
des dict\'ees. \\ $1^{re}$ dict\'ee : $20-2\times2=16$, pour une
note de 16 sur 20. \vskip 10pt $2^{e}$ dict\'ee : $20-1\times2=18$,
pour une note de 18 sur 20. \vskip 10pt $3^{e}$ dict\'ee :
$20-4\times2=12$, pour une note de 12 sur 20. \vskip 10pt $4^{e}$
dict\'ee : $20-3\times2=14$, pour une note de 14 sur 20. \vskip 10pt
\noindent Il ne reste qu'\`a calculer la moyenne de ces 4
r\'esultats.
$$(16+18+12+14)\div4=60\div4=15.$$
La moyenne de Julien est de $\frac{15}{20}$. La r\'eponse est donc d).\\

1478-- Un oiseau s'envole verticalement en partant du sol. Il s'aper\c
coit que la temp\'erature diminue de $8^{\circ}$C \`a chaque fois
qu'il s'\'el\`eve de trois kilom\`etres. Si la temp\'erature au sol
est $18^{\circ}$C, quelle temp\'erature l'oiseau ressentira-t-il au
moment o\`u il sera \`a 27
km de la Terre?\\
a) $-198^{\circ}$C\\
b) $-54^{\circ}$C\\
c) $90^{\circ}$C\\
d) $234^{\circ}$C\\

R\'eponse : b)\\

R\'etroaction :\\
La temp\'erature diminue de $8^{\circ}$C \`a chaque fois que
l'oiseau parcourt trois kilom\`etres. S'il parcourt 27 km, l'oiseau
sentira neuf diminutions de temp\'erature. Cette derni\`ere variera
donc de $9\times8^{\circ}$C= $72^{\circ}$C. Comme la temp\'erature
de base est de $18^{\circ}$C, elle atteindra $18^{\circ}$C - $72^{\circ}$C=
$-54^{\circ}$C.
La r\'eponse est donc b).\\

1479-- M. Charcutier vend les $\frac{3}{8}$ de ses paquets de viande
le matin et la moiti\'e durant le reste de la journ\'ee. S'il lui
reste 16 paquets \`a la fermeture de la boucherie, combien en avait-il
lors de l'ouverture de celle-ci?\\
a) 22 paquets\\
b) 48 paquets\\
c) 96 paquets\\
d) 128 paquets\\

R\'eponse : d)\\

R\'etroaction :\\
Si on nomme $x$ le nombre de paquets de viande que M. Charcutier
avait au d\'epart, le probl\`eme revient \`a r\'esoudre l'\'equation
suivante :
$x-\left(\frac{3}{8}\right)x-\left(\frac{1}{2}\right)x$ = 16. Mettons tous
ces termes sur le m\^eme
d\'enominateur et effectuons le calcul.
$$\left(\frac{8}{8}\right)x-\left(\frac{3}{8}\right)x-\left(\frac{4}{8}\right)x
= 16$$ $$\left(\frac{1}{8}\right)x = 16$$
$$x = 128$$ Par cons\'equent, la r\'eponse est d).\\

1480-- Dans un village de 3 500 habitants, 60\,\% sont des hommes et
parmi tous ces hommes, 20\,\% sont des fumeurs. Combien d'habitants
du village sont des
hommes fumeurs?\\
a) 280 personnes\\
b) 420 personnes\\
c) 1 680 personnes\\
d) 2 800 personnes\\

R\'eponse : b)\\

R\'etroaction :\\
Ce probl\`eme se r\'esout en deux \'etapes : \vskip 10pt
$1^{re}{\textrm{ \'etape}}$ : 60\,\% des 3 500 personnes sont des
hommes. Ainsi, il y a $\left(\frac{60}{100}\right)\times{\textrm{ 3
500}}={\textrm{ 2 100}}$ hommes dans ce village. \vskip 10pt $2^e
{\textrm{ \'etape}}$ : Parmi ces 2 100 hommes, 20\,\% sont des
fumeurs. Par cons\'equent, il y a
$\left(\frac{20}{100}\right)\times{\textrm{ 2 100}}=420$ hommes fumeurs dans
ce village. La r\'eponse est donc b).\\

1481-- Lequel des \'enonc\'es suivants est faux?\\
a) L'image d'une droite par une translation est une droite qui lui est
parall\`ele.\\
b) L'image d'une droite par une rotation est une droite qui lui est
congrue.\\
c) L'image d'une droite par une r\'eflexion est une droite qui est toujours
s\'ecante avec la droite initiale.\\
d) L'image d'une figure par une translation, une rotation ou une
r\'eflexion
est une figure qui est congrue avec la figure initiale.\\

R\'eponse : c)\\

R\'etroaction :\\
Prenons par exemple une droite parall\`ele \`a l'axe de r\'eflexion;
dans ce cas, l'image sera une droite parall\`ele \`a cet axe, mais
plac\'ee de l'autre c\^ot\'e. Ainsi, l'image sera une droite
parall\`ele \`a la droite initiale. Les droites ne seront donc pas
s\'ecantes.
La r\'eponse est c).\\

1482-- Un triangle qui poss\`ede exactement deux c\^ot\'es congrus est :\\
a) \'Equilat\'eral\\
b) Isoc\`ele\\
c) Rectangle\\
d) Scal\`ene\\

R\'eponse : b)\\

R\'etroaction :\\
Lorsqu'un triangle a 2 c\^ot\'es congrus, il est isoc\`ele. Un triangle
isoc\`ele poss\`ede \'egalement deux angles congrus. La r\'eponse est b).\\

1483-- Un quadrilat\`ere ayant exactement 2 c\^ot\'es oppos\'es
parall\`eles
est un :\\
a) Carr\'e\\
b) Losange\\
c) Parall\'elogramme\\
d) Trap\`eze\\

R\'eponse : d)\\

R\'etroaction :\\
Le trap\`eze est le seul de ces 4 quadrilat\`eres \`a poss\'eder
seulement 2 c\^ot\'es oppos\'es parall\`eles. Les 3 autres
quadrilat\`eres ont
2 paires de c\^ot\'es oppos\'es parall\`eles. La r\'eponse est d).\\

1484-- M. S. Ouffl\'e fait du jogging tous les jours sur une piste de
course rectangulaire dont les dimensions sont de 130 m par 75 m. Si
le premier jour, il fait trois tours et que les jours suivants, il
en fait cinq, combien de kilom\`etres aura-t-il parcourus apr\`es
cinq jours
d'entra\^inement?\\
a) 3,075 km\\
b) 4,715 km\\
c) 9,15 km\\
d) 9,43 km\\

R\'eponse : d)\\

R\'etroaction :\\
On doit tout d'abord calculer la distance que M. S. Ouffl\'e
parcourt lorsqu'il fait un tour de piste. Le p\'erim\`etre de
celle-ci est de $2\,\times(130{\textrm{ m}}+75{\textrm{ m}})=410$ m.
Au total, $3+5+5+5+5=23$ tours sont effectu\'es par notre champion.
M. S. Ouffl\'e parcourt 23 tours de 410 m chacun, c'est-\`a-dire
un total de $23\times410{\textrm{ m}}=9430$ m = $9,43$ km. La r\'eponse est
d).\\

1485-- Quelle est l'aire d'un triangle de 18,2 cm de base et de 10,1 cm de
hauteur?\\
a) 91,91 cm\\
b) 91,91 cm$^{2}$\\
c) 183,82 cm\\
d) 183,82 cm$^{2}$\\

R\'eponse : b)\\

R\'etroaction :\\
L'aire d'un triangle se calcule \`a l'aide de la formule suivante :
$${\LARGE \frac {base\cdot hauteur}{2}}.$$

Ainsi, nous avons ${\LARGE \frac {18,2\cdot 10,2}{2}}= 91,91$. Les
unit\'es sont bien des centim\`etres carr\'es car on est en
pr\'esence de l'aire d'un triangle; on doit donc
utiliser des unit\'es d'aire. La r\'eponse est b).\\

1486-- Quel est l'\^age de Pauline si elle est quatre fois plus
\^ag\'ee que
son fils Paul de 13 ans?\\
a) 17 ans\\
b) 42 ans\\
c) 52 ans\\
d) 62 ans\\

R\'eponse : c)\\

R\'etroaction :\\
L'op\'eration \`a effectuer dans ce cas est $4\times13=52$. la
r\'eponse
est donc c).\\

1487-- Trouve le nombre qui correspond au d\'eveloppement suivant :\\
$$(5\times10^4)+(7\times10^2)+(4\times10^1).$$\\
a) 574\\
b) 5 074\\
c) 5 740\\
d) 50 740\\

R\'eponse : d)\\

R\'etroaction :\\
Pour parvenir \`a la r\'eponse, il faut proc\'eder comme suit :
$$5\times10^4 + 7\times10^2 + 4\times10^1 =\\
50\,000 + 700 + 40 =50\,740.$$ La r\'eponse est d).\\

1488-- M. Al Cool a achet\'e un baril de 500 L de vin, qu'il revend en
bouteilles de 1,25 L. Quel sera le profit de M. Al Cool s'il a
pay\'e ce vin 6,75\,\$ le litre et qu'il le revend 12\,\$ la bouteille?\\
a) 0\,\$ car il le vendra au m\^eme prix qu'il l'a achet\'e.\\
b) 1\,425\,\$\\
c) 2\,100\,\$\\
d) 4\,125\,\$\\

R\'eponse : b)\\

R\'etroaction :\\
Quantit\'e de bouteilles vendues : $\frac{500}{1,25}$ = 400 bouteilles\\

Prix d'achat du vin : $500\times6,75=3\,375\,\$$\\

Prix de vente du vin : $400\times 12=4\,800\,\$$\\

Profit : $4\,800-3\,375=1\,425\,\$$\\[3mm]
La r\'eponse est donc b).\\

1489-- Tu ach\`etes une petite tortue de 80 grammes \`a l'animalerie.
Une semaine plus tard, la masse de ta tortue a augment\'e de 15\,\%
de sa masse initiale. Une autre semaine passe et la tortue a encore
pris du poids; sa masse s'est accrue d'exactement 50\,\% du total de
sa masse de la semaine pr\'ec\'edente. Combien
la petite tortue p\`ese-t-elle apr\`es ces deux semaines?\\
a) 132 grammes\\
b) 138 grammes\\
c) 142,5 grammes\\
d) 145 grammes\\

R\'eponse : b)\\

R\'etroaction :\\
Ce probl\`eme se r\'esoud en deux parties : \vskip 10pt
$1^{re}{\textrm{ partie}}$ : 80 g + 15\,\% de 80 g = 80 g + 12 g =
92 g. \vskip 10pt $2^e{\textrm{ partie}}$ : 92 g + 50\,\% de 92 g =
92 g + 46 g = 138 g. \vskip 20 pt \noindent
Au bout de ces deux semaines, la petite tortue p\`ese 138 g. La r\'eponse
est donc b).\\

1490-- Dans une usine, les cadres repr\'esentent $\frac{1}{10}$ du
personnel, le personnel de secr\'etariat compte pour 20\,\% du
personnel et le reste est compos\'e des ouvriers. Durant une
r\'eunion, on remarque l'absence du tiers des cadres. Il manque
aussi la moiti\'e du personnel de secr\'etariat ainsi que le
neuvi\`eme des ouvriers. Quelle fraction repr\'esente le nombre
exact d'employ\'es pr\'esents \`a la r\'eunion?\\
a) $\frac{10}{90}$\\[3mm]
b) $\frac{12}{90}$\\[3mm]
c) $\frac{71}{90}$\\[3mm]
d) $\frac{78}{90}$\\

R\'eponse : c)\\

R\'etroaction :\\
Il faut tout d'abord calculer la fraction du personnel
repr\'esentant les ouvriers.
$$1-\frac{1}{10}-\frac{2}{10}=1-\frac{3}{10}=\frac{7}{10}$$
\vskip 10pt \noindent Voici donc la fraction des gens absents de la
r\'eunion :
$$\frac{1}{3}\times\frac{1}{10} +
\frac{1}{2}\times\frac{2}{10}+\frac{1}{9}\times\frac{7}{10} =
\frac{1}{30}+\frac{1}{10}+\frac{7}{90}=\frac{19}{90}.$$ Nous
cherchons la fraction du personnel pr\'esent \`a la r\'eunion :
$1-\frac{19}{90}=\frac{71}{90}$. La r\'eponse est donc c).\\

1491-- Deux trains quittent la gare au m\^eme moment et vont dans exactement la m\^eme direction. Le train A se
d\'eplace \`a 125 km/h et le train B  \`a 175 km/h. Quelle distance
les s\'eparera
apr\`es deux heures trente?\\
a) 50 km\\
b) 100 km\\
c) 125 km\\
d) 150 km\\

R\'eponse : c)\\

R\'etroaction :\\
\`A chaque heure, le train B fait 50 km de plus que le train A.
En deux heures trente, le train B parcourra exactement $50+50+25=125$ km de
plus. La r\'eponse est c).\\

1492-- Deux trains quittent la gare au m\^eme moment. Le train A se
d\'eplace \`a 125 km/h et le train B \`a 175 km/h. Quelle distance
les s\'eparera apr\`es deux heures trente s'ils roulent en direction
oppos\'ee?\\
a) 50 km\\
b) 125 km\\
c) 300 km\\
d) 750 km\\

R\'eponse : d)\\

R\'etroaction :\\
En une heure, le train A aura fait 125 km dans une direction et le
train B, 175 km dans la direction oppos\'ee. Ils se seront
distanc\'es ainsi de 300 km \`a chaque heure parcourue. Apr\`es deux
heures trente, ils seront donc \`a une distance de 750 km l'un de
l'autre.
Par cons\'equent, la r\'eponse est d).\\

1493-- Ta m\`ere coupe la moiti\'e d'un g\^ateau en trois parties
\'egales et l'autre moiti\'e en quatre parties \'egales. Si tu
manges une pointe de la premi\`ere moiti\'e et une pointe de la
deuxi\`eme moiti\'e, quelle fraction
du g\^ateau auras-tu mang\'ee?\\
a) $\frac{2}{14}$\\[3mm]
b) $\frac{2}{7}$\\[3mm]
c) $\frac{7}{24}$\\[3mm]
d) $\frac{1}{3}$\\[3mm]

R\'eponse : c)\\

R\'etroaction :\\
La premi\`ere pointe que tu manges repr\'esente le
$\frac{1}{2}\cdot\frac{1}{3}=\frac{1}{6}$ du g\^ateau et la
deuxi\`eme le $\frac{1}{2}\cdot\frac{1}{4}=\frac{1}{8}$ du g\^ateau.
Tu dois
effectuer la somme de ces deux fractions.\\
$$\frac{1}{6}+\frac{1}{8} = \frac{4}{24}+\frac{3}{24} =
\frac{7}{24}.$$\\
La r\'eponse est donc c).\\

1494-- Un terrain rectangulaire a une longueur de 20 m et une largeur
de 16 m. Tu as des projets en t\^ete concernant ce terrain, mais ils
n\'ecessitent un peu plus d'espace. Tu d\'ecides d'augmenter la
longueur du terrain de 5 m et la largeur de 4 m. Quelle est l'aire
de la surface ajout\'ee?\\
a) 20 m$^2$\\
b) 100 m$^2$\\
c) 180 m$^2$\\
d) 500 m$^2$\\

R\'eponse : c)\\

R\'etroaction :\\
Il est plus facile de calculer l'aire en dessinant le rectangle
repr\'esentant le terrain mentionn\'e. En allongeant les c\^ot\'es
de ce terrain comme on l'a fait, on a ajout\'e
$(5\times16)+(4\times20)+(4\times5)$ = 180 m$^2$.
La r\'eponse est c).\\


1495-- On dessine sur une feuille un carr\'e dont l'un des c\^ot\'es est la
base d'un triangle \'equilat\'eral. Quel est le
p\'erim\`etre de cette
figure si le c\^ot\'e du carr\'e mesure six cm?\\
a) 30 cm\\
b) 32 cm\\
c) 36 cm\\
d) 42 cm\\

R\'eponse : a)\\

R\'etroaction :\\
On doit compter trois des quatre c\^ot\'es du carr\'e et deux des
trois c\^ot\'es du triangle car un c\^ot\'e du triangle et du
carr\'e ne font pas partie du p\'erim\`etre. Le triangle est
\'equilat\'eral, il est donc aussi de 6 cm de c\^ot\'e. On doit
donc compter cinq c\^ot\'es mesurant chacun 6 cm, soit 30 cm. La r\'eponse
est a).\\

1496-- Aujourd'hui, la somme des \^ages d'un p\`ere et de sa fille est
\'egale \`a 66 ans. Il y a dix ans, la fille \'etait \^ag\'ee de
quatre ans. Quel
\^age aura le p\`ere dans dix ans?\\
a) 62 ans\\
b) 72 ans\\
c) 76 ans\\
d) 86 ans\\

R\'eponse : a)\\

R\'etroaction :\\
Dans ce genre de probl\`eme, le plus important est de se fixer un
point de rep\`ere. Avec les indices donn\'es, on sait que la fille a
aujourd'hui 14 ans puisqu'elle avait quatre ans il y a 10 ans. Par
cons\'equent, la somme de leur \^age \'etant \'egale \`a 66 ans, le
p\`ere a 52 ans. \vskip 10pt \noindent
Dans dix ans, le p\`ere aura donc 52 + 10 ans, soit 62 ans. La r\'eponse est
a).\\

1497-- M. Crayola veut fabriquer une affiche sur un carton de couleur.
Il poss\`ede trois crayons (un noir, un bleu et un gris) et trois
cartons (un jaune, un vert et un rouge). M. Crayola ne doit utiliser
qu'une seule couleur de crayon sur le carton qu'il choisira. Combien
a-t-il de possibilit\'es diff\'erentes pour r\'ealiser son affiche?\\
a) 3\\
b) 6\\
c) 9\\
d) 27\\[3mm]
R\'eponse : c)\\[3mm]
R\'etroaction :\\
Pour calculer le nombre de possibilit\'es diff\'erentes offertes \`a
M. Crayola, il faut remarquer qu'\`a chaque carton diff\'erent, il a
le choix entre trois couleurs de crayon. Il a trois cartons, donc
$3\times3=9$ possibilit\'es
diff\'erentes. La r\'eponse est c).\\

1498-- Dans un club de p\^eche, on facture 8\,\$ de frais de base pour
l'acc\`es au terrain et on demande 3\,\$ pour chaque poisson
p\^ech\'e. Quel est le prix d'une journ\'ee de p\^eche pour un
vacancier qui attrape
six poissons?\\
a) 26\,\$\\
b) 51\,\$\\
c) 62\,\$\\
d) 66\,\$\\

R\'eponse : a)\\

R\'etroaction :\\
Le p\^echeur doit payer 8\,\$ + 3\,\$ par poisson, c'est-\`a-dire
qu'il doit payer
$8+3\cdot6 = 26\,\$$. La r\'eponse est donc a).\\

1499-- M. Dollar emprunte 1 000\,\$. Il doit remettre 50\,\$ par mois
\`a la banque jusqu'\ au remboursement total de sa dette. Quel est
le solde de sa dette apr\`es $x$ mois sachant qu'il n'a pas
d'int\'er\^et \`a payer?\\
a) ${\textrm{ 1 000}}-50\cdot x$\\
b) ${\textrm{ 1 000}}+50\cdot x$\\
c) ${\textrm{ 1 000}}\cdot x+50$\\
d) ${\textrm{ 1 000}}\cdot x-50$\\

R\'eponse : a)\\

R\'etroaction :\\
Lorsqu'on rembourse une dette, le montant de celle-ci doit diminuer
et non s'accro\^itre. Au d\'epart, la dette totalise 1 000\,\$ et
\`a chaque mois, M. Dollar rembourse 50\,\$, abaissant ainsi le
montant de sa dette.
L'\'equation recherch\'ee est donc : $${\textrm{ 1 000}}-50\cdot x.$$ \\La
r\'eponse est a).\\

1500-- R\'esous l'\'equation alg\'ebrique suivante :$$\frac{3}{5}a-
\frac{2}{3} a.$$\\
a)  $\frac{1}{2}$\\[3mm]
b)  $\frac{a}{2}$\\[3mm]
c)  $\frac{-1}{15}$\\[3mm]
d)  $\frac{-a}{15}$\\[3mm]

R\'eponse : d)\\

R\'etroaction :\\
Nous devons mettre les fractions sur un d\'enominateur commun.
\begin{eqnarray*}
\frac{3}{5}a-\frac{2}{3}a&=&\frac{9}{15}a-\frac{10}{15}a\\[3mm]&=&(\frac{9}{15}-\frac{10}{15})a\\[3mm]&=&\frac{-1}{15}a\\[3mm]&=&\frac{-a}{15}.
\end{eqnarray*}\\
La r\'eponse est donc d).\\

1501-- Ton professeur te demande d'aller simplifier l'\'equation
suivante au tableau :$$5a+2+3b-5c-7b-5+2c.$$ Que vas-tu inscrire au
tableau?\\
a) $5a-4b-3$\\
b) $-2abc-3$\\
c) $5a-4b-3c-3$\\
d) On ne peut rien simplifier car ce sont des additions.\\

R\'eponse : c)\\

R\'etroaction :\\
On doit regrouper les termes semblables.
\begin{eqnarray*}
5a+2+3b-5c-7b-5+2c &=& 5a+(3b-7b)+(2c-5c)+(2-5)\\&=& 5a-4b-3c-3.
\end{eqnarray*}\\Par cons\'equent, la r\'eponse est c).\\

1502-- M. Chlore veut faire repeindre l'int\'erieur de sa piscine. Si
celle-ci mesure 8  m\`etres de longueur, $x$  m\`etres de largeur et
$y$  m\`etres de profondeur, quelle expression
repr\'esente la surface totale \`a peindre?\\
a) $8x+16y+2xy$\\
b) $8xy$\\
c) $16x+16y+2xy$\\
d) $16y+2xy$\\

R\'eponse : a)\\

R\'etroaction :\\
Nous devons calculer l'aire des surfaces des c\^ot\'es ainsi que
celle du fond. Les c\^ot\'es les plus longs de la piscine ont une
aire de $8y$ chacun. Les plus courts ont pour leur part une aire de
$xy$ chacun. On doit aussi ajouter l'aire du fond de la piscine qui
est de $8x$. La somme de ces aires donne l'aire totale de la surface
\`a peindre : 2($8 y$)+2($xy$)+ $8 x$ =($8 x$ + $16 y$ + $2 xy$)
m$^2$. \\La r\'eponse est a).\\

1503-- M. Rona d\'esire faire cl\^oturer un terrain rectangulaire
dont la longueur est deux fois plus grande que la largeur. La
cl\^oture lui revient \`a 18\,\$ le m\`etre carr\'e. Si la largeur
du terrain est $x$, quelle expression d\'esigne le co\^ut total
qu'il devra d\'ebourser pour sa cl\^oture?\\
a) 18 + 6$x$\\
b) 54$x$\\
c) 72$x$\\
d) 108$x$\\

R\'eponse : d)\\

R\'etroaction :\\
La largeur du terrain est $x$ et la longueur $2x$ car elle
correspond au double de la largeur. Le p\'erim\`etre peut donc
\^etre repr\'esent\'e par l'expression $P=2(x+2x)=2(3x)=6x$ m. Comme
chaque m\`etre lui co\^ute $18\,\$$,
l'expression repr\'esentant le co\^ut total de la cl\^oture est
$$18(6x)=108x.$$ Ainsi, la r\'eponse est d).\\

1504-- Tu ach\`etes cinq cahiers et deux livres. Chaque cahier
co\^ute six dollars de moins qu'un livre. Si on d\'esigne le prix
d'un livre par $x$, quelle expression d\'esigne le montant total
d\'ebours\'e?\\
a) 42\,\$\\
b) $7x$\,\$\\
c) $(7x-12)$\,\$\\
d) $(7x-30)$\,\$\\

R\'eponse : d)\\

R\'etroaction :\\
Un livre co\^ute $x\,\$$ et un cahier $(x-6)\,\$$ car un cahier vaut
6\,\$ de moins qu'un livre. Si on ach\`ete cinq cahiers et deux
livres, le co\^ut de l'achat est donn\'e par ce qui suit :
$$C=5\cdot(x-6)+2\cdot(x)=5x-30+2x=7x-30.$$ La r\'eponse est(7$x$ - 30)\,\$,
soit d).\\

1505-- Un groupe de 30 jeunes va au jardin zoologique pour une sortie
scolaire. Les jeunes de 14 ans et moins doivent payer 4,25\,\$ pour
entrer et ceux de plus de 14 ans doivent payer 8\,\$. Si $x$
d\'esigne le nombre de jeunes de 14 ans et moins, quelle est
l'expression repr\'esentant le montant total
d\'epens\'e pour la sortie au jardin zoologique?\\
a) (8 + 4,25$x$)\,\$\\
b) (12,25$x$)\,\$\\
c) (240 $-$ 3,75$x$)\,\$\\
d) (240 $-$ 4,25$x$)\,\$\\

R\'eponse : c)\\

R\'etroaction :\\
Si les 30 jeunes \'etaient \^ag\'es de plus de 14 ans, le co\^ut de
la sortie aurait \'et\'e de 240\,\$. En partant de ce co\^ut maximal
et en consid\'erant que les moins de 14 ans doivent payer 4,25\,\$,
on doit retrancher un montant \'equivalent \`a  $8$ $-$ $4,25$ =
{\textrm{ 3,75\,\$}} pour chacun de ces jeunes. Comme $x$
repr\'esente le nombre de jeunes \^ag\'es de moins de 14 ans,
l'expression d\'esignant le
montant total d\'epens\'e pour la sortie est (240 $-$ 3,75$x$)\,\$. La
r\'eponse est c).\\

1506-- La longueur d'un rectangle \'etant 2$x$ + 1 et sa largeur
3$x$ $-$ 2, quel est son p\'erim\`etre?\\
a) $5x-1$\\
b) $6x^2-x-2$\\
c) $10x-2$\\
d) $10x+2$\\

R\'eponse : c)\\

R\'etroaction :\\
La formule pour calculer le p\'erim\`etre d'un rectangle est
$$P=2 (longueur+largeur).$$ Le p\'erim\`etre est
donc $$P=2 ((3x-2)+(2x+1))=2(5x-1)=10x-2.$$ La r\'eponse est c).\\

1507-- Tu trouves sur le sol un carton sur lequel sont inscrites des
indications de mesures. La longueur du carton est $2x+1$ et la
largeur $5$. Un ami te met au d\'efi de
trouver l'aire de ce carton. Que lui r\'eponds-tu?\\
a) $2x+5$\\
b) $4x+12$\\
c) $10x+1$\\
d) $10x+5$\\

R\'eponse : d)\\

R\'etroaction :\\
La formule pour trouver l'aire d'un rectangle est
$$A=longueur\times largeur.$$ On a donc $$5\cdot(2x+1)=10x+5.$$ Lorsqu'on
multiplie une constante par une
expression alg\'ebrique, on doit distribuer la constante et la
multiplier avec chacun des termes de
l'expression alg\'ebrique. La r\'eponse est d).\\

1508-- L'aire d'un rectangle est d\'etermin\'ee par l'expression
$24x+18$. Sachant que la largeur de ce rectangle vaut 6, trouve
l'expression qui en repr\'esente la longueur.\\
a) $4x+3$\\
b) $4x+18$\\
c) $24x+3$\\
d) $144x+108$\\

R\'eponse : a)\\

R\'etroaction :\\
La formule pour trouver l'aire d'un rectangle est $$A=longueur\times
largeur.$$ Si on conna\^it l'aire et la largeur de ce rectangle, on
trouve la longueur en faisant
$$longueur=\frac{A}{largeur}.$$ Il
est important de retenir que pour diviser une somme de termes par
une constante non nulle, il suffit de diviser chacun des termes par
cette constante. On doit donc effectuer le calcul suivant
:$$longueur=\frac{24x+18}{6}=\frac{24x}{6}+\frac{18}{6}=4x+3.$$\\  La
r\'eponse est a).\\

1509-- Que vaut l'op\'eration suivante ?
$$\frac{x+1}{2}-\frac{x-1}{3}$$
a) {\Large$\frac{x+5}{6} $}\\[3mm]
b) {\Large$\frac{5x+1}{6} $}\\[3mm]
c) {\Large$\frac{2x}{-1}$}\\[3mm]
d) {\Large$\frac{x+1}{6} $}\\[3mm]

R\'eponse : a)\\

R\'etroaction :\\
Nous devons tout d'abord mettre ces fractions sur un d\'enominateur
commun.
$$\frac{x+1}{2}-\frac{x-1}{3}\,=\,\frac{3\cdot(x+1)}{6}-\frac{2\cdot(x-1)}{6}\,=\,\frac{3x+3}{6}-\frac{2x-2}{6}=\frac{x+5}{6}.$$\\
La r\'eponse est donc a).\\

1510-- La base d'un triangle mesure $2x+4$ et sa hauteur $6$. Quelle
est
l'aire de ce triangle?\\
a) $6x+12$\\
b) $12x+24$\\
c) $12x+4$\\
d) $6x+4$\\

R\'eponse : a)\\

R\'etroaction :\\
L'aire d'un triangle se calcule par la formule suivante :
$$\frac{base\times hauteur}{2}.$$ On a donc $$\frac{(2x+4)\cdot
6}{2}=\frac{12x+24}{2}=\frac{12x}{2}+\frac{24}{2}=6x+12.$$\\
La r\'eponse est a).\\

1511-- Les tarifs exig\'es pour assister \`a un film au cin\'ema sont
de six dollars par adulte et quatre dollars par enfant. Le co\^ut
total pour un groupe de 18 personnes s'\'el\`eve \`a 96\,\$. Si l'on
d\'esigne par $x$ le nombre d'adultes pr\'esents \`a cette
repr\'esentation, traduis cette
situation par une expression alg\'ebrique.\\
a) $4x+6(18-x)=96$\\
b) $6x+4(18-x)=96$\\
c) $6x+4=96$\\
d) $18x=96$\\

R\'eponse : b)\\

R\'etroaction :\\
Le prix par adulte multipli\'e par le nombre d'adultes donne le
co\^ut total pour les adultes. Le prix par enfant multipli\'e par le
nombre d'enfants donne le co\^ut total pour les enfants. Si on
additionne le co\^ut total pour les adultes et le co\^ut total pour
les enfants, on obtient le co\^ut total de la sortie pour le groupe
entier. Si on traduit cela par une expression, on
obtient :$$6x+4(18-x)=96.$$ Par cons\'equent, la r\'eponse est b).\\

1512-- Le p\'erim\`etre d'un triangle ABC mesure 25 cm. Le c\^ot\'e
BC est deux fois plus long que le c\^ot\'e AB. En outre, le c\^ot\'e
AC mesure 5 cm de plus que le c\^ot\'e BC. On d\'esigne par $x$ la
mesure du c\^ot\'e AB. Traduis cette situation par une \'equation.\\
a) $3x=25$\\
b) $3x+5=25$\\
c) $4x+5=25$\\
d) $5x+5=25$\\

R\'eponse : d)\\

R\'etroaction :\\
La longueur du c\^ot\'e AB est $x$. Le c\^ot\'e BC, deux fois plus
long que AB, mesure 2$x$. Le c\^ot\'e AC, qui a 5 cm de plus que le
c\^ot\'e BC, mesure $2x+5$. Comme le p\'erim\`etre du triangle ABC
est de 25 cm, la somme des mesures des trois c\^ot\'es est 25,
c'est-\`a-dire $$x+ 2x+(2x+5)=25$$\
$$5x+5=25.$$ La r\'eponse est donc d).\\

1513-- R\'esous l'\'equation suivante :$$4x+12=24.$$\\
a) 2\\
b) 3\\
c) 4\\
d) Cette \'equation est impossible \`a r\'esoudre car on ne conna\^it pas la
valeur de $x$.\\

R\'eponse : b)\\

R\'etroaction :\\
Pour r\'esoudre une \'equation, il s'agit de trouver la ou les
valeurs de $x$ qui permettent d'obtenir l'\'egalit\'e donn\'ee. On
doit ensuite isoler le $x.$
\begin{eqnarray*}
4x+12&=&24 \\ 4x+12-12&=&24-12 \\ 4x &=& 12 \\  [3mm]
  \frac{4x}{4}&=&\frac{12}{4}\\[3mm]x&=&3
\end{eqnarray*}
Il est possible de v\'erifier notre r\'eponse en posant $x=3$ dans
l'\'equation de d\'epart. On obtient donc l'\'egalit\'e suivante
:$$4\cdot(3)+12=12+12=24.$$ La r\'eponse est
b).\\

1514-- D'un certain nombre, on retranche le triple du m\^eme nombre,
ce qui donne une
diff\'erence de $-12$. Quel est ce nombre?\\
a) $-9$\\
b) $-6$\\
c) 4\\
d) 6\\

R\'eponse : d)\\

R\'etroaction :\\
On peut traduire cette situation par :
\begin{eqnarray*}
x-3x&=&-12 \\ -2x&=&-12 \\ [3mm]\frac{-2x}{-2}&=& \frac{-12}{-2}\\
[3mm] x&=&6.
\end{eqnarray*}\\
La r\'eponse est d).\\

1515-- Un certain nombre est multipli\'e par 5. Au r\'esultat, on
ajoute le triple de
ce nombre et on obtient $-96$. Quel est ce nombre?\\
a) $-48$\\
b) $-12$\\
c) 12\\
d) 48\\

R\'eponse : b)\\

R\'etroaction :\\
Nous avons
\begin{eqnarray*}
5x+3x&=&-96 \\ 8x&=&-96 \\[3mm]\frac{8x}{8}&=& \frac{-96}{8}\\[3mm]
x&=&-12.
\end{eqnarray*}
On peut v\'erifier notre r\'eponse et on obtient bien que
$3(-12)+5(-12)=-96$. La r\'eponse est b).\\

1516-- La somme de trois nombres cons\'ecutifs est 51. Quel est le
plus petit
de ces trois nombres?\\
a) 16\\
b) 17\\
c) 18\\
d) Il est impossible de le trouver car on ne conna\^it aucun des trois
nombres en question.\\

R\'eponse : a)\\

R\'etroaction :\\
Si on pose $x$ comme \'etant le plus petit nombre, le deuxi\`eme
nombre est $x+1$ et le troisi\`eme $x+2$, car ils sont tous les
trois cons\'ecutifs. La somme des trois nombres est 51.
$$x+(x+1)+(x+2)=51$$\\ En regroupant les termes en $x$ et les
constantes, on obtient $$3x+3=51.$$\\Il ne nous reste qu'\`a isoler
$x$ dans la derni\`ere \'equation.
\begin{eqnarray*}
3x+3&=&51 \\ 3x+3-3&=&51-3 \\ 3x&=&48 \\[3mm]
\frac{3x}{3}&=&\frac{48}{3}\\[3mm] x&=&16
\end{eqnarray*}
Le plus petit des trois nombres est donc 16. On peut v\'erifier nos calculs
en faisant $$16+17+18=51.$$ La r\'eponse est a).\\

1517-- Ton p\`ere te demande de r\'esoudre pour lui l'\'equation
suivante : $3x-2=2x+1$. Que vas-tu lui r\'epondre?\\
a) $-3$\\
b) $-1$\\
c) 1\\
d) 3\\

R\'eponse : d)\\

R\'etroaction :\\
Pour isoler $x$, il faut envoyer tous les termes en $x$ du m\^eme
c\^ot\'e du signe d'\'egalit\'e et les constantes de l'autre
c\^ot\'e. Voici comment proc\'eder :
\begin{eqnarray*}
3x-2&=&2x+1 \\ 3x-2+2&=&2x+1+2 \\ 3x-2x&=&2x+3-2x \\ x&=&3.
\end{eqnarray*}
La r\'eponse est bien $x$ = 3. Il est toujours prudent de la
v\'erifier :
$$3(3)-2=2(3)+1$$ $$7=7.$$ On voit bien que 3 est la valeur de $x$. La
r\'eponse est donc d).\\

1518-- On ajoute dix au double d'un nombre et l'on obtient cinq de
moins que le triple de ce m\^eme nombre. Quel est ce nombre?\\
a) -15\\
b) -3\\
c) 1\\
d) 15\\

R\'eponse : d)\\

R\'etroaction :\\
L'important est de bien traduire l'\'enonc\'e en \'equation.
\begin{eqnarray*}
2x+10&=&3x-5 \\ 2x+10-10-3x&=&3x-5-10-3x \\ -x &=& -15 \\ x&=&15.
\end{eqnarray*}
La r\'eponse est donc d).\\

1519-- Que vaut $x$ dans l'\'equation suivante ? $$3x-(x+2)=5-(3-x)$$ \\
a) $\frac{-4}{3}$\\[3mm]
b) 0\\[3mm]
c) $\frac{4}{3}$\\[3mm]
d) 4\\

R\'eponse : d)\\

R\'etroaction :\\
\begin{eqnarray*}
3x-(x+2)&=&5-(3-x)  \\  3x-x-2 &=& 5-3+x \\ 2x-2&=&2+x \\ x&=&4
\end{eqnarray*}
La r\'eponse est d).\\

1520-- M. Bell vend des t\'el\'ephones dans une boutique. Il gagne un
salaire hebdomadaire de base de 175\,\$ et re\c coit 25\,\$ par
t\'el\'ephone vendu. Cette semaine, il a gagn\'e un salaire de
500\,\$. Combien de t\'el\'ephones a-t-il vendus?\\
a) 13 t\'el\'ephones\\
b) 20 t\'el\'ephones\\
c) 27 t\'el\'ephones\\
d) Il est impossible de le d\'eterminer, car on ne poss\`ede pas toutes les
donn\'ees n\'ecessaires.\\

R\'eponse : a)\\

R\'etroaction :\\
Si l'on pose $x$ comme \'etant le nombre de t\'el\'ephones vendus en
une semaine, on peut repr\'esenter le salaire de M. Bell par
l'\'equation $175+25x=500$, que l'on r\'esout comme suit :
\begin{eqnarray*}
175+25x&=&500 \\ 175+25x-175&=&500-175 \\25x&=&325 \\[3mm]
\frac{25x}{25}&=&\frac{325}{25}\\[3mm]
x&=&13.
\end{eqnarray*}
M. Bell a vendu 13 t\'el\'ephones durant la semaine. La r\'eponse est donc
a).\\

1521-- La diff\'erence d'\^age entre deux soeurs est de trois ans.
Sachant que la somme de leur \^age est de 33 ans, peux-tu
d\'eterminer l'\^age de
l'a\^in\'ee?\\
a) 15 ans\\
b) 16 ans\\
c) 17 ans\\
d) 18 ans\\

R\'eponse : d)\\

R\'etroaction  :\\
Traduisons tout d'abord ce probl\`eme en \'equation. Si on pose $x$
comme \'etant l'\^age de l'a\^in\'ee, sa soeur aura $x-3$ ans car la
diff\'erence d'\^age entre les deux soeurs est de 3 ans. Comme la
somme de leur \^age est 33, $$x+(x-3)=33.$$\\ Il ne nous reste
qu'\`a trouver la valeur de $x$ :
\begin{eqnarray*}
x+(x-3)&=&33 \\ 2x-3 &=&33 \\ 2x &=& 36 \\ x&=&18.
\end{eqnarray*}
L'\^age de la soeur a\^in\'ee est de 18 ans. La r\'eponse est donc d).\\

1522-- Anne est en train de lire un roman de 507 pages. Elle a lu 207
pages jusqu'\`a maintenant. Si elle parcourt en moyenne 15 pages par
jour, dans combien de jours aura-t-elle fini de lire son roman?\\
a) 14 jours\\
b) 20 jours\\
c) 30 jours\\
d) 34 jours\\

R\'eponse : b)\\

R\'etroaction :\\
Si on pose $x$ comme \'etant le nombre de jours de lecture
n\'ecessaires \`a Anne pour terminer son roman, la situation peut
\^etre repr\'esent\'ee par l'\'equation $207+15x=507$, que l'on
r\'esout comme suit :
\begin{eqnarray*}
207+15x&=&507 \\ 15x&=& 300  \\ x&=&20.
\end{eqnarray*}
Il lui reste donc 20 jours de lecture pour terminer son roman. La r\'eponse
est b).\\

1523-- Ton professeur te demande de trouver trois nombres
cons\'ecutifs dont
la somme est 495. Quel est le plus grand de ces nombres?\\
a) 165\\
b) 166\\
c) 167\\
d) 168\\

R\'eponse : b)\\

R\'etroaction :\\
Si on pose $x$ comme \'etant le plus grand de ces trois entiers, les
autres entiers seront $(x-1)$ et $(x-2)$. On obtient l'\'equation
$(x-2)+(x-1)+x=495$, que l'on r\'esout comme suit :
\begin{eqnarray*}
(x-2)+(x-1)+x&=&495 \\ 3x-3&=&495 \\ 3x&=&498 \\ x&=&166.
\end{eqnarray*}
Par cons\'equent, les trois nombres sont 164, 165 et 166. La somme
donne bien
$164+165+166=495$. Le plus grand de ces nombres est 166. La r\'eponse est
donc b).\\

1524-- Tu te rends chez Club Patio pour acqu\'erir une table et des
chaises. Si trois tables et 12 chaises sont vendus $ {\textrm{1
350}}\,\$$ et que le prix d'une table est le double de celui d'une
chaise, peux-tu trouver le prix
d'une table?\\
a) $75\,\$$\\
b) $112,5\,\$$\\
c) $150\,\$$\\
d) $225\,\$$\\

R\'eponse : c)\\

R\'etroaction :\\
Si on pose $x$ comme \'etant le prix d'une chaise, le prix d'une
table sera $2x$. On sait que 3 tables et 12 chaises co\^utent $
{\textrm{1 350}}\,\$$. Ainsi,

\begin{eqnarray*}
3\cdot(2x)+12\cdot(x)={\textrm{1 350}}\\ 6x+12x={\textrm{1 350}} \\
18x={\textrm{1 350}} \\[3mm]
x=\frac{1350}{18}
\\[3mm] x=75.
\end{eqnarray*}

Une chaise vaut donc $75\,\$$ et une
table $150\,\$$. La r\'eponse est c).\\

1525-- Tu prends un taxi pour retourner chez toi apr\`es une soir\'ee
chez des amis. Le tarif de base est $2,00\,\$$ et il en co\^ute
$0,60\,\$$ suppl\'ementaire par kilom\`etre parcouru. Quelle
distance a \'et\'e parcourue par le taxi si le prix total de ta
facture
s'\'el\`eve \`a $14\,\$$?\\
a) 20 km\\
b) 23,33 km\\
c) 26,66 km\\
d) 34 km\\

R\'eponse : a)\\

R\'etroaction :\\
Si on pose $x$ comme \'etant le nombre de kilom\`etres parcourus, le
co\^ut total peut \^etre repr\'esent\'e par l'\'equation
$2+0,60x=14$, que l'on r\'esout comme suit :
\begin{eqnarray*}
2+0,60x&=&14 \\ 0,60x &=&12 \\[3mm] x&=&\frac{12}{0,60}\\[3mm]
x&=&20.
\end{eqnarray*}
Le taxi a donc parcouru 20 km. La r\'eponse est a).\\

1526-- Maxime et Jo veulent se cotiser pour acheter un ballon valant
$30\,\$$. Jo poss\`ede $15\,\$$ de plus que Maxime. Jo d\'epense le
quart de tout son argent tandis que Maxime d\'epense le tiers du
sien. Quel \'etait l'avoir de Jo au tout d\'ebut?\\
a) $20,42\,\$$\\
b) $42,86\,\$$\\
c) $57,86\,\$$\\
d) $60,00\,\$$\\

R\'eponse : d)\\

R\'etroaction :\\
Posons $x$ comme \'etant l'avoir de Jo. Celui de Maxime peut \^etre
repr\'esent\'e pas ($x-15$) car il a $15\,\$$ de moins que Jo. De
plus, le quart de l'avoir de Jo plus le tiers de l'avoir de Maxime
est \'egal \`a $30\,\$$. On obtient donc l'\'equation
$(\frac{1}{4})x +\frac{1}{3}(x-15)=30$, que l'on r\'esout comme suit
:
\begin{eqnarray*}
\left(\frac{1}{4}\right)x +\frac{1}{3}(x-15)&=&30  \\[3mm] \frac{x}{4}+
(\frac{x}{3}-\frac{15}{3})&=&30 \\[3mm]
\frac{7x}{12}-5&=&30
\\[3mm]
\frac{7x}{12}&=&35 \\[3mm] x&=&\frac{12\cdot 35}{7}\\[3mm] x&=&60.
\end{eqnarray*}\\
La r\'eponse est d).\\

1527-- L'entreprise pour laquelle tu travailles a r\'ealis\'e
beaucoup de profit durant la derni\`ere ann\'ee. Ton patron d\'ecide
de redistribuer $15\,000\,\$$ en prime aux 40 employ\'es de
l'entreprise. Les employ\'es ayant plus de 15 ans d'anciennet\'e
b\'en\'eficieront de $500\,\$$ et ceux en ayant moins de 15
recevront $300\,\$$.
Combien y a-t-il d'employ\'es ayant plus de 15 ans d'anciennet\'e?\\
a) 10 employ\'es\\
b) 15 employ\'es\\
c) 20 employ\'es\\
d) 25 employ\'es\\

R\'eponse : b)\\

R\'etroaction :\\
Posons $x$  comme \'etant le nombre d'employ\'es ayant plus de 15
ans d'anciennet\'e. Exactement ($40-x$) employ\'es en ont donc moins
de 15. Voici comment proc\'eder:
\begin{eqnarray*}
{\textrm{500}}\cdot x+{\textrm{300}}\cdot (40-x)&=&{\textrm{15 000}}\$ \\
500x+{\textrm{12 000}}-300x&=&{\textrm{15 000}}\\
200x&=&{\textrm{3 000}} \\[3mm] x&=&\frac{\textrm{3 000}}{200} \\[3mm]
x&=&15.
\end{eqnarray*}
L'entreprise comptait donc exactement 15 employ\'es avec plus de 15 ans
d'anciennet\'e et 25 employ\'es en ayant moins de 15. La r\'eponse est b).\\

1528-- Lors d'un voyage en automobile, tu te d\'eplaces de la ville A
\`a la ville D en passant par les villes B et C. La distance entre
les villes A et B est le sixi\`eme de celle s\'eparant les villes A
et D, alors que la distance entre les villes B et C est deux foix
plus grande que celle entre les villes A et B. Finalement, on sait
que la distance entre les villes C et D est de 60 km. Quelle est la distance
totale parcourue durant ton voyage?\\
a) 40 km\\
b) 60 km\\
c) 90 km\\
d) 120 km\\

R\'eponse : d)\\

R\'etroaction :\\
La difficult\'e de ce probl\`eme r\'eside dans l'interpr\'etation
exacte des phrases, le vocabulaire \'etant choisi afin de compliquer
la t\^ache. Tout d'abord, posons $x$ comme \'etant la distance entre
les villes A et D, soit le trajet complet que tu as \`a parcourir.
La distance entre les villes A et B est de $\frac{x}{6}$. Celle
s\'eparant les villes B et C est de $\frac{x}{3}$ puisqu'elle est le
double de la distance entre A et B. Le probl\`eme pr\'ecise aussi
que la distance entre les villes C et D est de 60 km. On obtient
l'\'equation $\frac{x}{6}+\frac{x}{3}+60=x$, que l'on r\'esout comme
suit :
\begin{eqnarray*}
\frac{x}{6}+\frac{x}{3}+60&=&x  \\[3mm]  x-\frac{x}{6}-\frac{x}{3}&=&60
\\[3mm]
\frac{6x}{6}-\frac{x}{6}-\frac{2x}{6}&=&60 \\[3mm] \frac{3x}{6}&=&60
\\[3mm]
x&=&\frac{6\cdot 60}{3} \\[3mm] x&=&120.
\end{eqnarray*}\\
La r\'eponse est d).\\

1529-- M. O. Henry doit vendre des tablettes de chocolat pour
financer son tournoi de hockey. Il a d\'ej\`a \'ecoul\'e les trois
quarts de sa caisse de tablettes. S'il vend 10 tablettes de chocolat
de plus, il ne lui restera \`a vendre que le sixi\`eme du contenu de sa
caisse. Combien celle-ci contenait-elle de tablettes de chocolat \`a
l'origine?\\
a) 60 tablettes\\
b) 90 tablettes\\
c) 120 tablettes\\
d) 150 tablettes\\

R\'eponse : c)\\

R\'etroaction :\\
Posons $x$ comme \'etant le nombre de tablettes de chocolat
contenues au d\'epart dans la caisse de notre joueur de hockey.
L'\'equation traduisant cette situation est
$\frac{3x}{4}+10+\frac{x}{6}=x$, que l'on r\'esout comme suit :
\begin{eqnarray*}
\frac{3x}{4}+10+\frac{x}{6}&=&x \\[3mm] x-\frac{3x}{4}-\frac{x}{6}&=&10
\\[3mm]
\frac{12x}{12}-\frac{9x}{12}-\frac{2x}{12}&=&10 \\[3mm] \frac{x}{12}&=&10
\\[3mm]
x&=&120.
\end{eqnarray*}
M. O. Henry avait donc 120 tablettes de chocolats \`a vendre pour le
financement de son tournoi de hockey. La r\'eponse est c).\\

1530-- Un p\`ere de 30 ans est aujourd'hui trois fois plus \^ag\'e
que son fils. Quel
\^age aura le p\`ere lorsqu'il sera deux fois plus \^ag\'e que son fils?\\
a) 10 ans\\
b) 20 ans\\
c) 36 ans\\
d) 40 ans\\

R\'eponse : d)\\

R\'etroaction :\\
Posons $x$ comme \'etant le nombre d'ann\'ees \'ecoul\'ees \`a
partir d'aujourd'hui. Comme le p\`ere de 30 ans est trois fois plus
\^ag\'e que son fils, ce dernier a 10 ans.
\begin{eqnarray*}
(30+x)&=&2(10+x) \\ 30+x&=&20+2x \\ 30-20&=&2x-x \\ 10&=&x
\end{eqnarray*}
Donc, dans 10 ans, le p\`ere aura 40 ans,  c'est-\`a-dire le double
de l'\^age de son fils. La r\'eponse est d).\\

1531-- M. Quad Rill\'e ach\`ete six cahiers au m\^eme prix. Si chaque
cahier co\^utait un dollar de moins, il pourrait s'en acheter trois
de plus pour le
m\^eme prix. Quel est le prix d'un cahier?\\
a) 2\,\$\\
b) 2,50\,\$\\
c) 3\,\$\\
d) 4\,\$\\

R\'eponse : c)\\

R\'etroaction :\\
Il est important de bien comprendre l'\'enonc\'e du probl\`eme. Le
prix de six cahiers serait le m\^eme que le prix de neuf cahiers si
chacun de ces cahiers co\^utait un dollar de moins. Si on pose $x$
comme \'etant le prix d'un cahier, on obtient l'\'equation
$6x=9(x-1)$, que l'on r\'esout comme suit :
\begin{eqnarray*}
6x&=&9(x-1)  \\ 6x&=&9x-9  \\ 9&=&3x  \\  3&=& x.
\end{eqnarray*}
Un cahier co\^utant 3\,\$, le total de la facture pour six cahiers
est donc de 18\,\$. Si les cahiers co\^utaient 1\,\$ de moins, soit
2\,\$, M. Quad Rill\'e pourrait s'en
procurer neuf. La r\'eponse est donc c).\\

1532-- Un rectangle a un p\'erim\`etre de 30 cm. Quelle est la
longueur de ce rectangle si le rapport entre sa longueur et sa
largeur est de quatre. \\
a) 8 cm\\
b) 10 cm\\
c) 12 cm\\
d) 24 cm\\

R\'eponse : c)\\

R\'etroaction :\\
Le rapport entre deux c\^ot\'es est le quotient d'un de ces
c\^ot\'es sur l'autre c\^ot\'e. Dans le cas pr\'esent, ce rapport
est de quatre, c'est-\`a-dire que la longueur est quatre fois plus
grande que la largeur. \,Si on pose $x$ comme \'etant la longueur du
rectangle, on obtient l'\'equation
$2\cdot\Big(\frac{x}{4}+x\Big)=30$, que l'on r\'esout comme suit :
\begin{eqnarray*}
2\cdot\Big(\frac{x}{4}+x\Big)&=&30  \\[3mm] \frac{x}{2}+2x&=&30
\\
\frac{5x}{2}&=&30  \\[3mm] \frac{2\cdot 30}{5}&=&x \\[3mm] x&=&12.
\end{eqnarray*}
La longueur du rectangle est de 12 cm et sa largeur est de 4
cm. La r\'eponse est c). \\

1533-- Une automobile met neuf heures pour parcourir {\textrm{1 035
}}km. Quelle
est sa vitesse moyenne?\\
a) 115 km\\
b) 115 km/h\\
c) 120 km/h\\
d) La vitesse moyenne est impossible \`a trouver avec
les donn\'ees fournies dans l'\'enonc\'e du probl\`eme.\\

R\'eponse : b)\\

R\'etroaction :\\
La vitesse moyenne est donn\'ee par la formule suivante :
$$\frac{\textrm{nombre de km parcourus}}{\textrm{temps}}.$$\\[2mm]

On a donc
$$\frac{\textrm{1 035 km}}{\textrm{9 heures}}= {\textrm{115 km/h.}}$$\\[2mm]
Lorsqu'il est question de taux, les unit\'es sont tr\`es
importantes. La vitesse est souvent exprim\'ee
en km/h. La r\'eponse est donc 115 km/h, soit b).\\

1534-- Ton auto manque d'essence. Tu te rends donc \`a la
station-service pour faire le plein. Tu ajoutes exactement
35 L d'essence dans ton r\'eservoir, ce qui te co\^ute 25\,\$. Quel est le
co\^ut de l'essence au litre?\\
a) 0,71\,\$\\
b) 0,71\,\$/L\\
c) 1,40\,\$\\
d) 1,40\,\$/L\\

R\'eponse : b)\\

R\'etroaction :\\
Voil\`a comment il faut proc\'eder : $$\frac{\textrm{co\^ut de
l'essence}}{\textrm{quantit\'e d'essence}} =\frac{\textrm{25
dollars}}{\textrm{35 litres}}
=0,71\,\$/{\textrm{litre.}}$$\\

La r\'eponse est b).\\

1535-- Un avion supersonique atteint une vitesse de 380 m/s. Peux-tu
interpr\'eter cette vitesse en km/h?\\
a) 136,8 km/h\\
b) 380 km/h\\
c) 1 368 km/h\\
d) 13 680 km/h\\

R\'eponse : c)\\

R\'etroaction :\\
On doit convertir les m\`etres en kilom\`etres et les secondes en
heures. L'avion parcourt 380 m en 1 seconde, il parcourt donc 1 368
000 m en 1 heure. \,Il ne nous reste qu'\`a diviser par 1 000 pour
obtenir le nombre de kilom\`etres parcourus en 1 heure :
$$380\,{\textrm{m/s}}=1\,368\,000\,{\textrm{m/h}}=1\,368\,{\textrm{km/h.}}$$\\
La r\'eponse est c).\\

1536-- Deux stations-services sont en comp\'etition pour offrir le
meilleur prix aux clients : la station Ezzo offre 40 L pour 35\,\$
et la station P\'etro-Qu\'ebec 45 L pour 40\,\$. Quelle station
offre le meilleur
prix \`a ses clients?\\
a) La station Ezzo\\
b) La station P\'etro-Qu\'ebec\\
c) Les deux stations-services offrent l'essence au m\^eme prix.\\
d) Il manque des informations pour le d\'eterminer.\\

R\'eponse : a)\\

R\'etroaction :\\
Pour pouvoir comparer les prix offerts par les deux
stations-services, il faut trouver le prix du litre d'essence, ce
qui revient \`a trouver le taux pour chacune des stations. Pour la
station Ezzo :
$$\frac{\textrm{35\,\$}}{\textrm{40
L}}=0,88\,\$/\,{\textrm{L}}.$$\\[3mm]
Pour la station P\'etro-Qu\'ebec
:$$\frac{\textrm{40\,\$}}{\textrm{45
L}}=0,89\,\$/\,{\textrm{L}}.$$\\[3mm]
La station Ezzo offre donc un tarif un
peu plus bas que la station P\'etro-Qu\'ebec. La r\'eponse est a).\\

1537-- M. Bic veut se procurer des stylos. \`A la librairie, il peut
en acheter cinq pour 2,25\,\$, tandis qu'\`a l'\'ecole, il lui en
co\^utera 5,40\,\$ pour un paquet de 12. O\`u M. Bic
devrait-il acheter ses stylos? \\
a) \`A la librairie\\
b) \`A l'\'ecole\\
c) Les deux \'etablissements offrent le m\^eme taux unitaire.\\
d) Il est impossible de le d\'eterminer avec les indices qui sont fournis
dans l'\'enonc\'e.\\

R\'eponse : c)\\

R\'etroaction :\\
On doit comparer le prix unitaire des stylos pour chacune des
options de M. Bic. \`A la librairie, il paiera
$$\frac{\textrm{2,25\,\$}}{\textrm{5}}=0,45\,\$/{\textrm{stylo}}.$$ \`A son
\'ecole, il
paiera
$$\frac{\textrm{5,40\,\$}}{\textrm{12}}=0,45\,\$/{\textrm{stylo}}.$$\\
Le prix unitaire des stylos est donc identique dans les deux
\'etablissements. Peu importe la d\'ecision qu'il prendra, M. Bic
paiera ses stylos le m\^eme prix. La r\'eponse est c).\\

1538-- Jerry se rend au travail en 20 minutes. Il parcourt les cinq
premiers kilom\`etres \`a une vitesse moyenne de 60 km/h et il
augmente ensuite sa vitesse pour parcourir les 20 derniers
kilom\`etres. \`A quelle vitesse
moyenne parcourt-il la seconde partie du trajet?\\
a) 60 km/h\\
b) 80 km/h\\
c) 100 km/h\\
d) Cette vitesse est impossible \`a d\'eterminer avec les informations dont
on dispose.\\

R\'eponse : b)\\

R\'etroaction :\\
Nous devons calculer le temps mis par Jerry pour effectuer la
premi\`ere partie du parcours : $$60{\textrm{ km/h}}
=\frac{{\textrm{? km}}}{{\textrm{? h}}}=\frac{{\textrm{ 5 km}}}{{x
\textrm{ h}}}$$


$$x=\frac{5}{60}=\frac{1}{12}.\\[3mm] $$ Jerry a mis 5
minutes, c'est-\`a-dire $\frac{1}{12}$ d'heure, pour parcourir la
premi\`ere partie du trajet. Il lui reste donc 15 minutes,
c'est-\`a-dire $\frac{1}{4}$ d'heure, pour arriver au travail et 20
km \`a parcourir.
$${\textrm{vitesse}}=\frac{\textrm{ 20 km}}{\textrm{0,25 h}}={\textrm{ 80
km/h}}$$
La r\'eponse est b).\\

1539-- La cuisine d'une maison a \'et\'e repr\'esent\'ee sur un plan
\`a une \'echelle de 1:200. La longueur de la cuisine est de 6,5 cm
et la largeur de 3,8 cm. Quel est le p\'erim\`etre r\'eel de
cette cuisine?\\
a) 20,6 m\\
b) 41,2 m \\
c) 42,1 m\\
d) 49,2 m\\

R\'eponse : b)\\

R\'etroaction :\\
Les dimensions r\'eelles de la cuisine sont de $6,5\times
200=1\,300$ cm de longueur par $3,8\times 200=760$ cm de largeur. Le
p\'erim\`etre est donc $$P=2\cdot(L+l)=2\cdot(1\,300+760)={\textrm{
4 120 cm}}.$$ Il ne nous reste qu'\`a convertir les centim\`etres en
m\`etres, ce qui donne exactement 41,2
m. La r\'eponse est b).\\

1540-- Sur une carte, on mesure 4,8 cm entre deux villes qui, en
r\'ealit\'e, sont distantes de 2,4 km. Quelle est l'\'echelle de
cette carte?\\
a) 1:2 \\
b) 1:500\\
c) 1:5\,000\\
d) 1:50\,000\\

R\'eponse : d)\\

R\'etroaction :\\
Il faut convertir les deux distances pour qu'elles soient
exprim\'ees dans la m\^eme unit\'e de mesure. On a $2,4
\,{\textrm{km}}= 240\,000\,{\textrm{ cm}}.$\\ Ainsi, le rapport est
$$\frac{240\,000{\textrm{ cm}}}{4,8{\textrm{ cm}}}=50\,000.$$\\
L'\'echelle est donc 1:50\,000, c'est-\`a-dire que 1 cm sur la carte
est \'egal \`a 50 000
cm dans la r\'ealit\'e. La r\'eponse est d).\\

1541-- Sur un plan, la largeur d'un salon est de 3,7 cm et la
longueur de 5,4 cm. Si, dans la r\'ealit\'e, la largeur mesure 5,55
m, quelle est la longueur r\'eelle de ce salon?\\
a) 3,72 m\\
b) 7,9 m\\
c) 8,1 m\\
d) 9,2 m\\

R\'eponse : c)\\

R\'etroaction :\\
On doit d'abord trouver l'\'echelle de ce plan. La largeur du salon
sur le plan est de 3,7 cm et sa largeur r\'eelle est de 5,55 m,
c'est-\`a-dire 555 cm. Le rapport des largeurs est donc
$$\frac{\textrm{555 cm}}{\textrm{3,7 cm}}=150.$$ Ainsi, 1 cm sur le plan
mesure 150 cm dans la r\'ealit\'e. L'\'echelle est donc 1:150. Il
suffit maintenant de trouver la mesure r\'eelle de la longueur du
salon. Sur le plan, cette longueur est de 5,4 cm. Dans la
r\'ealit\'e, elle vaut $5,4\times 150 = 810$ cm = 8,1 m. \,\,La
r\'eponse est
donc c).\\

1542-- La masse de 11 sacs d'oranges est de 25 kilogrammes. Quelle
est la masse de six sacs d'oranges?\\
a) 12,5 kg\\
b) 13,6 kg\\
c) 18,1 kg\\
d) 150 kg\\

R\'eponse : b)\\

R\'etroaction :\\
Nous devons trouver la masse de chaque sac d'oranges. $$M =
\frac{\textrm{25 kg}}{\textrm{11  sacs}}=2,27 {\textrm{
kg/sac.}}$$\\
Afin de trouver la masse de six sacs, effectuons le
calcul suivant : $$6\times 2,27 = 13,6{\textrm{ kg}}.$$ La r\'eponse est
b).\\

1543-- Un sondage d\'emontre qu'en moyenne cinq \'etudiants sur neuf
pr\'ef\`erent l'\'ecole primaire au secondaire. Parmi les 243
\'etudiants interrog\'es, combien pr\'ef\`erent l'\'ecole primaire?\\
a) 122 \'etudiants\\
b) 133 \'etudiants\\
c) 135 \'etudiants\\
d) 2 187 \'etudiants\\

R\'eponse : c)\\

R\'etroaction :\\
Les $\frac{5}{9}$ des 243 \'el\`eves interrog\'es pr\'ef\`erent le
primaire au secondaire. $$\frac{5}{9}\times 243 = 135$$ Parmi les
\'el\`eves interrog\'es, il y en a donc 135 qui pr\'ef\`erent le
primaire. La r\'eponse est donc c).\\

1544-- Sur une photo, un homme mesure 8 cm et son chien 1,5 cm.
Quelle est la taille r\'eelle de l'homme si le chien mesure
35 cm en r\'ealit\'e?\\
a) 1,87 m\\
b) 1,92 m\\
c) 6,56 m\\
d) 186,7 m\\

R\'eponse : a)\\

R\'etroaction :\\
Le rapport de longueur entre l'homme et son chien est de
$\frac{8}{1,5}=5,33$. L'homme est donc 5,33 fois plus grand que son
chien. Si ce dernier mesure 35 cm, son propri\'etaire mesure
$35\times5,33=186,67$
cm, ce qui \'equivaut \`a 1,87 m. La r\'eponse est a).\\

1545-- Si une moto roule \`a une vitesse moyenne de 90 km/h, quelle
distance
parcourt-elle en 40 minutes?\\
a) 60 km\\
b) 60 km/h\\
c) 80 km\\
d) 80 km/h\\

R\'eponse : a)\\

R\'etroaction :\\
\`A cette vitesse, la moto parcourt 90 km en une heure. On veut
savoir combien de kilom\`etres elle parcourt en $\frac{40}{60}$
d'heure, soit $\frac{2}{3}$ d'heure.
$$ \frac{2}{3}\cdot 90=60$$ La moto parcourt donc 60
km. La r\'eponse est a).\\

1546-- Un robinet remplit une chaudi\`ere de six litres en neuf
secondes. Au m\^eme d\'ebit, combien de temps lui faudrait-il pour
remplir une baignoire
de 60 L?\\
a) 40 secondes\\
b) 60 secondes\\
c) 75 secondes\\
d) 90 secondes\\

R\'eponse : d)\\

R\'etroaction :\\
Le robinet prend neuf secondes pour remplir la chaudi\`ere de six
litres. Comme il doit remplir une baignoire dix fois plus grande que
la chaudi\`ere, cela lui prendra dix
fois plus de temps, soit $9\times 10=90$ secondes. La r\'eponse est donc
d).\\

1547-- Dans une classe, 16 des 25 \'el\`eves sont des filles. Quel
pourcentage des \'el\`eves de cette classe sont des gar\c cons?\\
a) 9\,\% \\
b) 16\,\% \\
c) 36\,\% \\
d) 64\,\% \\

R\'eponse : c)\\

R\'etroaction :\\
Il y a neuf gar\c cons sur un total de 25 \'el\`eves. Il nous reste
\`a d\'eterminer le pourcentage que repr\'esente $\frac{9}{25}.$
$$\frac{9}{25}\cdot 100=36\,\%$$\\
La r\'eponse est c).\\

1548-- M. Lazyboy obtient un rabais de 75\,\$ sur un confortable
fauteuil inclinable de
400\,\$. Quel est le pourcentage de rabais obtenu?\\
a) 18,75\,\%\\
b) 25\,\% \\
c) 75\,\%\\
d) 81,25\,\%\\

R\'eponse : a)\\

R\'etroaction :\\
M. Lazyboy obtient 75\,\$ de rabais sur un fauteuil de 400\,\$.
Cherchons \`a quel pourcentage \'equivaut la fraction
$\frac{75}{400}.$
$$\frac{75}{400}\cdot100=18,75\,\%$$\\
La r\'eponse est a).\\

1549-- Si le vendeur du magasin de souliers Les P'tits Pieds te dit
qu'il te donne 30\,\% de rabais sur des chaussures de
130\,\$, quel prix total paieras-tu pour tes chaussures?\\
a) 39\,\$\\
b) 70\,\$\\
c) 91\,\$\\
d) 100\,\$\\

R\'eponse : c)\\

R\'etroaction :\\Calculons le montant du rabais :
$$\frac{30}{100}\cdot130\,\$= 39\,\$.$$\\ Voyons maintenant le prix auquel
tu paieras tes
chaussures. Le prix r\'egulier des
chaussures \'etant de 130\,\$, tu dois les payer 130\,\$ - 39\,\$,
c'est-\`a-dire 91\,\$. La r\'eponse est donc c).\\

1550-- Si la population mondiale s'\'el\`eve \`a 4 845 millions de
personnes, quel pourcentage de cette population habite dans un pays ayant
492 millions d'habitants?\\
a) 10,2\,\%\\
b) 89,8\,\%\\
c) 4,92\,\%\\
d) 9,85\,\%\\

R\'eponse : a)\\

R\'etroaction :\\
On doit d'abord trouver le pourcentage \'equivalent \`a
$\frac{492}{4\,845}: $ \\
$$\frac{492}{4\,\,845}\cdot
100=10,2\,\%.$$\\
La r\'eponse est a).\\

1551-- Tu places un capital de 1 300\,\$ \`a la banque. Un an plus
tard, le solde de ton compte s'\'el\`eve \`a 1 372,80\,\$. Exprime
par un pourcentage
l'int\'er\^et annuel obtenu sur le capital investi.\\\\
a) 0,053\,\%\\
b) 0,056\,\%\\
c) 5,3\,\%\\
d) 5,6\,\%\\

R\'eponse : d)\\

R\'etroaction :\\
Pour calculer le pourcentage d'int\'er\^et, on doit diviser le
montant de l'int\'er\^et par le capital investi.
$$\frac{72,80\,\$}{1\,300\,\$}\cdot100=5,6\,\%.$$\\
La r\'eponse est donc d).\\

1552-- Tu adores la revue mensuelle Math-o-Math. Tu as le choix entre
acheter chaque mois cette revue en kiosque au co\^ut de 3,75\,\$, ou
bien t'abonner pour un an pour la somme de 36\,\$. Exprime par un
pourcentage l'\'economie que tu r\'ealiseras avec l'abonnement
comparativement au prix en kiosque.\\
a) 10\,\%\\
b) 15\,\%\\
c) 20\,\%\\
d) 25\,\%\\

R\'eponse : c)\\

R\'etroaction :\\
Calculons d'abord le co\^ut de l'achat des 12 revues en kiosque :
$$12\cdot{\textrm{3,75\,\$}}={\textrm{45\,\$}}$$\\
Si tu ach\`etes tes revues \`a l'unit\'e, tu devras payer 45\,\$ au total.
L'abonnement annuel co\^ute 36\,\$. Tu sauves donc 9\,\$ en
choisissant l'abonnement annuel plut\^ot que l'achat en kiosque.
Cherchons maintenant le pourcentage de rabais que cela repr\'esente.
$$\frac{9}{45}\cdot100=20\,\%.$$\\ La r\'eponse est c).\\

1553-- Dans une classe, on compte 55\,\% de filles. Parmi tous les
gar\c cons, 60\,\% ne jouent pas au ballon durant les
r\'ecr\'eations. Quel
pourcentage de la classe repr\'esente les gar\c cons jouant au ballon?\\
a) 18\,\%\\
b) 27\,\%\\
c) 40\,\%\\
d) 45\,\%\\

R\'eponse : a)\\

R\'etroaction :\\
Dans la classe, 45\,\% des \'el\`eves sont des gar\c cons et parmi
ceux-ci, 40\,\% jouent au ballon. On doit donc calculer $45\,\%$ de
$40\,\%$ :
\\$$\frac{45}{100}\cdot\frac{40}{100}=\frac{18}{100},{\textrm{ soit
18\,\%.}}$$\\
La r\'eponse est donc a).\\

1554-- Pour le d\'ebut des classes, Anne ach\`ete un sac d'\'ecole
\'etiquet\'e 58,60\,\$, mais sur lequel elle obtient 20\,\% de
rabais. Elle se procure \'egalement un stylo d'une valeur de
18,50\,\$ portant une \'etiquette indiquant un rabais de 15\,\%.
Combien Anne paye-t-elle pour ces deux articles?\\
a) 61,68\,\$\\
b) 62,61\,\$\\
c) 65,53\,\$\\
d) 77,10\,\$\\

R\'eponse : b)\\

R\'etroaction :\\
Nous devons calculer le prix de chacun des articles apr\`es
r\'eduction.\\

Le sac co\^ute $58,60\,\$-(\frac{20}{100}\cdot58,60\,\$)=46,88\,\$$
et le stylo revient \`a
$18,50\,\$-(\frac{15}{100}\cdot18,50\,\$)=15,73\,\$$.\\ Il est
maintenant possible de calculer la somme du prix des deux articles.
\\$$46,88\,\$+15,73\,\$=62,61\,\$.$$ La r\'eponse est donc b).\\

1555-- Dans un pays, 21\,\% de la population totale a comme langue
maternelle l'anglais. Si ce pourcentage repr\'esente exactement 5
047 346 personnes,
quelle est la population totale de ce pays?\\
a) 1 624 765 habitants\\
b) 23 144 796 habitants\\
c) 24 034 981 habitants\\
d) 24 996 380 habitants\\

R\'eponse : c)\\

R\'etroaction :\\
Tout d'abord, posons $x$ comme \'etant la population totale du pays.
On sait que $21\,\%$ de $x$ donne {\textrm{5 047 346}}. Il suffit
d'isoler $x$. \\$$x=\frac{\textrm{5 047 346}}{0,21}={\textrm{24 034
981}}$$\\
Par cons\'equent, la r\'eponse est c).\\

1556-- Une entreprise doit r\'eduire sa production de 2 070  produits
par ann\'ee. Si ce chiffre repr\'esente 9\,\% de la production
totale, quelle
\'etait cette production avant la diminution?\\
a) 920 produits\\
b) 8 280 produits \\
c) 18 400 produits\\
d) 23 000 produits\\

R\'eponse : d)\\

R\'etroaction :\\
Tout d'abord, posons $x$ comme \'etant la production initiale totale
de l'entreprise. On sait que 9\,\% de $x$ donne {\textrm{2 070}}. On
doit maintenant isoler $x$ et on obtient
\\$$x=\frac{2\,070}{9\,\%}=\frac{2\,070}{0,09}={\textrm{23 000.}}$$\\
La r\'eponse est donc d).\\

1557-- M. Bureau a parcouru 30\,\% du trajet de son bureau jusqu'\`a
sa maison. Si 96 km les s\'eparent, quelle distance lui reste-t-il
\`a
parcourir?\\
a) 28,8 km\\
b) 30 km\\
c) 66 km\\
d) 67,2 km \\

R\'eponse : d)\\

R\'etroaction :\\
En premier lieu, on doit calculer la distance parcourue par M.
Bureau. $$30\,\%{\textrm{ de 96 km}}=\frac{30}{96}\cdot
100={\textrm{28,8 km.}}$$\\ Il lui reste donc $96 - 28,8=
{\textrm{67,2
km }}$\`a parcourir. La r\'eponse est d).\\

1558-- Lors d'une temp\^ete de neige, la vitesse de la circulation
est r\'eduite \`a 25\,\% de sa vitesse r\'eguli\`ere. Si Olivier a
normalement besoin de 15 minutes pour se rendre au travail, combien
de temps cela lui prendra-t-il aujourd'hui
alors qu'il y a une temp\^ete \`a l'ext\'erieur?\\
a) 3,75 min\\
b) 11,25 min \\
c) 60 min\\
d) 75 min\\

R\'eponse : c)\\

R\'etroaction :\\
Lorsque la vitesse de la circulation est r\'eduite \`a 25\,\% de sa
vitesse r\'eguli\`ere, cela signifie que les gens circulent quatre
fois moins vite qu'\`a l'habitude. Olivier aura donc besoin de
quatre fois plus de temps pour se rendre au travail, soit
$4\cdot15=60${\textrm{ minutes.}}
La r\'eponse est c). \\

1559-- Tes voisins, M. et Mme House, ont vers\'e \`a leur agente
immobili\`ere une commission de 8,5\,\% du montant total de  la
vente de leur maison. Si la valeur de cette commission se monte \`a
11 050\,\$, quel est le
montant de vente de la maison de tes voisins?\\
a) 13 000\,\$\\
b) 122 777,78\,\$\\
c) 130 000\,\$\\
d) 138 125\,\$\\

R\'eponse : c)\\

R\'etroaction  :\\
Posons d'abord $x$ comme \'etant le montant total de la vente de la
maison des House. Alors, {\textrm{8,5\,\% de $x$ donne }}11 050\,\$.
$$x=\frac{\textrm{11 050}}{0,085}={\textrm{130 000}}\,\$$$\\
La r\'eponse est c).\\

1560-- Une skieuse a gagn\'e la m\'edaille d'or aux Jeux Olympiques.
Elle a parcouru 27 km en 93 minutes 46 secondes. Quelle a
\'et\'e la vitesse moyenne (en m/s) de cette athl\`ete?\\
a) 17,274 km/h\\
b) 0,480 m/s \\
c) 4,808 m/s\\
d) 4,799 m/s\\

R\'eponse  : d)\\

R\'etroaction :\\
Comme on d\'esire exprimer la r\'eponse en m/s, on doit convertir la
distance en m\`etres et le temps en secondes. \vskip 15pt 27 km = 27
000 m \vskip 15pt 93 min 46 s = $93\times60+46=${\textrm{ 5 626}}
secondes \vskip 20pt \noindent Il ne nous reste qu'\`a obtenir la
vitesse de la skieuse.
$$\frac{\textrm{27 000}}{\textrm{5 626}}=4,799$$ L'athl\`ete a donc
effectu\'e
la course \`a une vitesse moyenne de 4,799 m/s. La r\'eponse est donc d).\\

1561-- En 1991, la population du Canada \'etait de $26,40$ millions
d'habitants et celle des \'Etats-Unis de $252,69$ millions
d'habitants. La superficie du Canada est de {\textrm{9 976 139}}
km$^2$ et celle des \'Etats-Unis de {\textrm{9 363 123}} km$^2$. De
combien de fois la densit\'e de la population des \'Etats-Unis (en
hab/{km$^2$}) est-elle plus grande
que celle du Canada?\\
a) 5 fois\\
b) 8 fois\\
c) 9 fois\\
d) 10 fois\\

R\'eponse : d)\\

R\'etroaction :\\
Calculons la densit\'e de la population de chacun des deux pays.
Pour ce faire, il faut diviser le nombre d'habitants par la
superficie du pays. \vskip 10pt Pour les \'Etats-Unis
:$$\frac{\textrm{252 690 000}}{\textrm{9 363 123}} ={\textrm{26,987
79}} {\textrm{ hab/km$^2$.}}$$ \vskip 10pt Pour le Canada
:$$\frac{\textrm{26 400 000}}{\textrm{9 976 139}}={\textrm{2,646 314
4}} {\textrm{ hab/km$^2$.}}$$\\

Faisons ensuite le rapport entre les densit\'es des populations des
deux pays :
$$\frac{\textrm{26,987 79}}{\textrm{2,646 314
4}}={\textrm{10,2.}}$$\\
La population des \'Etats-Unis est donc environ dix fois plus
dense que celle du Canada.\, La r\'eponse est d).\\


1562-- Dans la confiserie Au Bonbon G\^ateau, on pr\'epare des
paquets contenant six chocolats et quatre bonbons. Combien de
bonbons doit-on avoir si on veut utiliser tous les 144 chocolats
dont on dispose? \\
a) 96 bonbons\\
b) 142 bonbons\\
c) 144 bonbons\\
d) 216 bonbons\\

R\'eponse : a)\\

R\'etroaction :\\
Le rapport entre les bonbons et les chocolats est de $\frac{4}{6}$.
Si on a 144 chocolats, on doit avoir exactement
$$144\cdot\frac{4}{6}={\textrm{96 bonbons.}}$$\\ La r\'eponse est a).\\

1563-- M. Tartempion doit cuisiner une tarte et sa recette est la
suivante : 200 grammes de farine, 160 grammes de beurre et 120
grammes de sucre. Il a cependant un petit probl\`eme : il ne reste
dans son garde-manger que 125 grammes de farine! Pour respecter les
proportions de la recette originale,
quelle masse de beurre doit-il utiliser?\\
a) 85 g\\
b) 100 g\\
c) 115 g\\
d) 160 g\\

R\'eponse : b)\\

R\'etroaction :\\
Il faut conserver les m\^emes proportions de chacun des
ingr\'edients pour ne pas modifier la recette. On doit passer de 200
grammes \`a 125 grammes de farine. On a donc comme rapport
$\frac{125}{200}=0,625$. Calculons maintenant la quantit\'e de
beurre n\'ecessaire : $$160\cdot0,625={\textrm{100 g.}}$$ La r\'eponse est
b).\\

1564-- Tu disposes de deux tuyaux d'arrosage pour remplir ta piscine.
Lorsque tu utilises uniquement un des deux tuyaux \`a la fois, le
premier tuyau (A) permet de remplir ta piscine en 12 heures, tandis
que le deuxi\`eme (tuyau B) effectue cette op\'eration en 8 heures.
Combien de temps faudra-t-il pour remplir ta piscine en utilisant
les deux tuyaux en m\^eme temps?\\
a) 4,8\,h\\
b) 8\,h\\
c) 6,8\,h\\
d) 20\,h\\

R\'eponse : a)\\

R\'etroaction :\\
Posons $x$ comme \'etant le nombre de litres d'eau dans ta piscine.
Pour le tuyau A, le taux de variation du volume d'eau en fonction du
temps est de $\frac{x}{12}$ L/h. Quant au tuyau B, ce taux est de
$\frac{x}{8}$ L/h. Pour calculer le temps n\'ecessaire pour
remplir la piscine avec les deux tuyaux d'arrosage, on doit isoler
$x$ dans l'\'equation suivante :$$\frac{x}{12}\cdot
x+\frac{x}{8}\cdot x=x.$$
\begin{eqnarray*}
2x^2+3x^2&=&24x \\ 5x&=&24 \\ x&=&4,8
\end{eqnarray*}\\
La r\'eponse est a).\\

1565-- Tu gagnes 24\,\$ par semaine en tondant la pelouse des
voisins. Tu d\'esires t'acheter une bicyclette valant 124,95\,\$
avec l'argent ainsi accumul\'e. Apr\`es combien de semaines
pourras-tu t'offrir cette bicyclette si le magasin V\'elo-Sport
t'accorde un rabais de 15\,\%
avant le calcul de la taxe et que celle-ci est de 9\,\%?\\
a) 4,8 semaines\\
b) 5,2 semaines \\
c) 5,5 semaines\\
d) 6,5 semaines\\

R\'eponse : a)\\

R\'etroaction :\\
Trouvons tout d'abord le prix de la bicyclette.
\begin{eqnarray*}
{\textrm{prix}}&=&124,95-(15\,\%\cdot124,95)+9\,\% \\ &=&124,95-18,74+9\,\%
\\
&=&106,21+(9\,\%\cdot106,21) \\&=&106,21+9,56 \\&=&115,77
\end{eqnarray*}
Divisons maintenant ce prix par le montant que tu gagnes chaque
semaine.
$$\frac{115,77}{24}=4,8$$\\
La r\'eponse est donc a).\\

1566-- Au cin\'ema, lors de la projection d'un certain film, 20\,\%
de femmes adultes et 25\,\% d'hommes adultes font partie de
l'auditoire. Combien y a-t-il
d'enfants si 40 personnes assistent \`a cette repr\'esentation?\\
a) 10 enfants\\
b) 18 enfants\\
c) 22 enfants\\
d) 55 enfants\\

R\'eponse : c)\\

R\'etroaction :\\
Il y a $100\,\%-(20\,\%+25\,\%)=55\,\%$ d'enfants dans l'auditoire.
Comme il y a 40 personnes, le nombre
d'enfants est $$\frac{55}{100}\cdot40=22.$$\\ Il y a donc 22 enfants dans la
salle de cin\'ema. La r\'eponse est c).\\

1567-- Que se passe-t-il si on fait une homoth\'etie de centre $O$ et
de rapport $-2$ sur une figure?\\
a) Elle rapetisse et change de c\^ot\'e par rapport au centre $O$.\\
b) Elle rapetisse et reste du m\^eme c\^ot\'e par rapport au centre $O$.\\
c) Elle s'agrandit et change de c\^ot\'e par rapport au centre $O$.\\
d) Elle s'agrandit et reste du m\^eme c\^ot\'e par rapport au centre $O$.\\

R\'eponse : c)\\

R\'etroaction :\\
Un rapport positif garde la figure du m\^eme c\^ot\'e du centre $O$.
Lorsqu'il est n\'egatif, la figure change de c\^ot\'e. \vskip 15pt
\noindent On doit regarder la valeur du rapport sans se soucier du
signe. Un rapport sup\'erieur \`a  1 agrandit la figure, alors qu'il
la rapetisse s'il est inf\'erieur \`a 1. \vskip 15pt \noindent Dans
le cas d'un rapport de -2, la figure change de c\^ot\'e et
s'agrandit. La r\'eponse est c).\\

1568-- Laquelle des r\`egles suivantes repr\'esente une homoth\'etie?\\
a) $(x,\,y)\mapsto(2x,\,4y)$\\
b) $(x,\,y)\mapsto(x,\,-y)$\\
c) $(x,\,y)\mapsto(x+1,\,y+2)$\\
d) $(x,\,y)\mapsto 2\cdot(x,\,y)$\\

R\'eponse : d)\\

R\'etroaction :\\
Par d\'efinition, pour avoir une homoth\'etie, on doit multiplier le $x$ et
le $y$ par un facteur identique pour obtenir une figure semblable. Le seul
choix qui respecte cette r\`egle est le d), dans lequel le rapport est de
deux. La r\'eponse est d).\\

1569-- Quelle est l'image du point (3, -2) s'il subit une translation
$(x,\,\,y-4)$?\\
a) (1, -6)\\
b) (3, 2) \\
c) (3, -6)\\
d) (3, 8)\\

R\'eponse : c)\\

R\'etroaction :\\
Il suffit d'appliquer la r\`egle $(x,\,y)\mapsto(x,\,y-4)$ au point
(3,\,-2). On a
donc $$(3, -2-4)=(3, -6).$$ La r\'eponse est c).\\

1570-- Si l'image du point (1, 8) apr\`es transformation est (4, 7), quelle
\'etait la r\`egle de cette transformation?\\
a) $(x,\,y)\mapsto(x+3,\,y+1)$\\
b) $(x,\,y)\mapsto(4x,\,y-1)$\\
c) $(x,\,y)\mapsto(x+3,\,y-1)$\\
d) $(x,\,y)\mapsto(x-1,\,y+3)$\\

R\'eponse : c)\\

R\'etroaction :\\
On doit trouver ce qu'on a ajout\'e pour passer de 1 \`a 4 en $x$ et
de 8 \`a 7 en $y$. En $x$, on a ajout\'e 3 unit\'es et en $y$, on a
enlev\'e 1 unit\'e. La r\`egle est donc
$$(x,\,y)\mapsto(x+3,\,y-1).$$ La r\'eponse est c).\\

1571-- Quel \'etait le point A si son image A' est (6, 10) et que la
transformation suivait la r\`egle $h$$(O ,\,2)$ (homoth\'etie de
centre
$O$ et de rapport 2)?\\
a) (3,\,5)\\
b) (3,\,10) \\
c) (4,\,8)\\
d) (12,\,20)\\

R\'eponse : a)\\

R\'etroaction :\\
On cherche $(x,\,y)$ tel que $(2x,\,2y)=(6,\,10)$. On a donc que
$x=3$
et $y=5$. Le point A est (3,\,5). La r\'eponse est a).\\

1572-- Quelle est l'image du point (2, 6) subissant l'homoth\'etie
$h$$\,(O ,\frac{1}{2})$?\\
a) (1, 2)\\
b) (1, 3)\\
c) (3, 1)\\
d) (4, 10)\\

R\'eponse : b) \\

R\'etroaction :\\
Voici le calcul \`a faire pour cette homoth\'etie :$$(2,\,6)\mapsto
\frac{1}{2}\,(2,\,6)=(\frac{1}{2}\,\cdot 2,\,\,\frac{1}{2}\cdot
6)=(1,\,3).$$\\
La r\'eponse est b).\\

1573-- Laquelle des transformations suivantes fait subir un
agrandissement \`a
une figure?\\
a) Homoth\'etie\\
b) R\'eflexion\\
c) Rotation\\
d) Translation \\

R\'eponse : a)\\

R\'etroaction :\\
L'homoth\'etie est la seule des quatre transformations qui peut
changer les
dimensions des figures. Les autres pr\'eservent la longueur des c\^ot\'es.
La r\'eponse est a).\\

1574-- Lequel des \'enonc\'es suivants est faux?\\
a)  Lors d'une translation, d'une r\'eflexion ou d'une homoth\'etie, on doit
modifier chaque point d'une figure par la m\^eme r\`egle pour obtenir
l'image de ce point.\\
b) Faire subir une rotation de $90^{\circ}$ ou de $-90^{\circ}$ \`a
une figure permet d'obtenir la m\^eme image.\\
c) Une homoth\'etie peut, selon son rapport, r\'eduire ou agrandir
des figures.\\
d) $(x,\,y)\mapsto(x,\,-y)$ est la r\`egle d'une r\'eflexion par
rapport \`a
l'axe des $x$.\\

R\'eponse : b)\\

R\'etroaction :\\
Une rotation de $180^{\circ}$ dans un sens ou dans l'autre engendre
exactement la m\^eme image, mais le fait de r\'ealiser des rotations
de $90^\circ$ et de
$-90^\circ$ g\'en\`ere deux images diff\'erentes. La r\'eponse est b).\\

1575-- Dans quel quadrant se trouve l'image du point (5, -6) si on
lui fait subir une r\'eflexion
par rapport \`a l'axe des $y$?\\
a) $1^{er}$ quadrant\\
b) $2^e$ quadrant\\
c) $3^e$ quadrant\\
d) $4^e$ quadrant\\

R\'eponse : c)\\

R\'etroaction :\\
Le point (5, -6) est dans le $4^e$ quadrant. Si on lui fait subir
une r\'eflexion par rapport \`a l'axe des $y$, on retrouve l'image
de ce point dans le $3^e$ quadrant. La r\'eponse est c).\\

1576-- Un rectangle ayant un p\'erim\`etre de 18 unit\'es mesure six
unit\'es de longueur et trois de largeur. Si on lui fait subir la
translation $t$$\,(x+3,\,\,y+1)$, quel est le p\'erim\`etre de l'image?\\
a) 18 unit\'es\\
b) 20 unit\'es\\
c) 22 unit\'es\\
d) 26 unit\'es\\

R\'eponse : a)\\

R\'etroaction :\\
Lorsqu'on fait une translation, on d\'eplace les points. Cependant,
les dimensions des c\^ot\'es de la figure restent les m\^emes. Le
p\'erim\`etre reste donc de 18 unit\'es. La r\'eponse est a).\\

1577-- Quelle est la mesure d'un angle int\'erieur d'un pentagone r\'egulier
?\\
a) $72^{\circ}$\\
b) $108^{\circ}$\\
c) $120^{\circ}$\\
d) $136^{\circ}$\\

R\'eponse : b)\\

R\'etroaction :\\
La somme des angles int\'erieurs d'un polygone r\'egulier \`a $n$
c\^ot\'es est d\'etermin\'ee par la formule
$$(n-2)\cdot180^{\circ}.$$\\\\
On a donc $$(5-2)\cdot180=540^{\circ}.$$\\

Comme la somme des cinq angles est de $540^{\circ}$, pour trouver la
mesure d'un angle, nous devons faire le calcul
$$\frac{540}{5}=108^{\circ}.$$ La r\'eponse est b).\\

1578-- Quelle est la somme des angles int\'erieurs d'un octogone
r\'egulier?\\
a) $360^{\circ}$\\
b) $900^{\circ}$\\
c) $1\,080^{\circ}$\\
d) $1\,260^{\circ}$\\

R\'eponse : c)\\

R\'etroaction :\\
La somme des angles int\'erieurs d'un polygone r\'egulier \`a $n$
c\^ot\'es est d\'etermin\'ee par la formule
$$(n-2)\cdot180^{\circ}.$$
On a donc $$(8-2)\cdot180^{\circ}=6\cdot180=1\,080^{\circ}.$$\\
La r\'eponse est c).\\

1579-- Quel est le nom d'un polygone r\'egulier \`a neuf c\^ot\'es?\\
a) Enn\'eagone\\
b) Hend\'ecagone\\
c) Nanod\'ecagone\\
d) Nanogone\\

R\'eponse : a)\\

R\'etroaction :\\
Le nom d'un polygone r\'egulier \`a neuf c\^ot\'es est enn\'eagone.
La r\'eponse est a).
\\

1580-- Que vaut la somme des mesures des angles ext\'erieurs d'un polygone?\\
a) $180^{\circ}$\\
b) $360^{\circ}$\\
c) $540^{\circ}$\\
d) Cela d\'epend de chaque polygone.\\

R\'eponse : b)\\

R\'etroaction :\\
La somme des angles ext\'erieurs d'un polygone est toujours $360^{\circ}.$
La r\'eponse est b).\\

1581-- Quel est le p\'erim\`etre d'un hexagone r\'egulier de 7 cm
de c\^ot\'e?\\
a) 35 cm\\
b) 42 cm\\
c) 49 cm\\
d) Il nous manque l'apoth\`eme pour le d\'eterminer.\\

R\'eponse : b)\\

R\'etroaction :\\
Chaque c\^ot\'e de l'hexagone r\'egulier mesure 7 cm. Un hexagone
est un polygone \`a six c\^ot\'es. Le p\'erim\`etre est donc de $6\times
7{\textrm{ cm}}$= $42{\textrm{ cm}}$. La r\'eponse est b).\\

1582-- Le p\'erim\`etre d'un polygone r\'egulier est de 56 cm. Si chaque
c\^ot\'e mesure 8 cm, quel est le nom de ce polygone?\\
a) Hexagone\\
b) Heptagone\\
c) Octogone\\
d) On ne peut le d\'eterminer car on ne conna\^it pas le nombre de
c\^ot\'es du polygone.\\

R\'eponse : b)\\

R\'etroaction :\\
Si on pose $n$ comme \'etant le nombre de c\^ot\'es et $m$ la mesure
d'un c\^ot\'e, la formule du p\'erim\`etre est $$P=n\times m.$$ On
doit donc isoler $n$ :
\begin{eqnarray*}
P&=&n\times m \\ 56&=&n\times 8 \\ \frac{56}{8}&=&n \\[3mm] 7&=&n.
\end{eqnarray*}
Le polygone poss\`ede sept c\^ot\'es et se nomme donc heptagone. La
r\'eponse est b).\\

1583-- Je suis un polygone r\'egulier dont la mesure de l'angle
ext\'erieur est
$30^{\circ}$. Qui suis-je?\\
a) Enn\'eagone\\
b) Pentagone \\
c) D\'ecagone\\
d) Dod\'ecagone\\

R\'eponse : d)\\

R\'etroaction :\\
La somme des angles ext\'erieurs d'un polygone est \'egale \`a
$360^{\circ}$. En divisant $360^{\circ}$ par la mesure d'un angle
ext\'erieur, on obtient le nombre de c\^ot\'es du polygone.
$$n=\frac{360^{\circ}}{30^{\circ}}=12$$ Le polygone poss\`ede 12
c\^ot\'es et se nomme donc dod\'ecagone. La r\'eponse est d).\\

1584-- Quel est le p\'erim\`etre d'un polygone r\'egulier dont la
mesure de l'angle int\'erieur est $135^{\circ}$ et dont la mesure
d'un c\^ot\'e est de 12 cm?\\
a) 96 cm\\
b) 84 cm\\
c) 108 cm\\
d) 124 cm\\

R\'eponse : a)\\

R\'etroaction :\\
\`A l'aide de la formule de la somme des angles int\'erieurs, on
trouve
\begin{eqnarray*}
(n-2)\cdot180^{\circ}&=&n\cdot135^{\circ} \\ 180n-360^{\circ}&=&135^{\circ}n
\\
45n&=&360 \\ n&=&8.
\end{eqnarray*}
On est en pr\'esence d'un octogone de 12 cm de c\^ot\'e. Le
p\'erim\`etre est donc $8\times12=96${\textrm{ cm}}. La r\'eponse est a).\\

1585-- Quelle est l'aire d'un d\'ecagone r\'egulier sachant que son
apoth\`eme mesure 12,5 cm et que chaque c\^ot\'e mesure 8,2 cm
?\\
a) $102,5{\textrm{ cm}}^2$\\
b) $512,5{\textrm{ cm}}^2$\\
c) $615{\textrm{ cm}}^2$\\
d) $637,5{\textrm{ cm}}^2$\\

R\'eponse : b)\\

R\'etroaction :\\
Posons $A$ comme \'etant l'aire du polygone, $P$ le p\'erim\`etre et
$ap$ l'apoth\`eme. L'aire d'un polygone se calcule par la formule
suivante :$$A=\frac{P\times ap}{2}.$$Nous avons
donc$$A=\frac{10\times8,2\times12,5}{2}=\frac{1\,025}{2}=512,5{\textrm{
cm}}^2.$$\\ La r\'eponse est b).\\

1586-- Calcule l'aire d'un polygone r\'egulier sachant que son
apoth\`eme mesure 6,88 cm, son c\^ot\'e, 10 cm et la somme des
mesures de ses angles int\'erieurs, $540^{\circ}.$\\
a) $103,2{\textrm{ cm}}^2$\\
b) $137,6{\textrm{ cm}}^2$\\
c) $172{\textrm{ cm}}^2$\\
d) $206,4{\textrm{ cm}}^2$\\

R\'eponse : c)\\

R\'etroaction :\\
Nous devons tout d'abord trouver le nombre de  c\^ot\'es constituant
le polygone. La somme des angles \'etant $540^{\circ}$, on peut
effectuer les calculs suivants :

\begin{eqnarray*}
(n-2)\cdot180^{\circ}&=&540^{\circ} \\
(n-2)&=&\frac{540^{\circ}}{180 ^{\circ}}\\[3mm]
n&=&\frac{540^{\circ}}{180 ^{\circ}}+ 2\\[3mm]
n&=&5.\\
\end{eqnarray*}

On est donc en pr\'esence d'un pentagone. Le p\'erim\`etre est de 50
cm et l'apoth\`eme de 6,88 cm. L'aire est donc $$A=\frac{P\times
ap}{2}=\frac{50\times6,88}{2}=172
{\textrm{ cm}}^2.$$\\ La r\'eponse est c).\\

1587-- Sur un terrain rectangulaire de 12 m par 14 m, on installe une
piscine en forme d'hexagone r\'egulier de 5 m de c\^ot\'e.
L'apoth\`eme de l'hexagone mesurant 4,3 m, quelle est l'aire du terrain non
occup\'e par la piscine?\\
a) $39{\textrm{ m}}^2$\\
b) $64,5{\textrm{ m}}^2$\\
c) $103,5{\textrm{ m}}^2$\\
d) $168{\textrm{ m}}^2$\\

R\'eponse : c)\\

R\'etroaction :\\
L'aire du terrain est $$A=L\times l=14\times 12=168{\textrm{
m}}^2.$$\\ L'aire de la piscine est $$A=\frac{P\times
ap}{2}=\frac{30\times 4,3}{2}=64,5{\textrm{ m}}^2.$$\\
L'aire du terrain restant est donc
$$168-64,5=103,5{\textrm{ m}}^2.$$\\
La r\'eponse est c).\\

1588-- Ton professeur te pose la question suivante : Combien
obtiens-tu de chiffres apr\`es la virgule lorsque tu mets au carr\'e
un nombre poss\'edant
$n$ d\'ecimales? \\
a) $n$\\
b) $2n$\\
c) $n^2$\\
d) Cela varie avec chaque nombre.\\

R\'eponse : b)\\

R\'etroaction :\\
Lorsqu'on met au carr\'e un nombre poss\'edant $n$ d\'ecimales, on
obtient un nombre poss\'edant $2n$ d\'ecimales. La r\'eponse est donc b).\\

1589-- Laquelle des \'egalit\'es suivantes est vraie?\\
a) $3^2+4^2=5^2$\\
b) $3^2+7^2=(3+7)^2$ \\
c) $4^2+5^2=6^2$\\
d) $5^2+6^2=11^2$\\

R\'eponse : a)\\

R\'etroaction :\\
On n'a qu'\`a v\'erifier :
\begin{eqnarray*}
3^2+4^2&=&5^2 \\ 9+16&=&25 \\ 25&=&25.
\end{eqnarray*}
L'\'enonc\'e a) est vrai. La r\'eponse est donc a).\\

1590-- Lequel des \'enonc\'es suivants est faux?\\
a) $3^2=9$\\
b) $(-6)^2=36$ \\
c) $-6^2=-36$\\
d) $-6^2=36$\\

R\'eponse : d)\\

R\'etroaction :\\
En calculant la valeur de $-6^2$, on obtient $-6\times6=-36$, et non
36.
L'\'enonc\'e d) est faux.\\

1591-- Sylvain poss\`ede un jardin de forme carr\'ee dont l'aire est
de $12,96{\textrm{ m}}^2.$\\ Quelle est la mesure d'un des c\^ot\'es de ce
jardin?\\
a) $2,16$ m\\
b) $3,24$ m\\
c) $3,60$ m\\
d) $6,48$ m\\

R\'eponse : c)\\

R\'etroaction :\\[3mm]
La formule de l'aire d'un carr\'e est donn\'ee par $$A=c^2.$$ En
isolant $c$ dans l'\'equation pr\'ec\'edente, on obtient $$c=\sqrt
A.$$ Comme l'aire est de $12,96{\textrm{ cm}}^2$, on trouve que
$$c=\sqrt A=\sqrt
12,96=3,60{\textrm{ m.}}$$\\ La r\'eponse est donc c).\\

1592-- Parmi les expressions suivantes, identifie celle qui est vraie.\\
a) $\sqrt 3+\sqrt 4=\sqrt 7$\\
b) $\sqrt 3\cdot\sqrt 4=\sqrt 7$\\
c) $\sqrt 3\cdot\sqrt 4=\sqrt {12}$\\
d) Impossible, on ne peut fusionner deux racines carr\'ees ensemble.\\

R\'eponse : c)\\

R\'etroaction :\\
Pour faire le produit de deux racines, on peut calculer la racine du
produit des nombres sous les racines. $$\sqrt a \cdot \sqrt b =
\sqrt {a\cdot b}$$ L'\'enonc\'e c)
est vrai.\\

1593-- L'aire d'un rectangle dont la base mesure $8,32{\textrm{ m}}$
est de $386,88{\textrm{ m}}^2.$\,
Quel est le p\'erim\`etre de ce rectangle?\\
a) 46,5 cm\\
b) 54,82 cm\\
c) 109,64 cm\\
d) 186 cm\\

R\'eponse : c)\\

R\'etroaction :\\
On trouve la hauteur du rectangle \`a l'aide du calcul suivant :
$$\frac{386,88}{8,32}=46,5.$$ On a donc un rectangle de 46,5 cm par
8,32 cm. Par cons\'equent, le p\'erim\`etre est
$$2\cdot(46,5+8,32)=109,64
{\textrm{ cm}}^2.$$ La r\'eponse est c).\\

1594-- Tu veux recouvrir le plancher de ton salon avec des tuiles
carr\'ees mesurant chacune 20 cm de c\^ot\'e. Si les dimensions de
la pi\`ece sont de 4,2 m par 3,75 m, combien de tuiles te
faudra-t-il
pour recouvrir tout le plancher?\\
a) 79 tuiles\\
b) 383 tuiles\\
c) 394 tuiles\\
d) 7 875 tuiles\\

R\'eponse : c)\\

R\'etroaction :\\
D\'eterminons d'abord l'aire du plancher : $$A=4,2\times
3,75=15,75{\textrm{ m}}^2.$$ Chaque tuile mesure $0,2\times
0,2=0,04{\textrm{ m}}^2.$ Pour trouver le nombre de tuiles
n\'ecessaires, il faut donc diviser l'aire du plancher par l'aire
d'une tuile. $$\frac{15,75}{0,04}=393,75$$
On aura donc besoin de 394 tuiles pour recouvrir le plancher. La r\'eponse
est c).\\

1595-- L'aire d'un losange est de ${\textrm{14 421 m.}}^2$ Si la
petite diagonale de ce losange mesure $126,5{\textrm{
m}}$, quelle est la mesure de sa grande diagonale?\\
a) 114 m\\
b) 126,5 m \\
c) 228 m\\
d) 328 m\\

R\'eponse : c)\\

R\'etroaction :\\
L'aire d'un losange se calcule \`a l'aide de la formule suivante :
$$A=\frac{D\times d}{2}.$$
On obtient donc $${\textrm{14 421 }}=\frac{D\times 126,5}{2}.$$\\[3mm]
Il ne nous reste qu'\`a isoler $D$ pour trouver sa valeur.
\begin{eqnarray*}
{\textrm{14 421 }}&=&\frac{{D\times 126,5}}{2}\\[3mm]
{\textrm{28 842 }}&=&D\times 126,5 \\[3mm]
\frac{\textrm{28 842 }}{126,5}&=&D\\[3mm]
228&=&D \end{eqnarray*}\
La r\'eponse est c).\\

1596-- Si la base d'un triangle mesure 8 cm et sa hauteur 12
cm, quelle est l'aire de ce triangle?\\
a) $48{\textrm{ cm}}$\\
b) $48{\textrm{ cm}}^2$ \\
c) $96{\textrm{ cm}}$\\
d) $96{\textrm{ cm}}^2$ \\

R\'eponse : b)\\

R\'etroaction :\\
L'aire d'un triangle se calcule par la formule $$A=\frac{B\times
h}{2}.$$
On a donc $$A=\frac{8\times 12}{2}=48{\textrm{ cm}}^2.$$\\
La r\'eponse est b).\\

1597-- L'aire d'un trap\`eze est de $1\,012{\textrm{ cm}}^2.$  \,La
hauteur de ce trap\`eze mesure 23 cm et la petite base 38 cm. Quelle
est la mesure de la grande base?\\
a) 33 cm\\
b) 44 cm\\
c) 50 cm\\
d) 55 cm\\

R\'eponse : c)\\

R\'etroaction :\\
L'aire d'un trap\`eze est donn\'ee par $$A=\frac{(B+b)\cdot
h}{2}.$$\\
On obtient donc $$1\,012=\frac{(B+38)\cdot 23}{2},$$\\d'o\`u l'on tire
$$B=\frac{1\,012\cdot2}{23}-38=50{\textrm{ cm.}}$$\\
La r\'eponse est c).\\

1598-- Laquelle des formules suivantes permet de calculer
l'aire d'un trap\`eze?\\[3mm]
a) $\frac{{(B\times h)}}{2}$\\[3mm]
b) $\frac{{((B+b)\times h)}}{2}$\\[3mm]
c) $\frac{{D\times d}}{2}$\\[3mm]
d) $\frac{{(P\times{\textrm{ apoth\`eme}})}}{2}$\\[3mm]

R\'eponse : b)\\

R\'etroaction :\\
En a), on mentionne la formule permettant de calculer l'aire d'un
triangle. En c)6 est donn\'ee la formule d'aire d'un losange. En d),
on est en pr\'esence de la formule servant \`a  calculer l'aire d'un
polygone r\'egulier. Finalement, en b) se retrouve la formule d'aire
d'un
trap\`eze. La r\'eponse est b).\\

1599-- Lequel des \'enonc\'es suivants est faux?\\
a) Tous les diam\`etres d'un cercle sont de la m\^eme
longueur.\\
b) L'ensemble des points \`a \'egale distance d'un centre $O$
repr\'esente un
cercle.\\
c) Le rayon d'un cercle est toujours \'egal au double de son diam\`etre.\\
d) Plus le rayon d'un cercle est grand, plus le diam\`etre de ce cercle est
grand.\\

R\'eponse : c)\\

R\'etroaction :\\
Le double du rayon est \'egal au diam\`etre et non l'inverse. La r\'eponse
est c).\\

1600-- Quelle est l'aire d'un cercle de 8 cm de diam\`etre?\\
a) $8\pi$\\
b) $16\pi$\\
c) $64\pi$\\
d) $32\pi$\\

R\'eponse : b)\\

R\'etroaction :\\Comme le diam\`etre du cercle mesure 8 cm, le
rayon, pour sa part, mesure 4 cm. L'aire d'un cercle est donn\'ee
par $$A=\pi\cdot r^2.$$On a donc $$A=\pi\cdot
4^2=16\cdot\pi.$$\\
La r\'eponse est b).\\

1601-- Lequel des \'enonc\'es suivants permet de
calculer la circonf\'erence d'un cercle?\\
a) $\pi r^2$\\
b) $r\pi^2 $\\
c) $2\pi r$\\
d) $4\pi r$\\

R\'eponse : c)\\

R\'etroaction :\\
La formule permettant de calculer la circonf\'erence d'un cercle est
$$A=2\pi r.$$ Par cons\'equent, la r\'eponse est c).\\

1602-- Lequel des \'enonc\'es suivants est vrai?\\
a) Le diam\`etre d'un cercle repr\'esente aussi un axe de sym\'etrie
de ce cercle.\\
b) Le rayon d'un cercle repr\'esente aussi un axe de sym\'etrie de
ce cercle.\\
c) Pour un cercle de rayon 2, l'aire du cercle est plus grande que
sa circonf\'erence.\\
d) Pour un cercle de rayon 2, la circonf\'erence du cercle est plus grande
que son aire.\\

R\'eponse : a)\\

R\'etroaction :\\
Le diam\`etre d'un cercle passe par le milieu de ce cercle. Il le s\'epare
donc en deux parties identiques. On peut donc en conclure que le diam\`etre
du cercle est aussi un axe de sym\'etrie de ce cercle. La r\'eponse est
a).\\

1603-- Combien mesure l'angle au centre interceptant un arc
repr\'esentant 40\,\% de la circonf\'erence totale d'un cercle?\\
a) $40^{\circ}$\\
b) $140^{\circ}$\\
c) $144^{\circ}$\\
d) $216^{\circ}$\\

R\'eponse : c)\\

R\'etroaction :\\
Sachant que l'arc de cercle repr\'esente 40\,\% de la mesure de la
circonf\'erence, on sait aussi que l'angle au centre mesurera 40\,\%
de l'angle total qui est de $360^{\circ}$. $$40\,\% \cdot
360^{\circ}=144^{\circ}.$$ La r\'eponse est donc c).\\

1604-- Un angle au centre mesurant $66^{\circ}$ intercepte un certain
arc. Quel est la mesure de cet arc?\\
a) $33^{\circ}$\\
b) $66^{\circ}$\\
c) $132^{\circ}$\\
d) On ne peut le d\'eterminer car on ne peut trouver la
circonf\'erence du cercle.\\

R\'eponse : b)\\

R\'etroaction :\\
La mesure d'un arc et la mesure de l'angle au centre qui
l'intercepte sont toujours \'egales. La r\'eponse est donc b).\\

1605-- Ton professeur te demande d'exprimer le diam\`etre d'un cercle
en fonction de son rayon. Que lui r\'eponds-tu?\\
a) $d=2r$\\
b) $d=\frac{r}{2}$\\[3mm]
c) $d=r^2$\\
d) $d=r\cdot r$\\

R\'eponse : a)\\

R\'etroaction :\\
Le diam\`etre d'un cercle vaut le double de son rayon. La r\'eponse est donc
a).\\

1606--Parmi les expressions suivantes, laquelle repr\'esente le rayon
d'un cercle en fonction de sa cironf\'erence?\\
a) $r=\frac{C}{2}$\\[3mm]
b) $r=\frac{C}{2\pi}$\\[3mm]
c) $r=C-2\pi$\\[3mm]
d) $r=2\pi C$\\

R\'eponse : b)\\

R\'etroaction :\\
La formule de la circonf\'erence est $$C=2\pi r.$$
En isolant $r$ dans l'\'equation pr\'ec\'edente, on obtient
$$r=\frac{C}{2\pi}.$$ La r\'eponse est b).\\

1607-- Une roue d'automobile a un rayon de 0,325 m. Quelle distance
parcourt cette automobile si chaque roue effectue 25 000 tours
complets?\\
(Utilise l'approximation suivante : $\pi =3,14.$)\\
a) 8,29 km\\
b) 16,25 km\\
c) 25,51 km\\
d) 51,025 km\\

R\'eponse : d)\\

R\'etroaction :\\
Calculons d'abord la circonf\'erence de la roue.
$$C=2\cdot3,14\cdot0,325=2,041{\textrm{ m.}}$$
Comme elle fait 25 000 tours, la roue parcourt
$$2,041\cdot{\textrm{25 000}}={\textrm{51 025 m}}= 51,025{\textrm{ km.}}$$
La r\'eponse est d).\\

1608-- Lors d'un voyage, tu parcours 250 km. Tes pneus ont un rayon
de 30 cm. Combien de tours seront effectu\'es par chacune de tes
roues?\\
(Utilise l'approximation suivante : $\pi =3,14.$)\\
a) 8 846 tours \\
b) 66 348 tours\\
c) 132 696 tours\\
d) 416 667 tours\\

R\'eponse : c)\\

R\'etroaction :\\
Lorsque la roue fait 1 tour, elle parcourt exactement $2\cdot 3,14
\cdot 30=188,4{\textrm{ cm.}}$ On cherche combien de tours fait
cette roue lors d'un trajet de 25 000 000 cm.
$$\frac{\textrm{25 000 000 cm}}{\textrm{188,4 cm}}={\textrm{132 696.}}$$ La
roue fait
donc 132 696 tours pour un trajet de 250 km. La r\'eponse est c).\\

1609-- Quel est le diam\`etre d'un disque si son aire vaut
$25\pi{\textrm{ cm}}^2$?\\
a) 5 cm\\
b) $5\pi$ cm\\
c) 10 cm \\
d) 25 cm\\

R\'eponse : a)\\

R\'etroaction :\\
La formule pour calculer l'aire d'un disque est $$A=\pi r^2.$$
Sachant que l'aire vaut $25\pi{\textrm{ cm}}^2$, on a
$$r^2=\frac{25\pi}{\pi}=25.$$ $$r=\sqrt 25=5.$$ Le rayon vaut donc 5
cm. La r\'eponse est a).\\

1610-- Tu trouves un disque sur le sol. La circonf\'erence de ce
disque est de $36\pi{\textrm{ cm}}$. Quelle est son aire?\\
a) $18\pi^2{\textrm{ cm}}^2$\\
b) $18\pi {\textrm{ cm}}^2$\\
c) $324\pi {\textrm{ cm}}^2$\\
d) $1\,296\pi {\textrm{ cm}}^2$\\

R\'eponse : c)\\

R\'etroaction :\\
La formule de la circonf\'erence est $$C=2\pi r.$$Ainsi,
$$r=\frac{C}{2\pi}=\frac{36\pi}{2\pi}=18.$$L'aire de ce disque est
donc $$A=\pi r^2=\pi (18)^2=324\pi{\textrm{ cm}}^2.$$\ La r\'eponse
est c).\\

1611-- Martin s'entra\^ine pour les Jeux Olympiques et court sur une
piste circulaire. Quel est le rayon de cette piste s'il a parcouru
exactement 3 140 m en effectuant huit tours complets?\\
a) 11,2 m \\
b) 39,6 m \\
c) 62,5 m\\
d) 124,9 m\\

R\'eponse : c)\\

R\'etroaction :\\
Martin parcourt 3 140 m en faisant huit tours complets. Un tour de
piste est par cons\'equent \'egal \`a
$\frac{3\,140}{8}=393,5{\textrm{ m.}}$ La piste a donc une
circonf\'erence de 393,5 m. Trouvons maintenant le rayon de la
piste. $$C=2\pi r\Rightarrow
r=\frac{C}{2\pi}=\frac{393,5}{2\pi}=62,5{\textrm{ m.}}$$\\
La r\'eponse est c).\\

1612-- Tu piges une bille dans une bo\^ite contenant trois billes
bleues et deux rouges. Quelle est la probabilit\'e de tirer une bille
bleue?\\
a) $\frac{1}{5}$\\[3mm]
b) $\frac{1}{3}$\\[3mm]
c) $\frac{2}{5}$\\[3mm]
d) $\frac{3}{5}$\\[3mm]

R\'eponse : d)\\

R\'etroaction :\\
Comme dans la  bo\^ite se trouvent trois billes bleues sur un total
de cinq, on a trois chances sur cinq de piger une bille bleue. La r\'eponse
est donc d).\\

1613-- Tu lances un d\'e. Quelle est la probabilit\'e d'obtenir un
chiffre impair?\\
a) $\frac{1}{3}$\\[3mm]
b) $\frac{1}{2}$\\[3mm]
c) $\frac{1}{6}+ \frac{3}{6}+ \frac{5}{6}= \frac{9}{6}$\\[3mm]
d) $\frac{2}{6}$\\[3mm]

R\'eponse : b)\\

R\'etroaction :\\
Sur un d\'e, il y a exactement trois chiffres impairs et trois
pairs. Sur un total de six chiffres, il y en a donc trois qui sont
impairs. On a par cons\'equent trois chances sur six d'obtenir un
chiffre impair.
$$\frac{3}{6}=\frac{1}{2}$$\\
La r\'eponse est b). \vskip 35pt



\end{document}

